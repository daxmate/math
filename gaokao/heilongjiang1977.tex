\section[1977年高考数学试卷及答案(黑龙江)理科]{1977年普通高等学校招生考试(黑龙江卷)\\\Huge{理科数字}}

\begin{questions}
	\question 解答下列各题:
	\begin{enumerate}[label=(\arabic*)]
		\item 解方程:$\sqrt{3x+4} = 4$.
		      \begin{solution}
			      方程两边平方得:
			      \begin{math}
				      3x + 4 = 16
			      \end{math}
			      移项整理得
			      \begin{align*}
				      3x & = 12 \\
				      x  & = 4
			      \end{align*}

		      \end{solution}
		\item 解不等式:$|x| < 5$
		      \begin{solution}
			      \begin{math}
				      -5 < x < 5
			      \end{math}

		      \end{solution}
		\item 已知正三角形的外接圆半径为$6\sqrt{3}$cm,求它的边长.
		      \begin{solution}
			      \begin{tikzpicture}[scale=.3]
				      \tkzDefPoints{0/0/O, 6sqrt(3)/0/A}
				      \tkzDefPoint(90:6sqrt(3)){B}
				      \tkzDefPoint(210:6sqrt(3)){C}
				      \tkzDefPoint(330:6sqrt(3)){D}
				      \tkzDefLine[orthogonal=through O](B,C) \tkzGetPoint{x}
				      \tkzInterLL(O,x)(B,C) \tkzGetPoint{E}

				      \tkzDrawCircle(O,A)
				      \tkzDrawPolygon(B,C,D)
				      \tkzDrawPoint(O)
				      \tkzLabelPoint[below right](O){$O$}
				      \tkzDrawSegments(B,O C,O O,E)

				      \tkzMarkSegment[dim={$6\sqrt{3}$, 16pt, right=6pt}, mark=](B,O)
				      \tkzMarkAngle[size=2](C,B,O)
				      \tkzLabelAngle[pos=3.5](C,B,O){$30^\circ$}
				      \tkzLabelPoint[left](E){$E$}
				      \tkzLabelPoint[above](B){$B$}
			      \end{tikzpicture}
			      可以计算出$BE=6$,则正三角形的边长为$12$.
		      \end{solution}

	\end{enumerate}

	\question 计算下列各题:
	\begin{enumerate}[label=(\arabic*)]
		\item $\sqrt{m^2 - 2ma + a^2} $
		      \begin{solution}
			      \begin{align*}
				      \sqrt{m^2 - 2ma + a^2} & = \sqrt{(m-a)^2} \\
				                             & = |m-a|
			      \end{align*}
		      \end{solution}
		\item $\cos78^\circ\cdot\cos3^\circ + \cos12^\circ\cdot\sin3^\circ$.
		      \begin{solution}
			      \begin{align*}
				      \cos78^\circ & = \cos(90^\circ - 12^\circ)                           \\
				                   & = \cos90^\circ\cos12^\circ + \sin90^\circ\sin12^\circ \\
				                   & = \sin12^\circ
			      \end{align*}
			      代入原式得:
			      \begin{align*}
				      \cos78^\circ\cdot\cos3^\circ + \cos12^\circ\cdot\sin3^\circ
				       & = \sin12^\circ\cos3^\circ + \cos12^\circ\sin3^\circ \\
				       & = \sin15^\circ                                      \\
				       & = \sin\left(\frac{30^\circ}{2}\right)               \\
				       & = \sqrt{\frac{1-\cos30^\circ}{2}}                   \\
				       & = \frac{\sqrt{2-\sqrt{3}}}{2}
			      \end{align*}
		      \end{solution}
		\item $\arcsin\left(\cos\dfrac\pi6\right)$.
		      \begin{solution}
			      \begin{align*}
				      \arcsin\left(\cos\dfrac\pi6\right) & = \arcsin(\dfrac{\sqrt{3}}2) \\
				                                         & = \frac{\pi}{3}
			      \end{align*}
		      \end{solution}
	\end{enumerate}
	\question 解下列各题:
	\begin{enumerate}[label=(\arabic*)]
		\item 解方程: $3^{x+1} - 9^{\frac{x}{2}} = 18$.
		      \begin{solution}
			      \begin{align*}
				      3^{x+1} - 9^{\frac{x}{2}}        & = 18 \\
				      3\cdot 3^x - (3^2)^{\frac{x}{2}} & = 18 \\
				      3^x                              & = 9  \\
				      x                                & = 2
			      \end{align*}
		      \end{solution}
		\item 求数列$2,4,8,16,\cdots $前十项的和.
		      \begin{solution}
			      \begin{align*}
				      S_n    & = a_0\frac{1-q^n}{1-q}, (q=2, a_0=2) \\
				      S_{10} & = 2\frac{1-2^{10}}{1-2}              \\
				             & = 2^{11} - 2                         \\
				             & = 2046
			      \end{align*}
		      \end{solution}
	\end{enumerate}

	\question 解下列各题:
	\begin{enumerate}[label=(\arabic*)]
		\item 圆锥的高为$6$cm,母线和底面半径的夹角为$30^\circ$,求它的侧面积.
		      \begin{solution}
			      根据题目中提供的信息可计算得底面半径为$r = 6\sqrt{3}$,母线的长度为$l = 12$,则侧面积为
			      \begin{math}
				      S = \pi r l = 72\sqrt{3} \text{cm}^2
			      \end{math}
		      \end{solution}
		\item 求过点$(1,4)$且与直线$2x - 5y + 3 = 0$垂直的直线方程.
		      \begin{solution}
			      直线$2x - 5y + 3 + 0$的斜率$k=\frac25$,则与之垂直的直线的斜率为$-\frac52$.

			      设所求直线方程为$y =
				      -\frac52x+b$, 并将点$(1,4)$代入得$b = \frac{13}{2}$.
			      所以直线方程为:$2y + 5x - 13 = 0$.
		      \end{solution}
	\end{enumerate}

	\question 如果$\triangle{ABC}$的$\angle{A}$的平分线交$BC$于$D$,交它的外接圆于$E$,那么$AB\cdot AC = AD \cdot AE$.
	\begin{figure*}[htbp]
		\centering
		\begin{tikzpicture}
			\tkzDefPoints{0/0/A, 3/0/B, 2.5/2.5/C}
			\tkzDefLine[bisector](B,A,C) \tkzGetPoint{x}
			\tkzDefCircle(A,B,C) \tkzGetPoint{O}
			\tkzInterLL(A,x)(B,C) \tkzGetPoint{D}
			\tkzInterLC(A,x)(O,A) \tkzGetSecondPoint{E}

			\tkzDrawPolygon(A,B,C)
			\tkzDrawCircle(O,A)
			\tkzDrawSegment(A,E)

			\tkzLabelPoint[left](A){$A$}
			\tkzLabelPoint[right](B){$B$}
			\tkzLabelPoint[above](C){$C$}
			\tkzLabelPoint[above left](D){$D$}
			\tkzLabelPoint[right](E){$E$}

			\tkzDrawSegments[red, dashed](C,E E,B)

		\end{tikzpicture}
	\end{figure*}
	\begin{solution}
		\begin{mathenum}
			\item 作辅助线$CE$和$BE$
			\item \because $\overset{\frown}{CE}$对应圆周角$\angle{CAE}$和$\angle{CBE}$
			\\ \therefore\ $\angle{CAE} = \angle{CBE}$
			\\ \because\ $\angle{CDA} = \angle{EDB} $
			\\ \therefore\ $\triangle{CAD} \sim \triangle{EBD}$
			\item \because\ $\overset{\frown}{BE}$对应圆周角$\angle{BAE} = \angle{BCE}$
			\\ \therefore\ $\angle{BAE} = \angle{EAC}$
			\\ \because $\angle{BAE} = \angle{BAC}$
			\\ \therefore\ $\overset{\frown}{CE} = \overset{\frown}{BE}$
			\\ \therefore\ $\angle{BCE} = \angle{CBE}$
			\\ \therefore\ $\angle{EBD} = \angle{BAE}$
			\\ \because\ $\angle{BEA} = \angle{AEB}$
			\\ \therefore\ $\triangle{EBD} \sim \triangle{EAB}$
			\\ \therefore\ $\triangle{EAB} \sim \triangle{CAD}$
			\\ \therefore\ $\dfrac{AC}{AE}={AD}{AB}$
			\\ \therefore\ $AC\cdot AB = AD\cdot AE$
		\end{mathenum}

	\end{solution}
	\question
	前进大队响应毛主席关于\enquote{绿化祖国}的伟大号召,1975年造林$200$亩,又知1975年到1977年三年内共造林$728$亩,求后两年造林面积的年平均增长率是多少?
	\begin{solution}
		设年平均增长率为$q$,根据等比数列求和公式
		\begin{math}
			S_n = a_0\frac{1-q^n}{1-q} (a_0 = 200, n = 3, S_3 = 728)
		\end{math}

		代入得:
		\begin{math}
			728 = 200 \frac{1-q^3}{1-q} = 200 \frac{(1-q)(1 + q + q^2)}{1-q} = 200(1 + q + q^2)
		\end{math}
		化简得:
		\begin{align*}
			25q^2 + 25q - 66  & = 0 \\
			(5q - 6)(5q + 11) & = 0 \\
		\end{align*}
		因为$q > 0$所以$q=1.2$
	\end{solution}

	\question 解方程:
	\begin{math}
		\lg(2^x + 2x - 16) = x(1-\lg5)
	\end{math}.
	\begin{solution}
		\begin{align*}
			\lg(2^x + 2x - 16) & = x(\lg10 - \lg5) \\
			                   & = x\lg2           \\
			                   & = \lg2^x          \\
			2^x + 2x - 16      & = 2^x             \\
			2x - 16            & = 0               \\
			x                  & = 8
		\end{align*}
	\end{solution}

	\question 已知三角形的三边成等差数列,周长为$80$cm,面积为$54$cm$^2$,求三边的长.
	\begin{solution}
		% \begin{tikzpicture}[scale=.5,baseline=(current bounding box.north)]
		% 	\tkzDefPoints{0/0/A, 12/0/B}
		% 	\tkzDefPoint(75:8){C}
		% 	\tkzDefLine[orthogonal=through A](C,B) \tkzGetPoint{x}
		% 	\tkzInterLL(C,B)(A,x) \tkzGetPoint{D}

		% 	\tkzDrawPolygon(A,B,C)
		% 	\tkzMarkSegment[dim={$12$cm, 16pt, above=6pt}, mark=](C,B)
		% 	\tkzMarkSegment[dim={$(12+x)$cm, -16pt, above=6pt}, mark=](A,B)
		% 	\tkzMarkSegment[dim={$(12-x)$cm, 16pt, above=6pt}, sloped, mark=](A,C)
		% 	\tkzDrawSegment(A,D)
		% 	\tkzMarkRightAngle(A,D,B)
		% 	\tkzMarkSegment[dim={$y$cm, 16pt, above=6pt}, sloped, mark=](D,A)

		% 	\tkzLabelPoint[below left](A){$A$}
		% 	\tkzLabelPoint[right](B){$B$}
		% 	\tkzLabelPoint[above](C){$C$}
		% 	\tkzLabelPoint[left](D){$D$}
		% \end{tikzpicture}

		由三边成等差数列,可得中间长度的边长为$12$cm,设另外两边分别为$(12+x)$cm和$(12-x)$cm.由海伦公式可得面积为:
		\begin{math}
			S = \sqrt{p(p-a)(p-b)(p-c)} = 54\; \text{其中} p = \frac{36}{2} = 18
		\end{math}
		代入可得:
		\begin{align*}
			S = \sqrt{18(18 - 12 + x)(18 - 12 - x)(18 - 12)} & = 54              \\
			\sqrt{18\cdot 6 (36-x^2)}                        & = 54              \\
			18\cdot 6 (36- x^2)                              & = 54^2            \\
			x^2                                              & = 9               \\
			x                                                & = 3 (\mathrm{cm})
		\end{align*}

	\end{solution}

	\begin{center}
		\subsubsection*{附加题}
	\end{center}

	\question
	如图,$AP$表示发动机的连杆,$OA$表示它的曲柄.当$A$在圆上作圆周运动时,$P$在$x$轴上作直线运动,求$P$点的横坐标.为什么当$\alpha$是直角时,$\angle{P}$是最大?

	\begin{tikzpicture}
		\tkzInit[xmax=4, xmin=0]
		\tkzDrawX
		\tkzDefPoints{0/0/O, 3/0/P, 1/1.5/A}

		\tkzDrawCircle(O,A)
		\tkzDrawSegment(O,A)
		\tkzMarkAngle[size=.5](P,O,A)
		\tkzLabelAngle(P,O,A){$\alpha$}

		\tkzDrawSegment(A,P)
		\tkzLabelSegment[above](A,P){$l$}
		\tkzLabelSegment[left](O,A){$R$}

		\tkzLabelPoint(O){$O$}
		\tkzLabelPoint(P){$P$}
		\tkzLabelPoint[above](A){$A$}
	\end{tikzpicture}
	\begin{solution}
		根据三角关系有$\sin\alpha = \frac{l}{OP}$,所以有$OP =
			\frac{l}{\sin\alpha}$,即$P$点的横坐标为$\frac{l}{\sin\alpha}$

		另有$\sin\angle{P} = \frac{R}{OP} =
			\frac{R\sin\alpha}{l}$,可以得出在$\alpha$为直角时$\sin\angle{P}$有最大值,即$\angle{P}$有最大值.
	\end{solution}
	\question 求曲线$y=\sin{x}$在$[0,\pi]$上的曲边梯形绕$x$轴旋转一周所形成的旋转体的体积.
	\begin{solution}
		\begin{align*}
			V & = \int_0^{\pi}\pi \sin^2{x}\mathrm{d}x                              \\
			  & = \pi\int_0^{\pi}\frac{1-\cos{2x}}{2}\mathrm{d}x                    \\
			  & = \pi(\left[\frac{x}{2}\right]_0^{\pi} - \frac12[\sin(2x)]_0^{\pi}) \\
			  & = \pi\frac{\pi}{2}
			  & = \frac{\pi^2}{2}
		\end{align*}
	\end{solution}

\end{questions}
