%!tex program = lualatex
\documentclass[answers]{exam}
% 横向超过10pt之内不警告
\hfuzz=10pt
% 纵向超过10pt之内不警告
\vfuzz=10pt
% 将页面风格设置为空,去除页眉页脚和页码等
\pagestyle{empty}
% exam文档中题解的标题设置
\renewcommand{\solutiontitle}{\textbf{解:}}
% exam文档part的标签样式
\renewcommand{\thepartno}{\arabic{partno}}
% 修改选项标号的样式
\renewcommand{\choicelabel}{(\Alph{choice})}
% 设置填空题答案与横线之间的间距
\setlength{\answerclearance}{6pt}
% hfs用于添加水平空白于选择题干与答案之间,最后加上一个2cm的括号
\newcommand\hfs{\hfill (\hspace{2cm})}
% 设置正确答案的样式
\CorrectChoiceEmphasis{\color{magenta}\bfseries}
% 汉化
% 因为文档类型本身的限制,在有多个questions环境时,计分表无法使用
% TODO 以后看能不能解决这个问题
\pointname{分}
\hsword{得分}
\hqword{题目}
\hpword{分值}
\htword{总计}
\newenvironment{proofsolution}
% 定义 proofsolution 环境
{\renewcommand{\solutiontitle}{\textbf{证明:}}\begin{solution}} % 环境开始部分
{\end{solution}}            % 环境结束部分
% 添加中式有点倾斜的平行符号
\renewcommand{\parallel}{
	\mathrel{ %设置平行符号为数学中的关系符
		\tikz \draw[baseline=(base)] (0,0) -- (70:.8em) [xshift=2pt](0,0) -- (70:.8em);
	}
}

% 屏蔽掉一些不影响的警告
\usepackage{silence}
\WarningFilter{latexfont}{Font shape}
\WarningFilter{latexfont}{Some font}
\WarningFilter{latex}{Label `question}
\WarningFilter{latex}{There were multiply-defined labels}
\WarningFilter{latex}{Label `part}
\WarningFilter{latex}{Command `\theHfootnote'}
\usepackage{
ctex, 	 		% 中文支持
amsmath, 		% ams数学包
amssymb, 		% ams数学符号
amsthm,  		% ams定理环境
csquotes,		% 处理引号
arcs,	 		% 添加弧的符号
enumitem,		% 列表环境增强
tikz,	 		% 绘图工具
pgfplots,		% 绘制函数曲线
tikz-3dplot, 	% 3D 图像
tkz-euclide, 	% 简单几何绘图
siunitx,	 	% 数字单位处理
titlesec,	 	% 标题设置
caption,	 	% 图表标题
multicol,	 	% 多列环境
hyperref, 	 	% 处理超链接
tocloft,	 	% 格式化目录
}
% 修改section的标题样式
\titleformat{name=\section}
  {\centering\bfseries} % 格式:居中、大号字体、加粗
  {}              		% 编号样式
  {0pt}                 % 编号与标题间距
  {}                    % 标题前无额外内容
\setcounter{secnumdepth}{0} % 取消section的计数
\renewcommand{\cftsecleader}{\cftdotfill{\cftdotsep}} % 设置目录中用点线连接目录内容与页码
\renewcommand{\cfttoctitlefont}{\hfill\Huge\bfseries} % 设置目录标题的样式
% 在solution环境内重置公式计数器
\NewCommandCopy\oldsolution\solution
\NewCommandCopy\oldendsolution\endsolution
\renewenvironment{solution}
{\setcounter{equation}{0}
\oldsolution}
{\oldendsolution}
% 重新定义eqref命令
\renewcommand{\eqref}[1]{式~(\ref{#1})}
% 允许公式环境换页
% \allowdisplaybreaks

% 定义sgn为操作符
\DeclareMathOperator{\sgn}{sgn}

% 设置页面的边距为2cm
\usepackage[margin=2cm]{geometry}

% 定义一个数字加点标号的列表环境
\newenvironment{mathenum}
{\begin{enumerate}[label=\arabic*.]}
		{\end{enumerate}}

% 定义一个括号数字标号的列表环境
\newenvironment{penum}
{\begin{enumerate}[label=(\arabic*)]}
		{\end{enumerate}}

% 解决弧标记显示的问题
% TODO 这个要了解一下有没有其他实现的方式
\makeatletter
\providecommand\@gobblethree[3]{}
\patchcmd{\over@under@arc}
{\@gobbletwo}
{\@gobblethree}
{}{}
\makeatother


% 带圈数字用tikz实现
\newcommand*\circled[1]{\tikz[baseline=(char.base)]{
		\node[shape=circle,draw,inner sep=1pt] (char) {#1};}}
% 定义一个带圈数字标号的列表环境
\newenvironment{cenum} {\begin{enumerate}[label=\protect\circled{\arabic*}]} {\end{enumerate}}

% 加载tikz库
\usetikzlibrary{
	angles,
	backgrounds,
	calc,
	decorations.pathmorphing,
	decorations.pathreplacing,
	decorations.text,
	intersections,
	patterns,
	petri,
	positioning,
	quotes,
	shapes,
	shapes.symbols,
}
\pgfplotsset{
compat=1.18, 		 		% 设置pgfplots兼容最新的功能
axis lines = middle, 		% 坐标轴原点在中心
ticks = none,  		 		% 不显示刻度
xlabel = $x$, ylabel = $y$, % 坐标轴符号
xlabel style={anchor=west},
ylabel style={anchor=south}, % 坐标轴符号的位置
samples = 100, 				% 采样率,数目越大曲线越平滑
}

% 修复pgfplots的fillbetween出现missing characters的警告
\makeatletter
\patchcmd\pgfplotsforeachtodomain@@{\ifnum}{\ifdim}{}{\fail}
\makeatother
\usepgfplotslibrary{
	fillbetween,  		% 在两个曲线之间填充
}

\hypersetup{
	hidelinks,
}



% TODO 设计封面样式
% \title{历年高考数学试卷及答案}
% \author{收集于网络 by 大象同学}

\includeonly{tianjing1977}
\begin{document}
% \maketitle
\begin{titlepage}
	\centering

	\begin{tikzpicture}[remember picture, overlay]
		% 1. 绘制柔和背景
		\fill[green!10] (current page.south west) rectangle (current page.north east); % 浅绿背景

		% 2. 添加几何图案
		% 左侧三角形
		% \fill[orange!30] (current page.south west) -- ++(6,18) -- ++(-6,0) -- cycle;
		% 右侧三角形
		% \fill[yellow!30] (current page.south east) -- ++(-6,18) -- ++(6,0) -- cycle;

		% 圆形装饰
		\draw[fill=cyan, draw=none, opacity=.25] ($(current page.center)+(5,-5)$) circle (3cm);
		\draw[fill=magenta, draw=none, rounded corners=25pt, opacity=.15] ($(current page.center)+(-5,-8)$) rectangle +(5cm, 3cm);
		\begin{scope}[shift=($(current page.west)$)]
			\tkzDefPoints{0/0/c,1/0/d,3/0/a0}
			\def\tkzRadius{1}
			\tkzDrawCircle(c,d)
			\foreach \an in {0,10,...,350}
				{
					\tkzDefPointBy[rotation=center c angle \an](a0) \tkzGetPoint{a}
					\tkzDefLine[tangent from = a](c,d)
					\tkzGetPoints{e}{f}
					\tkzDrawLines[white, very thick](a,f a,e)
					\tkzDrawSegments[white, very thick](c,e c,f)
				}
		\end{scope}

		\begin{scope}[yshift=-4cm]
			\tkzInit[ymin=-3,ymax=5,xmin=-5,xmax=7]
			\tkzClip
			\tkzDefPoints{-2.5/-2/A,2/4/B,5/-1/C}
			\tkzFindAngle(C,A,B)
			\tkzGetAngle{anglea}
			\tkzDefPointBy[rotation=center A angle 1*\anglea/3](C)
			\tkzGetPoint{TA1}
			\tkzDefPointBy[rotation=center A angle 2*\anglea/3](C)
			\tkzGetPoint{TA2}
			\tkzFindAngle(A,B,C)
			\tkzGetAngle{angleb}
			\tkzDefPointBy[rotation=center B angle 1*\angleb/3](A)
			\tkzGetPoint{TB1}
			\tkzDefPointBy[rotation=center B angle 2*\angleb/3](A)
			\tkzGetPoint{TB2}
			\tkzFindAngle(B,C,A)
			\tkzGetAngle{anglec}
			\tkzDefPointBy[rotation=center C angle 1*\anglec/3](B)
			\tkzGetPoint{TC1}
			\tkzDefPointBy[rotation=center C angle 2*\anglec/3](B)
			\tkzGetPoint{TC2}
			\tkzInterLL(A,TA1)(B,TB2)
			\tkzGetPoint{U1}
			\tkzInterLL(A,TA2)(B,TB1)
			\tkzGetPoint{V1}
			\tkzInterLL(B,TB1)(C,TC2)
			\tkzGetPoint{U2}
			\tkzInterLL(B,TB2)(C,TC1)
			\tkzGetPoint{V2}
			\tkzInterLL(C,TC1)(A,TA2)
			\tkzGetPoint{U3}
			\tkzInterLL(C,TC2)(A,TA1)
			\tkzGetPoint{V3}
			\tkzDrawPolygons(A,B,C U1,U2,U3 V1,V2,V3)
			\tkzDrawLines[add=2 and 2,very thin,dashed](A,TA1 B,TB1 C,TC1 A,TA2 B,TB2 C,TC2)
			\tkzDrawPoints(U1,U2,U3,V1,V2,V3)
			\tkzLabelPoint[left](V1){$s_a$}
			\tkzLabelPoint[right](V2){$s_b$}
			\tkzLabelPoint[below](V3){$s_c$}
			\tkzLabelPoint[above left](A){$A$}
			\tkzLabelPoints[above right](B,C)
			\tkzLabelPoint(U1){$t_a$}
			\tkzLabelPoint[below left](U2){$t_b$}
			\tkzLabelPoint[above](U3){$t_c$}
		\end{scope}

		% \begin{scope}[yshift=-17cm, xshift=-5cm, color=green!30!red!30]
		% 	\begin{axis}[axis lines=middle,
		% 			ticks = none,
		% 			xmin = -1.5*pi,
		% 			xmax = 1.5*pi,
		% 			ymin = -1.5,
		% 			ymax = 1.5,
		% 			xlabel = $x$,
		% 			ylabel = $y$,
		% 			xlabel style={anchor=west},
		% 			ylabel style={anchor=south},
		% 			samples=100,
		% 		]
		% 		\addplot[domain=-pi:pi]{sin(deg(x))};
		% 		\path node[below] at ({pi}, 0){$\pi$};
		% 		\path node[above] at ({-pi}, 0){$-\pi$};
		% 	\end{axis}
		% \end{scope}

		\tdplotsetmaincoords{65}{165}
		\begin{scope}[tdplot_main_coords, yshift=-16cm, scale=2.3, color=black!55]
			\draw[tdplot_screen_coords] (0,0) circle [radius=1];
			\foreach \t in {0, 10, ..., 350}{
					\tdplotsetrotatedcoords{90}{\t}{0}
					\draw[tdplot_rotated_coords,very thin] (0,0) circle [radius=1];
					\tdplotsetrotatedcoords{180}{90}{0}
					\draw[tdplot_rotated_coords,very thin] (0,0,{sin(\t)}) circle [radius={cos(\t)}];
				}
			\clip[tdplot_screen_coords] (-5,-3) rectangle (5,3);
			\draw[tdplot_screen_coords,white] (-4.9,-2.9) rectangle (4.9,2.9);
			\foreach \k in {10,20,...,80}{
					% steiner circles on the plane
					\draw[] ({1/cos(\k)},0) circle [radius={sqrt(1/(cos(\k)^2)-cos(\k))}];
					\draw[] (0,{-sin(\k)/cos(\k)}) circle [radius={1/cos(\k)}];
				}
			\foreach \k in {100,110,...,170}{
					% steiner circles on the plane
					\draw[] ({1/cos(\k)},0) circle [radius={sqrt(1/(abs(cos(\k))^2)-abs(cos(\k)))}];
					\draw[] (0,{-sin(\k)/cos(\k)}) circle [radius={1/cos(\k)}];
				}
			\draw[] (0,0) circle [radius=1]; % central steiner circle
			\draw[] (0,-7,0) -- (0,7,0); % circle at infinity
			\draw[] (-7,0,0) -- (7,0,0); % circle at infinity
			% axes
			\draw[-latex,thick] (-2,0,0) -- (2,0,0) node[pos=1,below left]{$x,\xi$}; % x-axis
			\draw[-latex,thick] (0,-3.5,0) -- (0,3.5,0) node[pos=1,below right]{$y,\eta$}; % y-axis
			\draw[-latex,thick] (0,0,-2) -- (0,0,1.5) node[pos=1,above right]{$z,\zeta$}; % z-axis
		\end{scope}

		% 3. 添加代数方程
		\node[anchor=south west, text=blue!50!black, font=\large, rotate=30] at ($(current page.center)+(-7,5)$) {
			$y = ax^2 + bx + c$
		};
		\node[anchor=north east, text=black!55, font=\zihao{1}, rotate=90] at ($(current page.center)+(5,-3)$) {
			$e^{i\pi}+1=0$
		};
		\node[anchor=north west, text=blue!50!black, font=\large] at ($(current page.center)+(-6,-6)$) {
			$\int_a^b f(x)\,\mathrm{d}x$
		};

		% 4. 添加标题文字
		\node [
			text=green!50!blue,
			font=\bfseries\zihao{0},
			align=center
		] at ($(current page.center)+(0,4)$) {历年高考数学真题集};

		% 副标题文字
		\node [
			text=black,
			font=\Large,
			align=center
		] at ($(current page.center)+(0,1.5)$) {精选全国各省市试卷};

		% 出版社信息
		\node [
			text=gray,
			font=\normalsize,
			align=center
		] at ($(current page.center)+(0,-10)$) {大象同学整理};

		% 5. 添加装饰线条
		% \draw[line width=1mm, blue!50] ($(current page.center)+(-8,-8)$) -- ($(current page.center)+(8,8)$);
		% \draw[line width=1mm, orange!50] ($(current page.center)+(-8,8)$) -- ($(current page.center)+(8,-8)$);
	\end{tikzpicture}
\end{titlepage}
\thispagestyle{empty}
\setlength{\columnsep}{1cm}
% \begin{multicols}{2}
% TODO 设置目录样式
{\color{cyan}
	\tableofcontents
	\thispagestyle{empty}
}
% \end{multicols}

%!tex program = lualatex
\documentclass[answers]{exam}
\usepackage{ctex}
\usepackage{graphicx}
\usepackage[margin=2cm]{geometry}
\usepackage{amsmath, amssymb}
\usepackage{csquotes}
\usepackage{tikz, pgfplots}
\usetikzlibrary{
	angles,
	backgrounds,
	calc,
	decorations.pathmorphing,
	decorations.pathreplacing,
	decorations.text,
	intersections,
	patterns,
	quotes,
	shapes,
	shapes.symbols,
}
\pagestyle{empty}
\newcounter{xcord}
\newcounter{ycord}
\newcounter{total}
\renewcommand{\labelenumi}{\textbf{\ifnum\value{enumi}<10 0\fi\arabic{enumi})}}

\pgfplotsset{compat=1.18}

\CorrectChoiceEmphasis{\color{blue!70!green}\bfseries}
\renewcommand{\solutiontitle}{\textbf{解:}}

\usepackage{array, tabularx}
\newcolumntype{C}{>{\centering\arraybackslash}X}
\newcolumntype{B}{>{\centering\bfseries\arraybackslash}X}
\catcode`\幺=0

\begin{document}
\begin{center}
	\textbf{\large{1977 年普通高等学校招生考试(北京卷) }}

	\textbf{\LARGE{理科数学}}
\end{center}
\begin{questions}
	\question 解方程: \( \sqrt{x - 1} = 3 - x \).
	\begin{solution}
		\begin{align*}
			\sqrt{x - 1}  & = 3 - x        \\
			x - 1         & = 9 - 6x + x^2 \\
			x^2 - 7x + 10 & = 0            \\
			(x-2)(x-5)    & = 0            \\
			x_1 = 2, x_2 = 5
		\end{align*}
	\end{solution}

	\question 计算: \( 2^{-\frac12} + \frac{2^0}{\sqrt{2}} + \frac{1}{\sqrt{2} - 1}. \)

	\begin{solution}
		\begin{align*}
			\text{原式} & = \frac{1}{\sqrt{2}} + \frac{1}{\sqrt{2}} + \sqrt{2} + 1 \\
			          & = \frac{2}{\sqrt{2}} + \sqrt{2} + 1                      \\
			          & = 2\sqrt{2} + 1
		\end{align*}
	\end{solution}

	\question 已知 \( \lg2 = 0.3010, \lg3 = 0.4771, \) 求 \( \lg\sqrt{45} \).

	\begin{solution}
		\begin{align*}
			\lg\sqrt{45} & = \frac12\lg{45}                        \\
			             & = \frac12(\lg5 + \lg9)                  \\
			             & = \frac12(\lg5 + 2\lg3)                 \\
			             & = \frac12(\lg\frac{10}{2} + 2\lg3)      \\
			             & = \frac12(\lg10 - \lg2 + 2\lg3)         \\
			             & = \frac12(1 - 0.3010 + 2 \times 0.4771) \\
			             & = 0.8266
		\end{align*}
	\end{solution}

	\question 证明: \( (1 + \tan\alpha)^2 = \dfrac{1 + \sin2\alpha}{\cos^2\alpha} \)。
	\begin{solution}
		\begin{align*}
			(1+\tan\alpha)^2
			           & = \left(1 + \frac{\sin\alpha}{\cos\alpha}\right)^2  \text{(将 \(\tan\alpha = \frac{\sin\alpha}{\cos\alpha}\) 代入)} \\
			           & = \left(\frac{\cos\alpha + \sin\alpha}{\cos\alpha}\right)^2 \text{(通分并化简分母)}                                     \\
			           & = \frac{(\cos\alpha + \sin\alpha)^2}{\cos^2\alpha}                                                               \\
			           & = \frac{\cos^2\alpha + 2\cos\alpha\sin\alpha + \sin^2\alpha}{\cos^2\alpha}                                       \\
			           & = \frac{1 + 2\cos\alpha\sin\alpha}{\cos^2\alpha}                                                                 \\
			           & \text{(根据三角恒等式 \(\cos^2\alpha + \sin^2\alpha = 1\) 化简)}                                                          \\
			           & = \frac{1 + \sin2\alpha}{\cos^2\alpha}                                                                           \\
			           & \text{(利用 \(\sin2\alpha = 2\cos\alpha\sin\alpha\))}                                                              \\
			\therefore & \, (1+\tan\alpha)^2 = \frac{1 + \sin2\alpha}{\cos^2\alpha}。
		\end{align*}
	\end{solution}

	\question 求过两直线 \( x + y - 7 = 0 \) 和 \( 3x - y - 1 = 0 \) 的交点且过 \( (1, 1) \) 点的直线方程。
	\begin{solution}
		\begin{align*}
			 & \begin{cases}
				   x + y - 7 = 0, \\
				   3x - y - 1 = 0
			   \end{cases}                                                                \\
			 & \text{解得:} x = 2, \, y = 5,\text{即交点为 } (2, 5)。                               \\
			 & \text{过点 } (2,5) \text{ 和 } (1,1) \text{ 的直线斜率:} k = \frac{5 - 1}{2 - 1} = 4。 \\
			 & \text{设直线方程为:} y = 4x + a,\text{将 } (1,1) \text{ 代入,得:}                       \\
			 & a = -3。                                                                       \\
			 & \text{因此,所求直线方程为:} y = 4x - 3。
		\end{align*}
	\end{solution}

	\question 某工厂今年七月份的产值为 \( 100 \)万元,以后每月产值比上月增加 \( 20\%
	\),问今年七月份到十月份总产值是多少?

	\begin{solution}
		\begin{align*}
			\text{根据等比数列求和公式} S_n = a_1\frac{1- q^n}{1-q}                    \\
			\text{代入}  a_1 = 100, q = 1.2, n = 4  \text{得:}                  \\
			S_4 & = 100 \times \frac{1 - 1.2^4}{1 - 1.2}                     \\
			    & = 100 \times \frac{(1 - 1.2)(1 + 1.2)(1 + 1.2^2)}{1 - 1.2} \\
			    & = 100 \times 2.2 \times 2.44                               \\
			    & = 536.8 \text{(万元)}
		\end{align*}

	\end{solution}

	\question 已知二次函数 \( y = x^2 - 6x + 5 \)
	\begin{enumerate}[label=(\arabic*)]
		\item 求出它的图像的顶点坐标和对称轴方程;
		\item 画出它的图像;
		\item 分别求出它的图像和 \( x \)轴、$ y $的交点坐标.
	\end{enumerate}

	\begin{solution}
		\begin{enumerate}[label=(\arabic*)]
			\item
			      \begin{align*}
				      \text{对称轴的方程为} x = -\frac{b}{2a} & = 3,  \\
				      \text{将} x = 3 \text{代入得:} y     & = -4, \\
			      \end{align*}
			\item
			      \begin{center}
				      \begin{tikzpicture}
					      \begin{axis}[
							      axis lines=middle,          % 坐标轴通过原点
							      xlabel={$x$}, ylabel={$y$}, % 坐标轴标签
							      grid=major,                 % 显示网格
							      xmin=-2, xmax=8,            % x轴范围
							      ymin=-5, ymax=10,           % y轴范围
							      domain=-4:8,                % 定义函数绘制范围
							      samples=100,                % 样本点数,越大曲线越平滑
							      enlargelimits=true,         % 扩展范围
						      ]
						      % 绘制抛物线 y = ax^2 + bx + c
						      \addplot[smooth, thick, blue] {x^2 - 6*x + 5};
						      % 添加图例
						      \addlegendentry{$y = x^2 - 6x + 5$}
					      \end{axis}
				      \end{tikzpicture}
			      \end{center}

			\item \begin{align*}
				      y & = x^2 - 6x + 5                            \\
				        & = (x - 1)(x - 5)                          \\
				        & \text{则有两个交点分别为:} (1, 0) \text{和} (5, 0).
			      \end{align*}
		\end{enumerate}

	\end{solution}

	\question 一只船以 $20$ 海里/小时的速度向正东航行, 起初船在 $A$处看见一灯塔 $B$ 在船的北 $45^\circ$ 东方向, 一小时后船在
	$C$ 处看见这个灯塔在船的北 $15^\circ$ 东方向, 求这时船和灯塔的距离 $CB$.

	\begin{solution}\\
		\begin{tikzpicture}
			\coordinate(A) at (0,0);
			\coordinate(B) at (4,4);
			\coordinate(C) at (2,0);
			\coordinate(D) at (0,4);
			\coordinate(E) at (2,4);
			\draw[dashed] (D)  -- (A)node[left]{$A$};
			\draw[name path=AB] (A) --  (B)node[right]{$B$};
			\pic["$45^\circ$", draw, angle eccentricity=2] {angle=B--A--D};
			\draw (A) -- (C)node[right] {$C$};
			\draw[decorate, decoration={brace, mirror, raise=5pt}, blue, draw] (A) --node[midway, below=8pt]{20} (C);
			\draw (C)-- (B)node[right]{$B$};
			\draw[dashed, name path=CE] (E) -- (C);
			\pic["$15^\circ$", draw, angle eccentricity=2] {angle=B--C--E};
			\path [name intersections={of=AB and CE, by=F}];
			\node[above left] at (F) {$F$};
			\coordinate(G) at ($(C)!(F)!(B)$);
			\draw (F) -- (G)node[right]{$G$};
		\end{tikzpicture}
		\begin{tikzpicture}
			\tkzDefPoint(0,0){A}  % 定义点 A
			\tkzDefPoint(4,4){B}  % 定义点 B
			\tkzDefPoint(2,0){C}  % 定义点 C
			\tkzDefPoint(0,4){D}  % 定义点 D
			\tkzDefPoint(2,4){E}  % 定义点 E

			% 绘制点
			\tkzDrawPoints(A,B,C,D,E)

			% 绘制线段
			\tkzDrawSegments(A,B B,C C,D)
			\tkzDrawSegment[draw,dashed](A,D)  % 虚线
			\tkzDrawSegment[draw,dashed](C,E)  % 虚线

			% 标记角度
			\tkzMarkAngle[draw=blue,fill=blue,angle radius=1cm](B,A,D)
			\tkzLabelAngle[sloped, pos=1.5](B,A,D){$45^\circ$}
			\tkzMarkAngle[draw=red,fill=red,angle radius=1cm](B,C,E)
			\tkzLabelAngle[sloped, pos=1.5](B,C,E){$15^\circ$}

			% 在某些点上标记标签
			\tkzLabelPoint[below left](A){$A$}
			\tkzLabelPoint[above right](B){$B$}
			\tkzLabelPoint[below](C){$C$}
			\tkzLabelPoint[above](D){$D$}
			\tkzLabelPoint[above](E){$E$}

			% 绘制带有标签的标注
			\tkzDrawSegment(A,C)
			\tkzLabelSegment[below](A,C){20}

			% 寻找交点 F,并绘制直线
			\tkzInterLL(A,B)(C,E) \tkzGetPoint{F}
			\tkzDrawPoints(F)
			\tkzLabelPoint[above left](F){$F$}

			% 绘制直线 FG
			\tkzDefPointBy[projection=onto B--C](F) \tkzGetPoint{G}
			\tkzDrawSegment(F,G)
			\tkzLabelPoint[right](G){$G$}
		\end{tikzpicture}

		\begin{itemize}
			\item \textbf{确定三角形} $\triangle{ACF}$ 是等腰直角三角形,得出:
			      \[
				      CF = AC = 20 \text{(海里)}.
			      \]
			\item \textbf{角度关系}:设 $\alpha = 15^\circ$,则:
			      \[
				      FG = \sin{\alpha} \cdot FC, \quad CG = \cos{\alpha} \cdot FC.
			      \]
			\item \textbf{角度推导}:由三角形角度关系得:
			      \[
				      \angle ABC = 180^\circ - \angle CAB - 90^\circ - 15^\circ = 30^\circ.
			      \]
			\item \textbf{计算 BG}:使用三角关系得:
			      \[
				      BG = \sqrt{3} \cdot FG.
			      \]
			\item \textbf{求解BC}: 综合前述公式得:
			      \[
				      BC = BG + CG = \sqrt{3} \cdot 20 \cdot \sin{\alpha} + 20 \cdot \cos{\alpha}.
			      \]
			\item \textbf{最终结果}:
			      \[
				      BC^2 = 400 \left( 3 \sin^2{\alpha} + 2\sqrt{3} \sin{\alpha} \cos{\alpha} + \cos^2{\alpha} \right).
			      \]
			\item \textbf{化简倍角公式}:
			      \[
				      \text{倍角公式:} \sin{2\alpha} = 2 \sin{\alpha} \cos{\alpha}, \quad \sin^2{\alpha} + \cos^2{\alpha} = 1,
			      \]
			      得到:
			      \[
				      BC^2 = 400 \left( 2 \sin^2{\alpha} + \sqrt{3} \sin{2\alpha} + 1 \right).
			      \]
			\item \textbf{最后化简}:
			      \[
				      BC^2 = 400 \left( 1 - \frac{\sqrt{3}}{2} + \frac{\sqrt{3}}{2} + 1 \right),
			      \]
			      最终得出:
			      \[
				      BC = 20\sqrt{2} \text{(海里)}.
			      \]
		\end{itemize}
	\end{solution}

	\question 一个圆内接三角形 \( ABC \),$\angle A$的平分线交$BC$于$D$,交外接圆于$E$,求证: \( AD \cdot AE = AC \cdot
	AB\).

	\begin{solution}
		\begin{minipage}{.4\textwidth}
			\begin{tikzpicture}
				\tkzDefPoint (0,0){A}
				\tkzDefPoint (5,0){B}
				\tkzDefPoint (4,4){C}

				% 外接圆
				\tkzCircumCenter(A,B,C) \tkzGetPoint{O}
				\tkzDrawCircle(O,A)

				% 角平分线
				\tkzDefLine[bisector](B,A,C) \tkzGetPoint{a}
				\tkzInterLL(A,a)(B,C) \tkzGetPoint{D}
				\tkzInterLC(A,a)(O,A) \tkzGetPoints{A}{E}
				\tkzDrawSegment(A,E)

				\tkzMarkAngle[mark=|](B,A,E)
				\tkzMarkAngle[mark=|](E,A,C)
				\tkzDrawPolygon(A,B,C)
				\tkzLabelPoints(A,B)
				\tkzLabelPoint[above](C){$C$}
				\tkzLabelPoint[above left](D){$D$}
				\tkzLabelPoint[right](E){$E$}

				\tkzDrawSegment[dashed, red](B,E)
				\tkzDrawSegment[dashed, red](C,E)
			\end{tikzpicture}
		\end{minipage}
		\begin{minipage}{0.55\textwidth}
			\begin{math}
				\because \angle{ACB}\text{和}\angle{AEB}\text{都是弦}AB\text{对应的圆周角} \\
				\therefore \angle{ACB} = \angle{AEB}                                       \\
				\because \angle{CDA}\text{和}\angle{BDE}是对顶角                        \\
				\therefore \angle{CDA} = \angle{BDE}                                       \\
				\therefore \triangle{CDA} \sim \triangle{EDB} \\
				\therefore \dfrac{AC}{AD} = \dfrac{BE}{BD} \\
				\because \angle{EBD} = \angle{EAB}, \angle{AEB} = \angle{BEA}\\
				\therefore \triangle{ABE} \sim \triangle{BDE} \\
				\therefore \dfrac{AB}{AE} = \dfrac{BD}{BE} = \dfrac{AD}{AC} \\
				\therefore AD \cdot AE = AC \cdot AB.
			\end{math}
		\end{minipage}

	\end{solution}

\end{questions}

\end{document}

%!tex program = lualatex
\documentclass[answers]{exam}
\usepackage{ctex}
\usepackage{graphicx}
\usepackage[margin=2cm]{geometry}
\usepackage{amsmath, amssymb}
\usepackage{csquotes}
\usepackage{tikz, pgfplots}
\usetikzlibrary{
	angles,
	backgrounds,
	calc,
	decorations.pathmorphing,
	decorations.pathreplacing,
	decorations.text,
	intersections,
	patterns,
	quotes,
	shapes,
	shapes.symbols,
}
\pagestyle{empty}
\newcounter{xcord}
\newcounter{ycord}
\newcounter{total}
\renewcommand{\labelenumi}{\textbf{\ifnum\value{enumi}<10 0\fi\arabic{enumi})}}

\pgfplotsset{compat=1.18}

\CorrectChoiceEmphasis{\color{blue!70!green}\bfseries}
\renewcommand{\solutiontitle}{\textbf{解:}}

\usepackage{array, tabularx}
\newcolumntype{C}{>{\centering\arraybackslash}X}
\newcolumntype{B}{>{\centering\bfseries\arraybackslash}X}
\catcode`\幺=0

\begin{document}
\begin{center}
	\textbf{1977年普通高等学校招生考试(福建卷)}

	\textbf{\huge{理科数学}}
\end{center}
\begin{questions}
	\question
	\begin{enumerate}[label=(\arabic*)]
		\item 计算: \( 5 - 3 \times \left[(-3\frac38)^{-\frac13} + 1031 \times (0.25 - 2^{-2})\right] \div 9^0 \).
		      \begin{solution}
			      \begin{align*}
				      \text{原式} & = 5 - 3 \times \left[ (-3\frac38)^{-\frac13} + 1031 \times 0 \right] \div 1 \\
				                & = 5 - 3 \times (-\frac{8}{27})^{\frac13}                                    \\
				                & = 5 - \sqrt[3]{3^3 \times (-\frac{8}{27}})                                  \\
				                & = 5 + 2                                                                     \\
				                & = 7
			      \end{align*}
		      \end{solution}
	\end{enumerate}
\end{questions}
\end{document}

\section[1977年高考数学试卷(河北卷)理科]{1977年普通高等学校招生考试(河北卷)\\\Huge{理科数学}}

\begin{questions}
	\question 解答下列各题
	\begin{parts}
		\part[10] 叙述函数的定义
			\begin{solution}
				函数是从一个集合到另一个集合之间的映射,使得每个输入值有且仅有一个输出值.
			\end{solution}
		\part[10] 求函数$y=1-\frac{1}{\sqrt{2-3x}}$的定义域.
			\begin{solution}
				由函数的定义有:
				\begin{math}
					2-3x > 0
				\end{math},则函数的定义域为:$\{x|x<\frac23\}$.
			\end{solution}
		\part[10] 计算:$\left[1-(0.5)^{-2}\right] \div \left(-\frac{27}{8}\right)^{\frac13}$.
			\begin{solution}
				\begin{align*}
					\left[1-(0.5)^{-2}\right] \div \left(-\frac{27}{8}\right)^{\frac13} & = \left(1-2^{-1\times (-2)}\right)
					\div \left(-\frac32\right)^{3\times\frac13}                                                              \\
					                                                                    & = (1 - 4) \cdot (-\frac23)         \\
					                                                                    & = 2
				\end{align*}
			\end{solution}
		\part[10] 计算:$\log_42$.
			\begin{solution}
				\begin{align*}
					\log_42 & = \log_44^\frac12 \\
					        & = \frac12
				\end{align*}
			\end{solution}
		\part[10] 分解因式:$x^2y - 2y^3$.
			\begin{solution}
				\begin{align*}
					x^2y - 2y^3 & = y(x^2 - 2y^2)               \\
					            & = y(x+\sqrt{2}y)(x-\sqrt{2}y)
				\end{align*}
			\end{solution}
		\part[10] 计算:$\sin\dfrac{4\pi}{3}\cdot\cos\dfrac{25\pi}{6}\cdot\tan\left(-\dfrac{3\pi}{4}\right)$.
			\begin{solution}
				\begin{align*}
					\sin\frac{4\pi}{3}\cdot\cos\frac{25\pi}{6}\cdot\tan\left(-\frac{3\pi}{4}\right)
					 & = -\sin\frac{\pi}{3}\cdot\cos\frac{\pi}{6}\cdot \tan\frac{3\pi}{4} \\
					 & = -\frac{\sqrt{3}}{2} \cdot \frac{\sqrt{3}}{2} \cdot 1             \\
					 & = - \frac34
				\end{align*}
			\end{solution}
	\end{parts}

	\question[10]
	\begin{minipage}[t]{.3\textwidth}
		证明:如图,$AB$是圆$O$的直径,$CB$是圆$O$的切线,切点为$B$,$OC$平行于弦$AD$,求证:$DC$是圆$O$的切线.
	\end{minipage}\hspace{3cm}
	\begin{tikzpicture}[scale=.5, baseline=(current bounding box.center)]
		\tkzDefPoints{-2/0/A, 0/0/O, 2/0/B, 2/5/C}
		\tkzDefLine[parallel=through A](O,C) \tkzGetPoint{x}
		\tkzInterLC(A,x)(O,B) \tkzGetPoints{D}{A}

		\tkzDrawCircle(O,B)
		\tkzDrawSegment(A,B)
		\tkzDrawSegment(C,B)
		\tkzDrawSegment(C,O)
		\tkzDrawSegment(A,D)
		\tkzDrawSegment(C,D)

		\tkzLabelPoint[left](A){$A$}
		\tkzLabelPoint[right](B){$B$}
		\tkzLabelPoint[above](C){$C$}
		\tkzLabelPoint[below](O){$O$}
		\tkzLabelPoint[left](D){$D$}
	\end{tikzpicture}

	\begin{proofsolution}
		\begin{center}
			\begin{tikzpicture}[scale=.5, baseline=(current bounding box.center)]
				\tkzDefPoints{-2/0/A, 0/0/O, 2/0/B, 2/5/C}
				\tkzDefLine[parallel=through A](O,C) \tkzGetPoint{x}
				\tkzInterLC(A,x)(O,B) \tkzGetPoints{D}{A}

				\tkzDrawCircle(O,B)
				\tkzDrawSegment(A,B)
				\tkzDrawSegment(C,B)
				\tkzDrawSegment(C,O)
				\tkzDrawSegment(A,D)
				\tkzDrawSegment(C,D)
				\tkzDrawSegment[dashed, red](D,O)

				\tkzMarkAngle[mark=|, size=15pt](O,A,D)
				\tkzMarkAngle[mark=|, size=15pt](B,O,C)
				\tkzMarkAngle[mark=||, size=20pt](A,D,O)
				\tkzMarkAngle[mark=||, size=20pt](C,O,D)

				\tkzLabelPoint[left](A){$A$}
				\tkzLabelPoint[right](B){$B$}
				\tkzLabelPoint[above](C){$C$}
				\tkzLabelPoint[below](O){$O$}
				\tkzLabelPoint[left](D){$D$}
			\end{tikzpicture}
		\end{center}
		\begin{enumerate}[label=\protect\circled{\arabic*}]
			\item 因为 $CB$ 是圆 $O$ 的切线且切点为 $B$,所以有 $AB \perp CB$.
			\item 由于 $AD \parallel OC$,可以得到 $\angle DAO = \angle COB$.
			\item 作辅助线 $DO$.因为 $\angle ADO$ 和 $\angle COD$ 是平行线的内错角,所以 $\angle ADO = \angle COD$.
			\item 在 $\triangle AOD$ 中,由于 $AD \parallel OC$,且 $OA = OD$(都是圆的半径),所以 $\triangle AOD$ 是等腰三角形,由此得到 $\angle OAD = \angle ODA$.
			\item 进一步分析 $\triangle DOC$ 和 $\triangle BOC$:
			      \begin{itemize}
				      \item $OB = OD$(都是圆的半径);
				      \item $OC = OC$(公共边);
				      \item $\angle BOC = \angle DOC$(由 $\angle COB = \angle DAO$ 和 $\angle ADO = \angle COD$ 推导).
			      \end{itemize}
			      因此,$\triangle BOC \cong \triangle DOC$(根据边角边准则).
			\item 由 $\triangle BOC \cong \triangle DOC$,可以得到 $\angle ODC = \angle CBO = 90^\circ$.
			\item 因为 $\angle ODC = 90^\circ$,所以 $DC$ 是圆 $O$ 的切线.
		\end{enumerate}
	\end{proofsolution}
	\question[10] 证明: $\dfrac{\sin2\alpha + 1}{1+\cos2\alpha + \sin2\alpha} = \dfrac12\tan\alpha + \dfrac12$.
	\begin{proofsolution}
		\begin{cenum}
			\item 根据倍角公式$\sin2\alpha = 2\sin\alpha\cos\alpha$和$\cos2\alpha=\cos^2\alpha -
				      \sin^2\alpha$来化简等式的左边得:
			      \begin{align*}
				      \frac{\sin2\alpha + 1}{1+\cos2\alpha + \sin2\alpha}
				       & = \frac{2\sin\alpha\cos\alpha + 1}{1 + \cos^2\alpha - \sin^2\alpha + 2\sin\alpha\cos\alpha} \\
				       & = \frac{\cos^2\alpha + \sin^2\alpha + 2\sin\alpha\cos\alpha}{\cos^2\alpha + \sin^2\alpha +
				      \cos^2\alpha - \sin^2\alpha + 2\sin\alpha\cos\alpha}                                           \\
				       & = \frac{(\sin\alpha + \cos\alpha)^2}{2\cos\alpha(\sin\alpha + \cos\alpha)}                  \\
				       & = \frac{\sin\alpha + \cos\alpha}{2\cos\alpha}                                               \\
				       & = \frac12\tan\alpha + \frac12
			      \end{align*}
			\item 综上,等式成立.
		\end{cenum}
	\end{proofsolution}

	\question[10] 已知$2\lg{x} + \lg{2} = \lg(x+6)$,求$x$.
	\begin{solution}
		\begin{align*}
			2\lg{x} + \lg2 & = \lg{x^2} + \lg2 \\
			               & = \lg{2x^2}
		\end{align*}
		根据等式则有:
		\begin{equation*}
			2x^2 = x + 6 \tag{a}
		\end{equation*}
		对式(a)整理得:
		\begin{align*}
			2x^2 - x - 6  & = 0 \\
			(2x + 3)(x-2) & = 0 \\
		\end{align*}
		因为$x>0$所以$x=2$.
	\end{solution}

	\question[10]
	\begin{minipage}[t]{.6\textwidth}
		某生产队要建立一个形状是直角梯形的苗圃,其两邻边借用夹角为$135^\circ$的两面墙,另外两边是总长为$30$米的篱笆(如图,$AD$和$DC$为墙),问篱笆的两边各多长时,苗圃的面积最大?最大面积是多少?
	\end{minipage}
	\hspace{1cm}
	\begin{tikzpicture}[baseline=(current bounding box.north)]
		\tkzDefPoints{0/0/A, 3/0/B}
		\tkzDefPoint(45:2){D}
		\tkzDefLine[parallel=through D](A,B) \tkzGetPoint{x}
		\tkzDefLine[orthogonal=through B](A,B) \tkzGetPoint{y}
		\tkzInterLL(D,x)(B,y) \tkzGetPoint{C}

		\tkzDrawPolygon(A,B,C,D)
		\tkzLabelPoints[below](A,B)
		\tkzLabelPoints[above](C,D)
	\end{tikzpicture}

	\begin{solution}
		\begin{center}
			\begin{tikzpicture}[baseline=(current bounding box.north)]
				\tkzDefPoints{0/0/A, 3/0/B}
				\tkzDefPoint(45:2){D}
				\tkzDefLine[parallel=through D](A,B) \tkzGetPoint{x}
				\tkzDefLine[orthogonal=through B](A,B) \tkzGetPoint{y}
				\tkzInterLL(D,x)(B,y) \tkzGetPoint{C}

				\tkzDefLine[orthogonal=through D](A,B) \tkzGetPoint{x}
				\tkzInterLL(D,x)(A,B) \tkzGetPoint{E}

				\tkzDrawPolygon(A,B,C,D)
				\tkzLabelPoints[below](A,B,E)
				\tkzLabelPoints[above](C,D)
				\tkzDrawSegment[blue, dashed](D,E)
			\end{tikzpicture}
		\end{center}

		\begin{cenum}
			\item 从$D$点向$AB$作垂线,垂足为$E$.
			\item 从条件可知$AE=ED=BC$
			\item 设$AE=x$,则有$DC=EB=30-2x$
			\item 梯形的面积可以表示为:
			      \begin{align*}
				      S & = (30 - 2x + x + 30 - 2x)\cdot x / 2 \\
				        & = (60 - 3x)x/2                       \\
				        & = -\frac32x^2 + 30x \tag{1}
			      \end{align*}
			\item 根据抛物线的性质在$x=-\frac{b}{2a}$时有极值,代入可得$x=10$,此时$AB=20,BC=10$
			\item 将$x=10$代入式(1)可得最大面积为\qty{150}{\meter\squared}.
		\end{cenum}
	\end{solution}

	\question[10]
	工人师傅要用铁皮做一个上大下小的正四棱台形容器(上面开口),使其容积为$208$立方分米,高为$4$分米,上口边长与下底面边长的比为$5:2$,做这样的容器需要多少平方米的铁皮?(不计容器的厚度和加工余量,不要求写出已知,
	求解,直接求解并画图即可)

	\begin{solution}
		\begin{cenum}
			\item 设下底面的边长为$2x$,则上底面的边长为$5x$,上下底面的面积分别为$25x^2$和$4x^2$;
			\item 根据棱台体积公式:
			      \begin{equation*}
				      V = \frac13h(S_1 + S_2 + \sqrt{S_1S_2})
			      \end{equation*}
			      代入可得:
			      \begin{align*}
				      V & = \frac13\cdot4(25x^2 + 4x^2 + 10x^2) \\
				        & = 52x^2                               \\
				        & = 208
			      \end{align*}
			      解得:
			      \begin{equation*}
				      x = 2
			      \end{equation*}
			      则有上边长为\qty{10}{\dm},底边长为\qty{4}{\dm}.
			\item 容器沿平行于边的中心线剖面如下图所示:
			      \tdplotsetmaincoords{60}{130}
			      \begin{center}
				      \begin{tikzpicture}[tdplot_main_coords]
					      \coordinate(A) at (1,1,0);
					      \coordinate(B) at (-1,1,0);
					      \coordinate(C) at (-1,-1,0);
					      \coordinate(D) at (1,-1,0);
					      \coordinate(A') at (2.5,2.5,2);
					      \coordinate(B') at (-2.5,2.5,2);
					      \coordinate(C') at (-2.5,-2.5,2);
					      \coordinate(D') at (2.5,-2.5,2);
					      \coordinate(E) at (-2.5, 0, 2);
					      \coordinate(F) at (2.5, 0, 2);
					      \coordinate(G) at (1, 0, 0);
					      \coordinate(H) at (-1, 0, 0);
					      \coordinate(I) at (-1, 0, 2);

					      \draw (D) -- (A) -- (B);
					      \draw [dashed](B)-- (C) -- (D);

					      \draw (A') -- (B') -- (C') -- (D') -- cycle;
					      \draw (A) -- (A');
					      \draw (B) -- (B');
					      \draw[dashed] (C) -- (C');
					      \draw (D) -- (D');
					      \draw[dashed](E) -- (F) -- (G) -- (H) -- cycle;
					      \draw[dashed](H) -- (I);

					      \tkzLabelPoint[above](E){$E$}
					      \tkzLabelPoint[above](F){$F$}
					      \tkzLabelPoint[below left](G){$G$}
					      \tkzLabelPoint[above left](H){$H$}
					      \tkzLabelPoint[above](I){$I$}

					      % \tkzMarkSegments(C',E E,B')
					      % \tkzMarkSegments[mark=||](A',F F,D')
					      \tkzMarkRightAngle(H,I,E)

				      \end{tikzpicture}
			      \end{center}

			      可得$HE=\sqrt{EI^2 + IH^2}=\qty{5}{\dm}$.

			\item 因此铁皮的面积为
			      \begin{align*}
				      (4+10)\times 5 \div 2 \times 4  +
				      4 \times 4 % 底面积
				      = \qty{156}{\dm\squared}
			      \end{align*}
		\end{cenum}
	\end{solution}
	\question[10]
	如图,$MN$为圆的直径,$P$、$C$为圆上两点,连$PM,PN$,过$C$作$MN$的垂线与$MN,MP$和$NP$的延长线依次相交于$A,B,D$,求证:$AC^2=AB\cdot AD.$

	\begin{center}
		\begin{tikzpicture}[baseline=(current bounding box.north), scale=.8]
			\tkzDefPoints{-2/0/M,0/0/O,2/0/N}
			\tkzDefPoint(65:2){P}
			\tkzDefPoint(110:2){C}

			\tkzDefLine[orthogonal=through C](M,N) \tkzGetPoint{x}
			\tkzInterLL(C,x)(M,N) \tkzGetPoint{A}
			\tkzInterLL(C,x)(M,P) \tkzGetPoint{B}
			\tkzInterLL(C,x)(N,P) \tkzGetPoint{D}

			\tkzDrawCircle(O,N)
			\tkzDrawSegments(M,N M,P A,D N,D C,N M,C)
			\tkzLabelPoint(A){$A$}
			\tkzLabelPoint[right](B){$B$}
			\tkzLabelPoint[above left](C){$C$}
			\tkzLabelPoint[above](D){$D$}
			\tkzLabelPoint[left](M){$M$}
			\tkzLabelPoint[right](N){$N$}
			\tkzLabelPoint[above right](P){$P$}
		\end{tikzpicture}
	\end{center}

	\begin{solution}
		\begin{align*}
			 & \because \angle{AMP} = \angle{PMA} \text{并且有} \angle{MAB} = \angle{MPN} = \ang{90}                  \\
			 & \therefore \triangle{AMB} \sim \triangle{PMN}                                                          \\
			 & \because \angle{PNM} = \angle{MNP} \text{和} \angle{MPN} = \angle{DAN} = \ang{90}                      \\
			 & \therefore \triangle{PMN} \sim \triangle{ADN}                                                          \\
			 & \therefore \triangle{AMB} \sim \triangle{ADN}                                                          \\
			 & \therefore \frac{AB}{MA} = \frac{AN}{AD}                                                               \\
			 & \therefore AB\cdot AD = MA \cdot AN                                                                    \\
			 & \because \angle{MCA} = \angle{CNA} = \ang{90} - \angle{NMC} \land \angle{CAN} = \angle{CAM} = \ang{90}
		\end{align*}

		\begin{align*}
			 & \therefore \triangle{AMC} \sim \triangle{ACN} \\
			 & \therefore \frac{AC}{MA} = \frac{NA}{AC}      \\
			 & \therefore AC^2 = MA \cdot CA                 \\
			 & \therefore AC^2 = AB \cdot AD
		\end{align*}
	\end{solution}
	\question[10] 下列两题选做一题.

	【甲】已知椭圆短轴长为$2$,中心与抛物线$y^2=4x$的顶点重合,椭圆的一个焦点恰是此抛物线的焦点,求椭圆方程及其长轴的长.
	\begin{solution}
		\begin{center}
			\begin{tikzpicture}
				\begin{axis}[xmin=-4,
						xmax=4,
						ymin=-3,
						ymax=3
					]
					\addplot[domain=0:3]{sqrt(4*x)};
					\addplot[domain=0:3]{-sqrt(4*x)};
					\filldraw(1,0) circle(1pt);
					\addplot[domain=0:2*pi]({sqrt(2)*cos(deg(x))}, {sin(deg(x))});
				\end{axis}
			\end{tikzpicture}
		\end{center}

		\begin{cenum}
			\item 抛物线的顶点为$(0,0)$;
			\item 抛物线$y^2=4px$的焦点为$(p,0)$,则有焦点的坐标为$(1,0)$;
			\item 设椭圆的方程为:
			      \begin{equation*}
				      \frac{x^2}{a^2} + \frac{y^2}{b^2} = 1
			      \end{equation*}
			      则有半焦距$c=\sqrt{a^2-b^2}$,因为短轴长为$2$,所以$b=1$.计算得$a=\sqrt{2}$,即长轴的长为$2\sqrt{2}$.
			      椭圆方程为
			      \begin{equation*}
				      \frac{x^2}{2} + \frac{y^2}{1} = 1
			      \end{equation*}
		\end{cenum}
	\end{solution}

	【乙】已知菱形的一对内角各为$60^\circ$,边长为$4$,以菱形为对角线所在的直线为坐标轴建立直角坐标系,以菱形$60^\circ$角的两个顶点为焦点,并且过菱形的另外两个顶点作椭圆,求椭圆方程.

	\begin{solution}
		\begin{center}
			\begin{tikzpicture}
				\tkzInit[xmax=5,xmin=-5, ymax=2.5, ymin=-2.5]
				\tkzDrawX\tkzDrawY
				\tkzDefPoints{{2sqrt(3)}/0/A,{-2sqrt(3)}/0/C, 0/2/B, 0/-2/D, 0/0/O}

				\tkzDrawPolygon(A,B,C,D)
				\tkzMarkSegment[dim={$4$, 16pt, right=6pt}, mark=](B,A)
				\tkzMarkSegment[dim={$2\sqrt{3}$, -16pt, below=6pt}, mark=](O,A)
				\tkzMarkSegment[dim={$2$, 16pt, left=6pt}, mark=](O,B)

				\draw[x radius=4, y radius=2] ellipse;
			\end{tikzpicture}
		\end{center}
		设椭圆的方程为:
		\begin{equation*}
			\frac{x^2}{a^2} + \frac{y^2}{b^2} = 1
		\end{equation*}
		则有:
		\begin{align*}
			b = 2
		\end{align*}
		半焦距为:
		\begin{align*}
			c = 2\sqrt{3}
		\end{align*}
		长半轴为
		\begin{align*}
			a & = \sqrt{b^2 + c^2} \\
			  & = 4
		\end{align*}
		故椭圆的方程为
		\begin{equation*}
			\frac{x^2}{16} + \frac{y^2}{4} = 1
		\end{equation*}
	\end{solution}

	\begin{center}
		\large\bf 附加题
	\end{center}
	\question 将函数$f(x)=\ue^x$展开为$x$的幂级数,并求出收敛区间.($\ue=2.718$为自然对数的底数)
	\begin{solution}
		$\ue^x$的幂级数展开式为$\displaystyle\sum_{n=0}^{\infty}\frac{x^n}{n!}$,使用比值判别法:
		\begin{equation*}
			\lim_{n\to\infty}\left| \frac{\frac{x^{n+1}}{(n+1)!}}{\frac{x^n}{n!}}\right| =
			\lim_{n\to\infty}\left|\frac{x}{n+1}\right|
			= 0
		\end{equation*}
		所以收敛区间为$(-\infty, \infty)$.
	\end{solution}

	\question 利用定积分计算椭圆$\frac{x^2}{a^2} + \frac{y^2}{b^2} = 1 (a > b > 0)$所围成的面积.
	\begin{solution}
		对于任意一处的$x$对应的$y=\pm b\sqrt{1 -
				\frac{x^2}{a^2}}$,则在此处的形成的宽度为$\ud{x}$,高度为$2|y|$的微小矩形的面积表达式为:$2b\sqrt{1-\frac{x^2}{a^2}}\ud{x}$,则椭圆的面积为
		\begin{equation*}
			A = \int_{-a}^{a}2b\sqrt{1-\frac{x^2}{a^2}}\ud{x} = 4b\int_{0}^{a}\sqrt{1-\frac{x^2}{a^2}}\ud{x}
		\end{equation*}
		令$x=a\sin\theta$,则$\ud{x} = a\cos\theta\ud{\theta}$,其中$\theta \in [0, \frac\pi2]$.
		代入得:
		\begin{align*}
			A = 4b\int_{0}^{a}\sqrt{1-\frac{x^2}{a^2}}\ud{x}
			 & = 4b\int_0^{\frac\pi2}\sqrt{1 -
			\frac{a^2\sin^2\theta}{a^2}}a\cos\theta\ud{\theta}                                   \\
			 & = 4ab\int_0^{\frac\pi2}\cos^2\theta\ud{\theta}                                    \\
			 & = 4ab\int_0^{\frac\pi2}\frac{1+\cos(2\theta)}{2}\ud{\theta}                       \\
			 & = 4ab\left(\int_0^{\frac\pi2}\frac12\ud{\theta} +
			 \int_0^{\frac\pi2}\frac{\cos(2\theta)}{2}\ud{\theta}\right)                     \\
			 & = 4ab\left(\frac\pi4  + \left[\frac{\sin(2\theta)}{4}\right]_0^{\frac\pi2}\right) \\
			 & = \pi ab
		\end{align*}
	\end{solution}

\end{questions}

%!tex program = lualatex
\documentclass[answers]{exam}
\usepackage{ctex}
\usepackage{graphicx}
\usepackage[margin=2cm]{geometry}
\usepackage{amsmath, amssymb}
\usepackage{csquotes}
\usepackage{tikz, pgfplots}
\usetikzlibrary{
	angles,
	backgrounds,
	calc,
	decorations.pathmorphing,
	decorations.pathreplacing,
	decorations.text,
	intersections,
	patterns,
	quotes,
	shapes,
	shapes.symbols,
}
\pagestyle{empty}
\newcounter{xcord}
\newcounter{ycord}
\newcounter{total}
\renewcommand{\labelenumi}{\textbf{\ifnum\value{enumi}<10 0\fi\arabic{enumi})}}

\pgfplotsset{compat=1.18}

\CorrectChoiceEmphasis{\color{blue!70!green}\bfseries}
\renewcommand{\solutiontitle}{\textbf{解:}}

\usepackage{array, tabularx}
\newcolumntype{C}{>{\centering\arraybackslash}X}
\newcolumntype{B}{>{\centering\bfseries\arraybackslash}X}
\catcode`\幺=0

\usepackage[lua]{tkz-euclide}

\begin{document}
\begin{center}
	\textbf{1977年普通高等学校招生考试(黑龙江卷)}

	\textbf{\Large{数学试卷}}
\end{center}

\begin{questions}
	\question 解答下列各题:
	\begin{enumerate}[label=(\arabic*)]
		\item 解方程:$\sqrt{3x+4} = 4$。
		      \begin{solution}
			      方程两边平方得:
			      \begin{math}
				      3x + 4 = 16
			      \end{math}
			      移项整理得
			      \begin{align*}
				      3x & = 12 \\
				      x  & = 4
			      \end{align*}

		      \end{solution}
		\item 解不等式:$|x| < 5$
		      \begin{solution}
			      \begin{math}
				      -5 < x < 5
			      \end{math}

		      \end{solution}
		\item 已知正三角形的外接圆半径为$6\sqrt{3}$cm,求它的边长。
		      \begin{solution}
			      \begin{tikzpicture}[scale=.3]
				      \tkzDefPoints{0/0/O, 6sqrt(3)/0/A}
				      \tkzDefPoint(90:6sqrt(3)){B}
				      \tkzDefPoint(210:6sqrt(3)){C}
				      \tkzDefPoint(330:6sqrt(3)){D}
				      \tkzDefLine[orthogonal=through O](B,C) \tkzGetPoint{x}
				      \tkzInterLL(O,x)(B,C) \tkzGetPoint{E}

				      \tkzDrawCircle(O,A)
				      \tkzDrawPolygon(B,C,D)
				      \tkzDrawPoint(O)
				      \tkzLabelPoint[below right](O){$O$}
				      \tkzDrawSegments(B,O C,O O,E)

				      \tkzMarkSegment[dim={$6\sqrt{3}$, 16pt, right=6pt}, mark=](B,O)
				      \tkzMarkAngle[size=2](C,B,O)
				      \tkzLabelAngle[pos=3.5](C,B,O){$30^\circ$}
				      \tkzLabelPoint[left](E){$E$}
				      \tkzLabelPoint[above](B){$B$}
			      \end{tikzpicture}
			      可以计算出$BE=6$,则正三角形的边长为$12$。
		      \end{solution}

	\end{enumerate}

	\question 计算下列各题:
	\begin{enumerate}[label=(\arabic*)]
		\item $\sqrt{m^2 - 2ma + a^2} $
		      \begin{solution}
			      \begin{align*}
				      \sqrt{m^2 - 2ma + a^2} & = \sqrt{(m-a)^2} \\
				                             & = |m-a|
			      \end{align*}
		      \end{solution}
		\item $\cos78^\circ\cdot\cos3^\circ + \cos12^\circ\cdot\sin3^\circ$。
		      \begin{solution}
			      \begin{align*}
				      \cos78^\circ & = \cos(90^\circ - 12^\circ)                           \\
				                   & = \cos90^\circ\cos12^\circ + \sin90^\circ\sin12^\circ \\
				                   & = \sin12^\circ
			      \end{align*}
			      代入原式得:
			      \begin{align*}
				      \cos78^\circ\cdot\cos3^\circ + \cos12^\circ\cdot\sin3^\circ
				       & = \sin12^\circ\cos3^\circ + \cos12^\circ\sin3^\circ \\
				       & = \sin15^\circ                                      \\
				       & = \sin\left(\frac{30^\circ}{2}\right)               \\
				       & = \sqrt{\frac{1-\cos30^\circ}{2}}                   \\
				       & = \frac{\sqrt{2-\sqrt{3}}}{2}
			      \end{align*}
		      \end{solution}
		\item $\arcsin\left(\cos\dfrac\pi6\right)$.
		      \begin{solution}
			      \begin{align*}
				      \arcsin\left(\cos\dfrac\pi6\right) & = \arcsin(\dfrac{\sqrt{3}}2) \\
				                                         & = \frac{\pi}{3}
			      \end{align*}
		      \end{solution}
	\end{enumerate}
	\question 解下列各题:
	\begin{enumerate}[label=(\arabic*)]
		\item 解方程: $3^{x+1} - 9^{\frac{x}{2}} = 18$.
		      \begin{solution}
			      \begin{align*}
				      3^{x+1} - 9^{\frac{x}{2}}        & = 18 \\
				      3\cdot 3^x - (3^2)^{\frac{x}{2}} & = 18 \\
				      3^x                              & = 9  \\
				      x                                & = 2
			      \end{align*}
		      \end{solution}
		\item 求数列$2,4,8,16,\cdots $前十项的和。
		      \begin{solution}
			      \begin{align*}
				      S_n    & = a_0\frac{1-q^n}{1-q}, (q=2, a_0=2) \\
				      S_{10} & = 2\frac{1-2^{10}}{1-2}              \\
				             & = 2^{11} - 2                         \\
				             & = 2046
			      \end{align*}
		      \end{solution}
	\end{enumerate}

	\question 解下列各题:
	\begin{enumerate}[label=(\arabic*)]
		\item 圆锥的高为$6$cm,母线和底面半径的夹角为$30^\circ$,求它的侧面积。
		      \begin{solution}
			      根据题目中提供的信息可计算得底面半径为$r = 6\sqrt{3}$,母线的长度为$l = 12$,则侧面积为
			      \begin{math}
				      S = \pi r l = 72\sqrt{3} \text{cm}^2
			      \end{math}
		      \end{solution}
		\item 求过点$(1,4)$且与直线$2x - 5y + 3 = 0$垂直的直线方程。
		      \begin{solution}
			      直线$2x - 5y + 3 + 0$的斜率$k=\frac25$,则与之垂直的直线的斜率为$-\frac52$。

			      设所求直线方程为$y =
				      -\frac52x+b$, 并将点$(1,4)$代入得$b = \frac{13}{2}$。
			      所以直线方程为:$2y + 5x - 13 = 0$。
		      \end{solution}
	\end{enumerate}

	\question 如果$\triangle{ABC}$的$\angle{A}$的平分线交$BC$于$D$,交它的外接圆于$E$,那么$AB\cdot AC = AD \cdot AE$。
	\begin{figure*}[htbp]
		\centering
		\begin{tikzpicture}
			\tkzDefPoints{0/0/A, 3/0/B, 2.5/2.5/C}
			\tkzDefLine[bisector](B,A,C) \tkzGetPoint{x}
			\tkzDefCircle(A,B,C) \tkzGetPoint{O}
			\tkzInterLL(A,x)(B,C) \tkzGetPoint{D}
			\tkzInterLC(A,x)(O,A) \tkzGetSecondPoint{E}

			\tkzDrawPolygon(A,B,C)
			\tkzDrawCircle(O,A)
			\tkzDrawSegment(A,E)

			\tkzLabelPoint[left](A){$A$}
			\tkzLabelPoint[right](B){$B$}
			\tkzLabelPoint[above](C){$C$}
			\tkzLabelPoint[above left](D){$D$}
			\tkzLabelPoint[right](E){$E$}

			\tkzDrawSegments[red, dashed](C,E E,B)

		\end{tikzpicture}
	\end{figure*}
	\begin{solution}
		\begin{mathenum}
			\item 作辅助线$CE$和$BE$
			\item \because $\overset{\frown}{CE}$对应圆周角$\angle{CAE}$和$\angle{CBE}$
			\\ \therefore\ $\angle{CAE} = \angle{CBE}$
			\\ \because\ $\angle{CDA} = \angle{EDB} $
			\\ \therefore\ $\triangle{CAD} \sim \triangle{EBD}$
			\item \because\ $\overset{\frown}{BE}$对应圆周角$\angle{BAE} = \angle{BCE}$
			\\ \therefore\ $\angle{BAE} = \angle{EAC}$
			\\ \because $\angle{BAE} = \angle{BAC}$
			\\ \therefore\ $\overset{\frown}{CE} = \overset{\frown}{BE}$
			\\ \therefore\ $\angle{BCE} = \angle{CBE}$
			\\ \therefore\ $\angle{EBD} = \angle{BAE}$
			\\ \because\ $\angle{BEA} = \angle{AEB}$
			\\ \therefore\ $\triangle{EBD} \sim \triangle{EAB}$
			\\ \therefore\ $\triangle{EAB} \sim \triangle{CAD}$
			\\ \therefore\ $\dfrac{AC}{AE}={AD}{AB}$
			\\ \therefore\ $AC\cdot AB = AD\cdot AE$
		\end{mathenum}

	\end{solution}
	\question
	前进大队响应毛主席关于\enquote{绿化祖国}的伟大号召,1975年造林$200$亩,又知1975年到1977年三年内共造林$728$亩,求后两年造林面积的年平均增长率是多少?
	\begin{solution}
		设年平均增长率为$q$,根据等比数列求和公式
		\begin{math}
			S_n = a_0\frac{1-q^n}{1-q} (a_0 = 200, n = 3, S_3 = 728)
		\end{math}

		代入得:
		\begin{math}
			728 = 200 \frac{1-q^3}{1-q} = 200 \frac{(1-q)(1 + q + q^2)}{1-q} = 200(1 + q + q^2)
		\end{math}
		化简得:
		\begin{align*}
			25q^2 + 25q - 66  & = 0 \\
			(5q - 6)(5q + 11) & = 0 \\
		\end{align*}
		因为$q > 0$所以$q=1.2$
	\end{solution}
\end{questions}

\end{document}

%!tex program = lualatex
\documentclass[answers]{exam}
\usepackage{ctex}
\usepackage{graphicx}
\usepackage[margin=2cm]{geometry}
\usepackage{amsmath, amssymb}
\usepackage{csquotes}
\usepackage{tikz, pgfplots}
\usetikzlibrary{
	angles,
	backgrounds,
	calc,
	decorations.pathmorphing,
	decorations.pathreplacing,
	decorations.text,
	intersections,
	patterns,
	quotes,
	shapes,
	shapes.symbols,
}
\pagestyle{empty}
\newcounter{xcord}
\newcounter{ycord}
\newcounter{total}
\renewcommand{\labelenumi}{\textbf{\ifnum\value{enumi}<10 0\fi\arabic{enumi})}}

\pgfplotsset{compat=1.18}

\CorrectChoiceEmphasis{\color{blue!70!green}\bfseries}
\renewcommand{\solutiontitle}{\textbf{解:}}

\usepackage{array, tabularx}
\newcolumntype{C}{>{\centering\arraybackslash}X}
\newcolumntype{B}{>{\centering\bfseries\arraybackslash}X}
\catcode`\幺=0


\begin{document}
\begin{center}
	\textbf{1977年普通高等学校招生考试(江苏卷)}
	\\ \textbf{\Large{数学试卷}}
\end{center}

\begin{questions}
	\question
	\begin{enumerate}[label=(\arabic*)]
		\item 计算:$ \left(2\frac14\right)^\frac12 + \left(\frac{1}{10}\right) - (3.14)^0 +
			      \left(-\frac{27}{8}\right)^{\frac13} $
		      \begin{solution}
			      \begin{align*}
				       & = \left(\frac32\right)^{2\times\frac12} + 10^2 - 1 - \left(\frac32\right)^{3\times(\frac13)} \\
				       & = \frac32 + 99 - \frac32                                                                     \\
				       & = 99
			      \end{align*}
		      \end{solution}

		\item 求函数$y=\sqrt{x-2} + \dfrac{1}{x-3} + \lg(5-x)$的定义域。
		      \begin{solution}
			      根据题意有:
			      \begin{math}
				      \begin{cases}
					      x - 2 >= 0   \\
					      x - 3 \neq 0 \\
					      5 - x > 0
				      \end{cases}
			      \end{math}
			      则有定义域为$[2,3),(3,5]$

		      \end{solution}
		\item 解方程:$ 5^{x^2+2x} = 125 $.
		      \begin{solution}
			      由 $5^{x^2 + 2x} = 125 = 5^3 $得: $x^2 + 2x = 3$

			      分解因式得$(x+3)(x-1) = 0$,所以$x_1 = -3, x_2 = 1$
		      \end{solution}
		\item 计算:$-\log_3(\log_3\sqrt[3]{\sqrt[3]{\sqrt[3]{3}}})$
		      \begin{solution}
			      \begin{align*}
				      -\log_3(\log_3\sqrt[3]{\sqrt[3]{\sqrt[3]{3}}}) & = -\log_3(\log_3\sqrt[3]{\sqrt[3]{3^\frac13}}) \\
				                                                     & = -\log_3(\log_3\sqrt[3]{3^\frac19})           \\
				                                                     & = -\log_3(\log_33^\frac1{27})                  \\
				                                                     & = -\log_3\left(\frac{1}{27}\right)             \\
				                                                     & = -\log_3\left(3^{-3}\right)                   \\
				                                                     & = 3
			      \end{align*}
		      \end{solution}
		\item 把直角坐标方程 \( (x-3)^2 + y^2 = 9 \)化为极坐标方程。
		      \begin{solution}
			      设动点为$(\rho, \theta)$,则有$x=\rho\cos\theta, y=\rho\sin\theta$,代入直角坐标方程得:
			      \begin{equation*}
				      (\rho\cos\theta - 3)^2 + (\rho\sin\theta)^2 = 9
			      \end{equation*}
			      展开得
			      \begin{equation*}
				      \rho^2\cos^2\theta - 6\rho\cos\theta + 9 + \rho^2\sin^2\theta = 9
			      \end{equation*}
			      合并同类项得:
			      \begin{equation*}
				      \rho^2(\cos^2\theta + \sin^2\theta) - 6\rho\cos\theta = 0
			      \end{equation*}
			      因为$\cos^2\theta + \sin^2\theta = 1$,简化为:
			      \begin{equation*}
				      \rho(\rho - 6\cos\theta) = 0
			      \end{equation*}
			      则极坐标方程为
			      \begin{equation*}
				      \rho = 6\cos\theta
			      \end{equation*}
		      \end{solution}
		\item 计算: \( \displaystyle \lim_{n\to\infty}\frac{1+2+3+\cdots+n}{n^2} \)
		      \begin{solution}
			      根据等差数列求和公式得
			      \begin{equation*}
				      1 + 2 + 3 + \cdots + n = \frac{n(n+1)}{2}
			      \end{equation*}
			      则原式
			      \begin{equation*}
				      \lim_{n\to\infty}\frac{1+2+3+\cdots+n}{n^2} = \lim_{n\to\infty}(\frac12 + \frac{1}{2n}) = \frac12
			      \end{equation*}
		      \end{solution}
		\item 分解因式:$ x^4 - 2x^2y - 3y^2 + 8y - 4 $.
		      \begin{solution}
			      \begin{align*}
				      x^4 - 2x^2y - 3y^2 + 8y - 4 & = x^4 - 2x^2y + y^2 - (4y^2 - 8y + 4)  \\
				                                  & = (x^2 - y)^2 - (2y - 2)^2             \\
				                                  & = (x^2 - y + 2y - 2)(x^2 - y - 2y + 2) \\
				                                  & = (x^2 + y - 2)(x^2 - 3y + 2)
			      \end{align*}
		      \end{solution}
	\end{enumerate}
	\question 过抛物线$ y^2 = 4x
	$的焦点作倾斜角为$\frac34\pi$的直线,它与抛物线相交于$A$、$B$两点。求$A$、$B$两点间的距离。
	\begin{solution}
		抛物线的焦点为$(0,1)$,斜角为$\frac34\pi$的直线的斜率为$-1$,设直线的方程为$ y = -x + b $,将点$(0,1)$代入解得
		$b = 1$,则直线方程为$ y=-x+1 $。将直线方程代入抛物线方程得:
		\begin{equation*}
			(-x + 1)^2 = 4x
		\end{equation*}
		展开得
		\begin{align*}
			x^2 - 2x + 1 & = 4x \\
			x^2 -6x + 1  & = 0  \\
		\end{align*}
		解得$ x_1,x_2 = \frac{6 \pm \sqrt{32}}{2} = 3 \pm 2\sqrt{2} $,代入直线方程可得$ y_1,y_2 = -(3\pm2\sqrt{2}) +
			1$,则点$A$的坐标为$(3+2\sqrt{2}, -2 - 2\sqrt{2})$,$B$点的坐标为$(3-2\sqrt{2}, -2 + 2\sqrt{2})$,$AB$的距离为:
		\begin{align*}
			\sqrt{(3-2\sqrt{2} - 3 - 2\sqrt{2})^2 + (-2 + 2\sqrt{2} + 2 + 2\sqrt{2})^2}
			 & = \sqrt{32 + 32} \\
			 & = 8
		\end{align*}
	\end{solution}
\end{questions}

\end{document}

%!tex program = lualatex
\documentclass[answers]{exam}
\usepackage{ctex}
\usepackage{graphicx}
\usepackage[margin=2cm]{geometry}
\usepackage{amsmath, amssymb}
\usepackage{csquotes}
\usepackage{tikz, pgfplots}
\usetikzlibrary{
	angles,
	backgrounds,
	calc,
	decorations.pathmorphing,
	decorations.pathreplacing,
	decorations.text,
	intersections,
	patterns,
	quotes,
	shapes,
	shapes.symbols,
}
\pagestyle{empty}
\newcounter{xcord}
\newcounter{ycord}
\newcounter{total}
\renewcommand{\labelenumi}{\textbf{\ifnum\value{enumi}<10 0\fi\arabic{enumi})}}

\pgfplotsset{compat=1.18}

\CorrectChoiceEmphasis{\color{blue!70!green}\bfseries}
\renewcommand{\solutiontitle}{\textbf{解:}}

\usepackage{array, tabularx}
\newcolumntype{C}{>{\centering\arraybackslash}X}
\newcolumntype{B}{>{\centering\bfseries\arraybackslash}X}
\catcode`\幺=0

\usepackage[lua]{tkz-euclide}

\begin{document}
\begin{center}
	\textbf{1977普通高等学校招生考试(上海卷)}

	\textbf{\Huge 理科数学}
\end{center}
\begin{questions}
	\question
	\begin{enumerate}[label=(\arabic*)]
		\item 化简:
		      \begin{math} \displaystyle
			      \left(\frac{a}{a+b} - \frac{a^2}{a^2 + 2ab + b^2}\right) \div \left(\frac{a}{a+b} - \frac{a^2}{a^2 - b^2}\right)
		      \end{math}.
		      \begin{solution}
			      \begin{align*}
				      \left(\frac{a}{a+b} - \frac{a^2}{a^2 + 2ab + b^2}\right) \div \left(\frac{a}{a+b} - \frac{a^2}{a^2 - b^2}\right)
				       & = \left[ \frac{a(a+b)}{(a+b)^2} - \frac{a^2}{(a+b)^2} \right] \div \frac{a(a-b) - a^2}{a^2-b^2}
				      \\
				       & = \frac{ab}{(a+b)^2} \cdot \frac{a^2 - b^2}{-ab}                                                \\
				       & = \frac{b-a}{a+b}
			      \end{align*}
		      \end{solution}
		\item 计算:
		      \begin{math}
			      \displaystyle
			      \frac12\lg25+\lg2 - \lg\sqrt{0.1} - \log_29 \times \log_32.
		      \end{math}
		      \begin{solution}
			      \begin{align*}
				       & = \frac12\lg5^2 + \lg2 - \lg(10)^{-\frac12} - \log_23^2 \times \log_32 \\
				       & = \lg5 + \lg2 + \frac12 - 2 \log_23 \times \log_32                     \\
				       & = \frac32 - 2 \cdot \frac{\ln3}{\ln2}\cdot \frac{\ln2}{\ln3}           \\
				       & = \frac12
			      \end{align*}
		      \end{solution}
		\item $\sqrt{-1} = i$,验算$i$是否方程$2x^4 + 3x^3 - 3x^2 + 3x - 5 = 0$的解。
		      \begin{solution}
			      将$i$代入方程得:
			      \begin{align*}
				      2i^4 + 3i^3 - 3x^2 + 3x - 5 & = 2 - 3i + 3  + 3i - 5 \\
				                                  & = 0
			      \end{align*}
			      所以$i$是原方程的解。
		      \end{solution}
		\item 求证:$\displaystyle
			      \frac{\sin\left(\frac{\pi}{4} + \theta \right)}{\sin \left( \frac{\pi}{4} - \theta \right)} +
			      \frac{\cos\left(\frac{\pi}{4} + \theta \right)}{\cos \left( \frac{\pi}{4} - \theta \right)} =
			      \frac{2}{\cos2\theta}
		      $.
		      \begin{solution}
			      根据和角公式
			      \[
				      \begin{array}{l}
					      \sin(\alpha+\beta) = \sin\alpha\cos\beta + \cos\alpha\sin\beta, \\
					      \cos(\alpha+\beta)=\cos\alpha\cos\beta - \sin\alpha\sin\beta
				      \end{array}
			      \]
			      及差角公式$$
				      \begin{array}{l}
					      \sin(\alpha-\beta) = \sin\alpha\cos\beta - \cos\alpha\sin\beta, \\
					      \cos(\alpha-\beta)=\cos\alpha\cos\beta + \sin\alpha\sin\beta
				      \end{array}
			      $$来化简等式的左边得:
			      \begin{align*}
				       & = 	\frac{\sin\frac{\pi}{4}\cos\theta +
					      \cos\frac{\pi}{4}\sin\theta}{\sin\frac{\pi}{4}\cos\theta-\cos\frac{\pi}{4}\sin\theta} +
				      \frac{\cos\frac{\pi}{4}\cos\theta - \sin\frac{\pi}{4}\sin\theta}{\cos\frac{\pi}{4}\cos\theta +
				      \sin\frac{\pi}{4}\sin\theta}                                                              \\
				       & =
				      \frac{\sin\theta + \cos\theta}{\cos\theta - \sin\theta} + \frac{\cos\theta -
				      \sin\theta}{\cos\theta + \sin\theta}                                                      \\
				       & =
				      \frac{1 + 2\sin\theta\cos\theta + 1 - 2\sin\theta\cos\theta}{\cos^2\theta - \sin^2\theta} \\
				       & = \frac{2}{\cos2\theta}
			      \end{align*}
			      因此原式成立。
		      \end{solution}
	\end{enumerate}
	\question 在$\triangle{ABC}$中,$\angle{C}$的平分线交$AB$于$D$,过$D$作$BC$的平行线交$AC$于$E$,已知$BC=a, AC=b$,求$DE$的长。

	\begin{figure}[htbp]
		\centering
		\begin{tikzpicture}
			\tkzDefPoints{0/0/A, 3/0/C, 2/2.5/B}
			\tkzDefLine[bisector](B,C,A) \tkzGetPoint{x}
			\tkzInterLL(C,x)(A,B) \tkzGetPoint{D}
			\tkzDefLine[parallel=through D](B,C) \tkzGetPoint{x}
			\tkzInterLL(D,x)(A,C) \tkzGetPoint{E}

			\tkzDrawPolygon(A,B,C)
			\tkzDrawSegments(D,E D,C)

			\tkzLabelPoints[below](A,E,C)
			\tkzLabelPoints[above left](D,B)
		\end{tikzpicture}
	\end{figure}

	\begin{solution}
		\begin{align*}
			 & \because \angle{ACD} = \angle{BCD}                  \\
			 & \therefore AD = BD                                  \\
			 & \because DE \parallel BC                            \\
			 & \therefore
			\begin{array}{l}
				AE = EC \\
				\dfrac{DE}{BC} = \dfrac{AE}{AC}
			\end{array}                         \\
			 & \therefore DE = BC\frac{\frac{1}{2}a}{a} = \frac12b
		\end{align*}
	\end{solution}
	\pagebreak
	\question
	已知圆$A$的直径为$2\sqrt{3}$,圆$B$的直径为$4-2\sqrt{3}$,圆$C$的直径为$2$,圆$A$与圆$B$外切,圆$A$又与圆$C$外切,$\angle{A}=60^\circ$,求$BC$及$\angle{C}$。

	\begin{figure*}[ht]
		\centering
		\begin{tikzpicture}
			\tkzDefPoints{{sqrt(3)}/0/A, {sqrt(3)-2}/0/B, 0/0/O}
			\tkzDefShiftPoint[A](120:{sqrt(3)+1}){C}
			\tkzDefLine[orthogonal=through C](B,A) \tkzGetPoint{x}
			\tkzInterLL(C,x)(A,B) \tkzGetPoint{D}
			\tkzInterLC(A,C)(A,O) \tkzGetSecondPoint{x}

			\tkzDrawCircles(A,O B,O)
			\tkzDrawCircle(C,x)
			\tkzDrawPolygon(A,B,C)

			\tkzLabelPoints(B,A,D)
			\tkzLabelPoint[above](C){$C$}
			\tkzMarkAngle[size=.5](C,A,B)
			\tkzLabelAngle(C,A,B){$60^\circ$}

			\tkzDrawSegment[dashed](C,D)
		\end{tikzpicture}
	\end{figure*}

	\begin{solution}
		由题意得:
		\begin{align*}
			AB & = 2            \\
			AC & = 1 + \sqrt{3}
		\end{align*}
		由余弦定理有:
		\begin{align*}
			BC^2 & = AB^2 + AC^2 - 2\cdot AB \cdot AC \cdot \cos60^\circ           \\
			     & = 4 + 2\sqrt{3} + 4 - 2\cdot 2 \cdot (1+\sqrt{3}) \cdot \frac12 \\
			     & = 6
		\end{align*}
		所以
		\begin{equation*}
			BC = \sqrt{6}
		\end{equation*}

		再由余弦定理有:
		\begin{align*}
			\cos{C} & = \frac{AC^2 + BC^2 - AB^2}{2AC\cdot BC}                            \\
			        & = \frac{(1+\sqrt{3})^2 + (\sqrt{6})^2 - 2^2}{2(1+\sqrt{3})\sqrt{6}} \\
			        & = \frac{4 + 2\sqrt{3} + 6 - 4}{2\sqrt{2}(3+\sqrt{3})}               \\
			        & = \frac{2(3+\sqrt{3})}{2\sqrt{2}(3+\sqrt{3})}                       \\
			        & = \frac{\sqrt{2}}{2}
		\end{align*}
		所以有:
		\begin{equation*}
			\angle{C} = 45^\circ
		\end{equation*}
	\end{solution}

\end{questions}

\end{document}

\section[1977年高考数学试卷及答案(天津卷)理科]{1977年普通高等学校招生考试(天津卷)\\\Huge 数学试卷}

\begin{questions}
	\question
	\begin{parts}
		\part 在什么条件下,$\dfrac{y}{2x}$
			\begin{cenum}
				\item 是正数;
				\item 是负数;
				\item 等于零;
				\item 没有意义?
			\end{cenum}

			\begin{solution}
				\begin{cenum}
					\item $\sgn(x)=\sgn(y)$且$x\neq 0$;
					\item $\sgn(x)\neq\sgn(y)$且$x\neq 0$;
					\item $y=0$且$x\neq 0$;
					\item $x=0$
				\end{cenum}
			\end{solution}

		\part 比较下列各组数的大小,并说明理由.
			\begin{cenum}
				\item $\cos\ang{31}$与$\cos\ang{30}$.
				      \begin{solution}
					      \begin{center}
						      \begin{tikzpicture}
							      \begin{axis}[
									      xmin = -pi/2, xmax = 2.5*pi,
									      ymin = -1.5, ymax = 1.5,
								      ]
								      \addplot[domain=0:2*pi]{cos(deg(x))};
								      \addlegendentry{$\cos(x)$}

								      \node[above] at ({pi/2},0) {$\frac{\pi}{2}$};
							      \end{axis}
						      \end{tikzpicture}
					      \end{center}
					      可以看到余弦函数在第一象限是单调递减的,所以$\cos\ang{31} < \cos\ang{30}$.

				      \end{solution}
				\item $\log_21$与$\log_2\frac14$.
				      \begin{solution}
					      对二者分别求值:
					      \begin{align*}
						      \log_21 = \log_2{2^0} = 0 \\
						      \log_2{\frac14} = \log_2{2^{-2}} = -2
					      \end{align*}
					      所以有$\log_21 > \log_2{\frac14}$.
				      \end{solution}
			\end{cenum}
		\part 求值:
			\begin{cenum}
				\item $\tan \left( 5\arcsin\frac{\sqrt{3}}{2} \right)$.
				      \begin{solution}
					      \begin{align*}
						       & = \tan \left( 5\times \frac{\pi}{3} \right) \\
						       & = \tan \left( \frac{5\pi}{3} \right)        \\
						       & = -\sqrt{3}
					      \end{align*}
				      \end{solution}
				\item $(-2)^0 \times (0.01)^\frac12$.
				      \begin{solution}
					      \begin{align*}
						       & = 1 \times 1^{-2\times\frac12} \\
						       & = 1
					      \end{align*}
				      \end{solution}
			\end{cenum}
		\part 计算:$\lg12.5-\lg\frac58+\lg\sin\ang{30}$.
			\begin{solution}
				\begin{align*}
					 & = \lg\frac{25}{2} - \lg\frac58 + \lg\frac12               \\
					 & = \lg \left( \frac{25}{2}\cdot\frac85\cdot\frac12 \right) \\
					 & = \lg10                                                   \\
					 & = 1
				\end{align*}
			\end{solution}

		\part 解方程:
			\begin{math}
				\frac{4x}{x^2-4} - \frac2{x-2} = 1 - \frac1{x+2}.
			\end{math}

			\begin{solution}
				两边同乘以$x^2 - 4, (x\neq\pm2)$得:
				\begin{equation*}
					4x - 2(x+2) = x^2 - 4 - (x-2)
				\end{equation*}
				整理得:
				\begin{equation*}
					x^2 -3x + 2 = 0
				\end{equation*}
				分解因式得:
				\begin{equation*}
					(x-2)(x-1) = 0
				\end{equation*}
				因为$x\neq\pm2$,所以$x=1$.
			\end{solution}

	\end{parts}

	\question
	\begin{parts}
		\part
			某工厂准备在仓库的一侧建立一个矩形储料场(如图),现有$50$米长的铁丝网,如果用它来围成这个储料场,那么长和宽各是多少时,这个储料场的面积最大?并求出这个最大的面积.
			\begin{center}
				\begin{tikzpicture}
					\draw[very thick](0,1) -| (1,.2) (0,-1) -| (1,-.2);
					\draw[thin](1,.7) -| (3, -.7) -- (1,-.7);

					\node at (.5,0){仓库};
					\node at (2,0){储料场};
					\node at (2, -1) {$x$};
					\node[right] at (3, 0) {$y$};
				\end{tikzpicture}
			\end{center}

			\begin{solution}
				面积可以表示成:
				\begin{align*}
					S & = xy          \\
					  & = x(50-2x)    \\
					  & = -2x^2 + 50x
				\end{align*}
				根据抛物线的性质,在$-\frac{b}{2a}=\frac{25}{2}$处面积有最大值,代入面积公式得:
				\begin{equation*}
					S_{max} = \frac{25}{2}(50 - 2\times\frac{25}2) = \qty{312.5}{\meter\squared}
				\end{equation*}
			\end{solution}

		\part 如图,已知$AB$、$DE$是圆$O$的直径,$AC$是弦,$AC\parallel DE$,求证:$CE=EB$.
			\begin{center}
				\begin{tikzpicture}[scale=1.3]
					\draw (0,0) circle (1);
					\draw (0,1) -- (0,-1) (50:1) -- (230:1) -- (130:1);

					\node[above] at (0,1) {$E$};
					\node[below] at (0,-1) {$D$};
					\node[above right] at (50:1) {$B$};
					\node[below left] at (230:1) {$A$};
					\node[above] at (130:1) {$C$};
				\end{tikzpicture}
			\end{center}

			\begin{proofsolution}
				\begin{center}
					\begin{tikzpicture}[scale=1.3]
						\draw (0,0) circle (1);
						\draw (0,1) -- (0,-1) (50:1) -- (230:1) -- (130:1);

						\node[above] at (0,1) {$E$};
						\node[below] at (0,-1) {$D$};
						\node[above right] at (50:1) {$B$};
						\node[below left] at (230:1) {$A$};
						\node[above] at (130:1) {$C$};

						\draw[dashed, red](0,0) -- (130:1);
						\node[right] at (0,0) {$O$};
					\end{tikzpicture}
				\end{center}
				\begin{cenum}
					\item 连接圆心$O$和$C$点;
					\item
					      \begin{align*}
						       & \because CO = AO                                                 \\
						       & \therefore \angle{A} = \angle{C}                                 \\
						       & \because AC \parallel ED                                         \\
						       & \therefore \angle{C} = \angle{COE} \land \angle{A} = \angle{EOB} \\
						       & \therefore \angle{COE} = \angle{EOB}                             \\
						       & \therefore CE = EB
					      \end{align*}
				\end{cenum}
			\end{proofsolution}
	\end{parts}

	\question 如果已知$bx^2 - 4bx + 2(a+c) = 0\quad (b\neq0)$有两个相等的实数根,求证:$a,b,c$成等差数列.
	\begin{proofsolution}
		因为方程有两个相等的实数根,所以其判别式$\Delta=0$:
		\begin{align*}
			\Delta  = \sqrt{16b^2 - 4b\cdot2(a+c)} = 0 \\
			16b^2 - 8b(a+c) = 0                        \\
			2b = a+ c                                  \\
			b - a = c - b
		\end{align*}
		所以$a,b,c$成等差数列.
	\end{proofsolution}

	\question
	\begin{parts}
		\part
			如图,为求河对岸某建筑的高$AB$,在地面上引一条基线$CD=a$,测得$\angle{ACB}=\alpha,\angle{BCD}=\beta,\angle{BDC=\gamma}$,求$AB$.

			\begin{center}
				\begin{tikzpicture}
					\coordinate(A) at (0,1);
					\coordinate(B) at (0,0);
					\coordinate(C) at (3,-2.5);
					\coordinate(D) at (4,0);
					\draw[fill=gray!10, very thin](3,2) -- (4.5,1.2) to[out=210, in=80] (.5, -2.7) -- (-1, -2) to[out=35, in=250]
					(3,2);
					\draw(C)node[below]{$C$} -- (A)node[above]{$A$} -- (B)node[left]{$B$} -- (C) --
					(D)node[right]{$D$};
					\draw[dashed] (0,0) -- (4,0);
				\end{tikzpicture}
			\end{center}

			\begin{solution}
				\begin{center}
					\begin{tikzpicture}
						\coordinate(A) at (0,1);
						\coordinate(B) at (0,0);
						\coordinate(C) at (3,-2.5);
						\coordinate(D) at (4,0);
						\draw[fill=gray!10, very thin](3,2) -- (4.5,1.2) to[out=210, in=80] (.5, -2.7) -- (-1, -2) to[out=35, in=250]
						(3,2);
						\draw(C)node[below]{$C$} -- (A)node[above]{$A$} -- (B)node[left]{$B$} -- (C) --node[midway,
							right]{$a$} (D)node[right]{$D$};
						\draw[dashed] (0,0) -- (4,0);

						\pic[draw=red, angle radius=16mm, angle eccentricity=1.2, "$\alpha$"]{angle=A--C--B};
						\pic[draw=blue, angle radius=4mm, angle eccentricity=1.5, "$\beta$"]{angle=D--C--B};
						\pic[draw=green, angle radius=4mm, angle eccentricity=1.5, "$\gamma$"]{angle=B--D--C};
					\end{tikzpicture}
				\end{center}
				根据正弦定理有:
				\begin{equation*}
					\frac{a}{\sin(\pi-\gamma-\beta)} = \frac{BC}{\sin\gamma}
				\end{equation*}
				解得
				\begin{equation*}
					BC = \frac{a\sin\gamma}{\sin{(\pi - \gamma - \beta)}}
				\end{equation*}
				由三角关系得:
				\begin{equation*}
					AB = BC\cdot \tan\alpha
				\end{equation*}
				代入$BC$得:
				\begin{equation*}
					AB = \frac{a\sin\gamma\tan\alpha}{\sin(\pi - \gamma - \beta)}
				\end{equation*}
			\end{solution}
		\part 如果$\alpha=\ang{30},\beta=\ang{75}, \gamma=\ang{45}, a=\qty{33}{米}$,求建筑物$AB$的高度.(保留一位小数)

			\begin{solution}
				\begin{align*}
					AB & = \frac{33\sin\ang{45}\tan\ang{30}}{\sin(\ang{180}-\ang{45} -\ang{75})} \\
					   & = \frac{33\frac{\sqrt{2}}{2}\frac{\sqrt{3}}{3}}{\frac{\sqrt{3}}{2}}     \\
					   & = 11\sqrt{2}                                                            \\
					   & = \qty{15.4}{米}
				\end{align*}
			\end{solution}
	\end{parts}

	\question
	\begin{parts}
		\part 求直线$3x-2y+1=0$和$x+3y+4=0$的交点坐标.

			\begin{solution}
				解方程组
				\begin{math}
					\left\{
					\begin{array}{ll}
						3x-2y+1= 0 & (1) \\
						x+3y+4 =0  & (2)
					\end{array}
					\right.
				\end{math}

				将式$(1)\times 3 +$式$(2)\times 2$得:
				\begin{equation*}
					11x + 11 = 0
				\end{equation*}
				即 $x=-1$,代入式$(1)$中得:
				\begin{equation*}
					y = 1
				\end{equation*}

				所以交点的坐标为$(-1,1)$.

			\end{solution}

		\part 求通过上述交点,并同直线$x+3y+4=0$垂直的直线方程.

			\begin{solution}
				直线$x+3y+4=0$的斜率为:
				\begin{equation*}
					k_1 = \frac13
				\end{equation*}
				两条垂直的直线的斜率的乘积为$-1$,所以得所求直线的斜率为:
				\begin{equation*}
					k_2 = -3
				\end{equation*}
				设所求直线方程为:
				\begin{equation*}
					y = -3x + b
				\end{equation*}
				将上述的交点$(-1,1)$代入直线方程得:
				\begin{equation*}
					b = -2
				\end{equation*}
				所以所求直线方程为:
				\begin{equation*}
					y + 3x + 2 = 0
				\end{equation*}
			\end{solution}

	\end{parts}
\end{questions}

%!tex program = lualatex
\documentclass[answers]{exam}
\usepackage{ctex}
\usepackage{graphicx}
\usepackage[margin=2cm]{geometry}
\usepackage{amsmath, amssymb}
\usepackage{csquotes}
\usepackage{tikz, pgfplots}
\usetikzlibrary{
	angles,
	backgrounds,
	calc,
	decorations.pathmorphing,
	decorations.pathreplacing,
	decorations.text,
	intersections,
	patterns,
	quotes,
	shapes,
	shapes.symbols,
}
\pagestyle{empty}
\newcounter{xcord}
\newcounter{ycord}
\newcounter{total}
\renewcommand{\labelenumi}{\textbf{\ifnum\value{enumi}<10 0\fi\arabic{enumi})}}

\pgfplotsset{compat=1.18}

\CorrectChoiceEmphasis{\color{blue!70!green}\bfseries}
\renewcommand{\solutiontitle}{\textbf{解:}}

\usepackage{array, tabularx}
\newcolumntype{C}{>{\centering\arraybackslash}X}
\newcolumntype{B}{>{\centering\bfseries\arraybackslash}X}
\catcode`\幺=0

\usepackage{tkz-euclide}

\begin{document}
\begin{center}
	\textbf{1997年普通高等学校招生考试(全国卷)}

	\textbf{\Large 理科数学}
\end{center}
\begin{questions}
	\question 设集合$M=\{x|0 \leqslant x < 2\}$,集合$N=\{x|x^2 - 2x - 3 <0\}$,集合$M\cap N=$ \hfill (\hspace{2cm})
	\begin{oneparchoices}
		\choice $\{x|0\leqslant x < 1\}$
		\CorrectChoice $\{x|0\leqslant x < 2\}$
		\choice $\{x|0\leqslant x \leqslant 1\}$
		\choice $\{x|0\leqslant x \leqslant 2\}$
	\end{oneparchoices}

	\begin{solution}
		集合$N=\{x|-1 < x < 3\}$,两个集合的范围如下图所示:

		\begin{tikzpicture}
			\tkzInit[xmin=-4, xmax=4]
			\tkzDrawX

			\draw[rounded corners=2pt, thick, black!50] (-1,0) |- (3,.5) -- (3,0);
			\draw[black!50, thick, pattern=north east lines,pattern color=black!50, rounded corners=2pt] (0,0) |- (2,.8) -- (2,0);
			\draw [fill=white](-1,0) circle (1pt);
			\draw [fill=white](3,0) circle (1pt);
			\draw [fill=black](0,0) circle (1pt);
			\draw [fill=white](2,0) circle (1pt);
		\end{tikzpicture}
	\end{solution}

	\question 如果直线$ax+2y+2=0$与直线$3x-y-2=0$平行,那么系数$a=$ \hfs

	\begin{oneparchoices}
		\choice $-3$
		\CorrectChoice $-6$
		\choice $-\dfrac32$
		\choice $\dfrac23$
	\end{oneparchoices}

	\begin{solution}
		两条直线平行则其斜率相等,可得:
		\begin{equation*}
			-\frac{a}{2} = 3
		\end{equation*}
		所以答案为$a=-6$
	\end{solution}

	\question 函数$y=\tan \left( \dfrac12x - \dfrac{\pi}{3} \right)$在一个周期内的图像是 \hfs

	\begin{oneparchoices}
		\pgfplotsset{
			xlabel={$x$},
			ylabel={$y$},
			x label style={at={(current axis.right of origin)}, anchor=north},
			axis lines=center,
			samples=100,
			ticks=none
		}
		\CorrectChoice
		\begin{tikzpicture}[scale=.4]
			\begin{axis}[
					xmin = {-2/3*pi},
					xmax = {2*pi},
					ymin = -3,
					ymax = 3,
				]
				\addplot[domain={-1/3*pi+0.1}:{5/3*pi-0.1}]{tan(deg(1/2*x - pi/3))};
				\draw[dashed] (-pi/3,3) -- (-pi/3, -3);
				\draw[dashed] (pi*5/3,3) -- (pi*5/3, -3);
				\node[below left] at (-pi/3, 0) {$-\frac{\pi}{3}$};
				\node[below right] at (2*pi/3, 0) {$\frac{2\pi}{3}$};
				\node[below left] at (5*pi/3, 0) {$\frac{5\pi}{3}$};
				\node[below left] at (0,0) {$O$};
			\end{axis}
		\end{tikzpicture}
		\choice
		\begin{tikzpicture}[scale=.4]
			\begin{axis}[
					xmin = {-1/6*pi},
					xmax = {8/6*pi},
					ymin = -3,
					ymax = 3,
				]
				\addplot[domain={1/6*pi+0.1}:{7/6*pi-0.1}]{tan(deg(x - 2/3*pi))};
				\draw[dashed] (pi/6,3) -- (pi/6, -3);
				\draw[dashed] (pi*7/6,3) -- (pi*7/6, -3);
				\node[below left] at (pi/6, 0) {$\frac{\pi}{6}$};
				\node[below right] at (2*pi/3, 0) {$\frac{2\pi}{3}$};
				\node[below left] at (7*pi/6, 0) {$\frac{7\pi}{6}$};
				\node[below left] at (0,0) {$O$};
			\end{axis}
		\end{tikzpicture}
		\choice
		\begin{tikzpicture}[scale=.4]
			\begin{axis}[
					xmin = {-pi},
					xmax = {5/3*pi},
					ymin = -3,
					ymax = 3,
				]
				\addplot[domain={-2/3*pi+0.1}:{4/3*pi-0.1}]{tan(deg(1/2*x - pi/6))};
				\draw[dashed] (-pi*2/3,3) -- (-pi*2/3, -3);
				\draw[dashed] (pi*4/3,3) -- (pi*4/3, -3);
				\node[below left] at (-pi*2/3, 0) {$-\frac{2\pi}{3}$};
				\node[below right] at (pi/3, 0) {$\frac{\pi}{3}$};
				\node[below left] at (4*pi/3, 0) {$\frac{4\pi}{3}$};
				\node[below left] at (0,0) {$O$};
			\end{axis}
		\end{tikzpicture}
		\choice
		\begin{tikzpicture}[scale=.4]
			\begin{axis}[
					xmin = {-1/3*pi},
					xmax = {pi},
					ymin = -3,
					ymax = 3,
				]
				\addplot[domain={-1/6*pi+0.1}:{5/6*pi-0.1}]{tan(deg(x - 1/3*pi))};
				\draw[dashed] (-pi/6,3) -- (-pi/6, -3);
				\draw[dashed] (pi*5/6,3) -- (pi*5/6, -3);
				\node[below left] at (-pi/6, 0) {$-\frac{\pi}{6}$};
				\node[below right] at (pi/3, 0) {$\frac{\pi}{3}$};
				\node[below left] at (5*pi/6, 0) {$\frac{5\pi}{6}$};
				\node[below left] at (0,0) {$O$};
			\end{axis}
		\end{tikzpicture}
	\end{oneparchoices}
	\begin{solution}
		$\tan{x}$的周期是$\pi$,所以$\tan{\frac12{x}}$的周期应该为$2\pi$.因此排除B和D.
		$\tan(\frac12\cdot\frac{\pi}{3}-\frac{\pi}{3}) \neq 0$,因此选A.
	\end{solution}
	\question
	已知三棱锥$D-ABC$的三个侧面与底面全等,且$AB=AC=\sqrt{3}$,$BC=2$,则以$BC$为棱,以面$BCD$与面$BCA$为面的二面角的大小是
	\hfs

	\begin{oneparchoices}
		\choice $\arccos\dfrac{\sqrt{3}}{3}$
		\choice $\arccos\dfrac{1}{3}$
		\CorrectChoice $\dfrac{\pi}{2}$
		\choice $\dfrac{2\pi}{3}$
	\end{oneparchoices}

	\begin{solution}

		\tdplotsetmaincoords{20}{120}
		\begin{center}
			\begin{tikzpicture}[tdplot_main_coords, baseline=(current bounding box.north)]
				\coordinate(A) at ({sqrt(2)}, 0);
				\coordinate(B) at (0,1);
				\coordinate(C) at (0,-1);
				\coordinate(D) at (0,0,{sqrt(2)});

				\draw(A)node[below]{$A$} -- (B)node[below]{$B$};
				\draw[dashed](B)-- (C)node[above]{$C$};
				\draw(C)-- (A);
				\draw(D)node[above]{$D$} -- (A);
				\draw(D) -- (B);
				\draw(D) -- (C);

				\draw[blue, dashed](A) -- (0,0)node[below right]{$O$} -- (D);
			\end{tikzpicture}
		\end{center}

		取$BC$的中点为$O$,连接$AO$和$DO$.
		因为$AC=AB$,所以有$AO\perp BC$.因为$\triangle{BCD}\cong\triangle{BCA}$,所以也有$DO\perp
			BC$.计算可得$AO=DO=\sqrt{2}$.另外有$AD=2$.可以看出有$AO^2 + DO^2 =
			AD^2$,所以$\triangle{AOD}$是直角三角形,即二面角为$\ang{90}$.
	\end{solution}

	\question 函数$y=\sin \left( \dfrac{\pi}{3} - 2x \right) + \cos2x$的最小正周期是 \hfs

	\begin{oneparchoices}
		\choice $\dfrac{\pi}{2}$
		\CorrectChoice $\pi$
		\choice $2\pi$
		\choice $\dfrac{2\pi}{3}$
	\end{oneparchoices}

	\begin{solution}
		因为$\sin \left( \dfrac{\pi}{3} -2x \right)$与$\cos2x$的周期均为$\pi$,所以函数的最小正周期也为$\pi$.
	\end{solution}

	\question 满足$\arccos(1-x) \geqslant \arccos{x}$的取值范围是 \hfs

	\begin{oneparchoices}
		\choice $\left[-1, -\dfrac12\right]$
		\choice $\left[-\dfrac12, 0\right]$
		\choice $\left[0, \dfrac12\right]$
		\CorrectChoice $\left[\dfrac12, 1\right]$
	\end{oneparchoices}

	\begin{solution}
		因为$\arccos$的定义域是$[-1,1]$,所以排队选项A和B.

		$\arccos(0) = \frac{\pi}{2}, \arccos(1-0)=0$,所以C也可以排除.答案为D.
	\end{solution}

	\question 将$y=2^x$的图像\underline{\hspace{.5cm}},再作关于直线$y=x$对称的图像,可得到函数$y=\log_2(x+1)$的图像. \hfs

	\begin{oneparchoices}
		\choice 先向左平移$1$个单位
		\choice 先向右平移$1$个单位
		\choice 先向上平移$1$个单位
		\CorrectChoice 先向下平移$1$个单位
	\end{oneparchoices}

	\begin{solution}
		点$(0,0)$在直线$y=x$上,并且也在函数$y=\log_2(x+1)$上,所以应该也在$y=2^x$平移后的图像上,可以得到平移后的图像为$y=2^x
		- 1$才能满足点$(0,0)$在图像上.
	\end{solution}

\end{questions}
\end{document}

%!tex program = lualatex
\documentclass[answers]{exam}
\usepackage{ctex}
\usepackage{graphicx}
\usepackage[margin=2cm]{geometry}
\usepackage{amsmath, amssymb}
\usepackage{csquotes}
\usepackage{tikz, pgfplots}
\usetikzlibrary{
	angles,
	backgrounds,
	calc,
	decorations.pathmorphing,
	decorations.pathreplacing,
	decorations.text,
	intersections,
	patterns,
	quotes,
	shapes,
	shapes.symbols,
}
\pagestyle{empty}
\newcounter{xcord}
\newcounter{ycord}
\newcounter{total}
\renewcommand{\labelenumi}{\textbf{\ifnum\value{enumi}<10 0\fi\arabic{enumi})}}

\pgfplotsset{compat=1.18}

\CorrectChoiceEmphasis{\color{blue!70!green}\bfseries}
\renewcommand{\solutiontitle}{\textbf{解:}}

\usepackage{array, tabularx}
\newcolumntype{C}{>{\centering\arraybackslash}X}
\newcolumntype{B}{>{\centering\bfseries\arraybackslash}X}
\catcode`\幺=0

\usepackage[lua]{tkz-euclide}

\renewcommand{\choicelabel}{(\Alph{choice})}
\begin{document}
\begin{center}
	\textbf{1998年普通高等学校招生考试(全国卷)}\\
	\textbf{\Large 理科数学}
\end{center}

\begin{questions}
	\question $\sin\ang{600}$的值是 \hfill (\hspace{1cm})

	\begin{oneparchoices}
		\choice $\frac12$
		\choice $-\frac12$
		\choice $\frac{\sqrt{3}}{2}$
		\CorrectChoice $-\frac{\sqrt{3}}{2}$
	\end{oneparchoices}

	\begin{solution}
		\begin{align*}
			\sin\ang{600} = \sin\ang{240} = -\frac{\sqrt{3}}{2}
		\end{align*}
	\end{solution}

	\question 函数$y=a^{|x|}(a>1)$的图像是\hfill (\hspace{1cm})

	\begin{oneparchoices}
		\choice
		\begin{tikzpicture}[scale=0.45]
			\begin{axis}[
					axis lines=middle,
					xlabel = $x$,
					ylabel = $y$,
					domain = -3:3,
					samples = 100,
					ymin = -1,
					ymax = 5,
					ticks = none
				]
				\addplot[thick, domain=0:3]{2^x - 1};
				\addplot[thick, domain=-3:0]{2^(-x) - 1};
				\node[below left] at(0,0) {$O$};
			\end{axis}
		\end{tikzpicture}
		\CorrectChoice
		\begin{tikzpicture}[scale=0.45]
			\begin{axis}[
					axis lines=middle,
					xlabel = $x$,
					ylabel = $y$,
					domain = -3:3,
					samples = 100,
					ymin = -1,
					ymax = 5,
					ticks = none
				]
				\addplot[thick, domain=0:3]{2^x};
				\addplot[thick, domain=-3:0]{2^(-x)};
				\node[below left] at(0,0) {$O$};
				\node[below left] at(0,1) {$1$};
			\end{axis}
		\end{tikzpicture}

		\choice
		\begin{tikzpicture}[scale=0.45]
			\begin{axis}[
					axis lines=middle,
					xlabel = $x$,
					ylabel = $y$,
					domain = -3:3,
					samples = 100,
					ymin = -1,
					ymax = 5,
					ticks = none
				]
				\addplot[thick, domain=0:3]{2^(-x)};
				\addplot[thick, domain=-3:0]{2^(x)};
				\node[below left] at(0,0) {$O$};
				\node[above left] at(0,1) {$1$};
			\end{axis}
		\end{tikzpicture}

		\choice
		\begin{tikzpicture}[scale=0.45]
			\begin{axis}[
					axis lines=middle,
					xlabel = $x$,
					ylabel = $y$,
					domain = -3:3,
					samples = 100,
					ymin = -1,
					ymax = 5,
					ticks = none
				]
				\addplot[thick, domain=0:3]{2-2^(-x)};
				\addplot[thick, domain=-3:0]{2-2^(x)};
				\node[below left] at(0,0) {$O$};
				\node[below left] at(0,1) {$1$};
			\end{axis}
		\end{tikzpicture}
	\end{oneparchoices}

	\begin{solution}
		$a^{|x|} (a>1)$的最小值是$1$,而且不收敛,所以答案是B。
	\end{solution}

	\question 曲线的极坐标方程$\rho=4\sin\theta$化成直角坐标方程为 \hfill (\hspace{1cm})

	\begin{oneparchoices}
		\choice $x^2 + (y+2)^2 = 4$ \CorrectChoice $x^2 + (y-2)^2 = 4$
		\choice $(x-2)^2 + y^2 = 4$ \choice $(x+2)^2 + y^2 = 4$
	\end{oneparchoices}
	\begin{solution}
		设曲线上的动点$P(x,y)$,则有:
		\begin{align*}
			y    & = \rho\sin\theta                 \\
			x    & = \rho\cos\theta                 \\
			\rho & = \sqrt{x^2 + y^2} = 4\sin\theta
		\end{align*}
		则有
		\begin{align*}
			\sqrt{x^2 + y^2} & = 4\frac{y}{\rho} \\
			x^2 + y^2        & = 4y              \\
			x^2 + (y-2)^2    & = 4
		\end{align*}
	\end{solution}

	\question 两条直线$A_1x + B_1y + C_1 = 0, A_2x + B_2y + C_2=0$垂直的充要条件是 \hfill(\hspace{1cm})

	\begin{oneparchoices}
		\choice $A_1A_2 + B_1B_2 =0$ \choice $A_1A_2 - B_1B_2 = 0$ \CorrectChoice $\dfrac{A_1A_2}{B_1B_2} = -1$
		\choice $\dfrac{A_1A_2}{B_1B_2} = 1$
	\end{oneparchoices}

	\begin{solution}
		直线的斜率分别为$-\dfrac{A_1}{B_1}$和$-\dfrac{A_2}{B_2}$,两条直线垂直的充要条件是斜率的乘积为$-1$,所以答案选择C。
	\end{solution}

	\question 函数$f(x)=\frac1x(x\neq0)$的反函数$f^{-1}(x)=$ \hfill (\hspace{1cm})

	\begin{oneparchoices}
		\choice $x ( x\neq0)$ \CorrectChoice $\frac1x(x\neq0)$ \choice $-x (x\neq0)$ \choice $-\frac1x (x\neq0)$
	\end{oneparchoices}
	\begin{solution}
		因为$y=\frac1x$关于$y=x$对称,所以反函数是其本身。
	\end{solution}

	\question 已知点$P(\sin\alpha-\cos\alpha, \tan\alpha)$在第一象限,则在$[0,2\pi]$内$\alpha$的取值范围是 \hfill
	(\hspace{1cm})

	\begin{choices}
		\choice $\left( \dfrac{\pi}{2}, \dfrac{3\pi}{4} \right) \cup \left( \pi, \dfrac{5\pi}{4} \right)$
		\CorrectChoice $\left( \dfrac{\pi}{4}, \dfrac{\pi}{2} \right) \cup \left( \pi, \dfrac{5\pi}{4} \right)$
		\choice $\left( \dfrac{\pi}{2}, \dfrac{3\pi}{4} \right) \cup \left( \dfrac{5\pi}{4}, \dfrac{3\pi}{2} \right)$
		\choice $\left( \dfrac{\pi}{4}, \dfrac{\pi}{2} \right) \cup \left( \dfrac{3\pi}{4}, \pi \right)$
	\end{choices}

	\begin{solution}
		根据条件有:
		\begin{align*}
			\sin\theta - \cos\theta > 0 \tag{a} \\
			\tan\theta = \frac{\sin\theta}{\cos\theta} > 0 \tag{b}
		\end{align*}
		根据式$(a)$有$\alpha \in \left( \dfrac{\pi}{4}, \dfrac{5\pi}{4} \right)$;根据式$(b)$有$\alpha \in \left( 0,
			\dfrac{\pi}{2} \right) \cup \left( \pi, \dfrac{3\pi}{2} \right)$

		综上可得取值范围为B。
	\end{solution}

	\question 已知圆锥的全面积是底面积的$3$倍,那么该圆锥的侧面展开图扇形的圆心角为\hfill (\hspace{1cm})

	\begin{oneparchoices}
		\choice $\ang{120}$
		\choice $\ang{150}$
		\choice $\ang{180}$
		\choice $\ang{240}$
	\end{oneparchoices}

	\begin{solution}
		设圆锥的底面半径为$r$,棱线长为$l$,则底面积为$\pi r^2$,侧面积为$\frac{2\pi r}{2\pi l}\pi l^2 = \pi
			rl$,由题意有$\pi rl = 2\pi r^2$,即$l=2r$,展开图的圆心角为$2\pi\frac{r}{l}=\pi$,选C。
	\end{solution}

	\question 复数$-i$的一个立方根是$i$,它的另外两个立方根是 \hfill (\hspace{1cm})

	\begin{oneparchoices}
		\choice $\dfrac{\sqrt{3}}{2} \pm \dfrac12i$
		\choice $-\dfrac{\sqrt{3}}{2} \pm \dfrac12i$
		\choice $\pm\dfrac{\sqrt{3}}{2} + \dfrac12i$
		\CorrectChoice $\pm\dfrac{\sqrt{3}}{2} - \dfrac12i$
	\end{oneparchoices}

	\begin{solution}
		根据欧拉恒等式$e^{i\theta} = \cos\theta + i\sin\theta$,当$\theta=\frac32\pi$时有$e^{i\frac32\pi} =
			-i$。其立方根为$e^{i(\frac{\frac32\pi + 2k\pi}{3})}, (k=0,1,2)$。

		\begin{enumerate}[label=(\arabic*)]
			\item 当$k=0$时,$e^{\frac12\pi} = \cos \left( \dfrac12\pi \right) + \sin \left( \dfrac12\pi \right)i = i$
			\item 当$k=1$时,$e^{\frac76\pi} = \cos \left( \dfrac76\pi \right) + \sin \left( \dfrac76\pi \right)i =
				      -\frac{\sqrt{3}}{2} - \frac12i$
			\item 当$k=2$时,$e^{\frac{11}6\pi} = \cos \left( \dfrac{11}6\pi \right) + \sin \left( \dfrac{11}6\pi \right)i =
				      \frac{\sqrt{3}}{2} - \frac12i$
		\end{enumerate}
		答案选D。
	\end{solution}

	\question 如果棱台的两底面积分别是$S,S'$,中截面的面积是$S_0$,那么 \hfill (\hspace{1cm})

	\begin{oneparchoices}
		\CorrectChoice $2\sqrt{S_0} = \sqrt{S} + \sqrt{S'}$
		\choice $S_0 = \sqrt{SS'}$
		\choice $2S_0 = S + S'$
		\choice $S_0^2 = 2SS'$
	\end{oneparchoices}

	\begin{solution}
		一维方向上有 $\sqrt{S_0} = \frac{\sqrt{S} + \sqrt{S'}}{2}$,所以选择A。
	\end{solution}

	\question 向高为$H$的水瓶中注水,注满为止,如果注水量$V$与水深$h$的函数关系的图像如下图所示,那么水瓶的形状是 \hfill
	(\hspace{1cm})

	\begin{tikzpicture}[scale=0.45]
		\begin{axis}[
				axis lines=middle,
				xlabel = $h$,
				ylabel = $V$,
				domain = -3:3,
				samples = 100,
				ymin = -1,
				ymax = 5,
				xmax = 3,
				xmin = -1,
				ticks = none
			]
			\addplot[thick, domain=0:2]{-x^2+ 4*x};
			\node[below left] at(0,0) {$O$};
			\draw [dashed] (2,4)  -- (2,0) node[below]{$H$};
		\end{axis}
	\end{tikzpicture}

	\tdplotsetmaincoords{80}{0}
	\begin{oneparchoices}
		\choice
		\begin{tikzpicture}[tdplot_main_coords, scale=.5]
			\tkzDefPoints{0/0/O, 1/0/A, -1/0/B}
			\coordinate(O') at (0,0,4);
			\coordinate(A') at (2,0,4);
			\coordinate(B') at (-2,0,4);

			\tkzDrawSegments(A,A' B,B')
			\tkzDrawSemiCircle(O',A')
			\tkzDrawSemiCircle(O',B')

			\tkzDrawSemiCircle[dashed](O,A)
			\tkzDrawSemiCircle(O,B)
		\end{tikzpicture}

		\CorrectChoice
		\begin{tikzpicture}[tdplot_main_coords, scale=.5]
			\tkzDefPoints{0/0/O, 2/0/A, -2/0/B}
			\coordinate(O') at (0,0,4);
			\coordinate(A') at (1,0,4);
			\coordinate(B') at (-1,0,4);

			\tkzDrawSegments(A,A' B,B')
			\tkzDrawSemiCircle(O',A')
			\tkzDrawSemiCircle(O',B')

			\tkzDrawSemiCircle[dashed](O,A)
			\tkzDrawSemiCircle(O,B)
		\end{tikzpicture}

		\choice
		\begin{tikzpicture}[tdplot_main_coords, scale=.5]
			\tkzDefPoints{0/0/O, 2/0/A, -2/0/B}
			\coordinate(O') at (0,0,4);
			\coordinate(A') at (2,0,4);
			\coordinate(B') at (-2,0,4);
			\coordinate(C) at (1.5,0, 2);
			\coordinate(C') at (-1.5,0, 2);

			\tkzDrawSegments(A,C C,A' B,C' C',B')
			\tkzDrawSemiCircle(O',A')
			\tkzDrawSemiCircle(O',B')

			\tkzDrawSemiCircle[dashed](O,A)
			\tkzDrawSemiCircle(O,B)
		\end{tikzpicture}

		\choice
		\begin{tikzpicture}[tdplot_main_coords, scale=.5]
			\tkzDefPoints{0/0/O, 2/0/A, -2/0/B}
			\coordinate(O') at (0,0,4);
			\coordinate(A') at (2,0,4);
			\coordinate(B') at (-2,0,4);

			\tkzDrawSegments(A,A' B,B')
			\tkzDrawSemiCircle(O',A')
			\tkzDrawSemiCircle(O',B')

			\tkzDrawSemiCircle[dashed](O,A)
			\tkzDrawSemiCircle(O,B)
		\end{tikzpicture}
	\end{oneparchoices}

	\begin{solution}
		由函数图像可以看出斜率是逐渐下降的,所以选择B。
	\end{solution}

	\question $3$名医生和$6$名护士被分配到$3$所学校为学生体检,每校分配$1$名医生和$2$名护士,不同的分配方法共有\hfill
	(\hspace{1cm})

	\begin{oneparchoices}
		\choice $90$种
		\choice $180$种
		\choice $270$种
		\CorrectChoice $540$种
	\end{oneparchoices}

	\begin{solution}
		$3$名医生分到三个学校共有$3!=6$种分配方法。

		$6$名护士分到三个学校,每个学校两名护士有$\displaystyle\binom{6}{2}\cdot\binom{4}{2}\cdot\binom{2}{2}=90$种分配方法。

		所以总共有$6\times90=540$种分配方法。
	\end{solution}

	\question 椭圆$\dfrac{x^2}{12} + \dfrac{y^2}{3} =
		1$的焦点为$F_1$和$F_2$,点$P$在椭圆上,如果线段$PF_1$的中点在$y$轴上,那么$|PF_1|$是$|PF_2|$的\hfill (\hspace{1cm})

	\begin{oneparchoices}
		\choice $7$倍
		\choice $5$倍
		\choice $4$倍
		\choice $3$倍
	\end{oneparchoices}

	\begin{solution}
		$c=\sqrt{12-3} = 3$,所以焦点分别为$(-3,0), (3,0)$。

		如果$PF_1$的中点在$y$轴上,则可知$P$点的$x$坐标为$3$。此时$\triangle{F_1F_2P}$是一个直角三角形,其中$|F_1F_2|=6$,$|PF_2|=\dfrac{\sqrt{3}}{2}$,则$|PF_1|=\sqrt{6^2+\frac{3}{4}}=\frac72\sqrt{3}$,所以答案选A。
	\end{solution}

	\question
	球面上有$3$个点,其中任意两点的球面距离都等于大圆周长的$\frac16$,经过这$3$个点的小圆的周长为$4\pi$,那么这个球的半径为
	\hfill (\hspace{1cm})

	\begin{oneparchoices}
		\choice $4\sqrt{3}$
		\CorrectChoice $2\sqrt{3}$
		\choice $2$
		\choice $\sqrt{3}$
	\end{oneparchoices}

	\begin{solution}
		经过三个点的小圆的周长为$4\pi$,则小圆的半径$r=2$。则可进一步推算出小圆内接等边三角形的边长为$2\sqrt{3}$。因为任意两点的球面距离是大圆的周长的$\frac16$,所以两个点之间的夹角为$\ang{60}$,因此球心与其中两个点的组成一个等边三角形,所以球的半径等于两点之间的距离$2\sqrt{3}$。
	\end{solution}

	\question 一个直角三角形三内角的正弦值成等比数列,其最小内角为\hfill (\hspace{1cm})

	\begin{oneparchoices}
		\choice $\arccos\dfrac{\sqrt{5}-1}{2}$
		\CorrectChoice $\arcsin\dfrac{\sqrt{5}-1}{2}$
		\choice $\arccos\dfrac{1-\sqrt{5}}{2}$
		\choice $\arcsin\dfrac{1-\sqrt{5}}{2}$
	\end{oneparchoices}

	\begin{solution}
		根据正弦定理有$\frac{\sin A}{a}=\frac{\sin B}{b}=\frac{\sin C}{c}$,所以三条边也成等比数列。
		\begin{equation*}
			\frac{a}{b} = \frac{b}{c} \tag{1}
		\end{equation*}
		又因为三角形是直角三角形,所以有
		\begin{equation*}
			a^2 + b^2 = c^2 \tag{2}
		\end{equation*}
		将式$(1)$转化为$b^2=ac$并代入式$(2)$可得:
		\begin{equation*}
			a^2 + ac = c^2
		\end{equation*}
		两边都除以$ac$得:
		\begin{equation*}
			\frac{a}{c} + 1 = \frac{c}{a}
		\end{equation*}

		设$\sin{A}=\frac{a}{c}=x$则原式可以表示为:
		\begin{align*}
			x + 1 = \frac{1}{x} \\
			x^2 + x - 1 = 0     \\
			x_1,x_2 = \frac{-1\pm\sqrt{5}}{2}
		\end{align*}
		因为$\angle{A}<\ang{90}$,所以$\sin{A}=\frac{-1+\sqrt{5}}{2}$,所以$\angle{A}$可以表示为$\arcsin{\frac{-1+\sqrt{5}}{2}}$。
	\end{solution}

	\question 在等比数列${a_n}$中,$a_1>1$,且前$n$项和$S_n$满足$\displaystyle
		\lim_{n\to\infty}S_n=\frac{1}{a_1}$,那么$a_1$的取值范围是 \hfill (\hspace{1cm})

	\begin{oneparchoices}
		\choice $(1,+\infty)$
		\choice $(1,4)$
		\choice $(1,2)$
		\CorrectChoice $(1,\sqrt{2})$
	\end{oneparchoices}

	\begin{solution}
		\begin{align*}
			\lim_{n\to\infty}S_n & = \lim_{n\to\infty}a_1\frac{1-q^n}{1-q} \\
			                     & = a_1\frac{1}{1-q}\qquad (|q|< 1)       \\
			                     & = a_1
		\end{align*}
		整理得:
		\begin{equation*}
			a_1^2 = 1 - q \qquad (|q|<1)
		\end{equation*}
		则有$0<a_1^2<2$,结合题中$a_1>1$可得$a_1$的取值范围为$(1,\sqrt{2})$。
	\end{solution}

	\question 设圆过双曲线$\frac{x^2}{9} -
		\frac{y^2}{16}=1$的一个顶点和一个焦点,圆心在此双曲线上,则圆心到双曲线中心的距离是\underline{\hspace{2cm}}。

	\begin{solution}
		双曲线的顶点为$(-3,0)$和$(3,0)$,焦点为$(-5,0)$和$(5,0)$,由圆的特性可知圆心必在过顶点和焦点中心的垂直线上。这样可以推断出圆是过同一侧的顶点和焦点,因为如果不在同一侧,中心线分别为$x=\pm1$,与双曲线没有交点。

		那么可以考察右侧的顶点和焦点,则圆心在$x=4$上,代入双曲线方程可以计算得$y^2=\dfrac{112}{9}$,则圆心到双曲线中心$(0,0)$的距离为:
		\begin{align*}
			\sqrt{x^2 + y^2} & = \sqrt{4^2 + \frac{112}{9}} \\
			                 & = \frac{16}{3}
		\end{align*}
	\end{solution}

	\question $(x+2)^{10}(x^2-1)$的展开式中$x^{10}$的系数为\fillin[179][2cm]。(用数字作答)

	\begin{solution}
		根据二项式定理$(x+2)^{10}$可以展开为
		\begin{equation*}
			\sum_{n=0}^{10}\binom{10}{n}x^{10-n}2^{n}
		\end{equation*}
		则有$x^10$的系数为$1$,$x^8$的系数为$\binom{10}{2}2^2 =
			180$,再乘以$(x^2-1)$后原来的$x^{10}$的系数变为$-1$,原来$x^8$变为$x^{10}$,系数为$180$不变,则有展开式的$x^{10}$的系数为$179$。
	\end{solution}

	\question 在直四棱柱$A_1B_1C_1D_1-ABCD$中,当底面四边形$ABCD$满足条件\fillin[正方形或菱形][2cm]时,有$A_1C\perp
		B_1D_1$。(注:填上你认为正确的一种条件即可,不必考虑所有可能的情形)

	\begin{solution}
		\tdplotsetmaincoords{60}{110}
		\begin{tikzpicture}[tdplot_main_coords]
			\tkzDefPoints{0/0/A,2/0/B,2/2/C,0/2/D}
			\coordinate (A1) at (0,0,4);
			\coordinate (B1) at (2,0,4);
			\coordinate (C1) at (2,2,4);
			\coordinate (D1) at (0,2,4);

			\tkzDrawPolygon(A1,B1,C1,D1)
			\tkzDrawSegments(B,B1 C,C1 D,D1 B,C C,D)
			\tkzDrawSegments[dashed](A,A1 B,A A,D A1,C)

			\tkzLabelPoints(A,B,C,D)
			\tkzLabelPoints[above](A1,B1,C1,D1)

			\tkzDrawSegments(B1,D1 A1,C1)

		\end{tikzpicture}
		可以看到$A_1C$在四边形$A_1B_1C_1D_1$上的投影为$A_1C_1$,如果$A_1C_1 \perp B_1D_1$,则有$A_1C \perp
			B_1D_1$,所以底面四边形$ABCD$如果是正方形或者菱形即可。
	\end{solution}

	\question 关于函数$f(x)=4\sin \left( 2x + \dfrac{\pi}{3} \right)\quad (x\in \mathbf{R})$,有下列命题:
	\begin{enumerate}[label=\protect\circled{\arabic*}]
		\item 由$f(x_1) = f(x_2) = 0$可得$x_1 - x_2$必是$\pi$的整数倍;
		\item $y=f(x)$的表达式可改写为$y=4\cos \left( 2x - \dfrac{\pi}{6} \right)$;
		\item $y=f(x)$的图像关于点$\left( -\dfrac{\pi}{6}, 0\right)$对称;
		\item $y=f(x)$的图像关于直线$x=-\dfrac{\pi}{6}$对称。
	\end{enumerate}
	其中正确的命题的序号是\fillin[\circled{2}][2cm]。(注:把你认为正确的序号都填上)

	\begin{solution}
		\begin{enumerate}[label=\protect\circled{\arabic*}]
			\item 由$f(x_1) = f(x_2) = 0$可知$2x_1 + \frac{\pi}{3} - 2x_2 - \frac{\pi}{3} = 2(x_1-x_2) =
				      k\pi$,所以$x_1-x_2=\frac{k}{2}\pi$  \textcolor{red}{\times}
			\item $y=4\sin \left( 2x + \dfrac{\pi}{3} \right)=4\cos \left( \dfrac{\pi}{2} - 2x - \dfrac{\pi}{3} \right)
				      = 4\cos \left( \dfrac{\pi}{6} - 2x \right) = 4\cos \left( 2x - \dfrac{\pi}{6} \right)$ \textcolor{red}{\checkmark}
			\item 如图,所以\circled{3}、\circled{4}错误。

			      \begin{center}
				      \begin{tikzpicture}
					      \begin{axis}[
							      axis lines=middle,
							      xlabel = $x$,
							      ylabel = $y$,
							      xlabel style={right},
							      ylabel style={above},
							      ymax = 5,
							      ymin = -5,
							      samples = 100,
							      ticks = none
						      ]
						      \addplot[thick, domain=-2*pi:2*pi]{4*sin(deg(2*x + pi/3))};
						      \draw[draw=red, fill=red]({-pi/6}, 0)node[above left]{$-\frac{\pi}{6}$}circle(2pt);
					      \end{axis}

				      \end{tikzpicture}
			      \end{center}

		\end{enumerate}
	\end{solution}
	\question 在$\triangle{ABC}$中,$a$,$b$,$c$分别是角$A,B,C$的对边,设$a+c=2b,
		A-C=\dfrac{\pi}{3}$,求$\sin{B}$的值。

	\begin{solution}
		由正弦定理有$\dfrac{a}{\sin{A}}=\dfrac{b}{\sin{B}}=\dfrac{c}{\sin{C}}$和题中的$a+c=2b$可得:
		\begin{equation*}
			\sin{A} + \sin{C} = 2\sin{B}
		\end{equation*}
		由和差化积公式$\sin\alpha + \sin\beta = 2\sin{\dfrac{\alpha+\beta}{2}}\cos{\dfrac{\alpha-\beta}{2}}$可得:
		\begin{align*}
			\sin{A} + \sin{C} & = 2\sin \left( \frac{A+C}{2} \right)\cos \left( \frac{A-C}{2} \right) \\
			                  & = 2\sin \left( \frac{A+C+B - B}{2} \right) \cos{\frac{\pi}{6}}        \\
			                  & = \sqrt{3}\sin(\ang{90}-\frac{B}{2})                                  \\
			                  & = \sqrt{3}\cos\left(\frac{B}{2}\right)
		\end{align*}
		又因为$\sin{B}=2\sin\left(\dfrac{B}{2}\right)\cos\left(\dfrac{B}{2}\right)$,代入得:
		\begin{align*}
			4\sin\left(\frac{B}{2}\right)\cos\left(\frac{B}{2}\right) = \sqrt{3}\cos\left(\frac{B}{2}\right) \\
			\sin\left(\frac{B}{2}\right) = \frac{\sqrt{3}}{4}
		\end{align*}
		因为$\frac{B}{2}$肯定为锐角,所以
		\begin{equation*}
			\cos\left(\frac{B}{2}\right) = \sqrt{1 - \sin^2 \left(\frac{B}{2}\right)}= \frac{\sqrt{13}}{4}
		\end{equation*}
		则有:
		\begin{equation*}
			\sin{B} = 2\sin{\left( \frac{B}{2} \right)} \cos{\left( \frac{B}{2}
				\right)}=2\cdot\frac{\sqrt{3}}{4}\cdot\frac{\sqrt{13}}{4} = \frac{\sqrt{39}}{8}
		\end{equation*}
	\end{solution}

	\question 如图,直线$l_1$和$l_2$相交于点$M$,$l_1 \perp l_2$,点$N\in
		l_1$。以$A,B$为端点的曲线段$C$上的任一点到$l_2$的距离与点$N$的距离相等。若$\triangle{AMN}$为锐角三角形,$|AM|=\sqrt{17},|AN|=3,$且$|BN|=6$。建立适当的坐标系,求曲线$C$的方程。
	\begin{figure*}[htbp]
		\centering
		\begin{tikzpicture}[scale=0.5]
			\tkzDefPoints{0/0/M, 3/0/N, 0/3/P}
			\tkzDefPoint(40:{sqrt(17)}){A}
			\tkzDefPoint(6,5){B}
			\tkzInit[xmin=-2, xmax=6]
			\tkzDrawX

			\tkzDefLine[orthogonal=through A](M,P) \tkzGetPoint{x}
			\tkzInterLL(M,P)(A,x) \tkzGetPoint{D}

			\tkzDrawSegments[add=.5 and .5](M,N M,P)
			\tkzDrawSegments(A,M A,N B,N)
			\draw[shorten >= -1cm , shorten <= -1cm, -Latex] ([xshift=1.5cm]M) --([xshift=1.5cm]P) node[above=1cm]{$y$};
			\draw[very thin] (A) .. controls (4,5) and (5.5,5) .. (B);
			\tkzDrawSegment[dashed](A,D)

			\tkzLabelSegment[right=1.5cm, below](M,N){$l_1$}
			\tkzLabelSegment[above=1.5cm](M,P){$l_2$}
			\tkzLabelPoint[below left](M){$M$}
			\tkzLabelPoint[below](N){$N$}
			\tkzLabelPoint[above left](A){$A$}
			\tkzLabelPoint[above right](B){$B$}
			\tkzLabelPoint[left](D){$D$}
		\end{tikzpicture}
	\end{figure*}

	\begin{solution}
		\begin{enumerate}[label=\protect\circled{\arabic*}]
			\item 以$l_1$为$x$轴,过$MN$的中线为$y$轴,设抛物线的方程为$x=\dfrac{y^2}{4p}$。
			\item 过$A$点作垂线到$l_2$,垂足为$D$
			      \begin{align*}
				       & \because AD = AN = 3                                                  \\
				       & \therefore A \text{点横坐标为} 3-p                                         \\
				       & \therefore  |DM| = \sqrt{|AM|^2 - |DA|^2} = \sqrt{17 - 9} = 2\sqrt{2} \\
				       & \therefore A\text{点的纵坐标为}2\sqrt{2}                                    \\
			      \end{align*}
			\item 将$A$点坐标$(3-p, 2\sqrt{2})$代入抛物线方程得:
			      \begin{align*}
				      3-p = \frac{(2\sqrt{2})^2}{4p} \\
				      p^2 -3p + 2 = 0                \\
				      (p-2)(p-1) = 0
			      \end{align*}
			      则有$p_1 = 1, p_2 = 2$
			\item 当$p=1$时,$MN=2$,此时有$|AM|^2 > |AN|^2 +
				      |MN|^2$,可以判定$\triangle{AMN}$是钝角三角形,与条件不符,所以$p=2$,即曲线的方程为$x=\dfrac{y^2}{8}$。
			\item 根据题意可知$B$点的横坐标为$6-p=4$,代入曲线方程得:
			      \begin{align*}
				      4 = \frac{y^2}{8}
				      y = \pm4\sqrt{2}
			      \end{align*}
			      因为曲线段$C$在第一象限,所以$B$点坐标为$(4,4\sqrt{2})$

			\item 综上,曲线的方程为:
			      \begin{equation*}
				      x = \frac{y^2}{8} \quad (1 \leqslant x \leqslant 4, y>0)
			      \end{equation*}
		\end{enumerate}

	\end{solution}

	\question
	如图,为处理含有某种杂质的污水,要制造一底宽为$2$米的均长方体沉淀箱,污水从$A$孔流入,经沉淀后从$B$孔流出。设箱体的长度为$a$米,调试为$b$米。已知流出的水中该杂质的质量分数与$a,b$的乘积$ab$成反比。现有制箱材料$60$平方米。问当$a,b$各为多少米时,经沉淀后流出的水路该杂质的质量分数最小。($A$、$B$孔的面积忽略不计)

	\begin{figure*}[ht]
		\centering
		\begin{tikzpicture}[scale=.5]
			\coordinate(A) at (0,0);
			\coordinate(B) at (5,0);
			\coordinate(C) at (5,3);
			\coordinate(D) at (0,3);
			\coordinate(A') at (0,0,3);
			\coordinate(B') at (5,0,3);
			\coordinate(C') at (5,3,3);
			\coordinate(D') at (0,3,3);

			\draw [dashed](D) -- (A) -- (B) ;
			\draw (B)	-- (C) -- (D);
			\draw (A') -- (B') -- (C') -- (D')--cycle;
			\draw[dashed](A) -- (A');
			\draw(B) -- (B');
			\draw(C) -- (C');
			\draw(D) -- (D');

			\draw (0,1.5,1.5) circle (2pt);
			\draw (5,1.5,1.5) circle (2pt);
			\node[below] at(0,1.5,1.5) {$A$};
			\node[above] at(5,1.5,1.5) {$B$};

			\tkzLabelSegment(A',B'){$a$}
			\tkzLabelSegment(B,B'){$2$}
			\tkzLabelSegment[left](C',B'){$b$}
		\end{tikzpicture}
	\end{figure*}

	\begin{solution}
		根据总面积为$60$平方米可得如下关系:
		\begin{equation*}
			60 = 2a + 2ab + 4b
		\end{equation*}
		化简得:
		\begin{equation*}
			30 = a + ab + 2b \tag{a}
		\end{equation*}
		设$y=ab$,则有$b=\frac{y}{a}$,代入上面的式子中可得:
		\begin{align*}
			30 = a + y + \frac{2}{a}y           \\
			(1+\frac2a)y = 30 - a               \\
			y = \frac{30-a}{1 + \frac2a}        \\
			y = \frac{a(30-a)}{a+2}     \tag{b} \\
		\end{align*}
		\subsection*{解法一}
		对式$(b)$求导:
		\begin{align*}
			y' & = \left( \frac{a(30-a)}{a+2} \right)'              \\
			   & = \frac{(30-2a)(a+2) - a(30-a)}{(a+2)^2}           \\
			   & = \frac{30a + 60 - 2a^2 - 4a - 30a + a^2}{(a+2)^2} \\
			   & = \frac{60 - 4a - a^2}{(a+2)^2}                    \\
			   & = \frac{(10+a)(6-a)}{(a+2)^2}
		\end{align*}
		因为$a>0$,所以只有$a=6$时$y'=0$,即此时$y=ab$有最大值。将$a=6$代入面积关系式$(a)$中可得:
		\begin{equation*}
			30 = 6 + 6b + 2b
		\end{equation*}
		解得$b=3$米。

		综上,当$a=6$米、$b=3$米时质量分数最小。

		\subsection*{解法二}
		化简式$(b)$:
		\begin{align*}
			y & = \frac{30a -a^2}{a+2}                      \\
			  & = \frac{30a + 4 + 4a - (4 + 4a + a^2)}{a+2} \\
			  & = \frac{34a + 4 - (a+2)^2}{a+2}             \\
			  & = \frac{34(a+2) -64 - (a+2)^2}{a+2}         \\
			  & = 34 - \frac{64}{a+2} - (a+2)               \\
		\end{align*}
		$\dfrac{64}{a+2} + (a+2) \geqslant \sqrt{\dfrac{64}{a+2}\cdot(a+2)} = 8$当且仅当$\dfrac{64}{a+2} =
		a+2$时有最小值,即$(a+2)^2 = 64$,因为$a>0$所以$a=6$,代入式$(a)$中求得$b=3$。
	\end{solution}
\end{questions}

\end{document}

\end{document}
\usepackage{graphicx, wrapfig}
\newcounter{xcord}
\newcounter{ycord}
\newcounter{total}
\renewcommand{\labelenumi}{\textbf{\ifnum\value{enumi}<10 0\fi\arabic{enumi})}}

\usepackage{array, tabularx}
\newcolumntype{C}{>{\centering\arraybackslash}X}
\newcolumntype{B}{>{\centering\bfseries\arraybackslash}X}

\usepackage{fontawesome5}
\newcommand\regularHandPointRight{\faHandPointRight[regular]}
