%!tex program = lualatex
\documentclass[answers, addpoints]{examz}
% exam文档中题解的标题设置
\renewcommand{\solutiontitle}{\textbf{解:}}
% 修改选项标号的样式
\renewcommand{\choicelabel}{(\Alph{choice})}
% 设置填空题答案与横线之间的间距
\setlength{\answerclearance}{6pt}
% hfs用于添加水平空白于选择题干与答案之间,最后加上一个2cm的括号
\newcommand\hfs{\hfill (\hspace{2cm})}
% 设置正确答案的样式
\CorrectChoiceEmphasis{\color{magenta}\bfseries}
% 汉化
% 因为文档类型本身的限制,在有多个questions环境时,计分表无法使用
% TODO 以后看能不能解决这个问题
\pointname{分}
\hsword{得分}
\hqword{题目}
\hpword{分值}
\htword{总计}
% 定义 proofsolution 环境
\newenvironment{proofsolution}
{\renewcommand{\solutiontitle}{\textbf{证明:}}\begin{solution}} % 环境开始部分
{\end{solution}}            % 环境结束部分

% 将页面风格设置为空,去除页眉页脚和页码等
\pagestyle{empty}
% 屏蔽掉一些不影响的警告
\usepackage{silence}
\WarningFilter{latexfont}{Font shape}
\WarningFilter{latexfont}{Some font}
\WarningFilter{latex}{Label `question}
\WarningFilter{latex}{There were multiply-defined labels}
\usepackage{ctex}
\usepackage{
ctex, 	 		% 中文支持
amsmath, 		% ams数学包
amssymb, 		% ams数学符号
amsthm,  		% ams定理环境
csquotes,		% 处理引号
arcs,	 		% 添加弧的符号
enumitem,		% 列表环境增强
tikz,	 		% 绘图工具
pgfplots,		% 绘制函数曲线
tikz-3dplot, 	% 3D 图像
tkz-euclide, 	% 简单几何绘图
siunitx,	 	% 数字单位处理
titlesec,	 	% 标题设置
caption,	 	% 图表标题
multicol,	 	% 多列环境
hyperref, 	 	% 处理超链接
}

% 修改section的标题样式
\titleformat{name=\section}
  {\centering\bfseries} % 格式:居中、大号字体、加粗
  {}              		% 编号样式
  {0pt}                 % 编号与标题间距
  {}                    % 标题前无额外内容

% 设置页面的边距为2cm
\usepackage[margin=2cm]{geometry}

% 定义一个数字加点标号的列表环境
\newenvironment{mathenum}
{\begin{enumerate}[label=\arabic*.]}
		{\end{enumerate}}

% 定义一个括号数字标号的列表环境
\newenvironment{penum}
{\begin{enumerate}[label=(\arabic*)]}
		{\end{enumerate}}

% 解决弧标记显示的问题
% TODO 这个要了解一下有没有其他实现的方式
\makeatletter
\providecommand\@gobblethree[3]{}
\patchcmd{\over@under@arc}
{\@gobbletwo}
{\@gobblethree}
{}{}
\makeatother


% 带圈数字用tikz实现
\newcommand*\circled[1]{\tikz[baseline=(char.base)]{
		\node[shape=circle,draw,inner sep=1pt] (char) {#1};}}
% 定义一个带圈数字标号的列表环境
\newenvironment{cenum} {\begin{enumerate}[label=\protect\circled{\arabic*}]} {\end{enumerate}}

% 加载tikz库
\usetikzlibrary{
	angles,
	backgrounds,
	calc,
	decorations.pathmorphing,
	decorations.pathreplacing,
	decorations.text,
	intersections,
	patterns,
	petri,
	positioning,
	quotes,
	shapes,
	shapes.symbols,
}
% 设置pgfplots兼容最新的功能
\pgfplotsset{compat=1.18}


% TODO 设计封面样式
\title{历年高考数学试卷及答案}
\author{收集于网络 by 大象同学}

\includeonly{beijing1977}
\begin{document}
\maketitle
\pagebreak
\begin{multicols}{2}
	% TODO 设置目录样式
	\color{cyan}
	\tableofcontents
\end{multicols}

\section[1977年高考数学试卷及答案(北京)理科]{1977 年普通高等学校招生考试(北京卷)\\\Huge{理科数学} }

\begin{questions}
	\question 解方程: \( \sqrt{x - 1} = 3 - x \).
	\begin{solution}
		\begin{align*}
			\sqrt{x - 1}  & = 3 - x        \\
			x - 1         & = 9 - 6x + x^2 \\
			x^2 - 7x + 10 & = 0            \\
			(x-2)(x-5)    & = 0            \\
			x_1 = 2, x_2 = 5
		\end{align*}
		验算去除$x_2=5$.所以方程的解为$x=2$.
	\end{solution}

	\question 计算: \( 2^{-\frac12} + \frac{2^0}{\sqrt{2}} + \frac{1}{\sqrt{2} - 1}. \)

	\begin{solution}
		\begin{align*}
			\text{原式} & = \frac{1}{\sqrt{2}} + \frac{1}{\sqrt{2}} + \sqrt{2} + 1 \\
			          & = \frac{2}{\sqrt{2}} + \sqrt{2} + 1                      \\
			          & = 2\sqrt{2} + 1
		\end{align*}
	\end{solution}

	\question 已知 \( \lg2 = 0.3010, \lg3 = 0.4771, \) 求 \( \lg\sqrt{45} \).

	\begin{solution}
		\begin{align*}
			\lg\sqrt{45} & = \frac12\lg{45}                        \\
			             & = \frac12(\lg5 + \lg9)                  \\
			             & = \frac12(\lg5 + 2\lg3)                 \\
			             & = \frac12(\lg\frac{10}{2} + 2\lg3)      \\
			             & = \frac12(\lg10 - \lg2 + 2\lg3)         \\
			             & = \frac12(1 - 0.3010 + 2 \times 0.4771) \\
			             & = 0.8266
		\end{align*}
	\end{solution}

	\question 证明: \( (1 + \tan\alpha)^2 = \dfrac{1 + \sin2\alpha}{\cos^2\alpha} \).
	\begin{solution}
		\begin{cenum}
			\item 展开左边表达式

			由三角恒等式 $\tan\alpha = \dfrac{\sin\alpha}{\cos\alpha}$,代入左边:
			\[
				(1 + \tan\alpha)^2 = \left(1 + \dfrac{\sin\alpha}{\cos\alpha}\right)^2.
			\]

			\item 通分并化简

			将括号内通分:
			\[
				= \left(\dfrac{\cos\alpha + \sin\alpha}{\cos\alpha}\right)^2.
			\]
			平方后展开分子:
			\[
				= \dfrac{(\cos\alpha + \sin\alpha)^2}{\cos^2\alpha}.
			\]

			\item 展开分子多项式

			应用完全平方公式:
			\[
				(\cos\alpha + \sin\alpha)^2 = \cos^2\alpha + 2\cos\alpha\sin\alpha + \sin^2\alpha.
			\]
			因此:
			\[
				= \dfrac{\cos^2\alpha + 2\cos\alpha\sin\alpha + \sin^2\alpha}{\cos^2\alpha}.
			\]

			\item 应用基本恒等式

			利用 $\cos^2\alpha + \sin^2\alpha = 1$:
			\[
				= \dfrac{1 + 2\cos\alpha\sin\alpha}{\cos^2\alpha}.
			\]

			\item 引入倍角公式

			根据 $\sin2\alpha = 2\cos\alpha\sin\alpha$,替换分子:
			\[
				= \dfrac{1 + \sin2\alpha}{\cos^2\alpha}.
			\]
			\item 因此,原式得证:
			\[
				(1 + \tan\alpha)^2 = \dfrac{1 + \sin2\alpha}{\cos^2\alpha}.
			\]
		\end{cenum}

	\end{solution}

	\question 求过两直线 \( x + y - 7 = 0 \) 和 \( 3x - y - 1 = 0 \) 的交点且过 \( (1, 1) \) 点的直线方程.
	\begin{solution}
		\begin{cenum}
			\item{求两直线的交点}
			联立方程组:
			\[
				\begin{cases}
					x + y - 7 = 0, \\
					3x - y - 1 = 0
				\end{cases}
			\]
			将第一式化为 $y = 7 - x$,代入第二式:
			\[
				3x - (7 - x) - 1 = 0,
			\]
			化简得:
			\[
				4x - 8 = 0 \quad \Rightarrow \quad x = 2.
			\]
			将 $x = 2$ 代入 $y = 7 - x$,得:
			\[
				y = 7 - 2 = 5.
			\]
			因此,两直线的交点为 $(2, 5)$。

			\item{求经过点 $(2,5)$ 和 $(1,1)$ 的直线方程}
			两点间直线的斜率公式为:
			\[
				k = \frac{y_2 - y_1}{x_2 - x_1}.
			\]
			将 $(x_1, y_1) = (1,1)$、$(x_2, y_2) = (2,5)$ 带入,得:
			\[
				k = \frac{5 - 1}{2 - 1} = 4.
			\]

			\item{写出直线方程}
			根据点斜式方程,设直线方程为:
			\[
				y - y_1 = k(x - x_1).
			\]
			将 $k = 4$、$(x_1, y_1) = (1,1)$ 带入,得:
			\[
				y - 1 = 4(x - 1).
			\]
			化简后得:
			\[
				y = 4x - 3.
			\]

			\item{结论}

			所求直线的方程为:
			\[
				y = 4x - 3.
			\]
		\end{cenum}
	\end{solution}

	\pagebreak
	\question 某工厂今年七月份的产值为 \( 100 \)万元,以后每月产值比上月增加 \( 20\%
	\),问今年七月份到十月份总产值是多少?

	\begin{solution}
		\begin{enumerate}[label=\protect\circled{\arabic*}]
			\item[] 根据等比数列求和公式
			      \begin{equation*}
				      S_n = a_1\frac{1- q^n}{1-q}
			      \end{equation*}

			      代入
			      \begin{equation*}
				      a_1 = 100, q = 1.2, n = 4
			      \end{equation*}
			      得:
			      \begin{align*}
				      S_4 & = 100 \times \frac{1 - 1.2^4}{1 - 1.2}                     \\
				          & = 100 \times \frac{(1 - 1.2)(1 + 1.2)(1 + 1.2^2)}{1 - 1.2} \\
				          & = 100 \times 2.2 \times 2.44                               \\
				          & = 536.8 \text{(万元)}
			      \end{align*}
		\end{enumerate}
	\end{solution}

	\question 已知二次函数 \( y = x^2 - 6x + 5 \)
	\begin{enumerate}[label=(\arabic*)]
		\item 求出它的图像的顶点坐标和对称轴方程;
		\item 画出它的图像;
		\item 分别求出它的图像和 \( x \)轴、$ y $轴的交点坐标.
	\end{enumerate}

	\begin{solution}
		\begin{enumerate}[label=(\arabic*)]
			\item
			      根据对称轴的方程
			      \begin{equation*}
				      x = -\frac{b}{2a}
			      \end{equation*}
			      代入
			      \begin{equation*}
				      a=1, b = -6
			      \end{equation*}
			      得对称轴的方程为:
			      \begin{equation*}
				      x = 3
			      \end{equation*}
			      顶点位于对称轴上,将$x=3$代入函数方程得:
			      \begin{equation*}
				      y = -4
			      \end{equation*}
			      所以顶点的坐标为$(3,-4)$.
			      \pagebreak
			\item
			      \hfill
			      \begin{tikzpicture}[baseline=(current bounding box.north)]
				      \begin{axis}[
						      axis lines=middle,          % 坐标轴通过原点
						      xlabel={$x$}, ylabel={$y$}, % 坐标轴标签
						      grid=major,                 % 显示网格
						      xmin=-2, xmax=8,            % x轴范围
						      ymin=-5, ymax=10,           % y轴范围
						      domain=-4:8,                % 定义函数绘制范围
						      samples=100,                % 样本点数,越大曲线越平滑
						      enlargelimits=true,         % 扩展范围
					      ]
					      % 绘制抛物线 y = ax^2 + bx + c
					      \addplot[smooth, thick, blue] {x^2 - 6*x + 5};
					      % 添加图例
					      \addlegendentry{$y = x^2 - 6x + 5$}
				      \end{axis}
			      \end{tikzpicture}
			      \hfill{}

			\item
			      \begin{enumerate}[label=\protect\circled{\arabic*}]
				      \item 与$x$轴相交时$y=0$,即:
				            \begin{align*}
					            x^2 - 6x + 5 = 0 \\
					            (x-1)(x-5) = 0   \\
					            x_1=1, x_2=5
				            \end{align*}
				            所以与$x$轴有两个交点,$(1,0)$和$(5,0)$.
				      \item 与$y$轴相交时$x=0$,即:
				            \begin{equation*}
					            y=0^2 - 6\times0 + 5 = 5
				            \end{equation*}
				            所以与$y$轴的交点为:$(0,5)$.
			      \end{enumerate}
		\end{enumerate}

	\end{solution}

	\question 一只船以 $20$ 海里/小时的速度向正东航行, 起初船在 $A$处看见一灯塔 $B$ 在船的北 $45^\circ$ 东方向, 一小时后船在
	$C$ 处看见这个灯塔在船的北 $15^\circ$ 东方向, 求这时船和灯塔的距离 $CB$.

	\begin{solution}
		\begin{center}
			\begin{tikzpicture}
				\coordinate(A) at (0,0);
				\coordinate(B) at (4,4);
				\coordinate(C) at (2,0);
				\coordinate(D) at (0,4);
				\coordinate(E) at (2,4);
				\draw[dashed] (D)  -- (A)node[left]{$A$};
				\draw[name path=AB] (A) --  (B)node[right]{$B$};
				\pic["$45^\circ$", draw, angle eccentricity=2] {angle=B--A--D};
				\draw (A) -- (C)node[right] {$C$};
				\draw[decorate, decoration={brace, mirror, raise=5pt}, blue, draw] (A) --node[midway, below=8pt]{20} (C);
				\draw (C)-- (B)node[right]{$B$};
				\draw[dashed, name path=CE] (E) -- (C);
				\pic["$15^\circ$", draw, angle eccentricity=2] {angle=B--C--E};
			\end{tikzpicture}

		\end{center}

		\begin{cenum}
			\item 计算$\angle{A}$和$\angle{B}$

			很明显$\angle{A}=\ang{45}$, 则有$\angle{B}=\ang{180}-\ang{45} - \ang{90} - \ang{15}=\ang{30}$.
			\item 由正弦定理:
			\begin{equation*}
				\frac{a}{\sin\angle{A}} = \frac{b}{\sin\angle{B}} = \frac{c}{\sin\angle{C}}
			\end{equation*}得
			\begin{align*}
				BC & = \frac{AC}{\sin\angle{B}}\sin\angle{A} \\
				   & = \frac{20}{\frac12}\frac{\sqrt{2}}{2}  \\
				   & = 20\sqrt{2}(\text{海里})
			\end{align*}
		\end{cenum}
	\end{solution}

	\question 一个圆内接三角形 \( ABC \),$\angle A$的平分线交$BC$于$D$,交外接圆于$E$,求证: \( AD \cdot AE = AC \cdot
	AB\).

	\renewcommand{\solutiontitle}{\textbf{证明:}}
	\begin{solution}

		\begin{minipage}{.4\textwidth}
			\begin{tikzpicture}
				\tkzDefPoint (0,0){A}
				\tkzDefPoint (5,0){B}
				\tkzDefPoint (4,4){C}

				% 外接圆
				\tkzCircumCenter(A,B,C) \tkzGetPoint{O}
				\tkzDrawCircle(O,A)

				% 角平分线
				\tkzDefLine[bisector](B,A,C) \tkzGetPoint{a}
				\tkzInterLL(A,a)(B,C) \tkzGetPoint{D}
				\tkzInterLC(A,a)(O,A) \tkzGetSecondPoint{E}
				\tkzDrawSegment(A,E)

				\tkzMarkAngle[mark=|](B,A,E)
				\tkzMarkAngle[mark=|](E,A,C)
				\tkzDrawPolygon(A,B,C)
				\tkzLabelPoints(A,B)
				\tkzLabelPoint[above](C){$C$}
				\tkzLabelPoint[above left](D){$D$}
				\tkzLabelPoint[right](E){$E$}

				\tkzDrawSegment[dashed, red](C,E)
			\end{tikzpicture}
		\end{minipage}
		\begin{minipage}{0.55\textwidth}
			连接$CE$
			\begin{align*}
				 & \because \angle{AEC}\text{和}\angle{ABC}\text{都是弦}AC\text{对应的圆周角} \\
				 & \therefore \angle{AEC} = \angle{ABC}                             \\
				 & \because \angle{CAE}=\angle{BAD}                                 \\
				 & \therefore \triangle{BAD} \sim \triangle{EAC}                    \\
				 & \therefore \frac{AC}{AE} = \frac{AD}{AB}                         \\
				 & \therefore AD \cdot AE = AC \cdot AB.
			\end{align*}
		\end{minipage}

	\end{solution}
	\renewcommand{\solutiontitle}{\textbf{解:}}

	\question 当 \( m \)取哪些值时,直线 \( y = x + m \)与椭圆 \( \dfrac{x^2}{16} + \dfrac{y^2}{9} = 1
	\)有一个交点?有两个交点?没有交点?当它们有一个交点时,画出它的图像.

	\begin{solution}
		\begin{cenum}
			\item 将 \( y = x + m \)代入 \( \dfrac{x^2}{16} + \dfrac{y^2}{9} = 1  \) 得:
			\begin{align*}
				9x^2 + 16(x+m)^2           & = 144 \\
				25x^2 + 32mx + 16m^2 - 144 & = 0   \\
			\end{align*}
			\item 计算 \( \Delta \)
			\begin{align*}
				\Delta & = b^2 - 4ac = (32m)^2 - 4\times25(16m^2 - 144) \\
				       & = 1024m^2 - 1600m^2 + 14400                    \\
				       & = 14400 - 576m^2
			\end{align*}
			\item 当 \( \Delta = 0 \)时有一个交点:
			\begin{align*}
				14400 - 576m^2         & = 0 \\
				(120 + 24m)(120 - 24m) & = 0 \\
			\end{align*}
			即 \( m = \pm5 \)时有一个交点.
			\item 当 	\(  \Delta > 0 \)时有两个交点,此时 \( -5 < m < 5 \).
			\item 当 \( \Delta < 0 \)时没有交点,此时 \( m > 5 \)或 \( m < -5 \).

		\end{cenum}
		当只有一个交点时,图像如下:
		\begin{center}
			\begin{tikzpicture}
				\begin{axis}[
						axis lines=middle,          % 坐标轴通过原点
						xlabel={$x$}, ylabel={$y$}, % 坐标轴标签
						% grid=major,                 % 显示网格
						xmin=-8, xmax=8,            % x轴范围
						ymin=-8, ymax=8,            % y轴范围
						xlabel style={anchor=west}, % x轴标签位置
						ylabel style={anchor=south},% y轴标签位置
						domain=-8:8,                % 定义函数绘制范围
						samples=100,                % 样本点数,越大曲线越平滑
						enlargelimits=true,         % 扩展范围
					]
					\draw (0, 0) ellipse [x radius= 4, y radius=3 ];
					\addplot[thick, red]{x + 5};
					\addplot[thick, blue]{x - 5};
				\end{axis}
			\end{tikzpicture}
		\end{center}
	\end{solution}

	\section*{附加题}
	\question
	\begin{enumerate}[label=(\arabic*)]
		\item 求函数 \( f(x) = \begin{cases}
			      x^2\sin\frac{\pi}{x}, (x \neq 0) \\
			      0, (x=0)
		      \end{cases} \)的导数.
		      \begin{solution}
			      \begin{cenum}
				      \item \( x \neq 0 \)
				      \begin{align*}
					      f'(x) = (x^2\sin\frac{\pi}{x})' & = (x^2)'\sin\frac{\pi}{x} + x^2\cdot(\sin\frac{\pi}{x})' \\
					                                      & = 2x\sin\frac{\pi}{x} + x^2\cdot\cos\frac{\pi}{x}
				      \end{align*}
				      \item \( x = 0 \)
				      \begin{align*}
					      f'(x) & = \lim_{\Delta{x}\to0}\frac{f(0+\Delta{x})-f(0)}{\Delta{x}}                  \\
					            & = \lim_{\Delta{x}\to0}\frac{\Delta{x}^2\sin\frac{\pi}{\Delta{x}}}{\Delta{x}} \\
					            & = \lim_{\Delta{x}\to0}\Delta{x}\sin\frac{\pi}{\Delta{x}}                     \\
					            & = 0
				      \end{align*}
			      \end{cenum}
			      综上:
			      \begin{math}
				      f'(x) = \left\{\begin{array}{ll}
					      2x\sin\frac{\pi}{x} + x^2\cos\frac{\pi}{x} & (x \neq 0) \\
					      0                                          & (x = 0)
				      \end{array}\right.
			      \end{math}
		      \end{solution}

		\item 求椭圆 \( \frac{x^2}{a^2} + \frac{y^2}{b^2} = 1 \)绕 $x$轴旋转而成的旋转体体积.
		      \begin{solution}
			      用切片法对旋转体垂直于 \( x \)轴方向的圆盘进行积分来计算旋转体的体积:\\
			      每个圆盘的面积为: \[ \pi y^2 = \pi (1-\frac{x^2}{a^2})b^2 \] \\
			      则体积为:
			      \begin{align*}
				      \int_{-a}^{a}\pi(1-\frac{x^2}{a^2})b^2 & = 2\pi ab^2 - \frac{\pi b^2\cdot 2a^3}{3a^2} \\
				                                             & = \frac{4}{3}\pi ab^2
			      \end{align*}

		      \end{solution}

	\end{enumerate}

	\question
	\begin{enumerate}[label=(\arabic*)]
		\item 试用 \( \epsilon - \delta \)语言叙述\enquote{函数 \( f(x) \)在点 \( x = x_0 \)处连续}的定义.
		      \begin{solution}
			      若对于任给的正数,总存在某一正数$\epsilon$,使得当$x-x_0<\delta$时, 总有$|f(x)-f(x_0)| <
				      \epsilon$,则称函数$f(x)$在点$x_0$处连续.
		      \end{solution}
		\item 试证明: 若 $f(x)$ 在点 $x = x_0$ 处连续, 且 $f(x_0) > 0$, 则存在一个 $x_0$ 的 $(x_0 − \delta, x_0 + \delta)$, 在这个邻域内, 处处有 $f(x) > 0$.
	\end{enumerate}

\end{questions}

%!tex program = lualatex
\documentclass[answers]{exam}
\usepackage{silence}
\WarningFilter{latexfont}{Font shape}
\WarningFilter{latexfont}{Some font}
\usepackage{ctex}
\usepackage[margin=2cm]{geometry}
\usepackage{amsmath, amssymb, amsthm, arcs}
\usepackage{siunitx}
\usepackage{csquotes}
\usepackage{tikz, pgfplots, tikz-3dplot}
\usepackage[lua]{tkz-euclide}
\usetikzlibrary{
	angles,
	backgrounds,
	calc,
	decorations.pathmorphing,
	decorations.pathreplacing,
	decorations.text,
	intersections,
	patterns,
	petri,
	positioning,
	quotes,
	shapes,
	shapes.symbols,
}
\usepackage{graphicx, wrapfig}
\usepackage{caption}
\pagestyle{empty}
\newcounter{xcord}
\newcounter{ycord}
\newcounter{total}
\renewcommand{\labelenumi}{\textbf{\ifnum\value{enumi}<10 0\fi\arabic{enumi})}}

\pgfplotsset{compat=1.18}

\CorrectChoiceEmphasis{\color{blue!70!green}\bfseries}
\renewcommand{\solutiontitle}{\textbf{解:}}
\renewcommand{\choicelabel}{(\Alph{choice})}

\usepackage{array, tabularx}
\newcolumntype{C}{>{\centering\arraybackslash}X}
\newcolumntype{B}{>{\centering\bfseries\arraybackslash}X}

\usepackage{fontawesome5}
\usepackage{enumitem}
\newcommand\regularHandPointRight{\faHandPointRight[regular]}

\newenvironment{mathenum}
{\begin{enumerate}[label=\arabic*.]}
		{\end{enumerate}}

\newenvironment{penum}
{\begin{enumerate}[label=(\arabic*)]}
		{\end{enumerate}}
\makeatletter
\providecommand\@gobblethree[3]{}
\patchcmd{\over@under@arc}
{\@gobbletwo}
{\@gobblethree}
{}{}
\makeatother

\newcommand*\circled[1]{\tikz[baseline=(char.base)]{
		\node[shape=circle,draw,inner sep=1pt] (char) {#1};}}
\setlength{\answerclearance}{6pt}
\newcommand\hfs{\hfill (\hspace{2cm})}

\usepackage[lua]{tkz-euclide}
\begin{document}
\begin{center}
	\textbf{1977年普通高等学校招生考试(福建卷)}

	\textbf{\huge{理科数学}}
\end{center}
\begin{questions}
	\question
	\begin{enumerate}[label=(\arabic*)]
		\item 计算: \( 5 - 3 \times \left[(-3\frac38)^{-\frac13} + 1031 \times (0.25 - 2^{-2})\right] \div 9^0 \).
		      \begin{solution}
			      \begin{align*}
				      \text{原式} & = 5 - 3 \times \left[ (-3\frac38)^{-\frac13} + 1031 \times 0 \right] \div 1 \\
				                  & = 5 - 3 \times (-\frac{8}{27})^{\frac13}                                    \\
				                  & = 5 - \sqrt[3]{3^3 \times (-\frac{8}{27}})                                  \\
				                  & = 5 + 2                                                                     \\
				                  & = 7
			      \end{align*}
		      \end{solution}

		\item \( y = \dfrac{\cos{160^\circ} - \cos{170^\circ}}{\tan{155^\circ}} \)的值是正的还是负的?为什么?
		      \begin{solution}
			      \begin{tikzpicture}[scale=3]
				      \tkzDefPoints{0/0/O, 1/0/A, -1/0/A'}
				      \tkzDefShiftPointCoord[O](160:1){B}
				      \tkzDefShiftPointCoord[O](170:1){C}
				      \tkzDefShiftPointCoord[O](155:1){D}

				      \tkzDefLine[orthogonal=through B](O,A) \tkzGetPoint{x}
				      \tkzInterLL(B,x)(O,A) \tkzGetPoint{B'}
				      \tkzDefLine[orthogonal=through C](O,A) \tkzGetPoint{x}
				      \tkzInterLL(C,x)(O,A) \tkzGetPoint{C'}
				      \tkzDefLine[orthogonal=through D](O,A) \tkzGetPoint{x}
				      \tkzInterLL(D,x)(O,A) \tkzGetPoint{D'}

				      \tkzDrawSegments(A,A' O,B)
				      \tkzDrawSegment[dim={$\cos{160^\circ}$, .25cm, below=2pt}](O,B')
				      \tkzDrawSegments(A,O O,C)
				      \tkzDrawSegment[dim={$\cos{170^\circ}$, .75cm, below=2pt}](O,C')
				      \tkzDrawSegments(A,O O,D)
				      \tkzDrawSegments[blue, thick](D,D' D',O)
			      \end{tikzpicture}

			      由三角函数的定义有\( \cos{160^\circ} < 0, \cos{170^\circ} < 0, \tan{155^\circ} < 0 \),并且有 \(
			      \cos{160^\circ} > \cos{170^\circ} \),则分子为正数,而且分母为负数,所以 \( y \)的值为负。
		      \end{solution}

		\item 求函数 \( y = \dfrac{\lg(2-x)}{\sqrt{x-1}} \)的定义域。
		      \begin{solution}
			      根据题意有
			      \begin{math}
				      \begin{cases}
					      2 - x > 0 \\
					      x - 1 > 0,
				      \end{cases}
			      \end{math}
			      则函数的定义域为 \( 1 < x < 2 \)
		      \end{solution}
		\item
		      \begin{minipage}[t]{0.5\textwidth}
			      如图,在梯形 \( ABCD \)中, \( DM = MP = PA, MN \parallel PQ \parallel AB \), \( DC = 2\text{cm}, AB = 3.5\text{cm} \),求 \( MN \)和 \( PQ \)的长。
		      \end{minipage}\hspace{5em}
		      \begin{tikzpicture}[baseline=(current bounding box.north)]
			      \tkzDefPoints{0/0/A, 3/0/B, 0.5/3/D, 2.5/3/C}
			      \tkzDefPointOnLine[pos={1/3}](A,D)\tkzGetPoint{P}
			      \tkzDefPointOnLine[pos={2/3}](A,D)\tkzGetPoint{M}
			      \tkzDefPointOnLine[pos={1/3}](C,B)\tkzGetPoint{N}
			      \tkzDefPointOnLine[pos={2/3}](C,B)\tkzGetPoint{Q}

			      \tkzDrawPolygon(A,B,C,D)
			      \tkzDrawSegments(P,Q M,N)

			      \tkzLabelSegment[above](D,C){$2$}
			      \tkzLabelSegment[below](A,B){$3.5$}

			      \tkzLabelPoints[left](D,M,P,A)
			      \tkzLabelPoints[right](C,N,Q,B)

		      \end{tikzpicture}

		      \begin{solution}
			      \begin{align*}
				       & \because MN \parallel PQ \parallel AB \text{并且} DM = MP = PA \\
				       & \therefore \frac{DC}{MN} = \frac{MN}{PQ} = \frac{PQ}{AB}       \\
				       & \text{将数值代入得:}                                          \\
				       & \frac{2}{MN} = \frac{MN}{PQ} = \frac{PQ}{3.5}                  \\
				       & \text{则有:} PQ = \frac{MN^2}{2} \text{和} PQ \cdot MN = 7    \\
				       & MN = \sqrt[3]{14}                                              \\
				       & PQ = \sqrt[3]{14^2} / 2                                        \\
			      \end{align*}
		      \end{solution}
		\item 已经 \( \lg3=0.4771, \lg{x}=-3.5229 \),求 \( x \)。
		      \begin{solution}
			      \begin{align*}
				       & \because \lg3                              = 0.4771                   \\
				       & \therefore \lg{\frac{3}{10}}               = \lg3 - \lg10 = -0.5229   \\
				       & \text{另有} \lg0.001                       = -3                       \\
				       & \text{则} \lg(0.001 \times \frac{3}{10} )  = -3 + (-0.5229) = -3.5229 \\
				       & \therefore x = \frac{3}{10000}
			      \end{align*}
		      \end{solution}
		\item 求 \(\displaystyle \lim_{x\to1}\frac{x-1}{x^2-3x+2} \)。
		      \begin{solution}
			      \begin{align*}
				      \frac{x-1}{x^2 - 3x + 2} & = \frac{x-1}{(x-2)(x-1)} \\
				                               & = \frac{1}{x-2}
			      \end{align*}
			      则 \( \displaystyle \lim_{x\to1}\frac{x-1}{x^2-3x+2}=-1 \)
		      \end{solution}
		\item 解方程: \( \sqrt{4x+1} - 2x + 1 = 0 \)。
		      \begin{solution}
			      移项得:
			      \begin{math}
				      \sqrt{4x+1} = 2x - 1
			      \end{math}

			      两边平方得:
			      \begin{math}
				      4x + 1 = 4x^2 - 4x + 1
			      \end{math}

			      整理得:
			      \begin{math}
				      4x^2 -8x = 0
			      \end{math}

			      提取同类项得:
			      \begin{math}
				      4x(x-2) = 0
			      \end{math}

			      则
			      \begin{math}
				      x_1 = 0, x_2 = 2
			      \end{math}

			      代入验算 \( x_1 = 0  \)不符合条件,所以解为 \( x=2 \)。
		      \end{solution}
		\item 化简: \( \displaystyle \frac{a^{2n+1} - 6a^{2n} + 9a^{2n-1}}{a^{n+1} - 4a^n + 3a^{n-1}} \)。
		      \begin{solution}
			      \begin{align*}
				      \frac{a^{2n+1} - 6a^{2n} + 9a^{2n-1}}{a^{n+1} - 4a^n + 3a^{n-1}} & = \frac{a^{2n-1}(a^2 - 6a + 9)}{a^{n-1}(a^2-4a+3)}
				      \\
				                                                                       & = \frac{a^n(a-3)^2}{(a-1)(a-3)}
				      \\
				                                                                       & = \frac{a^n(a-3)}{a-1}
			      \end{align*}
		      \end{solution}
		\item 求函数 \( y = 2 - 5x - 3x^2 \)的极值。
		      \begin{solution}
			      \begin{enumerate}[label=\alph*.]
				      \item 对函数求导
				            \begin{math}
					            \frac{\text{d}y}{\text{d}x} = -5 - 6x
				            \end{math}
				      \item 则有 \( x = -\frac56 \)时函数有极值。
				      \item 将 \( x = -\frac56 \)代入函数,得到极值:
				            \begin{math}
					            y = 2 - 5(-\frac56) - 3(-\frac56)^2 = \frac{49}{12}
				            \end{math}

			      \end{enumerate}
		      \end{solution}
		\item 画出下面 \( V \)形铁块的三视图(只要画草图)
		      \begin{tikzpicture}[scale=1]

			      % 底部立方体的前侧面
			      \fill[gray!30] (0,0,0) -- (4,0,0) -- (4,2,0) -- (0,2,0) -- cycle; % 前面
			      \fill[gray!20] (0,0,0) -- (0,2,0) -- (0,2,2) -- (0,0,2) -- cycle; % 左面
			      \fill[gray!40] (0,0,2) -- (4,0,2) -- (4,2,2) -- (0,2,2) -- cycle; % 顶面

			      % 凹陷部分左侧立方体
			      \fill[gray!50] (1,0,2) -- (1,1,2) -- (1,1,4) -- (1,0,4) -- cycle; % 左前面
			      \fill[gray!30] (1,0,2) -- (2,0,2) -- (2,0,4) -- (1,0,4) -- cycle; % 左顶面
			      \fill[gray!40] (1,0,4) -- (2,0,4) -- (2,1,4) -- (1,1,4) -- cycle; % 左侧上方面

			      % 凹陷部分右侧立方体
			      \fill[gray!50] (3,0,2) -- (3,1,2) -- (3,1,4) -- (3,0,4) -- cycle; % 右前面
			      \fill[gray!30] (3,0,2) -- (4,0,2) -- (4,0,4) -- (3,0,4) -- cycle; % 右顶面
			      \fill[gray!40] (3,0,4) -- (4,0,4) -- (4,1,4) -- (3,1,4) -- cycle; % 右侧上方面

			      % 凹陷部分底面
			      \fill[gray!20] (1,1,2) -- (3,1,2) -- (3,1,4) -- (1,1,4) -- cycle; % 凹底部面

			      % 边线
			      \draw[thick] (0,0,0) -- (4,0,0) -- (4,2,0) -- (0,2,0) -- cycle; % 底部前边线
			      \draw[thick] (0,0,0) -- (0,0,2) -- (4,0,2) -- (4,0,0);           % 底部顶面边线
			      \draw[thick] (0,2,0) -- (0,2,2) -- (4,2,2) -- (4,2,0);           % 底部后面边线

			      \draw[thick] (1,0,2) -- (1,1,2) -- (3,1,2) -- (3,0,2);           % 凹顶部边线
			      \draw[thick] (1,0,2) -- (1,0,4) -- (1,1,4) -- (1,1,2);           % 凹左侧边线
			      \draw[thick] (3,0,2) -- (3,0,4) -- (3,1,4) -- (3,1,2);           % 凹右侧边线
			      \draw[thick] (1,1,4) -- (3,1,4);                                 % 凹顶横线

		      \end{tikzpicture}
	\end{enumerate}
	\question
	\begin{enumerate}[label=(\arabic*)]
		\item 解不等式: \( \frac{x^2 - x - 6}{x^2 + 2x + 2} < 0 \).
		      \begin{solution}
			      化简得:
			      \begin{math}
				      \frac{x^2 - x - 6}{x^2 + 2x + 2}  = \frac{(x-3)(x+2)}{(x+1)^2 + 1}
			      \end{math}

			      因为分母 \( (x+1)^2 + 1  \)肯定大于0,所以有 \( (x-3)(x+2) < 0 \)

			      因此有 \( -2 < x < 3 \)
		      \end{solution}

		\item 证明: \( \dfrac{2\cos\theta - \sin2\theta}{2\cos\theta + \sin2\theta} = \tan^2\left(\dfrac{90^\circ -
			      \theta}{2}\right) \).
		      \begin{solution}
			      \begin{enumerate}[label=\arabic*. ]
				      \item 首先,根据倍角公式 \( \sin2\theta = 2\sin\theta\cos\theta \) 来化简等式的左边:
				            \begin{align*}
					            \frac{2\cos\theta - \sin2\theta}{2\cos\theta + \sin2\theta}
					             & = \frac{2\cos\theta - 2\sin\theta\cos\theta}{2\cos\theta + 2\sin\theta\cos\theta} \\
					             & = \frac{1-\sin\theta}{1+\sin\theta} \tag{a}
				            \end{align*}
				      \item 然后,结合半角公式 \( \tan\dfrac{\theta}{2} = \dfrac{\sin\theta}{1+\cos\theta} \)
				            来化简等式的右边:
				            \begin{align*}
					            \tan^2\left(\dfrac{90^\circ - \theta}{2}\right)
					             & = \left(\frac{\sin(90^\circ-\theta)}{1+\cos(90^\circ-\theta)}\right)^2 \tag{b}
				            \end{align*}
				      \item 利用差角公式 \( \sin(\alpha-\beta) = \sin\alpha\cos\beta - \cos\alpha\sin\beta \) 及 \(
				            \cos(\alpha - \beta) = \cos\alpha\cos\beta + \sin\alpha\sin\beta \) 进一步化简式(b):
				            \begin{align*}
					            \left(\frac{\sin(90^\circ-\theta)}{1+\cos(90^\circ-\theta)}\right)^2
					             & = \left(\frac{\sin90^\circ\cos\theta -
						            \cos90^\circ\sin\theta}{1+\cos90^\circ\cos\theta + \sin90^\circ\sin\theta}\right)^2
					            \\
					             & = \left(\frac{\cos\theta}{1+\sin\theta}\right)^2 \\
					             & = \frac{\cos^2\theta}{(1+\sin\theta)^2}          \\
					             & = \frac{1-\sin^2\theta}{(1+\sin\theta)^2}        \\
					             & = \frac{1-\sin\theta}{1+\sin\theta} \tag{c}
				            \end{align*}
				      \item 综上,由(a) = (c),原式成立。
			      \end{enumerate}
		      \end{solution}
		\item
		      某中学革命师生自己动手油漆一个直径为$1.2$米的地球仪,如果第平方米面积需要油漆$150$克,问共需油漆多少克?(答案保留整数。)
		      \begin{solution}
			      球体的面积公式为: \( A = 4\pi r^3 \),代入$r=1.2$得:
			      \begin{math}
				      A = 4\times 3.14 \times 1.2^2 = 18.0864
			      \end{math}

			      则共需油漆 \( 18.0864 \times 150 = 2712.96 \approx 2713 \text{g} \)
		      \end{solution}
		\item
		      某农机厂开展\enquote{工业学大庆}运动,在十月份生产拖拉机$1000$台,这样,一月至十月的产量恰好完成全年生产任务。工人同志为了加速农业机械化,计划在年底前再生产$2310$台,求十一月、十二月平均每月增长率。

		      \begin{solution}
			      设每月的平均增长率为$x$,则有 	\( 1000(1+x) + 1000(1+x)^2 = 2310 \)。
			      对式子进行化简:
			      \begin{align*}
				      1000(1+x) + 1000(1+x)^2 & = 1000 + 1000x + 1000 + 2000x + 1000x^2 \\
				                              & = 2000 + 3000x + 1000x^2 = 2310         \\
				      1000x^2 +3000x - 310    & = 0                                     \\
				      (10x + 31)(100x - 10)   & = 0                                     \\
				      x                       & = 10\%
			      \end{align*}
		      \end{solution}
	\end{enumerate}
	\question 在半径为$R$的圆内接正六边形内,依次连续各边的中点,得一正六边形,又在这一正六边形内,再依次连结各边的中点,又得一正六边形,这样无限地继续下去,求:
	\begin{enumerate}[label=(\arabic*)]
		\item 前$n$个正六边形的周长之和$S_n$;
		\item 所有这些正六边形的周长之和$S$。
	\end{enumerate}
	\begin{solution}
		\begin{enumerate}[label=\zhnum{*}、]
			\item 半径为$R$的圆的内接正六边形的边长也为$R$,所以第一个正六边形的边长为$6R$。
			\item
			      以第一个正六边形的边的中点连接而成的第二个正六边形的边长,根据三角关系得到边长为$\frac{\sqrt{3}}{2}R$,则边长为$3\sqrt{3}R$
			\item 则前$n$个正六边形的周长之和为:
			      \begin{align*}
				      S_n & = 6R + 3\sqrt{3}R + \cdots +  (\frac{\sqrt{3}}{2})^{n-1}R \\
				          & = 12\frac{1-(\frac{\sqrt{3}}{2})^n}{2-\sqrt{3}}R          \\
			      \end{align*}
			\item 所有这些正六边形的周长之和:
			      \begin{align*}
				      S & = \lim_{n\to\infty}12\frac{1-(\frac{\sqrt{3}}{2})^n}{2-\sqrt{3}}R \\
				        & = \frac{12}{2-\sqrt{3}}R                                          \\
				        & = 12(2+\sqrt{3})R
			      \end{align*}
		\end{enumerate}
	\end{solution}

	\question 某大队在农田基本建设的规划中,要测定被障碍物隔开的两点$A,
		P$之间的距离,他们土法上马,在障碍物的两侧,选取两点$B$和$C$(如图),测得$AB=AC=50$m,$ \angle{BAC}=60^\circ,
		\angle{ACP}=120^\circ, \angle{ACP}=135^\circ$,求$A$和$P$之间的距离。(答案可用最简根式表示)。

	\begin{figure}[htbp]
		\centering
		\begin{tikzpicture}[scale=1.5]
			\tkzDefPoint(0,0){A}
			\tkzDefShiftPointCoord[A](30:2){C}
			\tkzDefShiftPointCoord[A](-30:2){B}
			\tkzDefPoint(4,0.3){P}
			\tkzInterLL(A,P)(B,C) \tkzGetPoint{D}
			\tkzDefLine[orthogonal=through C](B,P) \tkzGetPoint{x}
			\tkzInterLL(B,P)(C,x) \tkzGetPoint{E}
			\tkzInterLL(A,P)(C,E) \tkzGetPoint{O}

			\tkzDrawPolygon(A,B,P,C)
			\draw[fill=gray!50] (D) ellipse[x radius=0.3, y radius=0.2];
			\tkzDrawSegments[dashed](A,P B,C)
			\tkzDrawSegment[red, dashed](C,E)
			\tkzDrawPoint(O)

			\tkzLabelAngle[pos=.8](E,C,P){$45^\circ$}
			\tkzMarkAngle[size=.5, mark=s|](E,C,P)
			\tkzMarkRightAngle[red](C,E,P)

			\tkzMarkAngle[size=.5, mark=s|](C,P,E)
			\tkzLabelAngle[pos=.8](C,P,E){$45^\circ$}

			\tkzMarkAngle[size=.6](B,C,E)
			\tkzLabelAngle[pos=.8](B,C,E){$30^\circ$}
			\tkzMarkRightAngle[red](A,C,E)

			\tkzLabelPoint[left](A){$A$}
			\tkzLabelPoint[below](B){$B$}
			\tkzLabelPoint[above](C){$C$}
			\tkzLabelPoint[right](P){$P$}
			\tkzLabelPoint[below](E){$E$}
			\tkzLabelPoint[above right](O){$O$}

			\tkzLabelSegment[above left](A,C){$x$}
			\tkzLabelSegment[below left](A,B){$x$}
			\tkzLabelSegment[below right](B,E){$\frac{x}{2}$}
			\tkzLabelSegment[below right](E,P){$\frac{\sqrt{3}}{2}x$}
		\end{tikzpicture}
	\end{figure}

	\begin{solution}
		\begin{enumerate}[label=\zhnum{*}、, noitemsep]
			\item 为了书写方便,将$AC$的长度记为$x$
			\item 由于 \( AB=AC\)且\( \angle{BAC}=60^\circ \),得 \( \triangle{ABC}是等边三角形 \),并有 \( BC=AB=AC=x \)
			\item 从$C$点作垂线到$BP$并与$BP$,垂足记为$E$,$CE$与$AP$交于$O$点
			\item 由 \( \angle{BAC}=60^\circ, \angle{ACP}=135^\circ, \angle{ABP}=120^\circ \)得$\angle{CPB}=45^\circ$
			\item 则有 \( \angle{ECP} = 45^\circ, \angle{BCE}=30^\circ \)
			\item 由 \( \angle{ACB}=60^\circ, \angle{BCE}=30^\circ \)得 \( \angle{ACO}=90^\circ \)
			\item 在 直角\( \triangle{BCE} \)中,有 \( BE=\frac{BC}{2} = \frac{x}{2}, CE=\frac{\sqrt{3}}{2}BC =
			      \frac{\sqrt{3}}{2}x \)
			\item 在等腰直角三角形 \( \triangle{EPC} \)中,有 \( EP = EC = \frac{\sqrt{3}}{2}x \)
			\item 对于 \( \triangle{ACO} \)和 \( \triangle{PEO} \),因为 \( \angle{ACO} = \angle{PEO} = 90^\circ,
			      \angle{AOC} = \angle{POE} \),所以 \( \triangle{AOC} \sim \triangle{POE} \)

			      根据相似三角形的性质有:
			      \begin{align*}
				      \frac{AC}{EP}                 & = \frac{CO}{OE} = \frac{AO}{OP}                     \\
				      \frac{x}{\frac{\sqrt{3}}{2}x} & = \frac{CO}{OE} = \frac{AO}{OP}= \frac{2}{\sqrt{3}} \\
			      \end{align*}
			\item 因为 \( CE=\frac{\sqrt{3}}{2}x \),则有 \( CO = (2\sqrt{3} - 3)x, OE = \frac{6 - 3\sqrt{3}}{2}x \)
			\item 则有 \( AO = \sqrt{AC^2 + CO^2} = \sqrt{x^2 - (2\sqrt{3} - 3)^2x^2} = \sqrt{12\sqrt{3} - 20}x \)
			\item
			      \begin{align*}
				      OP & = \frac{\sqrt{3}}{2} \cdot AO               \\
				      AP & = AO + OP                                   \\
				         & = (1+\frac{\sqrt{3}}{2})AO                  \\
				         & = \frac{2+\sqrt{3}}{2}\sqrt{12\sqrt{3}-20}x \\
				         & = \sqrt{\sqrt{3} + 1}x                      \\
				         & = 50\sqrt{\sqrt{3} + 1}
			      \end{align*}

		\end{enumerate}
	\end{solution}

	\question 已知双曲线 \( \frac{x^2}{24\tan\alpha} - \frac{y^2}{16\cot\alpha} = 1 (\alpha\text{为锐角})\)和圆 \( (x-m)^2 +
	y^2 = r^2 \)相切于点 \( A(4\sqrt{3},4) \),求 \( \alpha, m, r \)的值。
	\begin{solution}
		\begin{enumerate}[label=\arabic*., noitemsep]
			\item 将点 	\( A(4\sqrt{3}, 4) \)代入双曲线方程得:
			      \begin{equation*}
				      \frac{(4\sqrt{3})^2}{24\tan\alpha} - \frac{4^2}{16\cot\alpha} = 1
			      \end{equation*}

			      进一步化简得:
			      \begin{equation*}
				      \frac{2}{\tan\alpha} - \frac{1}{\cot\alpha} = 1
			      \end{equation*}
			      设 $\tan\alpha=x$,则有:
			      \begin{align*}
				      2\cdot\frac1x - x = 1 \\
				      x^2 +x - 2 = 0        \\
				      (x+2)(x-1) = 0        \\
			      \end{align*}
			      因为$\alpha$是锐角,所以 $x=1$,即 \( \alpha=45^\circ \)。
			\item 将点 \( A(4\sqrt{3}, 4) \)代入圆方程得:
			      \begin{equation*}
				      (4\sqrt{3} - m)^2 + 4^2 = r^2 \tag{a}
			      \end{equation*}
			\item 双曲线与圆在 $A$点相切,所以其切线斜率相同。
			      \begin{enumerate}[label=\alph*., noitemsep]
				      \item 求解双曲线在$A$点的切线的斜率

				            对双曲线求偏导:
				            \begin{equation*}
					            \frac{2x}{24\tan\alpha} - \frac{2y}{16\cot\alpha}\cdot\frac{\text{d}y}{\text{d} x} = 0
				            \end{equation*}
				            解得:
				            \begin{equation*}
					            \frac{\text{d}y}{\text{d}x} = \frac{\frac{x}{12\tan\alpha}}{\frac{y}{8\cot\alpha}}
					            = \frac{8\cot\alpha\cdot x}{12\tan\alpha\cdot y} = \frac{2x}{3y}
				            \end{equation*}
				            点 \( A(4\sqrt{3},4) \)的切线斜率为:
				            \begin{equation*}
					            k_1 = \frac{2\cdot4\sqrt{3}}{3\cdot4} =
					            \frac{2\sqrt{3}}{3}
				            \end{equation*}
				      \item 求解圆在$A$点的法线的斜率

				            对圆求偏导:
				            \begin{equation*}
					            2(x-m) + 2y\frac{\text{d}y}{\text{d}x} = 0
				            \end{equation*}

				            点$A(4\sqrt{3},4)$处的切线斜率为:
				            \begin{equation*}
					            k_2 = -\frac{4\sqrt{3} - m}{4}
				            \end{equation*}

				      \item 因为 \( k_1 = k_2 \),所以有
				            \begin{align*}
					            -\frac{4\sqrt{3} - m }{4} = \frac{2\sqrt{3}}{3}
				            \end{align*}
				            解得:
				            \begin{math}
					            m = \frac{20}{\sqrt{3}}
				            \end{math}

			      \end{enumerate}

			\item 将 \( m = \frac{20}{\sqrt{3}} \)代入式(a)得:
			      \begin{math}
				      r = \sqrt{\dfrac{112}{3}}
			      \end{math}

		\end{enumerate}
	\end{solution}

	\question 设数列 $1,2,4, \cdots$前$n$项和是$S_n = a + bn + cn^2 +
		dn^3$,求这数列的通项$a_n$的公式,并确定$a,b,c,d$的值。
	\begin{solution}
		\begin{enumerate}[label=\arabic*.]
			\item $a_n = S_n - S_{n-1}$
			      代入并化简:
			      \begin{align*}
				      a_n & = a + bn + cn^2 + dn^3 - a - b(n-1) - c(n-1)^2 - d(n-1)^3 \\
				          & = b + 2cn -c + dn^3 -d(n^3 -3n^2 + 3n - 1)                \\
				          & = b - c + d + (2c - 3d)n + 3dn^2
			      \end{align*}
			\item 根据题目中的条件有:
			      \begin{align*}
				      a_0 & = 1 = b - c + d \tag{1}                               \\
				      a_1 & = 2 = b - c + d + 2c - 3d + 3d = b + c + d\tag{2}     \\
				      a_2 & = 4 = b - c + d + 4c - 6d + 12d = b + 3c + 7d \tag{3}
			      \end{align*}
			      式(2) - 式(1)得 \[c=\frac12 \tag{4}\]
			      将$c=\frac12$代入式(1)得 \[b+d=\frac32\tag{5}\]
			      综合式(3)、(4)、(5)得
			      \begin{align*}
				      b & = \frac43 \\
				      d & = \frac16
			      \end{align*}
			      另外因为$a_0 = a$,所以$a=1$。
			\item 综上,$a,b,c,d$的值分别为:
			      \begin{align*}
				      a & = 1       \\
				      b & = \frac43 \\
				      c & = \frac12 \\
				      d & = \frac16
			      \end{align*}
			      $a_n$的通项公式为:
			      \begin{equation*}
				      a_n = \frac12n^2 + \frac12n + 1
			      \end{equation*}

		\end{enumerate}

	\end{solution}
\end{questions}
\end{document}

%!tex program = lualatex
\documentclass[answers]{exam}
\usepackage{silence}
\WarningFilter{latexfont}{Font shape}
\WarningFilter{latexfont}{Some font}
\usepackage{ctex}
\usepackage[margin=2cm]{geometry}
\usepackage{amsmath, amssymb, amsthm, arcs}
\usepackage{siunitx}
\usepackage{csquotes}
\usepackage{tikz, pgfplots, tikz-3dplot}
\usepackage[lua]{tkz-euclide}
\usetikzlibrary{
	angles,
	backgrounds,
	calc,
	decorations.pathmorphing,
	decorations.pathreplacing,
	decorations.text,
	intersections,
	patterns,
	petri,
	positioning,
	quotes,
	shapes,
	shapes.symbols,
}
\usepackage{graphicx, wrapfig}
\usepackage{caption}
\pagestyle{empty}
\newcounter{xcord}
\newcounter{ycord}
\newcounter{total}
\renewcommand{\labelenumi}{\textbf{\ifnum\value{enumi}<10 0\fi\arabic{enumi})}}

\pgfplotsset{compat=1.18}

\CorrectChoiceEmphasis{\color{blue!70!green}\bfseries}
\renewcommand{\solutiontitle}{\textbf{解:}}
\renewcommand{\choicelabel}{(\Alph{choice})}

\usepackage{array, tabularx}
\newcolumntype{C}{>{\centering\arraybackslash}X}
\newcolumntype{B}{>{\centering\bfseries\arraybackslash}X}

\usepackage{fontawesome5}
\usepackage{enumitem}
\newcommand\regularHandPointRight{\faHandPointRight[regular]}

\newenvironment{mathenum}
{\begin{enumerate}[label=\arabic*.]}
		{\end{enumerate}}

\newenvironment{penum}
{\begin{enumerate}[label=(\arabic*)]}
		{\end{enumerate}}
\makeatletter
\providecommand\@gobblethree[3]{}
\patchcmd{\over@under@arc}
{\@gobbletwo}
{\@gobblethree}
{}{}
\makeatother

\newcommand*\circled[1]{\tikz[baseline=(char.base)]{
		\node[shape=circle,draw,inner sep=1pt] (char) {#1};}}
\setlength{\answerclearance}{6pt}
\newcommand\hfs{\hfill (\hspace{2cm})}

\usepackage[lua]{tkz-euclide}

\begin{document}
\begin{center}
	\textbf{1977年普通高等学校招生考试(河北卷)}

	\section*{数学试卷}
\end{center}
\begin{questions}
	\question 解答下列各题
	\begin{enumerate}[label=(\arabic*)]
		\item 叙述函数的定义
		      \begin{solution}
			      函数是从一个集合到另一个集合之间的映射,使得每个输入值有且公有一个输出值.
		      \end{solution}
		\item 求函数$y=1-\frac{1}{\sqrt{2-3x}}$的定义域.
		      \begin{solution}
			      由函数的定义有:
			      \begin{math}
				      2-3x > 0
			      \end{math},则函数的定义域为:$x<\frac23$
		      \end{solution}
		\item 计算:$\left[1-(0.5)^{-2}\right] \div \left(-\frac{27}{8}\right)^{\frac13}$.
		      \begin{solution}
			      \begin{align*}
				      \left[1-(0.5)^{-2}\right] \div \left(-\frac{27}{8}\right)^{\frac13} & = \left(1-2^{-1\times (-2)}\right)
				      \div \left(-\frac32\right)^{3\times\frac13}                                                              \\
				                                                                          & = (1 - 4) \cdot (-\frac32)         \\
				                                                                          & = \frac92
			      \end{align*}
		      \end{solution}
		\item 计算:$\log_42$.
		      \begin{solution}
			      \begin{align*}
				      \log_42 & = \log_44^\frac12 \\
				              & = \frac12
			      \end{align*}
		      \end{solution}
		\item 分解因式:$x^2y - 2y^3$.
		      \begin{solution}
			      \begin{align*}
				      x^2y - 2y^3 & = y(x^2 - 2y^2)               \\
				                  & = y(x+\sqrt{2}y)(x-\sqrt{2}y)
			      \end{align*}
		      \end{solution}
		\item 计算:$\sin\dfrac{4\pi}{3}\cdot\cos\dfrac{25\pi}{6}\cdot\tan\left(-\dfrac{3\pi}{4}\right)$.
		      \begin{solution}
			      \begin{align*}
				      \sin\frac{4\pi}{3}\cdot\cos\frac{25\pi}{6}\cdot\tan\left(-\frac{3\pi}{4}\right)
				       & = -\sin\frac{\pi}{3}\cdot\cos\frac{\pi}{6}\cdot \tan\frac{3\pi}{4} \\
				       & = -\frac{\sqrt{3}}{2} \cdot \frac{\sqrt{3}}{2} \cdot 1             \\
				       & = - \frac43
			      \end{align*}
		      \end{solution}
	\end{enumerate}

	\question
	\begin{minipage}[t]{.3\textwidth}
		证明:如图,$AB$是圆$O$的直径,$CB$是圆$O$的切线,切点为$B$,$OC$平行于弦$AD$,求证:$DC$是圆$O$的切线.
	\end{minipage}\hspace{3cm}
	\begin{tikzpicture}[scale=.5, baseline=(current bounding box.center)]
		\tkzDefPoints{-2/0/A, 0/0/O, 2/0/B, 2/5/C}
		\tkzDefLine[parallel=through A](O,C) \tkzGetPoint{x}
		\tkzInterLC(A,x)(O,B) \tkzGetPoints{D}{A}

		\tkzDrawCircle(O,B)
		\tkzDrawSegment(A,B)
		\tkzDrawSegment(C,B)
		\tkzDrawSegment(C,O)
		\tkzDrawSegment(A,D)
		\tkzDrawSegment(C,D)
		\tkzDrawSegment[dashed, red](D,O)

		\tkzMarkAngle[mark=|, size=15pt](O,A,D)
		\tkzMarkAngle[mark=|, size=15pt](B,O,C)
		\tkzMarkAngle[mark=||, size=20pt](A,D,O)
		\tkzMarkAngle[mark=||, size=20pt](C,O,D)

		\tkzLabelPoint[left](A){$A$}
		\tkzLabelPoint[right](B){$B$}
		\tkzLabelPoint[above](C){$C$}
		\tkzLabelPoint[below](O){$O$}
		\tkzLabelPoint[left](D){$D$}
	\end{tikzpicture}

	\begin{solution}
		\begin{enumerate}[label=\arabic*.]
			\item 因为 $CB$ 是圆 $O$ 的切线且切点为 $B$,所以有 $AB \perp CB$.
			\item 由于 $AD \parallel OC$,可以得到 $\angle DAO = \angle COB$.
			\item 作辅助线 $DO$.因为 $\angle ADO$ 和 $\angle COD$ 是平行线的内错角,所以 $\angle ADO = \angle COD$.
			\item 在 $\triangle AOD$ 中,由于 $AD \parallel OC$,且 $OA = OD$(都是圆的半径),所以 $\triangle AOD$ 是等腰三角形,由此得到 $\angle OAD = \angle ODA$.
			\item 进一步分析 $\triangle DOC$ 和 $\triangle BOC$:
			      \begin{itemize}
				      \item $OB = OD$(都是圆的半径);
				      \item $OC = OC$(公共边);
				      \item $\angle BOC = \angle DOC$(由 $\angle COB = \angle DAO$ 和 $\angle ADO = \angle COD$ 推导).
			      \end{itemize}
			      因此,$\triangle BOC \cong \triangle DOC$(根据边角边准则).
			\item 由 $\triangle BOC \cong \triangle DOC$,可以得到 $\angle ODC = \angle CBO = 90^\circ$.
			\item 因为 $\angle ODC = 90^\circ$,所以 $DC$ 是圆 $O$ 的切线.
		\end{enumerate}
	\end{solution}

	\question 证明: $\dfrac{\sin2\alpha + 1}{1+\cos2\alpha + \sin2\alpha} = \dfrac12\tan\alpha + \dfrac12$.
	\begin{mathenum}
		\item 根据倍角公式$\sin2\alpha = 2\sin\alpha\cos\alpha$和$\cos2\alpha=\cos^2\alpha -
			\sin^2\alpha$来化简等式的左边得:
		\begin{align*}
			\frac{\sin2\alpha + 1}{1+\cos2\alpha + \sin2\alpha}
			 & = \frac{2\sin\alpha\cos\alpha + 1}{1 + \cos^2\alpha - \sin^2\alpha + 2\sin\alpha\cos\alpha} \\
			 & = \frac{\cos^2\alpha + \sin^2\alpha + 2\sin\alpha\cos\alpha}{\cos^2\alpha + \sin^2\alpha +
			\cos^2\alpha - \sin^2\alpha + 2\sin\alpha\cos\alpha}                                           \\
			 & = \frac{(\sin\alpha + \cos\alpha)^2}{2\cos\alpha(\sin\alpha + \cos\alpha)}                  \\
			 & = \frac{\sin\alpha + \cos\alpha}{2\cos\alpha}                                               \\
			 & = \frac12\tan\alpha + \frac12
		\end{align*}
		\item 综上,等式成立.
	\end{mathenum}

	\question 已知$2\lg{x} + \lg{2} = \lg(x+6)$,求$x$.
	\begin{solution}
		\begin{align*}
			2\lg{x} + \lg2 & = \lg{x^2} + \lg2 \\
			               & = \lg{2x^2}
		\end{align*}
		根据等式则有:
		\begin{equation*}
			2x^2 = x + 6 \tag{a}
		\end{equation*}
		对等式(a)整理得:
		\begin{align*}
			2x^2 - x - 6  & = 0 \\
			(2x + 3)(x-2) & = 0 \\
		\end{align*}
		因为$x>0$所以$x=2$.
	\end{solution}

	\question
	\begin{minipage}[t]{.6\textwidth}
		某生产队要建立一个形状是直角梯形的苗圃,其两邻边借用夹角为$135^\circ$的两面墙,另外两边是总长为$30$米的篱笆(如图,$AD$和$DC$为墙),问篱笆的两边各多长时,苗圃的面积最大?最大面积是多少?
	\end{minipage}
	\hspace{1cm}
	\begin{tikzpicture}[baseline=(current bounding box.north)]
		\tkzDefPoints{0/0/A, 3/0/B}
		\tkzDefPoint(45:2){D}
		\tkzDefLine[parallel=through D](A,B) \tkzGetPoint{x}
		\tkzDefLine[orthogonal=through B](A,B) \tkzGetPoint{y}
		\tkzInterLL(D,x)(B,y) \tkzGetPoint{C}

		\tkzDefLine[orthogonal=through D](A,B) \tkzGetPoint{x}
		\tkzInterLL(D,x)(A,B) \tkzGetPoint{E}

		\tkzDrawPolygon(A,B,C,D)
		\tkzLabelPoints[below](A,B,E)
		\tkzLabelPoints[above](C,D)
		\tkzDrawSegment[blue, dashed](D,E)
	\end{tikzpicture}

	\begin{solution}
		\begin{mathenum}
			\item 从$D$点向$AB$作垂线,垂足为$E$.
			\item 从条件可知$AE=ED=BC$
			\item 设$AE=x$,则有$DC=EB=30-2x$
			\item 梯形的面积可以表示为:
			\begin{align*}
				S & = (30 - 2x + x + 30 - 2x)\cdot x / 2 \\
				  & = (60 - 3x)x/2                       \\
				  & = -\frac32x^2 + 30x \tag{1}
			\end{align*}
			\item 根据抛物线的性质在$x=-\frac{b}{2a}$时有极值,代入可得$x=10$,此时$AB=20,BC=10$
			\item 将$x=10$代入式(1)可得最大面积为$150\mathrm{m}^2$
		\end{mathenum}
	\end{solution}

	\question
	工人师傅要用铁皮做一个上大下小的正四棱台形容器(上面开口),使其容积为$208$立方分米,高为$4$分米,上口边长与下底面边长的比为$5:2$,做这样的容器需要多少平方米的铁皮?(不计容器的厚度和加工余量,不要求写出已知,
	求解,直接求解并画图即可)

	\begin{solution}
		\begin{mathenum}
			\item 设下底面的边长为$2x$,则上底面的边长为$5x$,上下底面的面积分别为$25x^2$和$4x^2$
			\item 容器沿平行于边的中心线剖面是一个等腰梯形,沿着两个斜边作延长线,汇聚于点$E$

			\begin{tikzpicture}[scale=2]
				\tkzDefPoints{-2.5/0/A, 2.5/0/B, -1/-1/D, 1/-1/C, 0/-{5/3}/E, 0/0/O, 0/-1/F}
				\tkzDefLine[orthogonal=through C](A,B) \tkzGetPoint{x}
				\tkzInterLL(C,x)(A,B) \tkzGetPoint{G}

				\tkzDrawPolygon(A,B,C,D)
				\tkzDrawSegments[dashed](D,E C,E)
				\tkzDrawSegment[dim={$4$dm, 6pt,right=6pt}](O,F)
				\tkzDrawSegment[dim={$y$, 6pt, right=6pt}, dashed](F,E)
				\tkzLabelPoint[above](O){$O$}
				\tkzLabelPoint[below](E){$E$}
				\tkzLabelPoint[above left](F){$F$}

				\tkzDrawSegment[dashed](G,C)
				\tkzLabelPoint[above](G){$G$}
				\tkzLabelPoint[right](C){$C$}
				\tkzLabelPoint[right](B){$B$}
			\end{tikzpicture}

			\tdplotsetmaincoords{60}{130}
			\begin{tikzpicture}[tdplot_main_coords]
				\coordinate(A) at (1,1,0);
				\coordinate(B) at (-1,1,0);
				\coordinate(C) at (-1,-1,0);
				\coordinate(D) at (1,-1,0);
				\coordinate(A') at (2.5,2.5,2);
				\coordinate(B') at (-2.5,2.5,2);
				\coordinate(C') at (-2.5,-2.5,2);
				\coordinate(D') at (2.5,-2.5,2);
				\coordinate(E) at (-2.5, 0, 2);
				\coordinate(F) at (2.5, 0, 2);
				\coordinate(G) at (1, 0, 0);
				\coordinate(H) at (-1, 0, 0);
				\coordinate(I) at (-1, 0, 2);

				\draw (D) -- (A) -- (B);
				\draw [dashed](B)-- (C) -- (D);

				\draw (A') -- (B') -- (C') -- (D') -- cycle;
				\draw (A) -- (A');
				\draw (B) -- (B');
				\draw[dashed] (C) -- (C');
				\draw (D) -- (D');
				\draw[dashed](E) -- (F) -- (G) -- (H) -- cycle;
				\draw[dashed](H) -- (I);

				\tkzLabelPoint[above](E){$E$}
				\tkzLabelPoint[above](F){$F$}
				\tkzLabelPoint[below left](G){$G$}
				\tkzLabelPoint[above left](H){$H$}
				\tkzLabelPoint[above](I){$I$}

				% \tkzMarkSegments(C',E E,B')
				% \tkzMarkSegments[mark=||](A',F F,D')
				\tkzMarkRightAngle(H,I,E)

			\end{tikzpicture}
			\item 根据几何比例关系,$EF$的长度$y$有如下等式:
			\begin{equation*}
				\frac{y}{y+4} = \frac25
			\end{equation*}
			解得$y=\frac83$dm.
			\item 则容器的容积为:
			\begin{align*}
				\frac13 \left[ 25x^2 \cdot (4 + \frac83) - 4x^2 \cdot \frac83 \right]
				 & = 51x^2 \\
				 & = 208
			\end{align*}
			计算得 $x = 2$,则有下边长为$4$,上边长为$10$
			\item 在截面上从$B$点作垂线到$AB$,垂足为$G$,则可以求得棱面梯形的高$BC=5$,因此铁皮的面积为
			\begin{align*}
				(4+10)\times 5 \div 2 \times 4  +
				4 \times 4 % 底面积
				= 156 (\text{dm}^2)
			\end{align*}
		\end{mathenum}
	\end{solution}
	\question
	如图,$MN$为圆的直径,$P$、$C$为圆上两点,连$PM,PN$,过$C$作$MN$的垂线与$MN,MP$和$NP$的延长线依次相交于$A,B,D$,求证:$AC^2=AB\cdot AD.$
	\begin{tikzpicture}[baseline=(current bounding box.north), scale=.8]
		\tkzDefPoints{-2/0/M,0/0/O,2/0/N}
		\tkzDefPoint(65:2){P}
		\tkzDefPoint(110:2){C}

		\tkzDefLine[orthogonal=through C](M,N) \tkzGetPoint{x}
		\tkzInterLL(C,x)(M,N) \tkzGetPoint{A}
		\tkzInterLL(C,x)(M,P) \tkzGetPoint{B}
		\tkzInterLL(C,x)(N,P) \tkzGetPoint{D}

		\tkzDrawCircle(O,N)
		\tkzDrawSegments(M,N M,P A,D N,D C,N M,C)
		\tkzLabelPoint(A){$A$}
		\tkzLabelPoint[right](B){$B$}
		\tkzLabelPoint[above left](C){$C$}
		\tkzLabelPoint[above](D){$D$}
		\tkzLabelPoint[left](M){$M$}
		\tkzLabelPoint[right](N){$N$}
		\tkzLabelPoint[above right](P){$P$}
	\end{tikzpicture}
	\begin{solution}
		\begin{align*}
			 & \because \angle{AMP} = \angle{PMA} \text{并且有} \angle{MAB} = \angle{MPN} = 90^\circ \\
			 & \therefore \triangle{AMB} \sim \triangle{PMN}                                      \\
			 & \because \angle{PNM} = \angle{MNP} \text{和} \angle{MPN} = \angle{DAN} = 90^\circ   \\
			 & \therefore \triangle{PMN} \sim \triangle{ADN}                                      \\
			 & \therefore \triangle{AMB} \sim \triangle{ADN}                                      \\
			 & \therefore \frac{AB}{MA} = \frac{AN}{AD}                                           \\
			 & \therefore AB\cdot AD = MA \cdot AN                                                \\
			 & \because \angle{MCA} = \angle{CNA} = 90^\circ - \angle{NMC}                        \\
			 & \text{和} \angle{CAN} = \angle{CAM} = 90^\circ                                      \\
			 & \therefore \triangle{AMC} \sim \triangle{ACN}                                      \\
			 & \therefore \frac{AC}{MA} = \frac{NA}{AC}                                           \\
			 & \therefore AC^2 = MA \cdot CA                                                      \\
			 & \therefore AC^2 = AB \cdot AD
		\end{align*}
	\end{solution}
	\question 下列两题选做一题.

	【甲】已知椭圆短轴长为$2$,中心与抛物线$y^2=4x$的顶点重合,椭圆的一个焦点恰是此抛物线的焦点,求椭圆方程及其长轴的长.
	\begin{solution}
		抛物线的顶点为$(0,0)$,焦点为$(1, 0)$,设椭圆的方程为$\frac{x^2}{a^2} + \frac{y^2}{b^2} =
			1$,则椭圆的半焦距$c=\sqrt{a^2 - b^2} = 1$,其中$b=1$,解得$a=\sqrt{2}$,则长轴为$2\sqrt{2}$,椭圆的方程为:
		\( \frac{x^2}{2} + \frac{y^2}{1} = 1 \)
	\end{solution}

	【乙】已知菱形的一对内角各为$60^\circ$,边长为$4$,以菱形为对角线所在的直线为坐标轴建立直角坐标系,以菱形$60^\circ$角的两个顶点为焦点,并且过菱形的另外两个顶点作椭圆,求椭圆方程.
	\begin{figure}[htbp]
		\centering
		\begin{tikzpicture}
			\tkzInit[xmax=5,xmin=-5, ymax=5, ymin=-5]
			\tkzDrawX\tkzDrawY
			\tkzDefPoints{2/0/A,-2/0/C, 0/{2sqrt(3)}/B, 0/-{2sqrt(3)}/D, 0/0/O}

			\tkzDrawPolygon(A,B,C,D)
			\tkzMarkSegment[dim={$4$, 16pt, right=6pt}, mark=](B,A)
			\tkzMarkSegment[dim={$2$, -16pt, below=6pt}, mark=](O,A)
			\tkzMarkSegment[dim={$2\sqrt{3}$, 16pt, left=6pt}, mark=](O,B)

			\draw[x radius=4, y radius=2*sqrt(3)] ellipse;
		\end{tikzpicture}
	\end{figure}

	根据几何关系可知椭圆的短半轴为$b=2\sqrt{3}$,半焦距为$c=2$,长半轴$a=4$,所以椭圆的方程为$\frac{x^2}{16} +
		\frac{y^2}{12} = 1$
	\begin{center}
		\subsection*{附加题}
	\end{center}
	\question 将函数$f(x)=e^x$展开为$x$的幂级数,并求出收敛区间.($e=2.718$为自然对数的底数)
	\begin{solution}
		$e^x$的幂级数展开式为$\displaystyle\sum_{n=0}^{\infty}\frac{x^n}{n!}$,使用比值判别法:
		\begin{equation*}
			\lim_{n\to\infty}\left| \frac{\frac{x^{n+1}}{(n+1)!}}{\frac{x^n}{n!}}\right| =
			\lim_{n\to\infty}\left|\frac{x}{n+1}\right|
			= 0
		\end{equation*}
		所以收敛区间为$(-\infty, \infty)$.
	\end{solution}

	\question 利用定积分计算椭圆$\frac{x^2}{a^2} + \frac{y^2}{b^2} = 1 (a > b > 0)$所围成的面积.
	\begin{solution}
		对于任意一处的$x$对应的$y=\pm b\sqrt{1 -
				\frac{x^2}{a^2}}$,则在此处的形成的宽度为$\mathrm{d}x$,高度为$2|y|$的微小矩形的面积表达式为:$2b\sqrt{1-\frac{x^2}{a^2}}\mathrm{d}x$,则椭圆的面积为
		\begin{equation*}
			A = \int_{-a}^{a}2b\sqrt{1-\frac{x^2}{a^2}}\mathrm{d}x = 4b\int_{0}^{a}\sqrt{1-\frac{x^2}{a^2}}\mathrm{d}x
		\end{equation*}
		令$x=a\sin\theta$,则$\mathrm{d}x = a\cos\theta\mathrm{d}\theta$,其中$\theta \in [0, \frac\pi2]$.
		代入得:
		\begin{align*}
			A = 4b\int_{0}^{a}\sqrt{1-\frac{x^2}{a^2}}\mathrm{d}x
			 & = 4b\int_0^{\frac\pi2}\sqrt{1 -
			\frac{a^2\sin^2\theta}{a^2}}a\cos\theta\mathrm{d}\theta                              \\
			 & = 4ab\int_0^{\frac\pi2}\cos^2\theta\mathrm{d}\theta                               \\
			 & = 4ab\int_0^{\frac\pi2}\frac{1+\cos(2\theta)}{2}\mathrm{d}\theta                  \\
			 & = 4ab\left(\int_0^{\frac\pi2}\frac12\mathrm{d}\theta +
			\int_0^{\frac\pi2}\frac{\cos(2\theta)}{2}\mathrm{d}\theta\right)                     \\
			 & = 4ab\left(\frac\pi4  + \left[\frac{\sin(2\theta)}{4}\right]_0^{\frac\pi2}\right) \\
			 & = \pi ab
		\end{align*}
	\end{solution}

\end{questions}

\end{document}

%!tex program = lualatex
\documentclass[answers]{exam}
\usepackage{silence}
\WarningFilter{latexfont}{Font shape}
\WarningFilter{latexfont}{Some font}
\usepackage{ctex}
\usepackage[margin=2cm]{geometry}
\usepackage{amsmath, amssymb, amsthm, arcs}
\usepackage{siunitx}
\usepackage{csquotes}
\usepackage{tikz, pgfplots, tikz-3dplot}
\usepackage[lua]{tkz-euclide}
\usetikzlibrary{
	angles,
	backgrounds,
	calc,
	decorations.pathmorphing,
	decorations.pathreplacing,
	decorations.text,
	intersections,
	patterns,
	petri,
	positioning,
	quotes,
	shapes,
	shapes.symbols,
}
\usepackage{graphicx, wrapfig}
\usepackage{caption}
\pagestyle{empty}
\newcounter{xcord}
\newcounter{ycord}
\newcounter{total}
\renewcommand{\labelenumi}{\textbf{\ifnum\value{enumi}<10 0\fi\arabic{enumi})}}

\pgfplotsset{compat=1.18}

\CorrectChoiceEmphasis{\color{blue!70!green}\bfseries}
\renewcommand{\solutiontitle}{\textbf{解:}}
\renewcommand{\choicelabel}{(\Alph{choice})}

\usepackage{array, tabularx}
\newcolumntype{C}{>{\centering\arraybackslash}X}
\newcolumntype{B}{>{\centering\bfseries\arraybackslash}X}

\usepackage{fontawesome5}
\usepackage{enumitem}
\newcommand\regularHandPointRight{\faHandPointRight[regular]}

\newenvironment{mathenum}
{\begin{enumerate}[label=\arabic*.]}
		{\end{enumerate}}

\newenvironment{penum}
{\begin{enumerate}[label=(\arabic*)]}
		{\end{enumerate}}
\makeatletter
\providecommand\@gobblethree[3]{}
\patchcmd{\over@under@arc}
{\@gobbletwo}
{\@gobblethree}
{}{}
\makeatother

\newcommand*\circled[1]{\tikz[baseline=(char.base)]{
		\node[shape=circle,draw,inner sep=1pt] (char) {#1};}}
\setlength{\answerclearance}{6pt}
\newcommand\hfs{\hfill (\hspace{2cm})}

\usepackage[lua]{tkz-euclide}

\begin{document}
\begin{center}
	\textbf{1977年普通高等学校招生考试(黑龙江卷)}

	\textbf{\Large{数学试卷}}
\end{center}

\begin{questions}
	\question 解答下列各题:
	\begin{enumerate}[label=(\arabic*)]
		\item 解方程:$\sqrt{3x+4} = 4$。
		      \begin{solution}
			      方程两边平方得:
			      \begin{math}
				      3x + 4 = 16
			      \end{math}
			      移项整理得
			      \begin{align*}
				      3x & = 12 \\
				      x  & = 4
			      \end{align*}

		      \end{solution}
		\item 解不等式:$|x| < 5$
		      \begin{solution}
			      \begin{math}
				      -5 < x < 5
			      \end{math}

		      \end{solution}
		\item 已知正三角形的外接圆半径为$6\sqrt{3}$cm,求它的边长。
		      \begin{solution}
			      \begin{tikzpicture}[scale=.3]
				      \tkzDefPoints{0/0/O, 6sqrt(3)/0/A}
				      \tkzDefPoint(90:6sqrt(3)){B}
				      \tkzDefPoint(210:6sqrt(3)){C}
				      \tkzDefPoint(330:6sqrt(3)){D}
				      \tkzDefLine[orthogonal=through O](B,C) \tkzGetPoint{x}
				      \tkzInterLL(O,x)(B,C) \tkzGetPoint{E}

				      \tkzDrawCircle(O,A)
				      \tkzDrawPolygon(B,C,D)
				      \tkzDrawPoint(O)
				      \tkzLabelPoint[below right](O){$O$}
				      \tkzDrawSegments(B,O C,O O,E)

				      \tkzMarkSegment[dim={$6\sqrt{3}$, 16pt, right=6pt}, mark=](B,O)
				      \tkzMarkAngle[size=2](C,B,O)
				      \tkzLabelAngle[pos=3.5](C,B,O){$30^\circ$}
				      \tkzLabelPoint[left](E){$E$}
				      \tkzLabelPoint[above](B){$B$}
			      \end{tikzpicture}
			      可以计算出$BE=6$,则正三角形的边长为$12$。
		      \end{solution}

	\end{enumerate}

	\question 计算下列各题:
	\begin{enumerate}[label=(\arabic*)]
		\item $\sqrt{m^2 - 2ma + a^2} $
		      \begin{solution}
			      \begin{align*}
				      \sqrt{m^2 - 2ma + a^2} & = \sqrt{(m-a)^2} \\
				                             & = |m-a|
			      \end{align*}
		      \end{solution}
		\item $\cos78^\circ\cdot\cos3^\circ + \cos12^\circ\cdot\sin3^\circ$。
		      \begin{solution}
			      \begin{align*}
				      \cos78^\circ & = \cos(90^\circ - 12^\circ)                           \\
				                   & = \cos90^\circ\cos12^\circ + \sin90^\circ\sin12^\circ \\
				                   & = \sin12^\circ
			      \end{align*}
			      代入原式得:
			      \begin{align*}
				      \cos78^\circ\cdot\cos3^\circ + \cos12^\circ\cdot\sin3^\circ
				       & = \sin12^\circ\cos3^\circ + \cos12^\circ\sin3^\circ \\
				       & = \sin15^\circ                                      \\
				       & = \sin\left(\frac{30^\circ}{2}\right)               \\
				       & = \sqrt{\frac{1-\cos30^\circ}{2}}                   \\
				       & = \frac{\sqrt{2-\sqrt{3}}}{2}
			      \end{align*}
		      \end{solution}
		\item $\arcsin\left(\cos\dfrac\pi6\right)$.
		      \begin{solution}
			      \begin{align*}
				      \arcsin\left(\cos\dfrac\pi6\right) & = \arcsin(\dfrac{\sqrt{3}}2) \\
				                                         & = \frac{\pi}{3}
			      \end{align*}
		      \end{solution}
	\end{enumerate}
	\question 解下列各题:
	\begin{enumerate}[label=(\arabic*)]
		\item 解方程: $3^{x+1} - 9^{\frac{x}{2}} = 18$.
		      \begin{solution}
			      \begin{align*}
				      3^{x+1} - 9^{\frac{x}{2}}        & = 18 \\
				      3\cdot 3^x - (3^2)^{\frac{x}{2}} & = 18 \\
				      3^x                              & = 9  \\
				      x                                & = 2
			      \end{align*}
		      \end{solution}
		\item 求数列$2,4,8,16,\cdots $前十项的和。
		      \begin{solution}
			      \begin{align*}
				      S_n    & = a_0\frac{1-q^n}{1-q}, (q=2, a_0=2) \\
				      S_{10} & = 2\frac{1-2^{10}}{1-2}              \\
				             & = 2^{11} - 2                         \\
				             & = 2046
			      \end{align*}
		      \end{solution}
	\end{enumerate}

	\question 解下列各题:
	\begin{enumerate}[label=(\arabic*)]
		\item 圆锥的高为$6$cm,母线和底面半径的夹角为$30^\circ$,求它的侧面积。
		      \begin{solution}
			      根据题目中提供的信息可计算得底面半径为$r = 6\sqrt{3}$,母线的长度为$l = 12$,则侧面积为
			      \begin{math}
				      S = \pi r l = 72\sqrt{3} \text{cm}^2
			      \end{math}
		      \end{solution}
		\item 求过点$(1,4)$且与直线$2x - 5y + 3 = 0$垂直的直线方程。
		      \begin{solution}
			      直线$2x - 5y + 3 + 0$的斜率$k=\frac25$,则与之垂直的直线的斜率为$-\frac52$。

			      设所求直线方程为$y =
				      -\frac52x+b$, 并将点$(1,4)$代入得$b = \frac{13}{2}$。
					  所以直线方程为:$2y + 5x - 13 = 0$。
		      \end{solution}
	\end{enumerate}

\end{questions}

\end{document}

\section[1977年高考数学试卷及答案(江苏卷)理科]{1977年普通高等学校招生考试(江苏卷)\\\Huge{理科数学}}

\begin{questions}
	\question
	\begin{parts}
		\part[6] 计算:$ \left(2\frac14\right)^\frac12 + \left(\frac{1}{10}\right) - (3.14)^0 +
				\left(-\frac{27}{8}\right)^{\frac13} $
			\begin{solution}
				\begin{align*}
					 & = \left(\frac32\right)^{2\times\frac12} + 10^2 - 1 - \left(\frac32\right)^{3\times(\frac13)} \\
					 & = \frac32 + 99 - \frac32                                                                     \\
					 & = 99
				\end{align*}
			\end{solution}

		\part[6] 求函数$y=\sqrt{x-2} + \dfrac{1}{x-3} + \lg(5-x)$的定义域.
			\begin{solution}
				根据题意有:
				\begin{equation*}
					\begin{cases}
						x - 2 >= 0   \\
						x - 3 \neq 0 \\
						5 - x > 0
					\end{cases}
				\end{equation*}
				则有定义域为$[2,3),(3,5]$

			\end{solution}
		\part[8] 解方程:$ 5^{x^2+2x} = 125 $.
			\begin{solution}
				由 $5^{x^2 + 2x} = 125 = 5^3 $得:
				\begin{equation*}
					x^2 + 2x = 3
				\end{equation*}

				分解因式得
				\begin{equation*}
					(x+3)(x-1) = 0
				\end{equation*}
				所以
				\begin{equation*}
					x_1 = -3, x_2 = 1
				\end{equation*}
			\end{solution}
		\part[8] 计算:$-\log_3(\log_3\sqrt[3]{\sqrt[3]{\sqrt[3]{3}}})$
			\begin{solution}
				\begin{align*}
					-\log_3(\log_3\sqrt[3]{\sqrt[3]{\sqrt[3]{3}}}) & = -\log_3(\log_3\sqrt[3]{\sqrt[3]{3^\frac13}}) \\
					                                               & = -\log_3(\log_3\sqrt[3]{3^\frac19})           \\
					                                               & = -\log_3(\log_33^\frac1{27})                  \\
					                                               & = -\log_3\left(\frac{1}{27}\right)             \\
					                                               & = -\log_3\left(3^{-3}\right)                   \\
					                                               & = 3
				\end{align*}
			\end{solution}
		\part[8] 把直角坐标方程 \( (x-3)^2 + y^2 = 9 \)化为极坐标方程.
			\begin{solution}
				设动点为$(\rho, \theta)$,则有$x=\rho\cos\theta, y=\rho\sin\theta$,代入直角坐标方程得:
				\begin{equation*}
					(\rho\cos\theta - 3)^2 + (\rho\sin\theta)^2 = 9
				\end{equation*}
				展开得
				\begin{equation*}
					\rho^2\cos^2\theta - 6\rho\cos\theta + 9 + \rho^2\sin^2\theta = 9
				\end{equation*}
				合并同类项得:
				\begin{equation*}
					\rho^2(\cos^2\theta + \sin^2\theta) - 6\rho\cos\theta = 0
				\end{equation*}
				因为$\cos^2\theta + \sin^2\theta = 1$,简化为:
				\begin{equation*}
					\rho(\rho - 6\cos\theta) = 0
				\end{equation*}
				则极坐标方程为
				\begin{equation*}
					\rho = 6\cos\theta
				\end{equation*}
			\end{solution}
		\part[8] 计算: \( \displaystyle \lim_{n\to\infty}\frac{1+2+3+\cdots+n}{n^2} \)
			\begin{solution}
				根据等差数列求和公式得
				\begin{equation*}
					1 + 2 + 3 + \cdots + n = \frac{n(n+1)}{2}
				\end{equation*}
				则原式
				\begin{equation*}
					\lim_{n\to\infty}\frac{1+2+3+\cdots+n}{n^2} = \lim_{n\to\infty}(\frac12 + \frac{1}{2n}) = \frac12
				\end{equation*}
			\end{solution}
		\part[8] 分解因式:$ x^4 - 2x^2y - 3y^2 + 8y - 4 $.
			\begin{solution}
				\begin{align*}
					x^4 - 2x^2y - 3y^2 + 8y - 4 & = x^4 - 2x^2y + y^2 - (4y^2 - 8y + 4)  \\
					                            & = (x^2 - y)^2 - (2y - 2)^2             \\
					                            & = (x^2 - y + 2y - 2)(x^2 - y - 2y + 2) \\
					                            & = (x^2 + y - 2)(x^2 - 3y + 2)
				\end{align*}
			\end{solution}
	\end{parts}
	\question[8] 过抛物线$ y^2 = 4x
	$的焦点作倾斜角为$\frac34\pi$的直线,它与抛物线相交于$A$、$B$两点.求$A$、$B$两点间的距离.
	\begin{solution}
		\begin{cenum}
			\item 抛物线的焦点为$(1,0)$;
			\item 斜角为$\frac34\pi$的直线的斜率为$-1$;
			\item 设直线的方程为
			      \begin{equation*}
				      y = -x + b
			      \end{equation*}
			      将点$(1,0)$代入解得
			      \begin{equation*}
				      b = 1
			      \end{equation*}
			      则直线方程为
			      \begin{equation}
				      y = -x + 1
			      \end{equation}
			\item 将直线方程代入抛物线方程得:
			      \begin{equation*}
				      (-x + 1)^2 = 4x
			      \end{equation*}
			      展开得
			      \begin{equation*}
				      x^2 -6x + 1  = 0
			      \end{equation*}
			      解得
			      \begin{equation*}
				      x_1,x_2 = \frac{6 \pm \sqrt{32}}{2} = 3 \pm 2\sqrt{2}
			      \end{equation*}
			      代入直线方程可得
			      \begin{equation*}
				      y_1,y_2 = -(3\pm2\sqrt{2}) + 1
			      \end{equation*}
			      则点$A$的坐标为$(3+2\sqrt{2}, -2 - 2\sqrt{2})$,$B$点的坐标为$(3-2\sqrt{2}, -2 + 2\sqrt{2})$
			\item $AB$的距离为:
			      \begin{align*}
				      \sqrt{(3-2\sqrt{2} - 3 - 2\sqrt{2})^2 + (-2 + 2\sqrt{2} + 2 + 2\sqrt{2})^2}
				       & = \sqrt{32 + 32} \\
				       & = 8
			      \end{align*}
		\end{cenum}
	\end{solution}
	\question[8] 在直角三角形$ \triangle{ABC} $中,$ \angle{ACB} = 90^\circ
	$,$CD$、$CE$分别为斜边$AB$上的高和中线,且$\angle{BCD}$与$\angle{ACD}$之比为$3:1$,求证:$CD=DE$.

	\begin{center}
		\begin{tikzpicture}
			\tkzDefPoints{-1/0/A, 0/0/D, 0/2/C}
			\tkzDefLine[orthogonal=through C](A,C) \tkzGetPoint{x}
			\tkzInterLL(C,x)(A,D) \tkzGetPoint{B}
			\tkzDefPointOnLine[pos=0.5](A,B) \tkzGetPoint{E}

			\tkzDrawPolygon(A,B,C)
			\tkzDrawSegments(C,D C,E)

			\tkzLabelPoints[below](A,D,E,B)
			\tkzLabelPoint[above](C){$C$}
		\end{tikzpicture}
	\end{center}

	\begin{proofsolution}
		\begin{align*}
			 & \text{设}\angle{ACD} = \alpha                               \\
			 & \because CE\text{是直角三角形的中线}                                \\
			 & \therefore CE = EB                                         \\
			 & \therefore \angle{ECB} = \angle{CBE}                       \\
			 & \because \angle{DCB} = 3\angle{ACD}                        \\
			 & \because \angle{ACD} = \angle{CBE}                         \\
			 & \therefore \angle{DCE} = 3\alpha - \alpha = 2\alpha        \\
			 & \because \angle{DEC} = \angle{EBC} + \angle{ECB} = 2\alpha \\
			 & \therefore \angle{DEC} = \angle{DCE}                       \\
			 & \therefore DE = DC
		\end{align*}

	\end{proofsolution}
	\question[8] 在周长为 $300$ cm 的圆周上, 有甲、乙两球以大小不等的速度作匀速圆周 运动. 甲球从 $A
	$点出发按逆时针方向运动, 乙球从 $B$ 点出发按顺时针方向 运动, 两球相遇于 $C$ 点相遇后, 两球各自反方向作匀速圆周运动,
	但这时 甲球速度的大小是原来的 $2$ 倍, 乙球速度的大小是原来的一半, 以后他们 第二次相遇于 $D$ 点. 已知
	\overarc{$AmC$}  = 40 厘米,  \overarc{$BnD$} = 20 厘米, 求 \overarc{$ACB$} 的 长度.
	\begin{center}
		\begin{tikzpicture}
			\coordinate(O) at (0,0);
			\coordinate(m) at (210:2);
			\coordinate(n) at (300:2);
			\coordinate(A) at (180:2);
			\coordinate(B) at (280:2);
			\coordinate(C) at (240:2);
			\coordinate(D) at (330:2);

			\draw[radius=2] circle;
			\draw[-Latex](190:1.7)node[above]{甲}  arc[start angle=190, end angle=230, radius=1.7cm];
			\draw[-Latex](280:1.7)node[right]{乙}  arc[start angle=280, end angle=250, radius=1.7cm];

			\tkzDrawPoints(m, n, A, B, C, D)
			\tkzAutoLabelPoints[center=O](m,n,A,B,C,D)
		\end{tikzpicture}
	\end{center}
	\begin{solution}
		设甲的速度为$x$,乙的速度为$y$,\overarc{$ACB$}的长度为$a$,则第一次相遇有如下时间相等的关系:
		\begin{equation*}
			\frac{40}{x} = \frac{a-40}{y} \tag{1}
		\end{equation*}
		根据反弹后的速度变化有如下等式:
		\begin{equation*}
			\frac{300 - 20 - (a - 40)}{2x} = \frac{a - 40 + 20}{y/2} \tag{2}
		\end{equation*}
		将(1)式与(2)式相除可得:
		\begin{equation*}
			\frac{80}{320 -a} = \frac{a-40}{2(a-20)}
		\end{equation*}
		整理可得
		\begin{equation*}
			(a-80)(a-120) = 0
		\end{equation*}
		所以可得$a=80$或$a=120$.如果\overarc{$ACB$}的长度为\qty{80}{\cm}则两球的初始速度相同,与题目中的条件不符.所以\overarc{$ACB$}的长度为\qty{120}{\cm}.
	\end{solution}
	\question
	\begin{parts}
		\part[8] 若三角形三内角成等差数列,求证:必有一内角为$60^\circ$
			\begin{proofsolution}
				因为三个内角成等差数列,可以设中间的角为$x$,则其余两个角可以分别表示为$x-a,
					x+a$,那么三个角的和可以表示为$x-a + x + x + a = 3x = 180^\circ$,则有$x=60^\circ$
			\end{proofsolution}
		\part[8] 若三角形三内角成等差数列,而且三边又成等比数列,求证:三角形三内角都是$60^\circ$.
			\begin{proofsolution}
				\begin{center}
					\begin{tikzpicture}
						\tkzDefPoints{0/0/A, 2/0/B}
						\tkzDefPoint(60:2){C}
						\tkzDefLine[orthogonal=through C](A,B) \tkzGetPoint{x}
						\tkzInterLL(C,x)(A,B) \tkzGetPoint{D}

						\tkzDrawPolygon(A,B,C)
						\tkzDrawSegment[dashed](C,D)

						\tkzLabelPoints(A,B,D)
						\tkzLabelPoint[above](C){$C$}
						\tkzMarkAngle[size=0.5](B,A,C)
						\tkzLabelAngle(B,A,C){$60^\circ$}
						\tkzLabelSegment[above, sloped](A,C){$a$}
					\end{tikzpicture}
				\end{center}

				假设三角形不是等边三角形,并假设$\angle{A} = 60^\circ,
					AC=a$,且$AC$是最短的一条边,则$AB$为最长边设其长为$b$.从$C$点作垂线交$AB$于$D$.根据题目中的信息有:
				\begin{equation*}
					\frac{AC}{BC} = \frac{BC}{AB} \tag{1}
				\end{equation*}
				根据三角关系得$AD=\frac{a}{2},
					CD=\frac{\sqrt{3}}{2}a$,$DB=b-\frac{a}{2}$,则可以计算出$BC=\sqrt{(b-\frac{a}{2})^2 +
						(\frac{\sqrt{3}}{2}a)^2}$
				根据式(1)有
				\begin{align*}
					BC^2 = AC \cdot AB                                 \\
					(b - \frac{a}{2})^2 + (\frac{\sqrt{3}}{2}a)^2 = ab \\
					b^2 - ab + \frac{1}{4}a^2 + \frac34a^2 = ab        \\
					(b-a)^2 = 0                                        \\
					a = b
				\end{align*}
				所以假设不成立,三角形为等边三角形,所以三角形的三内角都是$60^\circ$
			\end{proofsolution}
	\end{parts}
	\question[8] 在两条平行直线 $AB$ 和 $CD$ 上分别取定一点 $M$ 和 $N$, 在直线 $AB$ 上取 一定线段 $ME = a$; 在线段
	$MN$上取一点 $K$, 连结 $EK$ 并延长交 $CD$ 于 $F$. 试问 $K$ 取在哪里时, $\triangle{EMK}$ 与 $\triangle{FNK}$ 的面积之和最小? 最小值是多少?

	\begin{solution}
		\begin{center}
			\begin{tikzpicture}[scale=2]
				\tkzDefPoints{0/0/A, 3/0/B, 0/2/C, 3/2/D, 1/0/M, 2/2/N, 2/0/E}
				\tkzDefPointOnLine[pos=0.4](M,N) \tkzGetPoint{K}
				\tkzInterLL(E,K)(C,D) \tkzGetPoint{F}
				\tkzDefLine[orthogonal=through K](A,B) \tkzGetPoint{x}
				\tkzInterLL(K,x)(A,B) \tkzGetPoint{G}
				\tkzInterLL(K,x)(C,D) \tkzGetPoint{H}

				\tkzDrawSegments(A,B C,D M,N E,F)
				\tkzLabelPoint[below left](M){$M$}
				\tkzLabelPoint[below right](E){$E$}
				\tkzLabelPoints[above](F,N)
				\tkzLabelPoint[right](K){$K$}
				\tkzMarkSegment[dim={$a$, -16pt, above=6pt}, mark=](M,E)
				\tkzDrawSegment(G,H)
				\tkzLabelPoint[above right](G){$G$}
				\tkzLabelPoint[above](H){$H$}
				\tkzMarkSegment[dim={$h$, 36pt, right=12pt}, mark=](H,G)
			\end{tikzpicture}
		\end{center}

		假设$KG=xh$,则$HK=(1-x)h$.由$\triangle{MKE} \sim \triangle{NKF}$可得$FN =
			\frac{1-x}{x}a$,则$\triangle{EMK}$与$\triangle{NKF}$的面积和为:
		\begin{align*}
			\frac12axh + \frac12\cdot\frac{1-x}{x}a\cdot(1-x)h
			\\                                                         & = \frac12ah(x + \frac{(1-x)^2}{x})
			\\                                                         & = \frac12ah(\frac{x^2 + 1 - 2x + x^2}{x})
			\\                                                       & = \frac12ah\frac{2x^2 - 2x + 1}{x}
			\\ & = \frac12ah(2x -2 + \frac1x)
		\end{align*}
		对其求导数得:
		\begin{align*}
			\frac12ah(2 - \frac{1}{x^2})
		\end{align*}
		此导数在$x =
			\frac{\sqrt{2}}{2}$时等于$0$,所以此时面积有极值.所以$K$在$MN$的$\frac{\sqrt{2}}{2}$处$\triangle{EMK}$和$\triangle{NKF}$有最小值.

	\end{solution}

	\begin{center}
		\textbf{附加题}
	\end{center}

	\question 求极限:$ \displaystyle \lim_{n\to\infty}\sqrt{x}(\sqrt{x+1} - \sqrt{x}) $.
	\begin{solution}
		\begin{align*}
			\text{原式} & = \lim_{n\to\infty}\frac{\sqrt{x}}{\sqrt{x+1} + \sqrt{x}} \\
			          & = \lim_{n\to\infty}\frac{1}{\sqrt{1 + \frac1x} + 1}       \\
			          & = \frac12
		\end{align*}
	\end{solution}
	\question 求不定积分:$\int\dfrac{\mathrm{d}x}{(1+e^x)^2}$.
	\begin{solution}
		设$u = 1 + e^x$,则有$\mathrm{d}u = e^x\mathrm{d}x$,即$\mathrm{d}x = \frac{\mathrm{d}u}{u-1}$

		代入得:
		\begin{align*}
			\int\frac{\mathrm{d}x}{(1+e^x)^2} & = \int\frac{1}{(u-1)u^2}\mathrm{d}u
		\end{align*}
		对于$\frac{1}{(u-1)u^2}$进行因式分解,假设分解后的形式为:
		\begin{equation*}
			\frac{1}{(u-1)u^2} = \frac{A}{u} + \frac{B}{u^2} + \frac{C}{u-1}
		\end{equation*}
		两边同乘以$(u-1)u^2$得到:
		\begin{equation*}
			1 = Au(u-1) + B(u-1) + Cu^2
		\end{equation*}
		合并同类项得:
		\begin{equation*}
			1 = (C+A)u^2 + (B-A)u - B
		\end{equation*}
		比较系数得:
		\begin{equation*}
			\begin{cases}
				B = -1 \\
				A = -1 \\
				C = 1  \\
			\end{cases}
		\end{equation*}
		因此
		\begin{align*}
			\int\frac{1}{(u-1)u^2}\mathrm{d}u
			 & = \int\left(-\frac{1}{u} - \frac{1}{u^2} + \frac{1}{u-1}\right)\mathrm{d}u \\
			 & = \int\left(-\frac1u\right)\mathrm{d}u +
			\int\left(-\frac1{u^2}\right)\mathrm{d}u +
			\int\left(\frac1{u-1}\right)\mathrm{d}u                                       \\
			 & = -\ln|u| + \frac1u + \ln|u-1| + C
		\end{align*}
		将$u=1+e^x$代入得
		\begin{equation*}
			-\ln|1+e^x| + \frac{1}{1+e^x} + \ln|e^x| + C
			= -\ln(1+e^x) + \frac{1}{1+e^x} + x + C
		\end{equation*}
		所以不定积分为:
		{
		\boldmath
		\color{blue}
		\begin{equation*}
			-\ln(1+e^x) + \frac{1}{1+e^x} + x + C
		\end{equation*}
		}
	\end{solution}
\end{questions}

%!tex program = lualatex
\documentclass[answers]{exam}
\usepackage{silence}
\WarningFilter{latexfont}{Font shape}
\WarningFilter{latexfont}{Some font}
\usepackage{ctex}
\usepackage[margin=2cm]{geometry}
\usepackage{amsmath, amssymb, amsthm, arcs}
\usepackage{siunitx}
\usepackage{csquotes}
\usepackage{tikz, pgfplots, tikz-3dplot}
\usepackage[lua]{tkz-euclide}
\usetikzlibrary{
	angles,
	backgrounds,
	calc,
	decorations.pathmorphing,
	decorations.pathreplacing,
	decorations.text,
	intersections,
	patterns,
	petri,
	positioning,
	quotes,
	shapes,
	shapes.symbols,
}
\usepackage{graphicx, wrapfig}
\usepackage{caption}
\pagestyle{empty}
\newcounter{xcord}
\newcounter{ycord}
\newcounter{total}
\renewcommand{\labelenumi}{\textbf{\ifnum\value{enumi}<10 0\fi\arabic{enumi})}}

\pgfplotsset{compat=1.18}

\CorrectChoiceEmphasis{\color{blue!70!green}\bfseries}
\renewcommand{\solutiontitle}{\textbf{解:}}
\renewcommand{\choicelabel}{(\Alph{choice})}

\usepackage{array, tabularx}
\newcolumntype{C}{>{\centering\arraybackslash}X}
\newcolumntype{B}{>{\centering\bfseries\arraybackslash}X}

\usepackage{fontawesome5}
\usepackage{enumitem}
\newcommand\regularHandPointRight{\faHandPointRight[regular]}

\newenvironment{mathenum}
{\begin{enumerate}[label=\arabic*.]}
		{\end{enumerate}}

\newenvironment{penum}
{\begin{enumerate}[label=(\arabic*)]}
		{\end{enumerate}}
\makeatletter
\providecommand\@gobblethree[3]{}
\patchcmd{\over@under@arc}
{\@gobbletwo}
{\@gobblethree}
{}{}
\makeatother

\newcommand*\circled[1]{\tikz[baseline=(char.base)]{
		\node[shape=circle,draw,inner sep=1pt] (char) {#1};}}
\setlength{\answerclearance}{6pt}
\newcommand\hfs{\hfill (\hspace{2cm})}

\usepackage[lua]{tkz-euclide}

\begin{document}
\begin{center}
	\textbf{1977普通高等学校招生考试(上海卷)}

	\textbf{\Huge 理科数学}
\end{center}
\begin{questions}
	\question
	\begin{enumerate}[label=(\arabic*)]
		\item 化简:
		      \begin{math} \displaystyle
			      \left(\frac{a}{a+b} - \frac{a^2}{a^2 + 2ab + b^2}\right) \div \left(\frac{a}{a+b} - \frac{a^2}{a^2 - b^2}\right)
		      \end{math}.
		      \begin{solution}
			      \begin{align*}
				      \left(\frac{a}{a+b} - \frac{a^2}{a^2 + 2ab + b^2}\right) \div \left(\frac{a}{a+b} - \frac{a^2}{a^2 - b^2}\right)
				       & = \left[ \frac{a(a+b)}{(a+b)^2} - \frac{a^2}{(a+b)^2} \right] \div \frac{a(a-b) - a^2}{a^2-b^2}
				      \\
				       & = \frac{ab}{(a+b)^2} \cdot \frac{a^2 - b^2}{-ab}                                                \\
				       & = \frac{b-a}{a+b}
			      \end{align*}
		      \end{solution}
		\item 计算:
		      \begin{math}
			      \displaystyle
			      \frac12\lg25+\lg2 - \lg\sqrt{0.1} - \log_29 \times \log_32.
		      \end{math}
		      \begin{solution}
			      \begin{align*}
				       & = \frac12\lg5^2 + \lg2 - \lg(10)^{-\frac12} - \log_23^2 \times \log_32 \\
				       & = \lg5 + \lg2 + \frac12 - 2 \log_23 \times \log_32                     \\
				       & = \frac32 - 2 \cdot \frac{\ln3}{\ln2}\cdot \frac{\ln2}{\ln3}           \\
				       & = \frac12
			      \end{align*}
		      \end{solution}
		\item $\sqrt{-1} = i$,验算$i$是否方程$2x^4 + 3x^3 - 3x^2 + 3x - 5 = 0$的解。
		      \begin{solution}
			      将$i$代入方程得:
			      \begin{align*}
				      2i^4 + 3i^3 - 3x^2 + 3x - 5 & = 2 - 3i + 3  + 3i - 5 \\
				                                  & = 0
			      \end{align*}
			      所以$i$是原方程的解。
		      \end{solution}
		\item 求证:$\displaystyle
			      \frac{\sin\left(\frac{\pi}{4} + \theta \right)}{\sin \left( \frac{\pi}{4} - \theta \right)} +
			      \frac{\cos\left(\frac{\pi}{4} + \theta \right)}{\cos \left( \frac{\pi}{4} - \theta \right)} =
			      \frac{2}{\cos2\theta}
		      $.
		      \begin{solution}
			      根据和角公式
			      \[
				      \begin{array}{l}
					      \sin(\alpha+\beta) = \sin\alpha\cos\beta + \cos\alpha\sin\beta, \\
					      \cos(\alpha+\beta)=\cos\alpha\cos\beta - \sin\alpha\sin\beta
				      \end{array}
			      \]
			      及差角公式$$
				      \begin{array}{l}
					      \sin(\alpha-\beta) = \sin\alpha\cos\beta - \cos\alpha\sin\beta, \\
					      \cos(\alpha-\beta)=\cos\alpha\cos\beta + \sin\alpha\sin\beta
				      \end{array}
			      $$来化简等式的左边得:
			      \begin{align*}
				       & = 	\frac{\sin\frac{\pi}{4}\cos\theta +
					      \cos\frac{\pi}{4}\sin\theta}{\sin\frac{\pi}{4}\cos\theta-\cos\frac{\pi}{4}\sin\theta} +
				      \frac{\cos\frac{\pi}{4}\cos\theta - \sin\frac{\pi}{4}\sin\theta}{\cos\frac{\pi}{4}\cos\theta +
				      \sin\frac{\pi}{4}\sin\theta}                                                              \\
				       & =
				      \frac{\sin\theta + \cos\theta}{\cos\theta - \sin\theta} + \frac{\cos\theta -
				      \sin\theta}{\cos\theta + \sin\theta}                                                      \\
				       & =
				      \frac{1 + 2\sin\theta\cos\theta + 1 - 2\sin\theta\cos\theta}{\cos^2\theta - \sin^2\theta} \\
				       & = \frac{2}{\cos2\theta}
			      \end{align*}
			      因此原式成立。
		      \end{solution}
	\end{enumerate}
	\question 在$\triangle{ABC}$中,$\angle{C}$的平分线交$AB$于$D$,过$D$作$BC$的平行线交$AC$于$E$,已知$BC=a, AC=b$,求$DE$的长。

	\begin{figure}[htbp]
		\centering
		\begin{tikzpicture}
			\tkzDefPoints{0/0/A, 3/0/C, 2/2.5/B}
			\tkzDefLine[bisector](B,C,A) \tkzGetPoint{x}
			\tkzInterLL(C,x)(A,B) \tkzGetPoint{D}
			\tkzDefLine[parallel=through D](B,C) \tkzGetPoint{x}
			\tkzInterLL(D,x)(A,C) \tkzGetPoint{E}

			\tkzDrawPolygon(A,B,C)
			\tkzDrawSegments(D,E D,C)

			\tkzLabelPoints[below](A,E,C)
			\tkzLabelPoints[above left](D,B)
		\end{tikzpicture}
	\end{figure}

	\begin{solution}
		\begin{align*}
			 & \because \angle{ACD} = \angle{BCD}                  \\
			 & \therefore AD = BD                                  \\
			 & \because DE \parallel BC                            \\
			 & \therefore
			\begin{array}{l}
				AE = EC \\
				\dfrac{DE}{BC} = \dfrac{AE}{AC}
			\end{array}                         \\
			 & \therefore DE = BC\frac{\frac{1}{2}a}{a} = \frac12b
		\end{align*}
	\end{solution}
	\pagebreak
	\question
	已知圆$A$的直径为$2\sqrt{3}$,圆$B$的直径为$4-2\sqrt{3}$,圆$C$的直径为$2$,圆$A$与圆$B$外切,圆$A$又与圆$C$外切,$\angle{A}=60^\circ$,求$BC$及$\angle{C}$。

	\begin{figure*}[ht]
		\centering
		\begin{tikzpicture}
			\tkzDefPoints{{sqrt(3)}/0/A, {sqrt(3)-2}/0/B, 0/0/O}
			\tkzDefShiftPoint[A](120:{sqrt(3)+1}){C}
			\tkzDefLine[orthogonal=through C](B,A) \tkzGetPoint{x}
			\tkzInterLL(C,x)(A,B) \tkzGetPoint{D}
			\tkzInterLC(A,C)(A,O) \tkzGetSecondPoint{x}

			\tkzDrawCircles(A,O B,O)
			\tkzDrawCircle(C,x)
			\tkzDrawPolygon(A,B,C)

			\tkzLabelPoints(B,A,D)
			\tkzLabelPoint[above](C){$C$}
			\tkzMarkAngle[size=.5](C,A,B)
			\tkzLabelAngle(C,A,B){$60^\circ$}

			\tkzDrawSegment[dashed](C,D)
		\end{tikzpicture}
	\end{figure*}

	\begin{solution}
		由题意得:
		\begin{align*}
			AB & = 2            \\
			AC & = 1 + \sqrt{3}
		\end{align*}
		由余弦定理有:
		\begin{align*}
			BC^2 & = AB^2 + AC^2 - 2\cdot AB \cdot AC \cdot \cos60^\circ           \\
			     & = 4 + 2\sqrt{3} + 4 - 2\cdot 2 \cdot (1+\sqrt{3}) \cdot \frac12 \\
			     & = 6
		\end{align*}
		所以
		\begin{equation*}
			BC = \sqrt{6}
		\end{equation*}

		再由余弦定理有:
		\begin{align*}
			\cos{C} & = \frac{AC^2 + BC^2 - AB^2}{2AC\cdot BC}                            \\
			        & = \frac{(1+\sqrt{3})^2 + (\sqrt{6})^2 - 2^2}{2(1+\sqrt{3})\sqrt{6}} \\
			        & = \frac{4 + 2\sqrt{3} + 6 - 4}{2\sqrt{2}(3+\sqrt{3})}               \\
			        & = \frac{2(3+\sqrt{3})}{2\sqrt{2}(3+\sqrt{3})}                       \\
			        & = \frac{\sqrt{2}}{2}
		\end{align*}
		所以有:
		\begin{equation*}
			\angle{C} = 45^\circ
		\end{equation*}
	\end{solution}

	\question 正六棱锥$V-ABCDEF$的高为$2$cm,底面边长为$2$cm。
	\begin{enumerate}[label=(\arabic*)]
		\item 按$1:1$画出它的二视图;
		\item 求其侧面积;
		\item 求它的侧棱和底面的夹角。
	\end{enumerate}

	\begin{solution}

		\begin{minipage}[b]{0.4\textwidth}
			\tdplotsetmaincoords{90}{0}
			\centering
			\begin{tikzpicture}[tdplot_main_coords]
				\tkzDefPoint(0:2){A}
				\tkzDefPoint(60:2){B}
				\tkzDefPoint(120:2){C}
				\tkzDefPoint(180:2){D}
				\tkzDefPoint(240:2){E}
				\tkzDefPoint(300:2){F}
				\coordinate(V) at (0,0,2);

				\draw (A) -- (B) -- (C) -- (D) -- (E) -- (F) -- cycle;
				\tkzDrawSegments(V,A V,B V,C V,D V,E V,F)

			\end{tikzpicture}
			\captionof*{figure}{正视图}

		\end{minipage}
		\begin{minipage}[b]{0.4\textwidth}
			\centering
			\tdplotsetmaincoords{0}{0}
			\begin{tikzpicture}[tdplot_main_coords]
				\tkzDefPoint(0:2){A}
				\tkzDefPoint(60:2){B}
				\tkzDefPoint(120:2){C}
				\tkzDefPoint(180:2){D}
				\tkzDefPoint(240:2){E}
				\tkzDefPoint(300:2){F}
				\coordinate(V) at (0,0,2);

				\draw (A) -- (B) -- (C) -- (D) -- (E) -- (F) -- cycle;
				\tkzDrawSegments(V,A V,B V,C V,D V,E V,F)
			\end{tikzpicture}
			\captionof*{figure}{顶视图}
		\end{minipage}

		\begin{minipage}{\textwidth}
			\tdplotsetmaincoords{70}{120}
			\centering
			\begin{tikzpicture}[tdplot_main_coords, scale=2]
				\tkzDefPoint(0,0){O}
				\tkzDefPoint(0:2){A}
				\tkzDefPoint(60:2){B}
				\tkzDefPoint(120:2){C}
				\tkzDefPoint(180:2){D}
				\tkzDefPoint(240:2){E}
				\tkzDefPoint(300:2){F}
				\coordinate(V) at (0,0,2);
				\tkzDefPointOnLine[pos=0.5](B,C) \tkzGetPoint{G}

				\draw (F) -- (A) -- (B) -- (C);
				\draw[dashed] (C)-- (D) -- (E) -- (F);
				\tkzDrawSegment[blue](V,G)
				\tkzDrawSegment[dashed, blue](O,G)
				\tkzDrawSegments(V,A V,B V,C  V,F)
				\tkzDrawSegments[dashed](V,D V,E)
				\tkzDrawSegments[dashed](V,O O,A O,B)
				\tkzLabelSegment[above, sloped](O,V){$h=2\text{cm}$}
				\tkzLabelSegment[above, sloped](O,A){$2\text{cm}$}
				\tkzLabelPoint(O){$O$}
				\tkzLabelPoint(G){$G$}
				\tkzLabelPoint(B){$B$}
				\tkzLabelPoint(A){$A$}
				\tkzLabelPoint(C){$C$}
				\tkzLabelPoint[above](V){$V$}
			\end{tikzpicture}
		\end{minipage}

		\begin{align*}
			 & \because \text{底面是正六边形}                                \\
			 & \therefore \angle{AOB} = 60^\circ                      \\
			 & \because AO = BO                                       \\
			 & \therefore \triangle{AOB} \text{是正三角形}                 \\
			 & \therefore AO = AB = 2cm                               \\
			 & \because \angle{AOV} = 90^\circ                        \\
			 & \therefore VA = \sqrt{2^2 + 2^2} = 2\sqrt{2} \text{cm} \\
			 & \therefore \text{侧棱长为}2\sqrt{2}\mathrm{cm}
		\end{align*}

		取$BC$的中点为$G$,连接$OG$和$VG$。
		\begin{align*}
			 & \because \angle{BOG} = 30^\circ                                             \\
			 & \therefore BG = 2\sin30^\circ = 1 \text{cm}                                 \\
			 & \therefore VG = \sqrt{VG^2 - BG^2} = \sqrt{7}\text{cm}                      \\
			 & \therefore S_{\text{侧}} = \frac12VG\cdot BC \times 6 = 6\sqrt{7}\text{cm}^2 \\
			 & \because
			\begin{array}{l}
				VO = OA \\
				\angle{VOA} = 90^\circ
			\end{array}                                                          \\
			 & \therefore \triangle{VOA}\text{是等腰直角三角形}                                    \\
			 & \therefore \angle{VAO} = 45^\circ (\text{即侧棱与底面的夹角为}45^\circ)
		\end{align*}
	\end{solution}

	\question 解不等式: 	\begin{math}
		\begin{cases}
			16 - x^2 \geqslant 0 \\
			x^2 - x - 6 > 0
		\end{cases},并在数轴上把它的解表示出来。
	\end{math}
	\begin{solution}
		\begin{enumerate}[label=\Roman*.]
			\item 解不等式$16 - x^2 \geqslant 0$
			      \begin{align*}
				      16  \geqslant x^2 \\
				      -4 \leqslant x \leqslant 4
			      \end{align*}
			\item 解不等式$x^2 - x - 6 > 0$
			      \begin{align*}
				      (x-3)(x+2) > 0 \\
				      x > 3, x < -2
			      \end{align*}
			\item 绘制数轴

			      \begin{tikzpicture}
				      % 绘制数轴
				      \draw[->] (-5,0) -- (5,0) node[right] {$x$};

				      % 标记关键点
				      \foreach \x in {-4, -2, 3, 4} {
						      \path  (\x, 0) node[below] {\x};
					      }

				      % 绘制闭合圆圈
				      \filldraw[black] (-4,0) circle (1pt);
				      \filldraw[black] (4,0) circle (1pt);

				      % 绘制开口圆圈
				      \draw[black] (-2,0) circle (1pt);
				      \draw[black] (3,0) circle (1pt);

				      % 绘制范围
				      \draw[thick] (-4,0)--(-4,0.5) -- (-2,0.5) -- (-2,0);
				      \draw[thick] (3,0)--(3,0.5) -- (4,0.5) -- (4,0);

			      \end{tikzpicture}
		\end{enumerate}
	\end{solution}

\end{questions}

\end{document}

%!tex program = lualatex
\documentclass[answers]{exam}
\usepackage{silence}
\WarningFilter{latexfont}{Font shape}
\WarningFilter{latexfont}{Some font}
\usepackage{ctex}
\usepackage[margin=2cm]{geometry}
\usepackage{amsmath, amssymb, amsthm, arcs}
\usepackage{siunitx}
\usepackage{csquotes}
\usepackage{tikz, pgfplots, tikz-3dplot}
\usepackage[lua]{tkz-euclide}
\usetikzlibrary{
	angles,
	backgrounds,
	calc,
	decorations.pathmorphing,
	decorations.pathreplacing,
	decorations.text,
	intersections,
	patterns,
	petri,
	positioning,
	quotes,
	shapes,
	shapes.symbols,
}
\usepackage{graphicx, wrapfig}
\usepackage{caption}
\pagestyle{empty}
\newcounter{xcord}
\newcounter{ycord}
\newcounter{total}
\renewcommand{\labelenumi}{\textbf{\ifnum\value{enumi}<10 0\fi\arabic{enumi})}}

\pgfplotsset{compat=1.18}

\CorrectChoiceEmphasis{\color{blue!70!green}\bfseries}
\renewcommand{\solutiontitle}{\textbf{解:}}
\renewcommand{\choicelabel}{(\Alph{choice})}

\usepackage{array, tabularx}
\newcolumntype{C}{>{\centering\arraybackslash}X}
\newcolumntype{B}{>{\centering\bfseries\arraybackslash}X}

\usepackage{fontawesome5}
\usepackage{enumitem}
\newcommand\regularHandPointRight{\faHandPointRight[regular]}

\newenvironment{mathenum}
{\begin{enumerate}[label=\arabic*.]}
		{\end{enumerate}}

\newenvironment{penum}
{\begin{enumerate}[label=(\arabic*)]}
		{\end{enumerate}}
\makeatletter
\providecommand\@gobblethree[3]{}
\patchcmd{\over@under@arc}
{\@gobbletwo}
{\@gobblethree}
{}{}
\makeatother

\newcommand*\circled[1]{\tikz[baseline=(char.base)]{
		\node[shape=circle,draw,inner sep=1pt] (char) {#1};}}
\setlength{\answerclearance}{6pt}
\newcommand\hfs{\hfill (\hspace{2cm})}

\usepackage{tkz-euclide}

\begin{document}
\begin{center}
	\textbf{1997年普通高等学校招生考试(全国卷)}

	\textbf{\Large 理科数学}
\end{center}
\begin{questions}
	\question 设集合$M=\{x|0 \leqslant x < 2\}$,集合$N=\{x|x^2 - 2x - 3 <0\}$,集合$M\cap N=$ \hfill (\hspace{2cm})
	\begin{oneparchoices}
		\choice $\{x|0\leqslant x < 1\}$
		\CorrectChoice $\{x|0\leqslant x < 2\}$
		\choice $\{x|0\leqslant x \leqslant 1\}$
		\choice $\{x|0\leqslant x \leqslant 2\}$
	\end{oneparchoices}

	\begin{solution}
		集合$N=\{x|-1 < x < 3\}$,两个集合的范围如下图所示:

		\begin{tikzpicture}
			\tkzInit[xmin=-4, xmax=4]
			\tkzDrawX

			\draw[rounded corners=2pt, thick, black!50] (-1,0) |- (3,.5) -- (3,0);
			\draw[black!50, thick, pattern=north east lines,pattern color=black!50, rounded corners=2pt] (0,0) |- (2,.8) -- (2,0);
			\draw [fill=white](-1,0) circle (1pt);
			\draw [fill=white](3,0) circle (1pt);
			\draw [fill=black](0,0) circle (1pt);
			\draw [fill=white](2,0) circle (1pt);
		\end{tikzpicture}
	\end{solution}

	\question 如果直线$ax+2y+2=0$与直线$3x-y-2=0$平行,那么系数$a=$ \hfs

	\begin{oneparchoices}
		\choice $-3$
		\CorrectChoice $-6$
		\choice $-\dfrac32$
		\choice $\dfrac23$
	\end{oneparchoices}

	\begin{solution}
		两条直线平行则其斜率相等,可得:
		\begin{equation*}
			-\frac{a}{2} = 3
		\end{equation*}
		所以答案为$a=-6$
	\end{solution}

	\question 函数$y=\tan \left( \dfrac12x - \dfrac{\pi}{3} \right)$在一个周期内的图像是 \hfs

	\begin{oneparchoices}
		\pgfplotsset{
			xlabel={$x$},
			ylabel={$y$},
			% x label style={at={(current axis.right of origin)}, anchor=north},
			axis lines=center,
			samples=100,
			ticks=none
		}
		\CorrectChoice
		\begin{tikzpicture}[scale=.4]
			\begin{axis}[
					xmin = {-2/3*pi},
					xmax = {2*pi},
					ymin = -3,
					ymax = 3,
				]
				\addplot[domain={-1/3*pi+0.1}:{5/3*pi-0.1}]{tan(deg(1/2*x - pi/3))};
				\draw[dashed] (-pi/3,3) -- (-pi/3, -3);
				\draw[dashed] (pi*5/3,3) -- (pi*5/3, -3);
				\node[below left] at (-pi/3, 0) {$-\frac{\pi}{3}$};
				\node[below right] at (2*pi/3, 0) {$\frac{2\pi}{3}$};
				\node[below left] at (5*pi/3, 0) {$\frac{5\pi}{3}$};
				\node[below left] at (0,0) {$O$};
			\end{axis}
		\end{tikzpicture}
		\choice
		\begin{tikzpicture}[scale=.4]
			\begin{axis}[
					xmin = {-1/6*pi},
					xmax = {8/6*pi},
					ymin = -3,
					ymax = 3,
				]
				\addplot[domain={1/6*pi+0.1}:{7/6*pi-0.1}]{tan(deg(x - 2/3*pi))};
				\draw[dashed] (pi/6,3) -- (pi/6, -3);
				\draw[dashed] (pi*7/6,3) -- (pi*7/6, -3);
				\node[below left] at (pi/6, 0) {$\frac{\pi}{6}$};
				\node[below right] at (2*pi/3, 0) {$\frac{2\pi}{3}$};
				\node[below left] at (7*pi/6, 0) {$\frac{7\pi}{6}$};
				\node[below left] at (0,0) {$O$};
			\end{axis}
		\end{tikzpicture}
		\choice
		\begin{tikzpicture}[scale=.4]
			\begin{axis}[
					xmin = {-pi},
					xmax = {5/3*pi},
					ymin = -3,
					ymax = 3,
				]
				\addplot[domain={-2/3*pi+0.1}:{4/3*pi-0.1}]{tan(deg(1/2*x - pi/6))};
				\draw[dashed] (-pi*2/3,3) -- (-pi*2/3, -3);
				\draw[dashed] (pi*4/3,3) -- (pi*4/3, -3);
				\node[below left] at (-pi*2/3, 0) {$-\frac{2\pi}{3}$};
				\node[below right] at (pi/3, 0) {$\frac{\pi}{3}$};
				\node[below left] at (4*pi/3, 0) {$\frac{4\pi}{3}$};
				\node[below left] at (0,0) {$O$};
			\end{axis}
		\end{tikzpicture}
		\choice
		\begin{tikzpicture}[scale=.4]
			\begin{axis}[
					xmin = {-1/3*pi},
					xmax = {pi},
					ymin = -3,
					ymax = 3,
				]
				\addplot[domain={-1/6*pi+0.1}:{5/6*pi-0.1}]{tan(deg(x - 1/3*pi))};
				\draw[dashed] (-pi/6,3) -- (-pi/6, -3);
				\draw[dashed] (pi*5/6,3) -- (pi*5/6, -3);
				\node[below left] at (-pi/6, 0) {$-\frac{\pi}{6}$};
				\node[below right] at (pi/3, 0) {$\frac{\pi}{3}$};
				\node[below left] at (5*pi/6, 0) {$\frac{5\pi}{6}$};
				\node[below left] at (0,0) {$O$};
			\end{axis}
		\end{tikzpicture}
	\end{oneparchoices}
	\begin{solution}
		$\tan{x}$的周期是$\pi$,所以$\tan{\frac12{x}}$的周期应该为$2\pi$.因此排除B和D.
		$\tan(\frac12\cdot\frac{\pi}{3}-\frac{\pi}{3}) \neq 0$,因此选A.
	\end{solution}
	\question
	已知三棱锥$D-ABC$的三个侧面与底面全等,且$AB=AC=\sqrt{3}$,$BC=2$,则以$BC$为棱,以面$BCD$与面$BCA$为面的二面角的大小是
	\hfs

	\begin{oneparchoices}
		\choice $\arccos\dfrac{\sqrt{3}}{3}$
		\choice $\arccos\dfrac{1}{3}$
		\CorrectChoice $\dfrac{\pi}{2}$
		\choice $\dfrac{2\pi}{3}$
	\end{oneparchoices}

	\begin{solution}

		\tdplotsetmaincoords{20}{120}
		\begin{center}
			\begin{tikzpicture}[tdplot_main_coords, baseline=(current bounding box.north)]
				\coordinate(A) at ({sqrt(2)}, 0);
				\coordinate(B) at (0,1);
				\coordinate(C) at (0,-1);
				\coordinate(D) at (0,0,{sqrt(2)});

				\draw(A)node[below]{$A$} -- (B)node[below]{$B$};
				\draw[dashed](B)-- (C)node[above]{$C$};
				\draw(C)-- (A);
				\draw(D)node[above]{$D$} -- (A);
				\draw(D) -- (B);
				\draw(D) -- (C);

				\draw[blue, dashed](A) -- (0,0)node[below right]{$O$} -- (D);
			\end{tikzpicture}
		\end{center}

		取$BC$的中点为$O$,连接$AO$和$DO$.
		因为$AC=AB$,所以有$AO\perp BC$.因为$\triangle{BCD}\cong\triangle{BCA}$,所以也有$DO\perp
			BC$.计算可得$AO=DO=\sqrt{2}$.另外有$AD=2$.可以看出有$AO^2 + DO^2 =
			AD^2$,所以$\triangle{AOD}$是直角三角形,即二面角为$\ang{90}$.
	\end{solution}

	\question 函数$y=\sin \left( \dfrac{\pi}{3} - 2x \right) + \cos2x$的最小正周期是 \hfs

	\begin{oneparchoices}
		\choice $\dfrac{\pi}{2}$
		\CorrectChoice $\pi$
		\choice $2\pi$
		\choice $\dfrac{2\pi}{3}$
	\end{oneparchoices}

	\begin{solution}
		因为$\sin \left( \dfrac{\pi}{3} -2x \right)$与$\cos2x$的周期均为$\pi$,所以函数的最小正周期也为$\pi$.
	\end{solution}

	\question 满足$\arccos(1-x) \geqslant \arccos{x}$的取值范围是 \hfs

	\begin{oneparchoices}
		\choice $\left[-1, -\dfrac12\right]$
		\choice $\left[-\dfrac12, 0\right]$
		\choice $\left[0, \dfrac12\right]$
		\CorrectChoice $\left[\dfrac12, 1\right]$
	\end{oneparchoices}

	\begin{solution}
		因为$\arccos$的定义域是$[-1,1]$,所以排队选项A和B.

		$\arccos(0) = \frac{\pi}{2}, \arccos(1-0)=0$,所以C也可以排除.答案为D.
	\end{solution}

	\question 将$y=2^x$的图像\underline{\hspace{.5cm}},再作关于直线$y=x$对称的图像,可得到函数$y=\log_2(x+1)$的图像. \hfs

	\begin{oneparchoices}
		\choice 先向左平移$1$个单位
		\choice 先向右平移$1$个单位
		\choice 先向上平移$1$个单位
		\CorrectChoice 先向下平移$1$个单位
	\end{oneparchoices}

	\begin{solution}
		点$(0,0)$在直线$y=x$上,并且也在函数$y=\log_2(x+1)$上,所以应该也在$y=2^x$平移后的图像上,可以得到平移后的图像为$y=2^x
			- 1$才能满足点$(0,0)$在图像上.
	\end{solution}

	\question 长方体一个顶点上三条棱的长分别是$3,4,5$,且它的八个顶点都在同一个球面上,这个球的表面积是

	\hfs

	\begin{oneparchoices}
		\choice $20\sqrt{2}\pi$
		\choice $25\sqrt{2}\pi$
		\choice $50\pi$
		\choice $200\pi$
	\end{oneparchoices}

	\begin{solution}

		\tdplotsetmaincoords{60}{120}
		\begin{center}
			\begin{tikzpicture}[tdplot_main_coords]
				\coordinate (A) at (0,0);
				\coordinate (B) at (3,0);
				\coordinate (C) at (3,4);
				\coordinate (D) at (0,4);
				\coordinate (A') at (0,0,5);
				\coordinate (B') at (3,0,5);
				\coordinate (C') at (3,4,5);
				\coordinate (D') at (0,4,5);

				\draw (B) -- (C) -- (D);
				\draw[dashed] (A) -- (B) (D) -- (A);
				\draw (A') -- (B') -- (C') -- (D') -- cycle;
				\draw[dashed] (A) -- (A');
				\draw (B) -- (B');
				\draw (C) -- (C');
				\draw (D) -- (D');
				\draw[dashed] (A) -- (C');
				\draw[dashed] (A) -- (C);

				\tkzLabelPoint[left](A){$A$}
				\tkzLabelPoint[left](B){$B$}
				\tkzLabelPoint[below](C){$C$}
				\tkzLabelPoint[above](C'){$C'$}
			\end{tikzpicture}
		\end{center}

		\begin{align*}
			AB = 3, BC = 4, CC'= 5             \\
			AC = \sqrt{3^2 + 4^2} = 5          \\
			AC' = \sqrt{5^2 + 5^2} = 5\sqrt{2} \\
		\end{align*}
		则球体的半径为$\frac12AC'=\frac{5\sqrt{2}}{2}$, 球体的表面积为$4\pi \left( \frac{5\sqrt{2}}{2} \right)^2=50\pi$.
	\end{solution}

	\question 曲线的参数方程是\begin{math}
		\begin{cases}
			x = 1 - \dfrac1t \\
			y = 1 - t^2
		\end{cases}(t是参数,t\neq0),它的普通方程是 \hfs
	\end{math}

	\begin{oneparchoices}
		\choice $(x-1)^2(y-1)=1$
		\CorrectChoice $y=\dfrac{x(x-2)}{(1-x)^2}$
		\choice $y=\dfrac{1}{(1-x)^2} -1$
		\choice $y=\dfrac{x}{1-x^2} + 1$
	\end{oneparchoices}

	\begin{solution}
		由$x = 1 - \dfrac1t$得$t=\dfrac{1}{1-x}$,代入$y=1-t^2$得
		\begin{align*}
			y & = 1- \frac{1}{(1-x)^2}           \\
			  & = \frac{1-2x + x^2 - 1}{(1-x)^2} \\
			  & = \frac{x(x-2)}{(1-x)^2}
		\end{align*}
	\end{solution}

	\question 函数$y=\cos^2x-3\cos{x} + 2$的最小值为 \hfs

	\begin{oneparchoices}
		\choice $2$
		\CorrectChoice $0$
		\choice $-\dfrac14$
		\choice $6$
	\end{oneparchoices}

	\begin{solution}
		\begin{align*}
			y & = \cos^2x - 3x + \frac94 - \frac14 \\
			  & = (\cos{x} - \frac32)^2 - \frac14  \\
		\end{align*}
		$(\cos{x}-\dfrac32)^2$最小值为$\dfrac14$,则函数的最小值为$0$.
	\end{solution}

	\question 椭圆$C$与椭圆$\dfrac{(x-3)^2}{9} + \dfrac{(y-2)^2}{4}=1$关于直线$x+y=0$对称,椭圆$C$的方程是 \hfs

	\begin{choices}
		\CorrectChoice $\dfrac{(x+2)^2}{4} + \dfrac{(y+3)^2}{9} = 1$
		\choice $\dfrac{(x-2)^2}{9} + \dfrac{(y-3)^2}{4} = 1$
		\choice $\dfrac{(x+2)^2}{9} + \dfrac{(y+3)^2}{4} = 1$
		\choice $\dfrac{(x-2)^2}{4} + \dfrac{(y-3)^2}{9} = 1$
	\end{choices}

	\begin{solution}
		对于椭圆$C$上的任一点$(x,y)$,其关于直线$x+y=0$对称的点为$(-y, -x)$,代入椭圆方程得:
		\begin{align*}
			\frac{(-y-3)^2}{9} + \frac{(-x-2)^2}{4} & = 1 \\
			\frac{(y+3)^2}{9} + \frac{(x+2)^2}{4}   & = 1
		\end{align*}
	\end{solution}

	\question 圆台上、下底面积分别为$\pi$、$4\pi$,侧面积为$6\pi$,这个圆台的体积是\hfs

	\begin{oneparchoices}
		\choice $\dfrac{2\sqrt{3}\pi}{3}$
		\choice $2\sqrt{3}\pi$
		\choice $\dfrac{7\sqrt{3}\pi}{6}$
		\CorrectChoice $\dfrac{7\sqrt{3}\pi}{3}$
	\end{oneparchoices}

	\begin{solution}
		\tdplotsetmaincoords{80}{0}
		\begin{center}
			\begin{tikzpicture}[tdplot_main_coords]
				\draw[dashed] (2,0) arc (0:180:2);
				\draw (2,0) arc (0:-180:2);
				\draw (0,0,1) circle(1);

				\draw[thin] (2,0,0) -- node[above, sloped]{$l$}(1,0,1);
				\draw[thin] (-2,0,0) -- (-1,0,1);

				\draw (0,0,1)node[left]{$r_2$} -- (1,0,1);
				\draw[dashed] (0,0)node[left]{$r_1$} -- (2,0);
				\draw[dashed] (0,0) -- (0,0,2);
				\draw[dashed] (0,0,2) --node[above, sloped]{$l'$} (1,0,1);
			\end{tikzpicture}
		\end{center}

		圆台的侧面积$=\pi(l+l')r_1 - \pi l'r_2$,而$\frac{l'}{l'+l}=\frac{r_2}{r_1}=\frac12$,即$l'=l$,代入面积公式中得:
		$3\pi l = 6\pi$,得到$l=2$.圆台的高$h=\sqrt{l^2-(r_1-r_2)^2}=\sqrt{3}$.
		\begin{align*}
			V & = \frac13h{(S_上 + S_下 + \sqrt{S_上S_下})} \\
			  & = \frac{7\sqrt{3}\pi}{3}
		\end{align*}
	\end{solution}

	\question
	定义在区间$(-\infty,+\infty)$的奇函数$f(x)$为增函数,偶函数$g(x)$在区间$[0,+\infty)$的图像与$f(x)$的图像重合,设$a>b>0$,给出下列不等式:

	\circled{1}$f(b)-f(-a)>g(a)-g(-b)$; \circled{2}$f(b)-f(-a) < g(a) - g(-b)$;
	\circled{3}$f(a)-f(-b) > g(b) - g(-a)$; \circled{4}$f(a)-f(-b) < g(b) - g(-a)$.
	其中成立的是 \hfs

	\begin{oneparchoices}
		\choice \circled{1}与\circled{4}
		\choice \circled{2}与\circled{3}
		\CorrectChoice \circled{1}与\circled{3}
		\choice \circled{2}与\circled{4}
	\end{oneparchoices}

	\begin{solution}
		\begin{enumerate}[label=\protect\circled{\arabic*}]
			\item $f(b) - f(-a) = f(b)+f(a)$, $g(a) - g(-b) = g(a) - g(b)$,所以有$f(b)-f(-a) > g(a) - g(-b)$;
			\item 不成立;
			\item $f(a) - f(-b) = f(a) + f(b)$, $g(b) - g(-a)= g(b) - g(a)$,所以有$f(a) - f(-b) > g(b) - g(-a)$
			\item 不成立
		\end{enumerate}
	\end{solution}

	\question 不等式组 \begin{math}
		\begin{cases}
			x> 0 \\
			\dfrac{3-x}{3+x} > \left|\dfrac{2-x}{2+x}\right|
		\end{cases}
	\end{math}的解集是 \hfs

	\begin{oneparchoices}
		\choice $\{x|0< x < 2\}$
		\choice $\{x|0< x < 2.5\}$
		\CorrectChoice $\{x|0< x < \sqrt{6}\}$
		\choice $\{x|0< x < 3\}$
	\end{oneparchoices}

	\begin{solution}
		从\begin{math}
			\begin{cases}
				x> 0 \\
				\dfrac{3-x}{3+x} > \left|\dfrac{2-x}{2+x}\right|
			\end{cases}
		\end{math}可得
		\begin{align*}
			\frac{3-x}{3+x} & > \frac{2-x}{2+x} \qquad (-2 < x \leqslant 2) \tag{1}          \\
			\frac{3-x}{3+x} & > \frac{x-2}{2+x}  \qquad (-2 > x(x\neq-3) \cup x > 2) \tag{2}
		\end{align*}
		由式$(1)$得
		\begin{align*}
			6+x-x^2> 6-x -x^2 \qquad (-2 \leqslant x \leqslant 2) \\
			0 < x \leqslant 2 \tag{a}
		\end{align*}
		由式$(2)$得
		\begin{align*}
			6 +x - x^2 > -6 +x + x^2 \qquad (-2 > x(x\neq-3) \cup x > 2) \\
			6 > x^2 \qquad (-2 > x (x\neq-3) \cup x > 2)                 \\
			-\sqrt{6} < x < -2, 2 < x < \sqrt{6} \tag{b}
		\end{align*}
		综合$(a)$和$(b)$可得 $-\sqrt{6}<x < -2, 0< x <
			\sqrt{6}$,再结合题目中$x>0$的条件,所以最后的解集是$\{x|0<x<\sqrt{6}\}$
	\end{solution}

	\question 四面体的顶点和各棱中点共$10$个点,在其中取$4$个不共面的点,不同的取法共有 \hfs

	\begin{oneparchoices}
		\CorrectChoice $150$种
		\choice $147$种
		\choice $144$种
		\choice $141$种
	\end{oneparchoices}

	\begin{solution}
		从$10$个点中取$4$个点一共有$\binom{10}{4}=\frac{10\times9\times8\times7}{4\times3\times2\times1}=210$种.四面体上一个面上有规定的$6$个点,从这$6$个点中取$4$个点的取法有$\binom{6}{4}=15$种,一共有$4$个面,所以取得在同一个面上的$4$个点的取法有$60$种.则取$4$个不共面的点的取法共有$150$种.
	\end{solution}

	\question 已经$\left( \dfrac{a}{x} -\sqrt{\dfrac{x}{2}}
		\right)^9$的展开式中$x^3$的系数为$\frac94$,常数$a$的值为\fillin[36][2cm].

	\begin{solution}
		\begin{align*}
			\left( \frac{a}{x} - \sqrt{\frac{x}{2}} \right)^9
			 & = \sum_{r=0}^9\binom{9}{r} \left( \frac{a}{x} \right)^{9-r} \left( -\sqrt{\frac{x}{2}} \right)^r \\
			 & = \sum_{r=0}^{9}\binom{9}{r} a^{9-r}(-1)^r2^{-\frac{r}{2}}x^{r-9 + \frac{r}{2}}
		\end{align*}
		$x$的指数为$3$时有$r-9+\frac{r}{2} = 3$,计算得$r=8$.此时系数为$a^{9-8}(-1)^82^{-\frac{8}{2}} = \frac94$.

		$a=36$.
	\end{solution}
	\question 已知直线的极坐标方程为$\rho\sin \left( \theta + \dfrac{\pi}{4} \right) =
		\dfrac{\sqrt{2}}{2}$,则极点到该直线的距离是 \fillin[$\dfrac{\sqrt{2}}{2}$][2cm].

	\begin{solution}
		\begin{align*}
			\rho\sin \left( \theta + \frac{\pi}{4} \right)
			 & = \rho( \sin\theta\cos\frac{\pi}{4} + \cos\theta\sin\frac{\pi}{4}) \\
			 & =  \frac{\sqrt{2}}{2}\rho(\sin\theta + \cos\theta)                 \\
			 & = \frac{\sqrt{2}}{2}
		\end{align*}
		所以极坐标方程可以化简为:
		\begin{equation*}
			\rho\sin\theta + \rho\cos\theta = 1
		\end{equation*}
		将$x=\cos\theta, y=\sin\theta$代入极坐标方程可得直线的直角坐标方程为
		\begin{equation*}
			x + y - 1 = 0
		\end{equation*}
		任意一点$(x_1, y_1)$到直线$Ax+By+C=0$的距离为:
		\begin{align*}
			d = \frac{|Ax_1+By_1+C|}{\sqrt{A^2+B^2}}
		\end{align*}
		将$A=1,B=1,C=-1,x_1=0, y_1=0$代入得
		\begin{align*}
			d = \frac{1}{\sqrt{2}} = \frac{\sqrt{2}}{2}
		\end{align*}
	\end{solution}

	\question
	$\dfrac{\sin{\ang{7}}+\cos{\ang{15}}\sin{\ang{8}}}{\cos{\ang{7}}-\sin{\ang{15}}\sin{\ang{8}}}$的值为\fillin[$2-\sqrt{3}$][2cm].

	\begin{solution}
		\begin{align*}
			\frac{\sin{\ang{7}}+\cos{\ang{15}}\sin{\ang{8}}}{\cos{\ang{7}}-\sin{\ang{15}}\sin{\ang{8}}}
			 & = \frac{\sin(\ang{15}-\ang{8}) + \cos\ang{15}\sin\ang8}{\cos(\ang{15}-\ang8) -\sin\ang{15}\sin\ang8} \\
			 & = \frac{\sin\ang{15}\cos8 - \cos\ang{15}\sin\ang8 + \cos\ang{15}\sin\ang8}
			{\cos\ang{15}\cos8 + \sin\ang{15}\sin\ang8 - \sin\ang{15}\sin\ang8}                                     \\
			 & = \frac{\sin\ang{15}}{\cos\ang{15}}                                                                  \\
			 & = \tan\ang{15}                                                                                       \\
			 & = \frac{\sin\ang{30}}{1+\cos\ang{30}}                                                                \\
			 & = \frac{\frac12}{1+\frac{\sqrt{3}}{2}}                                                               \\
			 & = 2 - \sqrt{3}
		\end{align*}
	\end{solution}

	\question 已知$m,l$是直线,$\alpha,\beta$是平面,给出下列命题:
	\begin{enumerate}[label=\protect\circled{\arabic*}]
		\item 若$l$垂直于$\alpha$内的两条相交直线,则$l\perp \alpha$;
		\item 若$l$平行于$\alpha$,则$l$平行于$\alpha$内的所有直线;
		\item 若$m\subset\alpha, l\subset\beta$,且$l\perp m$,则$\alpha\perp\beta$;
		\item 若$l\subset\beta$,且$l\perp\alpha$,则$\alpha\perp\beta$;
		\item 若$m\subset\alpha,l\subset\beta$,且$\alpha\parallel\beta$,则$m\parallel l$.
	\end{enumerate}
	其中正确的倒是的序号是\fillin[\circled{1}\,\circled{4}][2cm].(注:把你认为正确的倒是的序号都填上)

	\begin{solution}
		\begin{enumerate}[label=\protect\circled{\arabic*}]
			\item 一条直线与一个平面内的两条相交直线垂直,则可以判定直线与平面垂直.这个是判定定理.
			\item 错误,直线平行于平面不能推出直线平行于平面内的所有直线.
			\item 错误,两个平行平面之间也可以有垂直的直线.
			\item 如果一条直线垂直于一个平面$\alpha$,则直线所在的平面与$\alpha$垂直.
			\item (错误, 两个平面平行并不能推出两个平台内的任意直线都平行(异面直线).
		\end{enumerate}
	\end{solution}

	\question 已知复数$z=\dfrac{\sqrt{3}}{2} - \dfrac12\text{i}, \omega = \dfrac{\sqrt{2}}{2} +
		\dfrac{\sqrt{2}}{2}\text{i}$.复数$\overline{z\omega},z^2\omega^3$在复数平面上所对应的点分别为$P,Q$.证明
	$\triangle{OPQ}$是等腰直角三角形(其中$O$为原点).

	\begin{solution}
		计算可知$z,\omega$的模长均为$1$.由三角关系可知$z$的幅角是$-\dfrac{\pi}{6}$,$\omega$的幅度是$\dfrac{\pi}{4}$.

		则可知$\overline{z\omega}$是幅角为$-\dfrac{\pi}{12}$的单位向量.

		$z^2\omega^3$是$-\dfrac{\pi}{6}\cdot2 + \dfrac{\pi}{4}\cdot3=\dfrac{5\pi}{12}$的单位向量.

		可见,$\overline{z\omega}$与$z^2\omega^3$中间的夹角是$\dfrac{\pi}{2}$,所以$\triangle{OPQ}$是直角三角形.
	\end{solution}

	\question 已知数列$\{a_n\}, \{b_n\}$都是由正数组成的等比数列,公比分别为$p,q$,其中$p>q$,且$p\neq1, q\neq1$.设$c_n
		= a_n+b_n$, $S_n$为数列$\{c_n\}$的前$n$项和。求$\displaystyle\lim_{n\to\infty}\dfrac{S_n}{S_{n-1}}$.

	\begin{solution}
		\begin{align*}
			\lim_{n\to\infty}\frac{S_n}{S_{n-1}}
			 & = \lim_{n\to\infty}\frac{a_1\frac{1-p^n}{1-p}+b_1\frac{1-q^n}{1-q}}{a_1\frac{1-p^{n-1}}{1-p}+b_1\frac{1-q^{n-1}}{1-q}} \\
		\end{align*}
		\circled{1}若$q>1$
		\begin{align*}
			\lim_{n\to\infty}\frac{S_n}{S_{n-1}}
			 & =\lim_{n\to\infty}\frac{a_1\frac{\frac{1}{q^{n-1}}- p\left( \frac{p}{q} \right)^{n-1}}{1-p} +
				b_1\frac{\frac{1}{q{n-1}}-q}{1-q}}
			{a_1\frac{\frac{1}{q^{n-1}}- \left( \frac{p}{q} \right)^{n-1}}{1-p} + b_1\frac{\frac{1}{q^{n-1}}-1}{1-q}}
			\tag{1}
		\end{align*}
		此时有$\displaystyle\lim_{n\to\infty}\frac{1}{q^{n-1}}=0$,则式$(1)$可以进一步化简为:
		\begin{align*}
			\lim_{n\to\infty}\frac{S_n}{S_{n-1}} = \lim_{n\to\infty}\frac{a_1\frac{-p \left( \frac{p}{q}
					\right)^{n-1}}{1-p} + b_1\frac{-q}{1-q}}
			{a_1\frac{-\left( \frac{p}{q} \right)^{n-1}}{1-p} + b_1\frac{-1}{1-q}}
		\end{align*}
		因为$p>q$,所以有$\displaystyle\lim_{n\to\infty}\left( \frac{p}{q} \right)^{n-1}$趋于无穷,所以有:
		\begin{align*}
			\lim_{n\to\infty}\frac{S_n}{S_{n-1}} & = \lim_{n\to\infty}\frac{a_1\frac{-p \left( \frac{p}{q}
			\right)^{n-1}}{1-p}}{a_1\frac{-\left( \frac{p}{q}\right)^{n-1}}{1-p}}                          \\
			                                     & = p
		\end{align*}
		\circled{2}若$q<1$
		\begin{align*}
			\lim_{n\to\infty}\frac{S_n}{S_{n-1}}
			 & = \lim_{n\to\infty}\frac{a_1\frac{\frac{1}{p^{n-1}}-p}{1-p} + b_1\frac{\frac{1}{p^{n-1}}-q \left(
					\frac{q}{p} \right)^{n-1}}{1-q}}
			{a_1\frac{\frac{1}{p^{n-1}}-1}{1-p} + b_1\frac{\frac{1}{p^{n-1}}- \left( \frac{q}{p} \right)^{n-1}}{1-q}}
			\tag{3}
		\end{align*}
		因为$p> q$所以$\displaystyle\lim_{n\to\infty}\left(  \frac{q}{p}\right)^{n-1} = 0$
		式$(3)$可以进一步化简为
		\begin{align*}
			\lim_{n\to\infty}\frac{S_n}{S_{n-1}} & =
			\lim_{n\to\infty}\frac{a_1\frac{\frac{1}{p^{n-1}}-p}{1-p}+b_1\frac{\frac{1}{p^{n-1}}}{1-q}}
			{a_1\frac{\frac{1}{p^{n-1}}-1}{1-p} + b_1\frac{\frac{1}{p^{n-1}}}{1-q}} \tag{4}
		\end{align*}
		\circled{a}若$p>1$,则有$\displaystyle\lim_{n\to\infty}\frac{1}{p^{n-1}} = 0$, 式$(4)$可进一步化简为:
		\begin{align*}
			\lim_{n\to\infty}\frac{S_n}{S_{n-1}} & = \lim_{n\to\infty}\frac{a_1\frac{-p}{1-p}}{a_1\frac{-1}{1-p}} \\
			                                     & = p
		\end{align*}
		\circled{b}若$p<1$,则有$\displaystyle\lim_{n\to\infty}p^n=0$.进一步化简式$(4)$得:
		\begin{align*}
			\lim_{n\to\infty}\frac{S_n}{S_{n-1}} & =
			\lim_{n\to\infty}\frac{a_1\frac{1-p^n}{1-p} + b_1\frac{1}{1-q}}{a_1\frac{1-p^{n-1}}{1-p} + b_1\frac1{1-q}}
			\\
			                                     & = \lim_{n\to\infty}\frac{a_1\frac{1}{1-p}
			+b_1\frac1{1-q}}{a_1\frac{1}{1-p} + b_1\frac1{1-q}}                              \\
			                                     & = 1
		\end{align*}
		综上:
		\begin{equation*}
			\lim_{n\to\infty}\frac{S_n}{S_{n-1}}  =
			\begin{cases}
				p \qquad (q >1)              \\
				p \qquad (q < 1 \land p > 1) \\
				1 \qquad (q < 1 \land p < 1)
			\end{cases}
		\end{equation*}
	\end{solution}

	\question
	甲、乙两地相距$S$千米,汽车从甲地匀速行驶到乙地,速度不得超过$c$千米/时.已知汽车每小时的运输成本(以元为单位)由可变部分和固定部分组成:可变部分与速度$v$(千米/时)的平方成正比,比例系数为$b$;固定部分为$a$元.
	\begin{enumerate}[label=(\arabic*)]
		\item 把全程运输成本$y$(元)表示为速度$v$(千米/时)的函数,并指出这个函数的定义域;
		\item 为了使全程运输成本最小,汽车应以多大速度行驶?
	\end{enumerate}

	\begin{solution}
		\begin{enumerate}[label=(\arabic*)]
			\item 运输成本可表示为:
			      \begin{align*}
				      y & = (a + bv^2)\cdot \frac{S}{v} \qquad(v < c) \\
				        & = S(\frac{a}{v}+ bv)
			      \end{align*}
			\item $\frac{a}{v} + bv \geqslant 2\sqrt{\frac{a}{v}bv} = 2\sqrt{ab}$,当且仅当$\frac{a}{v} =
				      bv$时取最小值,此时 $v=\sqrt{\frac{a}{b}}$
		\end{enumerate}

	\end{solution}

	\question 如图,在正方体$ABCD-A_1B_1C_1D_1$中,$E,F$分别是$BB_1,CD$的中点.
	\begin{penum}
		\item 证明:$AD\perp D_1F$;
		\item 求$AE$与$D_1F$所成的角;
		\item 证明:面$AED\perp$面$A_1FD_1$;
		\item 设$AA_1=2$,求三棱锥$F-A_1ED_1$的体积.
	\end{penum}

	\tdplotsetmaincoords{80}{5}
	\begin{center}
		\begin{tikzpicture}[tdplot_main_coords, scale=1.3]
			\coordinate(A) at(0,0);
			\coordinate(B) at(2,0);
			\coordinate(C) at(2,2);
			\coordinate(D) at(0,2);
			\coordinate(A1) at(0,0,2);
			\coordinate(B1) at(2,0,2);
			\coordinate(C1) at(2,2,2);
			\coordinate(D1) at(0,2,2);
			\coordinate(E) at (2,0,1);
			\coordinate(F) at (1,2);
			\coordinate(G) at (2,2,1);
			\coordinate(H) at (0.8,2,0.4);

			\tkzDrawPolygon(A,B,B1,A1)
			\draw[very thin] (A1) -- (D1) -- (C1) -- (B1) (C1) -- (C) -- (B) (A)--(E);
			\draw[very thin, dashed] (D) -- (C) (D) -- (A) (D) -- (D1) (D) -- (E) -- (F) -- (A) (D1)--(F)--(A1) (D)--(G);
			\tkzLabelPoints(A,B)
			\tkzLabelPoints[right](C,C1)
			\tkzLabelPoint[above left](D1){$D_1$}
			\tkzLabelPoint[left](A1){$A_1$}
			\tkzLabelPoint[above left](D){$D$}
			\tkzLabelPoint[right](E){$E$}
			\tkzLabelPoint[right](G){$G$}
			\tkzLabelPoint[above right](H){$H$}
			\tkzLabelPoint[below right](F){$F$}
			\draw[fill=gray](H) circle (1pt);
		\end{tikzpicture}
	\end{center}

	\begin{solution}
		\begin{penum}
			\item
			\begin{align*}
				 & \because ABCD-A_1B_1C_1D_1\text{是正方体} \\
				 & \therefore AD \perp CDD_1C_1          \\
				 & \therefore AD \perp D_1F
			\end{align*}
			\item
			取$C_1C$的中点$G$,连接$DG$与$D_1F$交于$H$.
			\begin{align*}
				 & \because \triangle{DFD_1} \cong \triangle{CGD}  \\
				 & \therefore \angle{DD_1F} = \angle{CDG}          \\
				 & \because \angle{DFD_1} = \angle{D_1FD}          \\
				 & \therefore \angle{DHF}=\angle{D_1DF} = \ang{90} \\
				 & \therefore DG \text{与} D_1F \text{成}\ang{90}    \\
				 & \because DG \text{是}AE\text{在}DCC_1D1上的投影       \\
				 & \therefore AE \perp D_1F
			\end{align*}
			\item
			\begin{align*}
				 & \because D_1F \perp DG \land D_1F \perp DA \\
				 & \therefore D_1F \perp ADE                  \\
				 & \therefore AED \perp A_1FD_1
			\end{align*}
			\item
			\begin{align*}
				V_{F-A_1ED_1} & = V_{E-A_1D_1F}                        \\
				              & = \frac13S_{\triangle{A_1FD_1}}\cdot h \\
			\end{align*}
			\begin{align*}
				S_{\triangle{A_1FD_1}} & = \frac12A_1D_1\cdot D_1F          \\
				                       & = \frac12 \times 2 \times \sqrt{5} \\
				                       & = \sqrt{5}
			\end{align*}
			因为平面$AEGD \perp$ 平面$A_1FD_1$,所以$h=HG=DG - DH = \sqrt{5} - \frac25{\sqrt{5}} = \frac35\sqrt{5}$

			所以体积为:
			\begin{align*}
				V_{F-A_1ED_1} & = \frac13 \times \sqrt{5} \times \frac35\sqrt{5} \\
				              & = 1
			\end{align*}
		\end{penum}
	\end{solution}

	\question 设二次函数$f(x)=ax^2+bx+c(a>0)$,方程$f(x) - x = 0$的两个根$x_1,x_2$满足$0<x_1<x_2<\frac1a$.
	\begin{penum}
		\item 当$x\in(0,x_1)$时,证明$x<f(x)<x_1$;
		\item 设函数$f(x)$的图像关于直线$x=x_0$对称,证明$x_0<\frac{x_1}{2}$.
	\end{penum}

	\begin{solution}
		\begin{penum}
			\item 因为$a>0$,方程$f(x)-x=0$的图像大概的形状应该是:
			\begin{center}
				\begin{tikzpicture}
					\begin{axis}[xmin=-2,
							xmax=5,
							axis lines=center,
							ymin=-2,
							ymax=5,
							xlabel=$x$,
							ylabel=$y$,
							ticks=none
						]
						\addplot[domain=-1:4]{x^2/3-x +2/3};
						\node[below left] at (0,0) {$O$};
						\node[below] at (1,0) {$x_1$};
						\node[below] at (2,0) {$x_2$};
						\filldraw[red] (3,0) node[below, black] {$1/a$} circle(1pt);
					\end{axis}
				\end{tikzpicture}
			\end{center}
			可以看到$f(x) -x > 0$,所以有$f(x) > x$.

			因为$f(x) = x < x_1$,所以得证:
			\begin{equation*}
				x  < f(x) < x_1
			\end{equation*}

			\item
			\begin{enumerate}[label=\protect\circled{\arabic*}]
				\item $f(x)$关于直线$x=-\frac{b}{2a}$ 对称,即$x_0 = -\frac{b}{2a}$.
				\item $x_1, x_2=\frac{-b + 1 \pm \sqrt{(b-1)^2-4ac}}{2a}$
				\item 命题转换为证明:
				      \begin{equation*}
					      -\frac{b}{2a} < \frac{-b + 1 - \sqrt{(b-1)^2 - 4ac}}{4a} \tag{1}
				      \end{equation*}
				\item 将式$(1)$两边同乘以$4a$:
				      \begin{align*}
					      -2b < -b + 1 - \sqrt{(b-1)^2 - 4ac} \\
					      b + 1 > \sqrt{(b-1)^2 - 4ac}        \\
					      b^2 + 2b + 1 > b^2 - 2b + 1 - 4ac   \\
					      b + ac > 0                          \\
				      \end{align*}
				\item 由$x_2 < \frac1a$得
				      \begin{align*}
					      \frac{-b + 1 + \sqrt{(b-1)^2 - 4ac}}{2a} < \frac1a \\
					      -b + 1 + \sqrt{(b-1)^2 - 4ac} < 2                  \\
					      (b-1)^2 - 4ac < (b+1)^2                            \\
					      b + ac > 0
				      \end{align*}
				\item 得证.
			\end{enumerate}
		\end{penum}
	\end{solution}

\end{questions}
\end{document}

%!tex program = lualatex
\documentclass[answers]{exam}
\usepackage{silence}
\WarningFilter{latexfont}{Font shape}
\WarningFilter{latexfont}{Some font}
\usepackage{ctex}
\usepackage[margin=2cm]{geometry}
\usepackage{amsmath, amssymb, amsthm, arcs}
\usepackage{siunitx}
\usepackage{csquotes}
\usepackage{tikz, pgfplots, tikz-3dplot}
\usepackage[lua]{tkz-euclide}
\usetikzlibrary{
	angles,
	backgrounds,
	calc,
	decorations.pathmorphing,
	decorations.pathreplacing,
	decorations.text,
	intersections,
	patterns,
	petri,
	positioning,
	quotes,
	shapes,
	shapes.symbols,
}
\usepackage{graphicx, wrapfig}
\usepackage{caption}
\pagestyle{empty}
\newcounter{xcord}
\newcounter{ycord}
\newcounter{total}
\renewcommand{\labelenumi}{\textbf{\ifnum\value{enumi}<10 0\fi\arabic{enumi})}}

\pgfplotsset{compat=1.18}

\CorrectChoiceEmphasis{\color{blue!70!green}\bfseries}
\renewcommand{\solutiontitle}{\textbf{解:}}
\renewcommand{\choicelabel}{(\Alph{choice})}

\usepackage{array, tabularx}
\newcolumntype{C}{>{\centering\arraybackslash}X}
\newcolumntype{B}{>{\centering\bfseries\arraybackslash}X}

\usepackage{fontawesome5}
\usepackage{enumitem}
\newcommand\regularHandPointRight{\faHandPointRight[regular]}

\newenvironment{mathenum}
{\begin{enumerate}[label=\arabic*.]}
		{\end{enumerate}}

\newenvironment{penum}
{\begin{enumerate}[label=(\arabic*)]}
		{\end{enumerate}}
\makeatletter
\providecommand\@gobblethree[3]{}
\patchcmd{\over@under@arc}
{\@gobbletwo}
{\@gobblethree}
{}{}
\makeatother

\newcommand*\circled[1]{\tikz[baseline=(char.base)]{
		\node[shape=circle,draw,inner sep=1pt] (char) {#1};}}
\setlength{\answerclearance}{6pt}
\newcommand\hfs{\hfill (\hspace{2cm})}

\usepackage[lua]{tkz-euclide}

\renewcommand{\choicelabel}{(\Alph{choice})}
\begin{document}
\begin{center}
	\textbf{1998年普通高等学校招生考试(全国卷)}\\
	\textbf{\Large 理科数学}
\end{center}

\begin{questions}
	\question $\sin\ang{600}$的值是 \hfill (\hspace{1cm})

	\begin{oneparchoices}
		\choice $\frac12$
		\choice $-\frac12$
		\choice $\frac{\sqrt{3}}{2}$
		\CorrectChoice $-\frac{\sqrt{3}}{2}$
	\end{oneparchoices}

	\begin{solution}
		\begin{align*}
			\sin\ang{600} = \sin\ang{240} = -\frac{\sqrt{3}}{2}
		\end{align*}
	\end{solution}

	\question 函数$y=a^{|x|}(a>1)$的图像是\hfill (\hspace{1cm})

	\begin{oneparchoices}
		\choice
		\begin{tikzpicture}[scale=0.45]
			\begin{axis}[
					axis lines=middle,
					xlabel = $x$,
					ylabel = $y$,
					domain = -3:3,
					samples = 100,
					ymin = -1,
					ymax = 5,
					ticks = none
				]
				\addplot[thick, domain=0:3]{2^x - 1};
				\addplot[thick, domain=-3:0]{2^(-x) - 1};
				\node[below left] at(0,0) {$O$};
			\end{axis}
		\end{tikzpicture}
		\CorrectChoice
		\begin{tikzpicture}[scale=0.45]
			\begin{axis}[
					axis lines=middle,
					xlabel = $x$,
					ylabel = $y$,
					domain = -3:3,
					samples = 100,
					ymin = -1,
					ymax = 5,
					ticks = none
				]
				\addplot[thick, domain=0:3]{2^x};
				\addplot[thick, domain=-3:0]{2^(-x)};
				\node[below left] at(0,0) {$O$};
				\node[below left] at(0,1) {$1$};
			\end{axis}
		\end{tikzpicture}

		\choice
		\begin{tikzpicture}[scale=0.45]
			\begin{axis}[
					axis lines=middle,
					xlabel = $x$,
					ylabel = $y$,
					domain = -3:3,
					samples = 100,
					ymin = -1,
					ymax = 5,
					ticks = none
				]
				\addplot[thick, domain=0:3]{2^(-x)};
				\addplot[thick, domain=-3:0]{2^(x)};
				\node[below left] at(0,0) {$O$};
				\node[above left] at(0,1) {$1$};
			\end{axis}
		\end{tikzpicture}

		\choice
		\begin{tikzpicture}[scale=0.45]
			\begin{axis}[
					axis lines=middle,
					xlabel = $x$,
					ylabel = $y$,
					domain = -3:3,
					samples = 100,
					ymin = -1,
					ymax = 5,
					ticks = none
				]
				\addplot[thick, domain=0:3]{2-2^(-x)};
				\addplot[thick, domain=-3:0]{2-2^(x)};
				\node[below left] at(0,0) {$O$};
				\node[below left] at(0,1) {$1$};
			\end{axis}
		\end{tikzpicture}
	\end{oneparchoices}

	\begin{solution}
		$a^{|x|} (a>1)$的最小值是$1$,而且不收敛,所以答案是B.
	\end{solution}

	\question 曲线的极坐标方程$\rho=4\sin\theta$化成直角坐标方程为 \hfill (\hspace{1cm})

	\begin{oneparchoices}
		\choice $x^2 + (y+2)^2 = 4$ \CorrectChoice $x^2 + (y-2)^2 = 4$
		\choice $(x-2)^2 + y^2 = 4$ \choice $(x+2)^2 + y^2 = 4$
	\end{oneparchoices}
	\begin{solution}
		设曲线上的动点$P(x,y)$,则有:
		\begin{align*}
			y    & = \rho\sin\theta                 \\
			x    & = \rho\cos\theta                 \\
			\rho & = \sqrt{x^2 + y^2} = 4\sin\theta
		\end{align*}
		则有
		\begin{align*}
			\sqrt{x^2 + y^2} & = 4\frac{y}{\rho} \\
			x^2 + y^2        & = 4y              \\
			x^2 + (y-2)^2    & = 4
		\end{align*}
	\end{solution}

	\question 两条直线$A_1x + B_1y + C_1 = 0, A_2x + B_2y + C_2=0$垂直的充要条件是 \hfill(\hspace{1cm})

	\begin{oneparchoices}
		\CorrectChoice $A_1A_2 + B_1B_2 =0$ \choice $A_1A_2 - B_1B_2 = 0$ \choice $\dfrac{A_1A_2}{B_1B_2} = -1$
		\choice $\dfrac{A_1A_2}{B_1B_2} = 1$
	\end{oneparchoices}

	\begin{solution}
		直线的斜率分别为$-\dfrac{A_1}{B_1}$和$-\dfrac{A_2}{B_2}$,两条直线垂直的充要条件是斜率的乘积为$-1$,所以答案选择A.
	\end{solution}

	\question 函数$f(x)=\frac1x(x\neq0)$的反函数$f^{-1}(x)=$ \hfill (\hspace{1cm})

	\begin{oneparchoices}
		\choice $x ( x\neq0)$ \CorrectChoice $\frac1x(x\neq0)$ \choice $-x (x\neq0)$ \choice $-\frac1x (x\neq0)$
	\end{oneparchoices}
	\begin{solution}
		因为$y=\frac1x$关于$y=x$对称,所以反函数是其本身.
	\end{solution}

	\question 已知点$P(\sin\alpha-\cos\alpha, \tan\alpha)$在第一象限,则在$[0,2\pi]$内$\alpha$的取值范围是 \hfill
	(\hspace{1cm})

	\begin{choices}
		\choice $\left( \dfrac{\pi}{2}, \dfrac{3\pi}{4} \right) \cup \left( \pi, \dfrac{5\pi}{4} \right)$
		\CorrectChoice $\left( \dfrac{\pi}{4}, \dfrac{\pi}{2} \right) \cup \left( \pi, \dfrac{5\pi}{4} \right)$
		\choice $\left( \dfrac{\pi}{2}, \dfrac{3\pi}{4} \right) \cup \left( \dfrac{5\pi}{4}, \dfrac{3\pi}{2} \right)$
		\choice $\left( \dfrac{\pi}{4}, \dfrac{\pi}{2} \right) \cup \left( \dfrac{3\pi}{4}, \pi \right)$
	\end{choices}

	\begin{solution}
		根据条件有:
		\begin{align*}
			\sin\theta - \cos\theta > 0 \tag{a} \\
			\tan\theta = \frac{\sin\theta}{\cos\theta} > 0 \tag{b}
		\end{align*}
		根据式$(a)$有$\alpha \in \left( \dfrac{\pi}{4}, \dfrac{5\pi}{4} \right)$;根据式$(b)$有$\alpha \in \left( 0,
			\dfrac{\pi}{2} \right) \cup \left( \pi, \dfrac{3\pi}{2} \right)$

		综上可得取值范围为B.
	\end{solution}

	\question 已知圆锥的全面积是底面积的$3$倍,那么该圆锥的侧面展开图扇形的圆心角为\hfill (\hspace{1cm})

	\begin{oneparchoices}
		\choice $\ang{120}$
		\choice $\ang{150}$
		\CorrectChoice $\ang{180}$
		\choice $\ang{240}$
	\end{oneparchoices}

	\begin{solution}
		设圆锥的底面半径为$r$,棱线长为$l$,则底面积为$\pi r^2$,侧面积为$\frac{2\pi r}{2\pi l}\pi l^2 = \pi
			rl$,由题意有$\pi rl = 2\pi r^2$,即$l=2r$,展开图的圆心角为$2\pi\frac{r}{l}=\pi$,选C.
	\end{solution}

	\question 复数$-i$的一个立方根是$i$,它的另外两个立方根是 \hfill (\hspace{1cm})

	\begin{oneparchoices}
		\choice $\dfrac{\sqrt{3}}{2} \pm \dfrac12i$
		\choice $-\dfrac{\sqrt{3}}{2} \pm \dfrac12i$
		\choice $\pm\dfrac{\sqrt{3}}{2} + \dfrac12i$
		\CorrectChoice $\pm\dfrac{\sqrt{3}}{2} - \dfrac12i$
	\end{oneparchoices}

	\begin{solution}
		根据欧拉恒等式$e^{i\theta} = \cos\theta + i\sin\theta$,当$\theta=\frac32\pi$时有$e^{i\frac32\pi} =
			-i$.其立方根为$e^{i(\frac{\frac32\pi + 2k\pi}{3})}, (k=0,1,2)$.

		\begin{enumerate}[label=(\arabic*)]
			\item 当$k=0$时,$e^{\frac12\pi} = \cos \left( \dfrac12\pi \right) + \sin \left( \dfrac12\pi \right)i = i$
			\item 当$k=1$时,$e^{\frac76\pi} = \cos \left( \dfrac76\pi \right) + \sin \left( \dfrac76\pi \right)i =
				      -\frac{\sqrt{3}}{2} - \frac12i$
			\item 当$k=2$时,$e^{\frac{11}6\pi} = \cos \left( \dfrac{11}6\pi \right) + \sin \left( \dfrac{11}6\pi \right)i =
				      \frac{\sqrt{3}}{2} - \frac12i$
		\end{enumerate}
		答案选D.
	\end{solution}

	\question 如果棱台的两底面积分别是$S,S'$,中截面的面积是$S_0$,那么 \hfill (\hspace{1cm})

	\begin{oneparchoices}
		\CorrectChoice $2\sqrt{S_0} = \sqrt{S} + \sqrt{S'}$
		\choice $S_0 = \sqrt{SS'}$
		\choice $2S_0 = S + S'$
		\choice $S_0^2 = 2SS'$
	\end{oneparchoices}

	\begin{solution}
		一维方向上有 $\sqrt{S_0} = \frac{\sqrt{S} + \sqrt{S'}}{2}$,所以选择A.
	\end{solution}

	\question 向高为$H$的水瓶中注水,注满为止,如果注水量$V$与水深$h$的函数关系的图像如下图所示,那么水瓶的形状是 \hfill
	(\hspace{1cm})

	\begin{tikzpicture}[scale=0.45]
		\begin{axis}[
				axis lines=middle,
				xlabel = $h$,
				ylabel = $V$,
				domain = -3:3,
				samples = 100,
				ymin = -1,
				ymax = 5,
				xmax = 3,
				xmin = -1,
				ticks = none
			]
			\addplot[thick, domain=0:2]{-x^2+ 4*x};
			\node[below left] at(0,0) {$O$};
			\draw [dashed] (2,4)  -- (2,0) node[below]{$H$};
		\end{axis}
	\end{tikzpicture}

	\tdplotsetmaincoords{80}{0}
	\begin{oneparchoices}
		\choice
		\begin{tikzpicture}[tdplot_main_coords, scale=.5]
			\tkzDefPoints{0/0/O, 1/0/A, -1/0/B}
			\coordinate(O') at (0,0,4);
			\coordinate(A') at (2,0,4);
			\coordinate(B') at (-2,0,4);

			\tkzDrawSegments(A,A' B,B')
			\tkzDrawSemiCircle(O',A')
			\tkzDrawSemiCircle(O',B')

			\tkzDrawSemiCircle[dashed](O,A)
			\tkzDrawSemiCircle(O,B)
		\end{tikzpicture}

		\CorrectChoice
		\begin{tikzpicture}[tdplot_main_coords, scale=.5]
			\tkzDefPoints{0/0/O, 2/0/A, -2/0/B}
			\coordinate(O') at (0,0,4);
			\coordinate(A') at (1,0,4);
			\coordinate(B') at (-1,0,4);

			\tkzDrawSegments(A,A' B,B')
			\tkzDrawSemiCircle(O',A')
			\tkzDrawSemiCircle(O',B')

			\tkzDrawSemiCircle[dashed](O,A)
			\tkzDrawSemiCircle(O,B)
		\end{tikzpicture}

		\choice
		\begin{tikzpicture}[tdplot_main_coords, scale=.5]
			\tkzDefPoints{0/0/O, 2/0/A, -2/0/B}
			\coordinate(O') at (0,0,4);
			\coordinate(A') at (2,0,4);
			\coordinate(B') at (-2,0,4);
			\coordinate(C) at (1.5,0, 2);
			\coordinate(C') at (-1.5,0, 2);

			\tkzDrawSegments(A,C C,A' B,C' C',B')
			\tkzDrawSemiCircle(O',A')
			\tkzDrawSemiCircle(O',B')

			\tkzDrawSemiCircle[dashed](O,A)
			\tkzDrawSemiCircle(O,B)
		\end{tikzpicture}

		\choice
		\begin{tikzpicture}[tdplot_main_coords, scale=.5]
			\tkzDefPoints{0/0/O, 2/0/A, -2/0/B}
			\coordinate(O') at (0,0,4);
			\coordinate(A') at (2,0,4);
			\coordinate(B') at (-2,0,4);

			\tkzDrawSegments(A,A' B,B')
			\tkzDrawSemiCircle(O',A')
			\tkzDrawSemiCircle(O',B')

			\tkzDrawSemiCircle[dashed](O,A)
			\tkzDrawSemiCircle(O,B)
		\end{tikzpicture}
	\end{oneparchoices}

	\begin{solution}
		由函数图像可以看出斜率是逐渐下降的,所以选择B.
	\end{solution}

	\question $3$名医生和$6$名护士被分配到$3$所学校为学生体检,每校分配$1$名医生和$2$名护士,不同的分配方法共有\hfill
	(\hspace{1cm})

	\begin{oneparchoices}
		\choice $90$种
		\choice $180$种
		\choice $270$种
		\CorrectChoice $540$种
	\end{oneparchoices}

	\begin{solution}
		$3$名医生分到三个学校共有$3!=6$种分配方法.

		$6$名护士分到三个学校,每个学校两名护士有$\displaystyle\binom{6}{2}\cdot\binom{4}{2}\cdot\binom{2}{2}=90$种分配方法.

		所以总共有$6\times90=540$种分配方法.
	\end{solution}

	\question 椭圆$\dfrac{x^2}{12} + \dfrac{y^2}{3} =
		1$的焦点为$F_1$和$F_2$,点$P$在椭圆上,如果线段$PF_1$的中点在$y$轴上,那么$|PF_1|$是$|PF_2|$的\hfill (\hspace{1cm})

	\begin{oneparchoices}
		\CorrectChoice $7$倍
		\choice $5$倍
		\choice $4$倍
		\choice $3$倍
	\end{oneparchoices}

	\begin{solution}
		$c=\sqrt{12-3} = 3$,所以焦点分别为$(-3,0), (3,0)$.

		如果$PF_1$的中点在$y$轴上,则可知$P$点的$x$坐标为$3$.此时$\triangle{F_1F_2P}$是一个直角三角形,其中$|F_1F_2|=6$,$|PF_2|=\dfrac{\sqrt{3}}{2}$,则$|PF_1|=\sqrt{6^2+\frac{3}{4}}=\frac72\sqrt{3}$,所以答案选A.
	\end{solution}

	\question
	球面上有$3$个点,其中任意两点的球面距离都等于大圆周长的$\frac16$,经过这$3$个点的小圆的周长为$4\pi$,那么这个球的半径为
	\hfill (\hspace{1cm})

	\begin{oneparchoices}
		\choice $4\sqrt{3}$
		\CorrectChoice $2\sqrt{3}$
		\choice $2$
		\choice $\sqrt{3}$
	\end{oneparchoices}

	\begin{solution}
		经过三个点的小圆的周长为$4\pi$,则小圆的半径$r=2$.则可进一步推算出小圆内接等边三角形的边长为$2\sqrt{3}$.因为任意两点的球面距离是大圆的周长的$\frac16$,所以两个点之间的夹角为$\ang{60}$,因此球心与其中两个点的组成一个等边三角形,所以球的半径等于两点之间的距离$2\sqrt{3}$.
	\end{solution}

	\question 一个直角三角形三内角的正弦值成等比数列,其最小内角为\hfill (\hspace{1cm})

	\begin{oneparchoices}
		\choice $\arccos\dfrac{\sqrt{5}-1}{2}$
		\CorrectChoice $\arcsin\dfrac{\sqrt{5}-1}{2}$
		\choice $\arccos\dfrac{1-\sqrt{5}}{2}$
		\choice $\arcsin\dfrac{1-\sqrt{5}}{2}$
	\end{oneparchoices}

	\begin{solution}
		根据正弦定理有$\frac{\sin A}{a}=\frac{\sin B}{b}=\frac{\sin C}{c}$,所以三条边也成等比数列.
		\begin{equation*}
			\frac{a}{b} = \frac{b}{c} \tag{1}
		\end{equation*}
		又因为三角形是直角三角形,所以有
		\begin{equation*}
			a^2 + b^2 = c^2 \tag{2}
		\end{equation*}
		将式$(1)$转化为$b^2=ac$并代入式$(2)$可得:
		\begin{equation*}
			a^2 + ac = c^2
		\end{equation*}
		两边都除以$ac$得:
		\begin{equation*}
			\frac{a}{c} + 1 = \frac{c}{a}
		\end{equation*}

		设$\sin{A}=\frac{a}{c}=x$则原式可以表示为:
		\begin{align*}
			x + 1 = \frac{1}{x} \\
			x^2 + x - 1 = 0     \\
			x_1,x_2 = \frac{-1\pm\sqrt{5}}{2}
		\end{align*}
		因为$\angle{A}<\ang{90}$,所以$\sin{A}=\frac{-1+\sqrt{5}}{2}$,所以$\angle{A}$可以表示为$\arcsin{\frac{-1+\sqrt{5}}{2}}$.
	\end{solution}

	\question 在等比数列${a_n}$中,$a_1>1$,且前$n$项和$S_n$满足$\displaystyle
		\lim_{n\to\infty}S_n=\frac{1}{a_1}$,那么$a_1$的取值范围是 \hfill (\hspace{1cm})

	\begin{oneparchoices}
		\choice $(1,+\infty)$
		\choice $(1,4)$
		\choice $(1,2)$
		\CorrectChoice $(1,\sqrt{2})$
	\end{oneparchoices}

	\begin{solution}
		\begin{align*}
			\lim_{n\to\infty}S_n & = \lim_{n\to\infty}a_1\frac{1-q^n}{1-q} \\
			                     & = a_1\frac{1}{1-q}\qquad (|q|< 1)       \\
			                     & = a_1
		\end{align*}
		整理得:
		\begin{equation*}
			a_1^2 = 1 - q \qquad (|q|<1)
		\end{equation*}
		则有$0<a_1^2<2$,结合题中$a_1>1$可得$a_1$的取值范围为$(1,\sqrt{2})$.
	\end{solution}

	\question 设圆过双曲线$\frac{x^2}{9} -
		\frac{y^2}{16}=1$的一个顶点和一个焦点,圆心在此双曲线上,则圆心到双曲线中心的距离是\fillin[$\frac{16}{3}$][2cm].

	\begin{solution}
		双曲线的顶点为$(-3,0)$和$(3,0)$,焦点为$(-5,0)$和$(5,0)$,由圆的特性可知圆心必在过顶点和焦点中心的垂直线上.这样可以推断出圆是过同一侧的顶点和焦点,因为如果不在同一侧,中心线分别为$x=\pm1$,与双曲线没有交点.

		那么可以考察右侧的顶点和焦点,则圆心在$x=4$上,代入双曲线方程可以计算得$y^2=\dfrac{112}{9}$,则圆心到双曲线中心$(0,0)$的距离为:
		\begin{align*}
			\sqrt{x^2 + y^2} & = \sqrt{4^2 + \frac{112}{9}} \\
			                 & = \frac{16}{3}
		\end{align*}
	\end{solution}

	\question $(x+2)^{10}(x^2-1)$的展开式中$x^{10}$的系数为\fillin[179][2cm].(用数字作答)

	\begin{solution}
		根据二项式定理$(x+2)^{10}$可以展开为
		\begin{equation*}
			\sum_{n=0}^{10}\binom{10}{n}x^{10-n}2^{n}
		\end{equation*}
		则有$x^10$的系数为$1$,$x^8$的系数为$\binom{10}{2}2^2 =
			180$,再乘以$(x^2-1)$后原来的$x^{10}$的系数变为$-1$,原来$x^8$变为$x^{10}$,系数为$180$不变,则有展开式的$x^{10}$的系数为$179$.
	\end{solution}

	\question 在直四棱柱$A_1B_1C_1D_1-ABCD$中,当底面四边形$ABCD$满足条件\fillin[正方形或菱形][2cm]时,有$A_1C\perp
		B_1D_1$.(注:填上你认为正确的一种条件即可,不必考虑所有可能的情形)

	\begin{solution}
		\tdplotsetmaincoords{60}{110}
		\begin{tikzpicture}[tdplot_main_coords]
			\tkzDefPoints{0/0/A,2/0/B,2/2/C,0/2/D}
			\coordinate (A1) at (0,0,4);
			\coordinate (B1) at (2,0,4);
			\coordinate (C1) at (2,2,4);
			\coordinate (D1) at (0,2,4);

			\tkzDrawPolygon(A1,B1,C1,D1)
			\tkzDrawSegments(B,B1 C,C1 D,D1 B,C C,D)
			\tkzDrawSegments[dashed](A,A1 B,A A,D A1,C)

			\tkzLabelPoints(A,B,C,D)
			\tkzLabelPoints[above](A1,B1,C1,D1)

			\tkzDrawSegments(B1,D1 A1,C1)

		\end{tikzpicture}
		可以看到$A_1C$在四边形$A_1B_1C_1D_1$上的投影为$A_1C_1$,如果$A_1C_1 \perp B_1D_1$,则有$A_1C \perp
			B_1D_1$,所以底面四边形$ABCD$如果是正方形或者菱形即可.
	\end{solution}

	\question 关于函数$f(x)=4\sin \left( 2x + \dfrac{\pi}{3} \right)\quad (x\in \mathbf{R})$,有下列命题:
	\begin{enumerate}[label=\protect\circled{\arabic*}]
		\item 由$f(x_1) = f(x_2) = 0$可得$x_1 - x_2$必是$\pi$的整数倍;
		\item $y=f(x)$的表达式可改写为$y=4\cos \left( 2x - \dfrac{\pi}{6} \right)$;
		\item $y=f(x)$的图像关于点$\left( -\dfrac{\pi}{6}, 0\right)$对称;
		\item $y=f(x)$的图像关于直线$x=-\dfrac{\pi}{6}$对称.
	\end{enumerate}
	其中正确的命题的序号是\fillin[\circled{2}、\circled{3}][2cm].(注:把你认为正确的序号都填上)

	\begin{solution}
		\begin{enumerate}[label=\protect\circled{\arabic*}]
			\item 由$f(x_1) = f(x_2) = 0$可知$2x_1 + \frac{\pi}{3} - 2x_2 - \frac{\pi}{3} = 2(x_1-x_2) =
				      k\pi$,所以$x_1-x_2=\frac{k}{2}\pi$  \textcolor{red}{\times}
			\item $y=4\sin \left( 2x + \dfrac{\pi}{3} \right)=4\cos \left( \dfrac{\pi}{2} - 2x - \dfrac{\pi}{3} \right)
				      = 4\cos \left( \dfrac{\pi}{6} - 2x \right) = 4\cos \left( 2x - \dfrac{\pi}{6} \right)$ \textcolor{red}{\checkmark}
			\item 如图,所以\circled{3}正确、\circled{4}错误.

			      \begin{center}
				      \begin{tikzpicture}
					      \begin{axis}[
							      axis lines=middle,
							      xlabel = $x$,
							      ylabel = $y$,
							      xlabel style={right},
							      ylabel style={above},
							      ymax = 5,
							      ymin = -5,
							      samples = 100,
							      ticks = none
						      ]
						      \addplot[thick, domain=-2*pi:2*pi]{4*sin(deg(2*x + pi/3))};
						      \draw[draw=red, fill=red]({-pi/6}, 0)node[above left]{$-\frac{\pi}{6}$}circle(2pt);
					      \end{axis}

				      \end{tikzpicture}
			      \end{center}

		\end{enumerate}
	\end{solution}
	\question 在$\triangle{ABC}$中,$a$,$b$,$c$分别是角$A,B,C$的对边,设$a+c=2b,
		A-C=\dfrac{\pi}{3}$,求$\sin{B}$的值.

	\begin{solution}
		由正弦定理有$\dfrac{a}{\sin{A}}=\dfrac{b}{\sin{B}}=\dfrac{c}{\sin{C}}$和题中的$a+c=2b$可得:
		\begin{equation*}
			\sin{A} + \sin{C} = 2\sin{B}
		\end{equation*}
		由和差化积公式$\sin\alpha + \sin\beta = 2\sin{\dfrac{\alpha+\beta}{2}}\cos{\dfrac{\alpha-\beta}{2}}$可得:
		\begin{align*}
			\sin{A} + \sin{C} & = 2\sin \left( \frac{A+C}{2} \right)\cos \left( \frac{A-C}{2} \right) \\
			                  & = 2\sin \left( \frac{A+C+B - B}{2} \right) \cos{\frac{\pi}{6}}        \\
			                  & = \sqrt{3}\sin(\ang{90}-\frac{B}{2})                                  \\
			                  & = \sqrt{3}\cos\left(\frac{B}{2}\right)
		\end{align*}
		又因为$\sin{B}=2\sin\left(\dfrac{B}{2}\right)\cos\left(\dfrac{B}{2}\right)$,代入得:
		\begin{align*}
			4\sin\left(\frac{B}{2}\right)\cos\left(\frac{B}{2}\right) = \sqrt{3}\cos\left(\frac{B}{2}\right) \\
			\sin\left(\frac{B}{2}\right) = \frac{\sqrt{3}}{4}
		\end{align*}
		因为$\frac{B}{2}$肯定为锐角,所以
		\begin{equation*}
			\cos\left(\frac{B}{2}\right) = \sqrt{1 - \sin^2 \left(\frac{B}{2}\right)}= \frac{\sqrt{13}}{4}
		\end{equation*}
		则有:
		\begin{equation*}
			\sin{B} = 2\sin{\left( \frac{B}{2} \right)} \cos{\left( \frac{B}{2}
				\right)}=2\cdot\frac{\sqrt{3}}{4}\cdot\frac{\sqrt{13}}{4} = \frac{\sqrt{39}}{8}
		\end{equation*}
	\end{solution}

	\question 如图,直线$l_1$和$l_2$相交于点$M$,$l_1 \perp l_2$,点$N\in
		l_1$.以$A,B$为端点的曲线段$C$上的任一点到$l_2$的距离与点$N$的距离相等.若$\triangle{AMN}$为锐角三角形,$|AM|=\sqrt{17},|AN|=3,$且$|BN|=6$.建立适当的坐标系,求曲线$C$的方程.
	\begin{figure*}[htbp]
		\centering
		\begin{tikzpicture}[scale=0.5]
			\tkzDefPoints{0/0/M, 3/0/N, 0/3/P}
			\tkzDefPoint(40:{sqrt(17)}){A}
			\tkzDefPoint(6,5){B}
			\tkzInit[xmin=-2, xmax=6]
			\tkzDrawX

			\tkzDefLine[orthogonal=through A](M,P) \tkzGetPoint{x}
			\tkzInterLL(M,P)(A,x) \tkzGetPoint{D}

			\tkzDrawSegments[add=.5 and .5](M,N M,P)
			\tkzDrawSegments(A,M A,N B,N)
			\draw[shorten >= -1cm , shorten <= -1cm, -Latex] ([xshift=1.5cm]M) --([xshift=1.5cm]P) node[above=1cm]{$y$};
			\draw[very thin] (A) .. controls (4,5) and (5.5,5) .. (B);
			\tkzDrawSegment[dashed](A,D)

			\tkzLabelSegment[right=1.5cm, below](M,N){$l_1$}
			\tkzLabelSegment[above=1.5cm](M,P){$l_2$}
			\tkzLabelPoint[below left](M){$M$}
			\tkzLabelPoint[below](N){$N$}
			\tkzLabelPoint[above left](A){$A$}
			\tkzLabelPoint[above right](B){$B$}
			\tkzLabelPoint[left](D){$D$}
		\end{tikzpicture}
	\end{figure*}

	\begin{solution}
		\begin{enumerate}[label=\protect\circled{\arabic*}]
			\item 以$l_1$为$x$轴,过$MN$的中线为$y$轴,设抛物线的方程为$x=\dfrac{y^2}{4p}$.
			\item 过$A$点作垂线到$l_2$,垂足为$D$
			      \begin{align*}
				       & \because AD = AN = 3                                                  \\
				       & \therefore A \text{点横坐标为} 3-p                                    \\
				       & \therefore  |DM| = \sqrt{|AM|^2 - |DA|^2} = \sqrt{17 - 9} = 2\sqrt{2} \\
				       & \therefore A\text{点的纵坐标为}2\sqrt{2}                              \\
			      \end{align*}
			\item 将$A$点坐标$(3-p, 2\sqrt{2})$代入抛物线方程得:
			      \begin{align*}
				      3-p = \frac{(2\sqrt{2})^2}{4p} \\
				      p^2 -3p + 2 = 0                \\
				      (p-2)(p-1) = 0
			      \end{align*}
			      则有$p_1 = 1, p_2 = 2$
			\item 当$p=1$时,$MN=2$,此时有$|AM|^2 > |AN|^2 +
				      |MN|^2$,可以判定$\triangle{AMN}$是钝角三角形,与条件不符,所以$p=2$,即曲线的方程为$x=\dfrac{y^2}{8}$.
			\item 根据题意可知$B$点的横坐标为$6-p=4$,代入曲线方程得:
			      \begin{align*}
				      4 = \frac{y^2}{8}
				      y = \pm4\sqrt{2}
			      \end{align*}
			      因为曲线段$C$在第一象限,所以$B$点坐标为$(4,4\sqrt{2})$

			\item 综上,曲线的方程为:
			      \begin{equation*}
				      x = \frac{y^2}{8} \quad (1 \leqslant x \leqslant 4, y>0)
			      \end{equation*}
		\end{enumerate}

	\end{solution}

	\question
	如图,为处理含有某种杂质的污水,要制造一底宽为$2$米的均长方体沉淀箱,污水从$A$孔流入,经沉淀后从$B$孔流出.设箱体的长度为$a$米,调试为$b$米.已知流出的水中该杂质的质量分数与$a,b$的乘积$ab$成反比.现有制箱材料$60$平方米.问当$a,b$各为多少米时,经沉淀后流出的水路该杂质的质量分数最小.($A$、$B$孔的面积忽略不计)

	\begin{figure*}[ht]
		\centering
		\begin{tikzpicture}[scale=.5]
			\coordinate(A) at (0,0);
			\coordinate(B) at (5,0);
			\coordinate(C) at (5,3);
			\coordinate(D) at (0,3);
			\coordinate(A') at (0,0,3);
			\coordinate(B') at (5,0,3);
			\coordinate(C') at (5,3,3);
			\coordinate(D') at (0,3,3);

			\draw [dashed](D) -- (A) -- (B) ;
			\draw (B)	-- (C) -- (D);
			\draw (A') -- (B') -- (C') -- (D')--cycle;
			\draw[dashed](A) -- (A');
			\draw(B) -- (B');
			\draw(C) -- (C');
			\draw(D) -- (D');

			\draw (0,1.5,1.5) circle (2pt);
			\draw (5,1.5,1.5) circle (2pt);
			\node[below] at(0,1.5,1.5) {$A$};
			\node[above] at(5,1.5,1.5) {$B$};

			\tkzLabelSegment(A',B'){$a$}
			\tkzLabelSegment(B,B'){$2$}
			\tkzLabelSegment[left](C',B'){$b$}
		\end{tikzpicture}
	\end{figure*}

	\begin{solution}
		根据总面积为$60$平方米可得如下关系:
		\begin{equation*}
			60 = 2a + 2ab + 4b
		\end{equation*}
		化简得:
		\begin{equation*}
			30 = a + ab + 2b \tag{a}
		\end{equation*}
		设$y=ab$,则有$b=\frac{y}{a}$,代入上面的式子中可得:
		\begin{align*}
			30 = a + y + \frac{2}{a}y           \\
			(1+\frac2a)y = 30 - a               \\
			y = \frac{30-a}{1 + \frac2a}        \\
			y = \frac{a(30-a)}{a+2}     \tag{b} \\
		\end{align*}
		\subsection*{解法一}
		对式$(b)$求导:
		\begin{align*}
			y' & = \left( \frac{a(30-a)}{a+2} \right)'              \\
			   & = \frac{(30-2a)(a+2) - a(30-a)}{(a+2)^2}           \\
			   & = \frac{30a + 60 - 2a^2 - 4a - 30a + a^2}{(a+2)^2} \\
			   & = \frac{60 - 4a - a^2}{(a+2)^2}                    \\
			   & = \frac{(10+a)(6-a)}{(a+2)^2}
		\end{align*}
		因为$a>0$,所以只有$a=6$时$y'=0$,即此时$y=ab$有最大值.将$a=6$代入面积关系式$(a)$中可得:
		\begin{equation*}
			30 = 6 + 6b + 2b
		\end{equation*}
		解得$b=3$米.

		综上,当$a=6$米、$b=3$米时质量分数最小.

		\subsection*{解法二}
		化简式$(b)$:
		\begin{align*}
			y & = \frac{30a -a^2}{a+2}                      \\
			  & = \frac{30a + 4 + 4a - (4 + 4a + a^2)}{a+2} \\
			  & = \frac{34a + 4 - (a+2)^2}{a+2}             \\
			  & = \frac{34(a+2) -64 - (a+2)^2}{a+2}         \\
			  & = 34 - \frac{64}{a+2} - (a+2)               \\
		\end{align*}
		$\dfrac{64}{a+2} + (a+2) \geqslant 2\sqrt{\dfrac{64}{a+2}\cdot(a+2)} = 16$,当且仅当$\dfrac{64}{a+2} =
			a+2$时有最小值,即$(a+2)^2 = 64$,因为$a>0$所以$a=6$,代入式$(a)$中求得$b=3$.
	\end{solution}

	\question
	已知斜三棱柱$ABC-A_1B_1C_1$的侧面$A_1ACC_1$与底面$ABC$垂直,$\angle{ABC}=\ang{90}$,$BC=2,AC=2\sqrt{3}$,且$AA_1
		\perp A_1C, AA_1=A_1C$.
	\begin{penum}
		\item 求侧棱$A_1A$与底面$ABC$所成角的大小;
		\item 求侧面$A_1ABB_1$与底面$ABC$所成二面角的大小;
		\item 求顶点$C$到侧面$A_1ABB_1$的距离.
	\end{penum}
	\begin{figure*}[ht]
		\centering
		\tdplotsetmaincoords{60}{140}
		\begin{tikzpicture}[tdplot_main_coords, scale=2]
			\coordinate (A) at ({2*sqrt(3)},0);
			\coordinate (B) at (25:2);
			\coordinate (C) at (0,0);
			\coordinate (A1) at ({2*sqrt(3)-2},0,2);
			\coordinate (B1) at (B);
			\path (B1) ++(-2,0,2) coordinate (B1);
			\coordinate (C1) at (-2,0,2);

			\tkzDrawSegments(A,B B,C C,C1 C1,A1 A1,B1 B1,B A1,A B1,C1)
			\tkzDrawSegments[dashed](A,C A1,C)

			\tkzLabelPoints(A,B,C)
			\tkzLabelPoint[left](A1){$A_1$}
			\tkzLabelPoint[above](B1){$B_1$}
			\tkzLabelPoint[right](C1){$C_1$}

			\tkzMarkRightAngle[blue](C,B,A)
			\tkzMarkRightAngle[blue](A,A1,C)
			\tkzMarkSegments[mark=||](A,A1 A1,C)
			\tkzLabelSegment(B,C){$2$}
			\tkzLabelSegment[above right=.3cm, sloped](A,C){$2\sqrt{3}$}

			\coordinate(D) at (2,0);
			\tkzDefPointOnLine[pos=.5](A,B) \tkzGetPoint{E}
			\tkzDrawSegments[dashed, red](A1,D E,D A1,E)
			\tkzMarkRightAngle[dashed, red](A,D,A1)
			\tkzLabelPoint(D){$D$}
			\tkzLabelPoint(E){$E$}
		\end{tikzpicture}
	\end{figure*}

	\begin{solution}
		\begin{penum}
			\item
			\begin{align*}
				 & \because A_1ACC_1 \perp ABC                               \\
				 & \therefore AA_1\text{与}ABC\text{的夹角等于}\angle{A_1AC} \\
				 & \because
				\begin{array}{l}
					AA_1 = A_1C \\
					AA_1 \perp A_1C
				\end{array}                                              \\
				 & \therefore \triangle{A_1AC}\text{为等腰直角三角形}        \\
				 & \therefore \angle{A_1AC} = \ang{45}
			\end{align*}
			$A_1A$与底面$ABC$所成角的大小为$\ang{45}$
			\item 从点$A_1$向$AC$作垂线,垂足为$D$,从$D$作垂线到$AB$,垂足为$E$.连接$A_1E$.
			\begin{align*}
				 & \because \triangle{AA_1C}\text{是等腰直角三角形}                                               \\
				 & \therefore A_1D = \sqrt{3}                                                                     \\
				 & \because DE \perp AB \land BC \perp AB                                                         \\
				 & \therefore DE \parallel BC                                                                     \\
				 & \because AD = DC                                                                               \\
				 & \therefore DE = \frac12BC = 1                                                                  \\
				 & \therefore \sin\angle{A_1ED} = \frac{\sqrt{3}}{\sqrt{1^2 + (\sqrt{3})^2}} = \frac{\sqrt{3}}{2} \\
				 & \therefore \angle{A_1BD} = \ang{60}
			\end{align*}
			侧面$A_1ABB_1$与底面$ABC$所成的二面角为$\ang{60}$.
			\item
			从$C$点作垂线到侧面$A_1ABB_1$,垂足为$F$,则有$\triangle{BCF}$中有$\angle{CBF}=\ang{60}$.所以$C$到侧面$A_1ABB_1$的距离等于$\sin{\ang{60}} \cdot BC = \sqrt{3}$.
		\end{penum}
	\end{solution}
	\question 设曲线$C$的方程是$y=x^3-x$,将$C$沿$x$轴、$y$轴正向分别平行移动$t$、$s$单位长度后得曲线$C_1$.
	\begin{penum}
		\item 写出曲线$C_1$的方程.
		\item 证明曲线$C$与$C_1$关于点$A \left( \frac{t}{2},\frac{s}{2} \right)$对称.
		\item 如果曲线$C$与$C_1$有且仅有一个公共点,证明$s=\frac{t^3}{4}-t$且$t\neq0$.

		\begin{solution}
			\begin{penum}
				\item $C_1$的方程为:
				\begin{equation*}
					y - s = (x-t)^3 - (x-t)
				\end{equation*}
				整理得
				\begin{equation*}
					y = (x-t)^3 - x + s +t
				\end{equation*}
				\item
				如果$C$和$C_1$关于点$(\frac{t}{2},\frac{s}{2})$对称,对于$C$上的任意一点$P(x_1,y_1)$,其关于点$A$的对称点$P'(t-x_1,
					s-y_1)$必定在$C_1$上.将$P'$坐标代入$C_1$的方程中得:
				\begin{align*}
					s-y_1 = (-x_1)^3 +x_1 - t + s +t \\
					y_1 = x_1^3 - x_1
				\end{align*}
				所以$C$和$C_1$关于点$A$对称.
				\item 设$C$和$C_1$的交点坐标为$(x,y)$,则有
				\begin{math}
					\begin{cases}
						y = x^3 - x \\
						y = (x-t)^3 - x + s + t
					\end{cases}
				\end{math}

				消去$y$得到:
				\begin{align*}
					x^3 - x = x^3 - 3tx^2 + 3t^2x - t^3 - x + s + t \\
					3tx^2 - 3t^2x + t^3 - s - t = 0
				\end{align*}
				因为只有一个交点,所以
				\begin{align*}
					\Delta = 9t^4 - 4\cdot 3t(t^3 - s - t) = 0 (t\neq0)
				\end{align*}
				进一步化简得
				\begin{equation*}
					t^3 - 4t - 4s = 0
				\end{equation*}
				所以
				$s=\frac14t^3 - t\quad (t\neq0)$.
			\end{penum}
		\end{solution}
	\end{penum}
	\question 已知数列$\{b_n\}$是等差数列,$b_1=1, b_1+b_2+\cdots+b_{10}=145$.
	\begin{penum}
		\item 求数列$\{b_n\}$的通项$b_n$;
		\item 设数列$\{a_n\}$的通项$a_n = \log_a \left( 1 + \frac{1}{b_n}
			\right)$(其中$a>0$,且$a\neq1$),记$S_n$是数列$\{a_n\}$的前$n$项和.试比较$S_n$与$\frac13\log_ab_{n+1}$的大小,并证明你的结论.
	\end{penum}

	\begin{solution}
		\begin{penum}
			\item 设通项$b_n$为$b_n = b_1 + (n-1)d$,则其前$n$项和可以表示为
			\begin{math}
				\frac{n(b_1 + b_n)}{2}
			\end{math}.

			则有
			\begin{math}
				10(1+b_{10})/2=145
			\end{math},求得$b_{10}=28$
			将$n=10, b_{10}=28$代入通项公式得$d=3$,所以通项公式为:
			\begin{equation*}
				b_n = 3n - 2
			\end{equation*}
			\item 根据题意有
			\begin{equation*}
				S_n = \sum_{i=1}^n\log_a(1+\frac{1}{3i-2})
			\end{equation*}
			将上式两边同乘以$3$:
			\begin{align*}
				3S_n & = \sum_{i=1}^n3\log_a(1+\frac1{3i-2})                                        \\
				     & = \sum_{i=1}^n\log_a(1+\frac1{3i-2})^3                                       \\
				     & = \sum_{i=1}^n\log_a[1+\frac{3}{3i-2}+\frac{3}{(3i-2)^2} + \frac1{(3i-2)^3}]
			\end{align*}
			因为$3i-2>0$,所以$\frac3{(3i-2)^2} > 0, \frac1{(3i-2)^3}>0$,所以有
			\begin{align*}
				3S_n & > \sum_{i=1}^n\log_a(1+\frac3{3i-2})     \\
				     & = \sum_{i=1}^n\log_a\frac{3i+1}{3i-2}    \\
				     & = \log_a(\frac{3\times1 + 1}{3\times1-2}
				\cdot\frac{3\times2+1}{3\times2-2})
				\cdots
				\frac{3(n-1) +1}{3(n-1)+2}
				\cdot \frac{3n + 1}{3n-2}                       \\
				     & = \log_a(3n+1)
			\end{align*}
			所以有:
			\begin{equation*}
				S_n > \frac13\log_a(3n+1)
			\end{equation*}
		\end{penum}
	\end{solution}
\end{questions}

\end{document}

\end{document}
\usepackage{graphicx, wrapfig}
\usepackage{caption}
\newcounter{xcord}
\newcounter{ycord}
\newcounter{total}
\renewcommand{\labelenumi}{\textbf{\ifnum\value{enumi}<10 0\fi\arabic{enumi})}}

\usepackage{array, tabularx}
\newcolumntype{C}{>{\centering\arraybackslash}X}
\newcolumntype{B}{>{\centering\bfseries\arraybackslash}X}

\usepackage{fontawesome5}
\newcommand\regularHandPointRight{\faHandPointRight[regular]}
