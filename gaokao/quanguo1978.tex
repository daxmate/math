\section[1978年高考数学试卷及答案(全国卷)理科]{1978年普通高等学校招生考试(全国卷)\\\Huge{理科数学}}
\begin{questions}
	\question
	\begin{parts}
		\part 分解因式:$x^2 - 4xy + 4y^2 - 4z^2$.

			\begin{solution}
				\begin{align*}
					 & = (x-2y)^2 - 4z^2    \\
					 & = (x-2y+2z)(x-2y-2z)
				\end{align*}
			\end{solution}

		\part 已知正方形的边长为$a$,求侧面积等于这个正方形的面积,高等于这个正方形边长的直圆柱体的体积.

			\begin{solution}
				设直圆柱体的直径为$r$,则有侧面积:
				\begin{equation*}
					S_侧 = 2\pi{r}a
				\end{equation*}
				因为直圆柱体的侧面积等于正方形的面积,因此有:
				\begin{align*}
					2\pi{r}a = a^2 \\
					r = \frac{a}{2\pi}
				\end{align*}
				所以可得直圆柱体的体积为:
				\begin{align*}
					V & = \pi{r^2}a               \\
					  & = \pi \frac{a^2}{4\pi^2}a \\
					  & = \frac{a^3}{4\pi}
				\end{align*}
			\end{solution}

		\part 求函数$y=\sqrt{\lg(2+x)}$的定义域.

			\begin{solution}
				\begin{cenum}
					\item 由$\lg(2+x) \geqslant 0$得:
					      \begin{align*}
						      2+x \geqslant 1 \\
						      x \geqslant -1
					      \end{align*}

					      \begin{center}
						      \begin{tikzpicture}
							      \begin{axis}[
									      xmin = -4, xmax = 4,
									      ticks=both
								      ]
								      \addplot[domain=-3:3, blue!70, thick]{log10(x+2)};
								      \addlegendentry{$\lg(x+2)$}
							      \end{axis}
						      \end{tikzpicture}
					      \end{center}

					\item 由$2+x > 0$得:
					      \begin{equation*}
						      x > -2
					      \end{equation*}
					\item 综上,函数的定义域为$\{x|x\geqslant-1\}$.

				\end{cenum}
			\end{solution}

		\part 不查表求$\cos\ang{80}\cos\ang{35} + \cos\ang{10}\cos\ang{55}$的值.
			\begin{solution}
				\begin{align*}
					 & = \cos\ang{80}\cos\ang{35} + \sin\ang{80}\sin\ang{35} \\
					 & = \cos(\ang{80} - \ang{35})                           \\
					 & = \cos\ang{45}                                        \\
					 & = \frac{\sqrt{2}}{2}
				\end{align*}
			\end{solution}
	\end{parts}

\end{questions}
