\section[1977年高考数学试卷及答案(天津卷)理科]{1977年普通高等学校招生考试(天津卷)\\\Huge 数学试卷}

\begin{questions}
	\question
	\begin{parts}
		\part 在什么条件下,$\dfrac{y}{2x}$
			\begin{cenum}
				\item 是正数;
				\item 是负数;
				\item 等于零;
				\item 没有意义?
			\end{cenum}

			\begin{solution}
				\begin{cenum}
					\item $\sgn(x)=\sgn(y)$且$x\neq 0$;
					\item $\sgn(x)\neq\sgn(y)$且$x\neq 0$;
					\item $y=0$且$x\neq 0$;
					\item $x=0$
				\end{cenum}
			\end{solution}

		\part 比较下列各组数的大小,并说明理由.
			\begin{cenum}
				\item $\cos\ang{31}$与$\cos\ang{30}$.
				      \begin{solution}
					      \begin{center}
						      \begin{tikzpicture}
							      \begin{axis}[
									      xmin = -pi/2, xmax = 2.5*pi,
									      ymin = -1.5, ymax = 1.5,
								      ]
								      \addplot[domain=0:2*pi]{cos(deg(x))};
								      \addlegendentry{$\cos(x)$}

								      \node[above] at ({pi/2},0) {$\frac{\pi}{2}$};
							      \end{axis}
						      \end{tikzpicture}
					      \end{center}
					      可以看到余弦函数在第一象限是单调递减的,所以$\cos\ang{31} < \cos\ang{30}$.

				      \end{solution}
				\item $\log_21$与$\log_2\frac14$.
				      \begin{solution}
					      对二者分别求值:
					      \begin{align*}
						      \log_21 = \log_2{2^0} = 0 \\
						      \log_2{\frac14} = \log_2{2^{-2}} = -2
					      \end{align*}
					      所以有$\log_21 > \log_2{\frac14}$.
				      \end{solution}
			\end{cenum}
		\part 求值:
			\begin{cenum}
				\item $\tan \left( 5\arcsin\frac{\sqrt{3}}{2} \right)$.
				      \begin{solution}
					      \begin{align*}
						       & = \tan \left( 5\times \frac{\pi}{3} \right) \\
						       & = \tan \left( \frac{5\pi}{3} \right)        \\
						       & = -\sqrt{3}
					      \end{align*}
				      \end{solution}
			\end{cenum}

	\end{parts}
\end{questions}
