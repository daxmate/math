\section[1978年高考数学试卷(备用卷)文理]{1978年普通高等学校招生考试(备用卷)\\\Huge{数学试卷}}
\begin{questions}
	\question
	\begin{parts}
		\part 分解因式:$x^2 - 2xy + y^2 + 2x - 2y - 3$.

			\begin{solution}
				\begin{align*}
					 & = (x-y)^2 + 2(x-y) - 3 \\
					 & = (x-y +3)(x-y-1)
				\end{align*}
			\end{solution}

		\part 求$\sin\ang{30}-\tan\ang{0} + \cot\frac{\pi}{4} - \cos^2\frac{5\pi}{6}$.

			\begin{solution}
				\begin{align*}
					 & = \frac12 - 0 + 1 - \left( -\frac{\sqrt{3}}{2} \right)^2 \\
					 & = \frac34
				\end{align*}
			\end{solution}

		\part 求函数$y=\frac{\lg(25-5^x)}{x+1}$的定义域.
			\begin{solution}
				\begin{math}
					\begin{cases}
						25-5^x > 0 \\
						x + 1 \neq 0
					\end{cases}
				\end{math} \Rightarrow\quad
				\begin{math}
					\begin{cases}
						x < 2 \\
						x \neq - 1
					\end{cases}
				\end{math}

				所以定义域为$\{x|x<2\,(x\neq-1)\}$

			\end{solution}

	\end{parts}
\end{questions}
