\section[1978年高考数学试卷(备用卷)文理]{1978年普通高等学校招生考试(备用卷)\\\Huge{数学试卷}}
\begin{questions}
	\question
	\begin{parts}
		\part 分解因式:$x^2 - 2xy + y^2 + 2x - 2y - 3$.

			\begin{solution}
				\begin{align*}
					 & = (x-y)^2 + 2(x-y) - 3 \\
					 & = (x-y +3)(x-y-1)
				\end{align*}
			\end{solution}

		\part 求$\sin\ang{30}-\tan\ang{0} + \cot\frac{\pi}{4} - \cos^2\frac{5\pi}{6}$.

			\begin{solution}
				\begin{align*}
					 & = \frac12 - 0 + 1 - \left( -\frac{\sqrt{3}}{2} \right)^2 \\
					 & = \frac34
				\end{align*}
			\end{solution}

		\part 求函数$y=\frac{\lg(25-5^x)}{x+1}$的定义域.
			\begin{solution}
				\begin{math}
					\begin{cases}
						25-5^x > 0 \\
						x + 1 \neq 0
					\end{cases}
				\end{math} \Rightarrow\quad
				\begin{math}
					\begin{cases}
						x < 2 \\
						x \neq - 1
					\end{cases}
				\end{math}

				所以定义域为$\{x|x<2\,(x\neq-1)\}$

			\end{solution}

		\part 已知直圆锥体的底面半径等于\qty{1}{\cm},母线的长等于\qty{2}{\cm},求它的体积.

			\begin{solution}
				\begin{center}
					\tdplotsetmaincoords{70}{0}
					\begin{tikzpicture}[tdplot_main_coords, scale=2]
						\begin{scope}[canvas is xy plane at z=0]
							\draw (-1,0) coordinate(a) arc[start angle=180, end angle=360, radius=1cm];
							\draw[dashed] (-1,0) to[out=90, in=90] (1,0) coordinate(b);
						\end{scope}
						\draw (1,0) --node[sloped, above]{$\qty{2}{\cm}$} (0,0,{sqrt(3)})coordinate(c) -- (-1,0);
						\draw[dashed] (c) --node[left]{$h$} (0,0) --node[below]{$\qty{1}{\cm}$} (b);
					\end{tikzpicture}
				\end{center}
				可以计算出$h=\sqrt{3}$,根据体积公式有:
				\begin{align*}
					V & = \frac13\pi r^2h                                             \\
					  & = \frac13\pi\times1^2\times\sqrt{3}                           \\
					  & = \qty[parse-numbers=false]{\frac{\sqrt{3}}{3}\pi}{\cm\cubed}
				\end{align*}
			\end{solution}

	\end{parts}
\end{questions}
