\section*{1977年普通高等学校招生考试(河北卷)}

\maintitle{数学试卷}
\begin{questions}
	\question 解答下列各题
	\begin{enumerate}[label=(\arabic*)]
		\item 叙述函数的定义
		      \begin{solution}
			      函数是从一个集合到另一个集合之间的映射,使得每个输入值有且公有一个输出值.
		      \end{solution}
		\item 求函数$y=1-\frac{1}{\sqrt{2-3x}}$的定义域.
		      \begin{solution}
			      由函数的定义有:
			      \begin{math}
				      2-3x > 0
			      \end{math},则函数的定义域为:$x<\frac23$
		      \end{solution}
		\item 计算:$\left[1-(0.5)^{-2}\right] \div \left(-\frac{27}{8}\right)^{\frac13}$.
		      \begin{solution}
			      \begin{align*}
				      \left[1-(0.5)^{-2}\right] \div \left(-\frac{27}{8}\right)^{\frac13} & = \left(1-2^{-1\times (-2)}\right)
				      \div \left(-\frac32\right)^{3\times\frac13}                                                              \\
				                                                                          & = (1 - 4) \cdot (-\frac32)         \\
				                                                                          & = \frac92
			      \end{align*}
		      \end{solution}
		\item 计算:$\log_42$.
		      \begin{solution}
			      \begin{align*}
				      \log_42 & = \log_44^\frac12 \\
				              & = \frac12
			      \end{align*}
		      \end{solution}
		\item 分解因式:$x^2y - 2y^3$.
		      \begin{solution}
			      \begin{align*}
				      x^2y - 2y^3 & = y(x^2 - 2y^2)               \\
				                  & = y(x+\sqrt{2}y)(x-\sqrt{2}y)
			      \end{align*}
		      \end{solution}
		\item 计算:$\sin\dfrac{4\pi}{3}\cdot\cos\dfrac{25\pi}{6}\cdot\tan\left(-\dfrac{3\pi}{4}\right)$.
		      \begin{solution}
			      \begin{align*}
				      \sin\frac{4\pi}{3}\cdot\cos\frac{25\pi}{6}\cdot\tan\left(-\frac{3\pi}{4}\right)
				       & = -\sin\frac{\pi}{3}\cdot\cos\frac{\pi}{6}\cdot \tan\frac{3\pi}{4} \\
				       & = -\frac{\sqrt{3}}{2} \cdot \frac{\sqrt{3}}{2} \cdot 1             \\
				       & = - \frac43
			      \end{align*}
		      \end{solution}
	\end{enumerate}

	\question
	\begin{minipage}[t]{.3\textwidth}
		证明:如图,$AB$是圆$O$的直径,$CB$是圆$O$的切线,切点为$B$,$OC$平行于弦$AD$,求证:$DC$是圆$O$的切线.
	\end{minipage}\hspace{3cm}
	\begin{tikzpicture}[scale=.5, baseline=(current bounding box.center)]
		\tkzDefPoints{-2/0/A, 0/0/O, 2/0/B, 2/5/C}
		\tkzDefLine[parallel=through A](O,C) \tkzGetPoint{x}
		\tkzInterLC(A,x)(O,B) \tkzGetPoints{D}{A}

		\tkzDrawCircle(O,B)
		\tkzDrawSegment(A,B)
		\tkzDrawSegment(C,B)
		\tkzDrawSegment(C,O)
		\tkzDrawSegment(A,D)
		\tkzDrawSegment(C,D)
		\tkzDrawSegment[dashed, red](D,O)

		\tkzMarkAngle[mark=|, size=15pt](O,A,D)
		\tkzMarkAngle[mark=|, size=15pt](B,O,C)
		\tkzMarkAngle[mark=||, size=20pt](A,D,O)
		\tkzMarkAngle[mark=||, size=20pt](C,O,D)

		\tkzLabelPoint[left](A){$A$}
		\tkzLabelPoint[right](B){$B$}
		\tkzLabelPoint[above](C){$C$}
		\tkzLabelPoint[below](O){$O$}
		\tkzLabelPoint[left](D){$D$}
	\end{tikzpicture}

	\begin{solution}
		\begin{enumerate}[label=\arabic*.]
			\item 因为 $CB$ 是圆 $O$ 的切线且切点为 $B$,所以有 $AB \perp CB$.
			\item 由于 $AD \parallel OC$,可以得到 $\angle DAO = \angle COB$.
			\item 作辅助线 $DO$.因为 $\angle ADO$ 和 $\angle COD$ 是平行线的内错角,所以 $\angle ADO = \angle COD$.
			\item 在 $\triangle AOD$ 中,由于 $AD \parallel OC$,且 $OA = OD$(都是圆的半径),所以 $\triangle AOD$ 是等腰三角形,由此得到 $\angle OAD = \angle ODA$.
			\item 进一步分析 $\triangle DOC$ 和 $\triangle BOC$:
			      \begin{itemize}
				      \item $OB = OD$(都是圆的半径);
				      \item $OC = OC$(公共边);
				      \item $\angle BOC = \angle DOC$(由 $\angle COB = \angle DAO$ 和 $\angle ADO = \angle COD$ 推导).
			      \end{itemize}
			      因此,$\triangle BOC \cong \triangle DOC$(根据边角边准则).
			\item 由 $\triangle BOC \cong \triangle DOC$,可以得到 $\angle ODC = \angle CBO = 90^\circ$.
			\item 因为 $\angle ODC = 90^\circ$,所以 $DC$ 是圆 $O$ 的切线.
		\end{enumerate}
	\end{solution}

	\question 证明: $\dfrac{\sin2\alpha + 1}{1+\cos2\alpha + \sin2\alpha} = \dfrac12\tan\alpha + \dfrac12$.
	\begin{mathenum}
		\item 根据倍角公式$\sin2\alpha = 2\sin\alpha\cos\alpha$和$\cos2\alpha=\cos^2\alpha -
			\sin^2\alpha$来化简等式的左边得:
		\begin{align*}
			\frac{\sin2\alpha + 1}{1+\cos2\alpha + \sin2\alpha}
			 & = \frac{2\sin\alpha\cos\alpha + 1}{1 + \cos^2\alpha - \sin^2\alpha + 2\sin\alpha\cos\alpha} \\
			 & = \frac{\cos^2\alpha + \sin^2\alpha + 2\sin\alpha\cos\alpha}{\cos^2\alpha + \sin^2\alpha +
			\cos^2\alpha - \sin^2\alpha + 2\sin\alpha\cos\alpha}                                           \\
			 & = \frac{(\sin\alpha + \cos\alpha)^2}{2\cos\alpha(\sin\alpha + \cos\alpha)}                  \\
			 & = \frac{\sin\alpha + \cos\alpha}{2\cos\alpha}                                               \\
			 & = \frac12\tan\alpha + \frac12
		\end{align*}
		\item 综上,等式成立.
	\end{mathenum}

	\question 已知$2\lg{x} + \lg{2} = \lg(x+6)$,求$x$.
	\begin{solution}
		\begin{align*}
			2\lg{x} + \lg2 & = \lg{x^2} + \lg2 \\
			               & = \lg{2x^2}
		\end{align*}
		根据等式则有:
		\begin{equation*}
			2x^2 = x + 6 \tag{a}
		\end{equation*}
		对等式(a)整理得:
		\begin{align*}
			2x^2 - x - 6  & = 0 \\
			(2x + 3)(x-2) & = 0 \\
		\end{align*}
		因为$x>0$所以$x=2$.
	\end{solution}

	\question
	\begin{minipage}[t]{.6\textwidth}
		某生产队要建立一个形状是直角梯形的苗圃,其两邻边借用夹角为$135^\circ$的两面墙,另外两边是总长为$30$米的篱笆(如图,$AD$和$DC$为墙),问篱笆的两边各多长时,苗圃的面积最大?最大面积是多少?
	\end{minipage}
	\hspace{1cm}
	\begin{tikzpicture}[baseline=(current bounding box.north)]
		\tkzDefPoints{0/0/A, 3/0/B}
		\tkzDefPoint(45:2){D}
		\tkzDefLine[parallel=through D](A,B) \tkzGetPoint{x}
		\tkzDefLine[orthogonal=through B](A,B) \tkzGetPoint{y}
		\tkzInterLL(D,x)(B,y) \tkzGetPoint{C}

		\tkzDefLine[orthogonal=through D](A,B) \tkzGetPoint{x}
		\tkzInterLL(D,x)(A,B) \tkzGetPoint{E}

		\tkzDrawPolygon(A,B,C,D)
		\tkzLabelPoints[below](A,B,E)
		\tkzLabelPoints[above](C,D)
		\tkzDrawSegment[blue, dashed](D,E)
	\end{tikzpicture}

	\begin{solution}
		\begin{mathenum}
			\item 从$D$点向$AB$作垂线,垂足为$E$.
			\item 从条件可知$AE=ED=BC$
			\item 设$AE=x$,则有$DC=EB=30-2x$
			\item 梯形的面积可以表示为:
			\begin{align*}
				S & = (30 - 2x + x + 30 - 2x)\cdot x / 2 \\
				  & = (60 - 3x)x/2                       \\
				  & = -\frac32x^2 + 30x \tag{1}
			\end{align*}
			\item 根据抛物线的性质在$x=-\frac{b}{2a}$时有极值,代入可得$x=10$,此时$AB=20,BC=10$
			\item 将$x=10$代入式(1)可得最大面积为$150\mathrm{m}^2$
		\end{mathenum}
	\end{solution}

	\question
	工人师傅要用铁皮做一个上大下小的正四棱台形容器(上面开口),使其容积为$208$立方分米,高为$4$分米,上口边长与下底面边长的比为$5:2$,做这样的容器需要多少平方米的铁皮?(不计容器的厚度和加工余量,不要求写出已知,
	求解,直接求解并画图即可)

	\begin{solution}
		\begin{mathenum}
			\item 设下底面的边长为$2x$,则上底面的边长为$5x$,上下底面的面积分别为$25x^2$和$4x^2$
			\item 容器沿平行于边的中心线剖面是一个等腰梯形,沿着两个斜边作延长线,汇聚于点$E$

			\begin{tikzpicture}[scale=2]
				\tkzDefPoints{-2.5/0/A, 2.5/0/B, -1/-1/D, 1/-1/C, 0/-{5/3}/E, 0/0/O, 0/-1/F}
				\tkzDefLine[orthogonal=through C](A,B) \tkzGetPoint{x}
				\tkzInterLL(C,x)(A,B) \tkzGetPoint{G}

				\tkzDrawPolygon(A,B,C,D)
				\tkzDrawSegments[dashed](D,E C,E)
				\tkzDrawSegment[dim={$4$dm, 6pt,right=6pt}](O,F)
				\tkzDrawSegment[dim={$y$, 6pt, right=6pt}, dashed](F,E)
				\tkzLabelPoint[above](O){$O$}
				\tkzLabelPoint[below](E){$E$}
				\tkzLabelPoint[above left](F){$F$}

				\tkzDrawSegment[dashed](G,C)
				\tkzLabelPoint[above](G){$G$}
				\tkzLabelPoint[right](C){$C$}
				\tkzLabelPoint[right](B){$B$}
			\end{tikzpicture}

			\tdplotsetmaincoords{60}{130}
			\begin{tikzpicture}[tdplot_main_coords]
				\coordinate(A) at (1,1,0);
				\coordinate(B) at (-1,1,0);
				\coordinate(C) at (-1,-1,0);
				\coordinate(D) at (1,-1,0);
				\coordinate(A') at (2.5,2.5,2);
				\coordinate(B') at (-2.5,2.5,2);
				\coordinate(C') at (-2.5,-2.5,2);
				\coordinate(D') at (2.5,-2.5,2);
				\coordinate(E) at (-2.5, 0, 2);
				\coordinate(F) at (2.5, 0, 2);
				\coordinate(G) at (1, 0, 0);
				\coordinate(H) at (-1, 0, 0);
				\coordinate(I) at (-1, 0, 2);

				\draw (D) -- (A) -- (B);
				\draw [dashed](B)-- (C) -- (D);

				\draw (A') -- (B') -- (C') -- (D') -- cycle;
				\draw (A) -- (A');
				\draw (B) -- (B');
				\draw[dashed] (C) -- (C');
				\draw (D) -- (D');
				\draw[dashed](E) -- (F) -- (G) -- (H) -- cycle;
				\draw[dashed](H) -- (I);

				\tkzLabelPoint[above](E){$E$}
				\tkzLabelPoint[above](F){$F$}
				\tkzLabelPoint[below left](G){$G$}
				\tkzLabelPoint[above left](H){$H$}
				\tkzLabelPoint[above](I){$I$}

				% \tkzMarkSegments(C',E E,B')
				% \tkzMarkSegments[mark=||](A',F F,D')
				\tkzMarkRightAngle(H,I,E)

			\end{tikzpicture}
			\item 根据几何比例关系,$EF$的长度$y$有如下等式:
			\begin{equation*}
				\frac{y}{y+4} = \frac25
			\end{equation*}
			解得$y=\frac83$dm.
			\item 则容器的容积为:
			\begin{align*}
				\frac13 \left[ 25x^2 \cdot (4 + \frac83) - 4x^2 \cdot \frac83 \right]
				 & = 51x^2 \\
				 & = 208
			\end{align*}
			计算得 $x = 2$,则有下边长为$4$,上边长为$10$
			\item 在截面上从$B$点作垂线到$AB$,垂足为$G$,则可以求得棱面梯形的高$BC=5$,因此铁皮的面积为
			\begin{align*}
				(4+10)\times 5 \div 2 \times 4  +
				4 \times 4 % 底面积
				= 156 (\text{dm}^2)
			\end{align*}
		\end{mathenum}
	\end{solution}
	\question
	如图,$MN$为圆的直径,$P$、$C$为圆上两点,连$PM,PN$,过$C$作$MN$的垂线与$MN,MP$和$NP$的延长线依次相交于$A,B,D$,求证:$AC^2=AB\cdot AD.$
	\begin{tikzpicture}[baseline=(current bounding box.north), scale=.8]
		\tkzDefPoints{-2/0/M,0/0/O,2/0/N}
		\tkzDefPoint(65:2){P}
		\tkzDefPoint(110:2){C}

		\tkzDefLine[orthogonal=through C](M,N) \tkzGetPoint{x}
		\tkzInterLL(C,x)(M,N) \tkzGetPoint{A}
		\tkzInterLL(C,x)(M,P) \tkzGetPoint{B}
		\tkzInterLL(C,x)(N,P) \tkzGetPoint{D}

		\tkzDrawCircle(O,N)
		\tkzDrawSegments(M,N M,P A,D N,D C,N M,C)
		\tkzLabelPoint(A){$A$}
		\tkzLabelPoint[right](B){$B$}
		\tkzLabelPoint[above left](C){$C$}
		\tkzLabelPoint[above](D){$D$}
		\tkzLabelPoint[left](M){$M$}
		\tkzLabelPoint[right](N){$N$}
		\tkzLabelPoint[above right](P){$P$}
	\end{tikzpicture}
	\begin{solution}
		\begin{align*}
			 & \because \angle{AMP} = \angle{PMA} \text{并且有} \angle{MAB} = \angle{MPN} = 90^\circ \\
			 & \therefore \triangle{AMB} \sim \triangle{PMN}                                      \\
			 & \because \angle{PNM} = \angle{MNP} \text{和} \angle{MPN} = \angle{DAN} = 90^\circ   \\
			 & \therefore \triangle{PMN} \sim \triangle{ADN}                                      \\
			 & \therefore \triangle{AMB} \sim \triangle{ADN}                                      \\
			 & \therefore \frac{AB}{MA} = \frac{AN}{AD}                                           \\
			 & \therefore AB\cdot AD = MA \cdot AN                                                \\
			 & \because \angle{MCA} = \angle{CNA} = 90^\circ - \angle{NMC}                        \\
			 & \text{和} \angle{CAN} = \angle{CAM} = 90^\circ                                      \\
			 & \therefore \triangle{AMC} \sim \triangle{ACN}                                      \\
			 & \therefore \frac{AC}{MA} = \frac{NA}{AC}                                           \\
			 & \therefore AC^2 = MA \cdot CA                                                      \\
			 & \therefore AC^2 = AB \cdot AD
		\end{align*}
	\end{solution}
	\question 下列两题选做一题.

	【甲】已知椭圆短轴长为$2$,中心与抛物线$y^2=4x$的顶点重合,椭圆的一个焦点恰是此抛物线的焦点,求椭圆方程及其长轴的长.
	\begin{solution}
		抛物线的顶点为$(0,0)$,焦点为$(1, 0)$,设椭圆的方程为$\frac{x^2}{a^2} + \frac{y^2}{b^2} =
			1$,则椭圆的半焦距$c=\sqrt{a^2 - b^2} = 1$,其中$b=1$,解得$a=\sqrt{2}$,则长轴为$2\sqrt{2}$,椭圆的方程为:
		\( \frac{x^2}{2} + \frac{y^2}{1} = 1 \)
	\end{solution}

	【乙】已知菱形的一对内角各为$60^\circ$,边长为$4$,以菱形为对角线所在的直线为坐标轴建立直角坐标系,以菱形$60^\circ$角的两个顶点为焦点,并且过菱形的另外两个顶点作椭圆,求椭圆方程.
	\begin{figure}[htbp]
		\centering
		\begin{tikzpicture}
			\tkzInit[xmax=5,xmin=-5, ymax=5, ymin=-5]
			\tkzDrawX\tkzDrawY
			\tkzDefPoints{2/0/A,-2/0/C, 0/{2sqrt(3)}/B, 0/-{2sqrt(3)}/D, 0/0/O}

			\tkzDrawPolygon(A,B,C,D)
			\tkzMarkSegment[dim={$4$, 16pt, right=6pt}, mark=](B,A)
			\tkzMarkSegment[dim={$2$, -16pt, below=6pt}, mark=](O,A)
			\tkzMarkSegment[dim={$2\sqrt{3}$, 16pt, left=6pt}, mark=](O,B)

			\draw[x radius=4, y radius=2*sqrt(3)] ellipse;
		\end{tikzpicture}
	\end{figure}

	根据几何关系可知椭圆的短半轴为$b=2\sqrt{3}$,半焦距为$c=2$,长半轴$a=4$,所以椭圆的方程为$\frac{x^2}{16} +
		\frac{y^2}{12} = 1$
	\begin{center}
		\subsection*{附加题}
	\end{center}
	\question 将函数$f(x)=e^x$展开为$x$的幂级数,并求出收敛区间.($e=2.718$为自然对数的底数)
	\begin{solution}
		$e^x$的幂级数展开式为$\displaystyle\sum_{n=0}^{\infty}\frac{x^n}{n!}$,使用比值判别法:
		\begin{equation*}
			\lim_{n\to\infty}\left| \frac{\frac{x^{n+1}}{(n+1)!}}{\frac{x^n}{n!}}\right| =
			\lim_{n\to\infty}\left|\frac{x}{n+1}\right|
			= 0
		\end{equation*}
		所以收敛区间为$(-\infty, \infty)$.
	\end{solution}

	\question 利用定积分计算椭圆$\frac{x^2}{a^2} + \frac{y^2}{b^2} = 1 (a > b > 0)$所围成的面积.
	\begin{solution}
		对于任意一处的$x$对应的$y=\pm b\sqrt{1 -
				\frac{x^2}{a^2}}$,则在此处的形成的宽度为$\mathrm{d}x$,高度为$2|y|$的微小矩形的面积表达式为:$2b\sqrt{1-\frac{x^2}{a^2}}\mathrm{d}x$,则椭圆的面积为
		\begin{equation*}
			A = \int_{-a}^{a}2b\sqrt{1-\frac{x^2}{a^2}}\mathrm{d}x = 4b\int_{0}^{a}\sqrt{1-\frac{x^2}{a^2}}\mathrm{d}x
		\end{equation*}
		令$x=a\sin\theta$,则$\mathrm{d}x = a\cos\theta\mathrm{d}\theta$,其中$\theta \in [0, \frac\pi2]$.
		代入得:
		\begin{align*}
			A = 4b\int_{0}^{a}\sqrt{1-\frac{x^2}{a^2}}\mathrm{d}x
			 & = 4b\int_0^{\frac\pi2}\sqrt{1 -
			\frac{a^2\sin^2\theta}{a^2}}a\cos\theta\mathrm{d}\theta                              \\
			 & = 4ab\int_0^{\frac\pi2}\cos^2\theta\mathrm{d}\theta                               \\
			 & = 4ab\int_0^{\frac\pi2}\frac{1+\cos(2\theta)}{2}\mathrm{d}\theta                  \\
			 & = 4ab\left(\int_0^{\frac\pi2}\frac12\mathrm{d}\theta +
			\int_0^{\frac\pi2}\frac{\cos(2\theta)}{2}\mathrm{d}\theta\right)                     \\
			 & = 4ab\left(\frac\pi4  + \left[\frac{\sin(2\theta)}{4}\right]_0^{\frac\pi2}\right) \\
			 & = \pi ab
		\end{align*}
	\end{solution}

\end{questions}
