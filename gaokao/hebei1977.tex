\section[1977年高考数学试卷及答案(河北卷)理科]{1977年普通高等学校招生考试(河北卷)\\\Huge{理科数学}}

\begin{questions}
	\question 解答下列各题
	\begin{parts}
		\part[10] 叙述函数的定义
			\begin{solution}
				函数是从一个集合到另一个集合之间的映射,使得每个输入值有且仅有一个输出值.
			\end{solution}
		\part[10] 求函数$y=1-\frac{1}{\sqrt{2-3x}}$的定义域.
			\begin{solution}
				由函数的定义有:
				\begin{math}
					2-3x > 0
				\end{math},则函数的定义域为:$\{x|x<\frac23\}$.
			\end{solution}
		\part[10] 计算:$\left[1-(0.5)^{-2}\right] \div \left(-\frac{27}{8}\right)^{\frac13}$.
			\begin{solution}
				\begin{align*}
					\left[1-(0.5)^{-2}\right] \div \left(-\frac{27}{8}\right)^{\frac13} & = \left(1-2^{-1\times (-2)}\right)
					\div \left(-\frac32\right)^{3\times\frac13}                                                              \\
					                                                                    & = (1 - 4) \cdot (-\frac23)         \\
					                                                                    & = 2
				\end{align*}
			\end{solution}
		\part[10] 计算:$\log_42$.
			\begin{solution}
				\begin{align*}
					\log_42 & = \log_44^\frac12 \\
					        & = \frac12
				\end{align*}
			\end{solution}
		\part[10] 分解因式:$x^2y - 2y^3$.
			\begin{solution}
				\begin{align*}
					x^2y - 2y^3 & = y(x^2 - 2y^2)               \\
					            & = y(x+\sqrt{2}y)(x-\sqrt{2}y)
				\end{align*}
			\end{solution}
		\part[10] 计算:$\sin\dfrac{4\pi}{3}\cdot\cos\dfrac{25\pi}{6}\cdot\tan\left(-\dfrac{3\pi}{4}\right)$.
			\begin{solution}
				\begin{align*}
					\sin\frac{4\pi}{3}\cdot\cos\frac{25\pi}{6}\cdot\tan\left(-\frac{3\pi}{4}\right)
					 & = -\sin\frac{\pi}{3}\cdot\cos\frac{\pi}{6}\cdot \tan\frac{3\pi}{4} \\
					 & = -\frac{\sqrt{3}}{2} \cdot \frac{\sqrt{3}}{2} \cdot 1             \\
					 & = - \frac34
				\end{align*}
			\end{solution}
	\end{parts}

	\question[10]
	\begin{minipage}[t]{.3\textwidth}
		证明:如图,$AB$是圆$O$的直径,$CB$是圆$O$的切线,切点为$B$,$OC$平行于弦$AD$,求证:$DC$是圆$O$的切线.
	\end{minipage}\hspace{3cm}
	\begin{tikzpicture}[scale=.5, baseline=(current bounding box.center)]
		\tkzDefPoints{-2/0/A, 0/0/O, 2/0/B, 2/5/C}
		\tkzDefLine[parallel=through A](O,C) \tkzGetPoint{x}
		\tkzInterLC(A,x)(O,B) \tkzGetPoints{D}{A}

		\tkzDrawCircle(O,B)
		\tkzDrawSegment(A,B)
		\tkzDrawSegment(C,B)
		\tkzDrawSegment(C,O)
		\tkzDrawSegment(A,D)
		\tkzDrawSegment(C,D)

		\tkzLabelPoint[left](A){$A$}
		\tkzLabelPoint[right](B){$B$}
		\tkzLabelPoint[above](C){$C$}
		\tkzLabelPoint[below](O){$O$}
		\tkzLabelPoint[left](D){$D$}
	\end{tikzpicture}

	\begin{proofsolution}
		\begin{center}
			\begin{tikzpicture}[scale=.5, baseline=(current bounding box.center)]
				\tkzDefPoints{-2/0/A, 0/0/O, 2/0/B, 2/5/C}
				\tkzDefLine[parallel=through A](O,C) \tkzGetPoint{x}
				\tkzInterLC(A,x)(O,B) \tkzGetPoints{D}{A}

				\tkzDrawCircle(O,B)
				\tkzDrawSegment(A,B)
				\tkzDrawSegment(C,B)
				\tkzDrawSegment(C,O)
				\tkzDrawSegment(A,D)
				\tkzDrawSegment(C,D)
				\tkzDrawSegment[dashed, red](D,O)

				\tkzMarkAngle[mark=|, size=15pt](O,A,D)
				\tkzMarkAngle[mark=|, size=15pt](B,O,C)
				\tkzMarkAngle[mark=||, size=20pt](A,D,O)
				\tkzMarkAngle[mark=||, size=20pt](C,O,D)

				\tkzLabelPoint[left](A){$A$}
				\tkzLabelPoint[right](B){$B$}
				\tkzLabelPoint[above](C){$C$}
				\tkzLabelPoint[below](O){$O$}
				\tkzLabelPoint[left](D){$D$}
			\end{tikzpicture}
		\end{center}
		\begin{enumerate}[label=\protect\circled{\arabic*}]
			\item 因为 $CB$ 是圆 $O$ 的切线且切点为 $B$,所以有 $AB \perp CB$.
			\item 由于 $AD \parallel OC$,可以得到 $\angle DAO = \angle COB$.
			\item 作辅助线 $DO$.因为 $\angle ADO$ 和 $\angle COD$ 是平行线的内错角,所以 $\angle ADO = \angle COD$.
			\item 在 $\triangle AOD$ 中,由于 $AD \parallel OC$,且 $OA = OD$(都是圆的半径),所以 $\triangle AOD$ 是等腰三角形,由此得到 $\angle OAD = \angle ODA$.
			\item 进一步分析 $\triangle DOC$ 和 $\triangle BOC$:
			      \begin{itemize}
				      \item $OB = OD$(都是圆的半径);
				      \item $OC = OC$(公共边);
				      \item $\angle BOC = \angle DOC$(由 $\angle COB = \angle DAO$ 和 $\angle ADO = \angle COD$ 推导).
			      \end{itemize}
			      因此,$\triangle BOC \cong \triangle DOC$(根据边角边准则).
			\item 由 $\triangle BOC \cong \triangle DOC$,可以得到 $\angle ODC = \angle CBO = 90^\circ$.
			\item 因为 $\angle ODC = 90^\circ$,所以 $DC$ 是圆 $O$ 的切线.
		\end{enumerate}
	\end{proofsolution}
	\question[10] 证明: $\dfrac{\sin2\alpha + 1}{1+\cos2\alpha + \sin2\alpha} = \dfrac12\tan\alpha + \dfrac12$.
	\begin{proofsolution}
		\begin{cenum}
			\item 根据倍角公式$\sin2\alpha = 2\sin\alpha\cos\alpha$和$\cos2\alpha=\cos^2\alpha -
				      \sin^2\alpha$来化简等式的左边得:
			      \begin{align*}
				      \frac{\sin2\alpha + 1}{1+\cos2\alpha + \sin2\alpha}
				       & = \frac{2\sin\alpha\cos\alpha + 1}{1 + \cos^2\alpha - \sin^2\alpha + 2\sin\alpha\cos\alpha} \\
				       & = \frac{\cos^2\alpha + \sin^2\alpha + 2\sin\alpha\cos\alpha}{\cos^2\alpha + \sin^2\alpha +
				      \cos^2\alpha - \sin^2\alpha + 2\sin\alpha\cos\alpha}                                           \\
				       & = \frac{(\sin\alpha + \cos\alpha)^2}{2\cos\alpha(\sin\alpha + \cos\alpha)}                  \\
				       & = \frac{\sin\alpha + \cos\alpha}{2\cos\alpha}                                               \\
				       & = \frac12\tan\alpha + \frac12
			      \end{align*}
			\item 综上,等式成立.
		\end{cenum}
	\end{proofsolution}

	\question[10] 已知$2\lg{x} + \lg{2} = \lg(x+6)$,求$x$.
	\begin{solution}
		\begin{align*}
			2\lg{x} + \lg2 & = \lg{x^2} + \lg2 \\
			               & = \lg{2x^2}
		\end{align*}
		根据等式则有:
		\begin{equation*}
			2x^2 = x + 6 \tag{a}
		\end{equation*}
		对式(a)整理得:
		\begin{align*}
			2x^2 - x - 6  & = 0 \\
			(2x + 3)(x-2) & = 0 \\
		\end{align*}
		因为$x>0$所以$x=2$.
	\end{solution}

	\question[10]
	\begin{minipage}[t]{.6\textwidth}
		某生产队要建立一个形状是直角梯形的苗圃,其两邻边借用夹角为$135^\circ$的两面墙,另外两边是总长为$30$米的篱笆(如图,$AD$和$DC$为墙),问篱笆的两边各多长时,苗圃的面积最大?最大面积是多少?
	\end{minipage}
	\hspace{1cm}
	\begin{tikzpicture}[baseline=(current bounding box.north)]
		\tkzDefPoints{0/0/A, 3/0/B}
		\tkzDefPoint(45:2){D}
		\tkzDefLine[parallel=through D](A,B) \tkzGetPoint{x}
		\tkzDefLine[orthogonal=through B](A,B) \tkzGetPoint{y}
		\tkzInterLL(D,x)(B,y) \tkzGetPoint{C}

		\tkzDrawPolygon(A,B,C,D)
		\tkzLabelPoints[below](A,B)
		\tkzLabelPoints[above](C,D)
	\end{tikzpicture}

	\begin{solution}
		\begin{center}
			\begin{tikzpicture}[baseline=(current bounding box.north)]
				\tkzDefPoints{0/0/A, 3/0/B}
				\tkzDefPoint(45:2){D}
				\tkzDefLine[parallel=through D](A,B) \tkzGetPoint{x}
				\tkzDefLine[orthogonal=through B](A,B) \tkzGetPoint{y}
				\tkzInterLL(D,x)(B,y) \tkzGetPoint{C}

				\tkzDefLine[orthogonal=through D](A,B) \tkzGetPoint{x}
				\tkzInterLL(D,x)(A,B) \tkzGetPoint{E}

				\tkzDrawPolygon(A,B,C,D)
				\tkzLabelPoints[below](A,B,E)
				\tkzLabelPoints[above](C,D)
				\tkzDrawSegment[blue, dashed](D,E)
			\end{tikzpicture}
		\end{center}

		\begin{cenum}
			\item 从$D$点向$AB$作垂线,垂足为$E$.
			\item 从条件可知$AE=ED=BC$
			\item 设$AE=x$,则有$DC=EB=30-2x$
			\item 梯形的面积可以表示为:
			      \begin{align*}
				      S & = (30 - 2x + x + 30 - 2x)\cdot x / 2 \\
				        & = (60 - 3x)x/2                       \\
				        & = -\frac32x^2 + 30x \tag{1}
			      \end{align*}
			\item 根据抛物线的性质在$x=-\frac{b}{2a}$时有极值,代入可得$x=10$,此时$AB=20,BC=10$
			\item 将$x=10$代入式(1)可得最大面积为\qty{150}{\meter\squared}.
		\end{cenum}
	\end{solution}

	\question[10]
	工人师傅要用铁皮做一个上大下小的正四棱台形容器(上面开口),使其容积为$208$立方分米,高为$4$分米,上口边长与下底面边长的比为$5:2$,做这样的容器需要多少平方米的铁皮?(不计容器的厚度和加工余量,不要求写出已知,
	求解,直接求解并画图即可)

	\begin{solution}
		\begin{cenum}
			\item 设下底面的边长为$2x$,则上底面的边长为$5x$,上下底面的面积分别为$25x^2$和$4x^2$;
			\item 根据棱台体积公式:
			      \begin{equation*}
				      V = \frac13h(S_1 + S_2 + \sqrt{S_1S_2})
			      \end{equation*}
			      代入可得:
			      \begin{align*}
				      V & = \frac13\cdot4(25x^2 + 4x^2 + 10x^2) \\
				        & = 52x^2                               \\
				        & = 208
			      \end{align*}
			      解得:
			      \begin{equation*}
				      x = 2
			      \end{equation*}
			      则有上边长为\qty{10}{\dm},底边长为\qty{4}{\dm}.
			\item 容器沿平行于边的中心线剖面如下图所示:
			      \tdplotsetmaincoords{60}{130}
			      \begin{center}
				      \begin{tikzpicture}[tdplot_main_coords]
					      \coordinate(A) at (1,1,0);
					      \coordinate(B) at (-1,1,0);
					      \coordinate(C) at (-1,-1,0);
					      \coordinate(D) at (1,-1,0);
					      \coordinate(A') at (2.5,2.5,2);
					      \coordinate(B') at (-2.5,2.5,2);
					      \coordinate(C') at (-2.5,-2.5,2);
					      \coordinate(D') at (2.5,-2.5,2);
					      \coordinate(E) at (-2.5, 0, 2);
					      \coordinate(F) at (2.5, 0, 2);
					      \coordinate(G) at (1, 0, 0);
					      \coordinate(H) at (-1, 0, 0);
					      \coordinate(I) at (-1, 0, 2);

					      \draw (D) -- (A) -- (B);
					      \draw [dashed](B)-- (C) -- (D);

					      \draw (A') -- (B') -- (C') -- (D') -- cycle;
					      \draw (A) -- (A');
					      \draw (B) -- (B');
					      \draw[dashed] (C) -- (C');
					      \draw (D) -- (D');
					      \draw[dashed](E) -- (F) -- (G) -- (H) -- cycle;
					      \draw[dashed](H) -- (I);

					      \tkzLabelPoint[above](E){$E$}
					      \tkzLabelPoint[above](F){$F$}
					      \tkzLabelPoint[below left](G){$G$}
					      \tkzLabelPoint[above left](H){$H$}
					      \tkzLabelPoint[above](I){$I$}

					      % \tkzMarkSegments(C',E E,B')
					      % \tkzMarkSegments[mark=||](A',F F,D')
					      \tkzMarkRightAngle(H,I,E)

				      \end{tikzpicture}
			      \end{center}

			      可得$HE=\sqrt{EI^2 + IH^2}=\qty{5}{\dm}$.

			\item 因此铁皮的面积为
			      \begin{align*}
				      (4+10)\times 5 \div 2 \times 4  +
				      4 \times 4 % 底面积
				      = \qty{156}{\dm\squared}
			      \end{align*}
		\end{cenum}
	\end{solution}
	\question[10]
	如图,$MN$为圆的直径,$P$、$C$为圆上两点,连$PM,PN$,过$C$作$MN$的垂线与$MN,MP$和$NP$的延长线依次相交于$A,B,D$,求证:$AC^2=AB\cdot AD.$

	\begin{center}
		\begin{tikzpicture}[baseline=(current bounding box.north), scale=.8]
			\tkzDefPoints{-2/0/M,0/0/O,2/0/N}
			\tkzDefPoint(65:2){P}
			\tkzDefPoint(110:2){C}

			\tkzDefLine[orthogonal=through C](M,N) \tkzGetPoint{x}
			\tkzInterLL(C,x)(M,N) \tkzGetPoint{A}
			\tkzInterLL(C,x)(M,P) \tkzGetPoint{B}
			\tkzInterLL(C,x)(N,P) \tkzGetPoint{D}

			\tkzDrawCircle(O,N)
			\tkzDrawSegments(M,N M,P A,D N,D C,N M,C)
			\tkzLabelPoint(A){$A$}
			\tkzLabelPoint[right](B){$B$}
			\tkzLabelPoint[above left](C){$C$}
			\tkzLabelPoint[above](D){$D$}
			\tkzLabelPoint[left](M){$M$}
			\tkzLabelPoint[right](N){$N$}
			\tkzLabelPoint[above right](P){$P$}
		\end{tikzpicture}
	\end{center}

	\begin{solution}
		\begin{align*}
			 & \because \angle{AMP} = \angle{PMA} \text{并且有} \angle{MAB} = \angle{MPN} = \ang{90}                     \\
			 & \therefore \triangle{AMB} \sim \triangle{PMN}                                                          \\
			 & \because \angle{PNM} = \angle{MNP} \text{和} \angle{MPN} = \angle{DAN} = \ang{90}                       \\
			 & \therefore \triangle{PMN} \sim \triangle{ADN}                                                          \\
			 & \therefore \triangle{AMB} \sim \triangle{ADN}                                                          \\
			 & \therefore \frac{AB}{MA} = \frac{AN}{AD}                                                               \\
			 & \therefore AB\cdot AD = MA \cdot AN                                                                    \\
			 & \because \angle{MCA} = \angle{CNA} = \ang{90} - \angle{NMC} \land \angle{CAN} = \angle{CAM} = \ang{90}
		\end{align*}

		\begin{align*}
			 & \therefore \triangle{AMC} \sim \triangle{ACN} \\
			 & \therefore \frac{AC}{MA} = \frac{NA}{AC}      \\
			 & \therefore AC^2 = MA \cdot CA                 \\
			 & \therefore AC^2 = AB \cdot AD
		\end{align*}
	\end{solution}
	\question[10] 下列两题选做一题.

	【甲】已知椭圆短轴长为$2$,中心与抛物线$y^2=4x$的顶点重合,椭圆的一个焦点恰是此抛物线的焦点,求椭圆方程及其长轴的长.
	\begin{solution}
		\begin{center}
			\begin{tikzpicture}
				\begin{axis}[xmin=-4,
						xmax=4,
						ymin=-3,
						ymax=3
					]
					\addplot[domain=0:3]{sqrt(4*x)};
					\addplot[domain=0:3]{-sqrt(4*x)};
					\filldraw(1,0) circle(1pt);
					\addplot[domain=0:2*pi]({sqrt(2)*cos(deg(x))}, {sin(deg(x))});
				\end{axis}
			\end{tikzpicture}
		\end{center}

		\begin{cenum}
			\item 抛物线的顶点为$(0,0)$;
			\item 抛物线$y^2=4px$的焦点为$(p,0)$,则有焦点的坐标为$(1,0)$;
			\item 设椭圆的方程为:
			      \begin{equation*}
				      \frac{x^2}{a^2} + \frac{y^2}{b^2} = 1
			      \end{equation*}
			      则有半焦距$c=\sqrt{a^2-b^2}$,因为短轴长为$2$,所以$b=1$.计算得$a=\sqrt{2}$,即长轴的长为$2\sqrt{2}$.
			      椭圆方程为
			      \begin{equation*}
				      \frac{x^2}{2} + \frac{y^2}{1} = 1
			      \end{equation*}
		\end{cenum}
	\end{solution}

	【乙】已知菱形的一对内角各为$60^\circ$,边长为$4$,以菱形为对角线所在的直线为坐标轴建立直角坐标系,以菱形$60^\circ$角的两个顶点为焦点,并且过菱形的另外两个顶点作椭圆,求椭圆方程.

	\begin{solution}
		\begin{center}
			\begin{tikzpicture}
				\tkzInit[xmax=5,xmin=-5, ymax=2.5, ymin=-2.5]
				\tkzDrawX\tkzDrawY
				\tkzDefPoints{{2sqrt(3)}/0/A,{-2sqrt(3)}/0/C, 0/2/B, 0/-2/D, 0/0/O}

				\tkzDrawPolygon(A,B,C,D)
				\tkzMarkSegment[dim={$4$, 16pt, right=6pt}, mark=](B,A)
				\tkzMarkSegment[dim={$2\sqrt{3}$, -16pt, below=6pt}, mark=](O,A)
				\tkzMarkSegment[dim={$2$, 16pt, left=6pt}, mark=](O,B)

				\draw[x radius=4, y radius=2] ellipse;
			\end{tikzpicture}
		\end{center}
		设椭圆的方程为:
		\begin{equation*}
			\frac{x^2}{a^2} + \frac{y^2}{b^2} = 1
		\end{equation*}
		则有:
		\begin{align*}
			b = 2
		\end{align*}
		半焦距为:
		\begin{align*}
			c = 2\sqrt{3}
		\end{align*}
		长半轴为
		\begin{align*}
			a & = \sqrt{b^2 + c^2} \\
			  & = 4
		\end{align*}
		故椭圆的方程为
		\begin{equation*}
			\frac{x^2}{16} + \frac{y^2}{4} = 1
		\end{equation*}
	\end{solution}

	\begin{center}
		\large\bf 附加题
	\end{center}
	\question 将函数$f(x)=e^x$展开为$x$的幂级数,并求出收敛区间.($e=2.718$为自然对数的底数)
	\begin{solution}
		$e^x$的幂级数展开式为$\displaystyle\sum_{n=0}^{\infty}\frac{x^n}{n!}$,使用比值判别法:
		\begin{equation*}
			\lim_{n\to\infty}\left| \frac{\frac{x^{n+1}}{(n+1)!}}{\frac{x^n}{n!}}\right| =
			\lim_{n\to\infty}\left|\frac{x}{n+1}\right|
			= 0
		\end{equation*}
		所以收敛区间为$(-\infty, \infty)$.
	\end{solution}

	\question 利用定积分计算椭圆$\frac{x^2}{a^2} + \frac{y^2}{b^2} = 1 (a > b > 0)$所围成的面积.
	\begin{solution}
		对于任意一处的$x$对应的$y=\pm b\sqrt{1 -
				\frac{x^2}{a^2}}$,则在此处的形成的宽度为$\mathrm{d}x$,高度为$2|y|$的微小矩形的面积表达式为:$2b\sqrt{1-\frac{x^2}{a^2}}\mathrm{d}x$,则椭圆的面积为
		\begin{equation*}
			A = \int_{-a}^{a}2b\sqrt{1-\frac{x^2}{a^2}}\mathrm{d}x = 4b\int_{0}^{a}\sqrt{1-\frac{x^2}{a^2}}\mathrm{d}x
		\end{equation*}
		令$x=a\sin\theta$,则$\mathrm{d}x = a\cos\theta\mathrm{d}\theta$,其中$\theta \in [0, \frac\pi2]$.
		代入得:
		\begin{align*}
			A = 4b\int_{0}^{a}\sqrt{1-\frac{x^2}{a^2}}\mathrm{d}x
			 & = 4b\int_0^{\frac\pi2}\sqrt{1 -
			\frac{a^2\sin^2\theta}{a^2}}a\cos\theta\mathrm{d}\theta                              \\
			 & = 4ab\int_0^{\frac\pi2}\cos^2\theta\mathrm{d}\theta                               \\
			 & = 4ab\int_0^{\frac\pi2}\frac{1+\cos(2\theta)}{2}\mathrm{d}\theta                  \\
			 & = 4ab\left(\int_0^{\frac\pi2}\frac12\mathrm{d}\theta +
			\int_0^{\frac\pi2}\frac{\cos(2\theta)}{2}\mathrm{d}\theta\right)                     \\
			 & = 4ab\left(\frac\pi4  + \left[\frac{\sin(2\theta)}{4}\right]_0^{\frac\pi2}\right) \\
			 & = \pi ab
		\end{align*}
	\end{solution}

\end{questions}
