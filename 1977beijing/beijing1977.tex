%!tex program = lualatex
\documentclass[answers]{exam}
\usepackage{ctex}
\usepackage{graphicx}
\usepackage[margin=2cm]{geometry}
\usepackage{amsmath, amssymb}
\usepackage{csquotes}
\usepackage{tikz, pgfplots}
\usetikzlibrary{
	angles,
	backgrounds,
	calc,
	decorations.pathmorphing,
	decorations.pathreplacing,
	decorations.text,
	intersections,
	patterns,
	quotes,
	shapes,
	shapes.symbols,
}
\pagestyle{empty}
\newcounter{xcord}
\newcounter{ycord}
\newcounter{total}
\renewcommand{\labelenumi}{\textbf{\ifnum\value{enumi}<10 0\fi\arabic{enumi})}}

\pgfplotsset{compat=1.18}

\CorrectChoiceEmphasis{\color{blue!70!green}\bfseries}
\renewcommand{\solutiontitle}{\textbf{解:}}

\usepackage{array, tabularx}
\newcolumntype{C}{>{\centering\arraybackslash}X}
\newcolumntype{B}{>{\centering\bfseries\arraybackslash}X}
\catcode`\幺=0

\begin{document}
\begin{center}
	\textbf{\large{1977 年普通高等学校招生考试(北京卷) }}

	\textbf{\LARGE{理科数学}}
\end{center}
\begin{questions}
	\question 解方程: \( \sqrt{x - 1} = 3 - x \).
	\begin{solution}
		\begin{align*}
			\sqrt{x - 1}  & = 3 - x        \\
			x - 1         & = 9 - 6x + x^2 \\
			x^2 - 7x + 10 & = 0            \\
			(x-2)(x-5)    & = 0            \\
			x_1 = 2, x_2 = 5
		\end{align*}
	\end{solution}

	\question 计算: \( 2^{-\frac12} + \frac{2^0}{\sqrt{2}} + \frac{1}{\sqrt{2} - 1}. \)

	\begin{solution}
		\begin{align*}
			\text{原式} & = \frac{1}{\sqrt{2}} + \frac{1}{\sqrt{2}} + \sqrt{2} + 1 \\
			          & = \frac{2}{\sqrt{2}} + \sqrt{2} + 1                      \\
			          & = 2\sqrt{2} + 1
		\end{align*}
	\end{solution}

	\question 已知 \( \lg2 = 0.3010, \lg3 = 0.4771, \) 求 \( \lg\sqrt{45} \).

	\begin{solution}
		\begin{align*}
			\lg\sqrt{45} & = \frac12\lg{45}                        \\
			             & = \frac12(\lg5 + \lg9)                  \\
			             & = \frac12(\lg5 + 2\lg3)                 \\
			             & = \frac12(\lg\frac{10}{2} + 2\lg3)      \\
			             & = \frac12(\lg10 - \lg2 + 2\lg3)         \\
			             & = \frac12(1 - 0.3010 + 2 \times 0.4771) \\
			             & = 0.8266
		\end{align*}
	\end{solution}

\end{questions}

\end{document}
