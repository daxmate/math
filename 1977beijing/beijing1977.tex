%!tex program = lualatex
\documentclass[answers]{exam}
\usepackage{ctex}
\usepackage{graphicx}
\usepackage[margin=2cm]{geometry}
\usepackage{amsmath, amssymb}
\usepackage{csquotes}
\usepackage{tikz, pgfplots}
\usetikzlibrary{
	angles,
	backgrounds,
	calc,
	decorations.pathmorphing,
	decorations.pathreplacing,
	decorations.text,
	intersections,
	patterns,
	quotes,
	shapes,
	shapes.symbols,
}
\pagestyle{empty}
\newcounter{xcord}
\newcounter{ycord}
\newcounter{total}
\renewcommand{\labelenumi}{\textbf{\ifnum\value{enumi}<10 0\fi\arabic{enumi})}}

\pgfplotsset{compat=1.18}

\CorrectChoiceEmphasis{\color{blue!70!green}\bfseries}
\renewcommand{\solutiontitle}{\textbf{解:}}

\usepackage{array, tabularx}
\newcolumntype{C}{>{\centering\arraybackslash}X}
\newcolumntype{B}{>{\centering\bfseries\arraybackslash}X}
\catcode`\幺=0

\begin{document}
\begin{center}
	\textbf{\large{1977 年普通高等学校招生考试(北京卷) }}

	\textbf{\LARGE{理科数学}}
\end{center}
\begin{questions}
	\question 解方程: \( \sqrt{x - 1} = 3 - x \).
	\begin{solution}
		\begin{align*}
			\sqrt{x - 1}  & = 3 - x        \\
			x - 1         & = 9 - 6x + x^2 \\
			x^2 - 7x + 10 & = 0            \\
			(x-2)(x-5)    & = 0            \\
			x_1 = 2, x_2 = 5
		\end{align*}
	\end{solution}

	\question 计算: \( 2^{-\frac12} + \frac{2^0}{\sqrt{2}} + \frac{1}{\sqrt{2} - 1}. \)

	\begin{solution}
		\begin{align*}
			\text{原式} & = \frac{1}{\sqrt{2}} + \frac{1}{\sqrt{2}} + \sqrt{2} + 1 \\
			            & = \frac{2}{\sqrt{2}} + \sqrt{2} + 1                      \\
			            & = 2\sqrt{2} + 1
		\end{align*}
	\end{solution}

	\question 已知 \( \lg2 = 0.3010, \lg3 = 0.4771, \) 求 \( \lg\sqrt{45} \).

	\begin{solution}
		\begin{align*}
			\lg\sqrt{45} & = \frac12\lg{45}                        \\
			             & = \frac12(\lg5 + \lg9)                  \\
			             & = \frac12(\lg5 + 2\lg3)                 \\
			             & = \frac12(\lg\frac{10}{2} + 2\lg3)      \\
			             & = \frac12(\lg10 - \lg2 + 2\lg3)         \\
			             & = \frac12(1 - 0.3010 + 2 \times 0.4771) \\
			             & = 0.8266
		\end{align*}
	\end{solution}

	\question 证明: \( (1 + \tan\alpha)^2 = \dfrac{1 + \sin2\alpha}{\cos^2\alpha} \)。
	\begin{solution}
		\begin{align*}
			(1+\tan\alpha)^2
			           & = \left(1 + \frac{\sin\alpha}{\cos\alpha}\right)^2  \text{(将 \(\tan\alpha = \frac{\sin\alpha}{\cos\alpha}\) 代入)} \\
			           & = \left(\frac{\cos\alpha + \sin\alpha}{\cos\alpha}\right)^2 \text{(通分并化简分母)}                                 \\
			           & = \frac{(\cos\alpha + \sin\alpha)^2}{\cos^2\alpha}                                                                    \\
			           & = \frac{\cos^2\alpha + 2\cos\alpha\sin\alpha + \sin^2\alpha}{\cos^2\alpha}                                            \\
			           & = \frac{1 + 2\cos\alpha\sin\alpha}{\cos^2\alpha}                                                                      \\
			           & \text{(根据三角恒等式 \(\cos^2\alpha + \sin^2\alpha = 1\) 化简)}                                                    \\
			           & = \frac{1 + \sin2\alpha}{\cos^2\alpha}                                                                                \\
			           & \text{(利用 \(\sin2\alpha = 2\cos\alpha\sin\alpha\))}                                                               \\
			\therefore & \, (1+\tan\alpha)^2 = \frac{1 + \sin2\alpha}{\cos^2\alpha}。
		\end{align*}
	\end{solution}

	\question 求过两直线 \( x + y - 7 = 0 \) 和 \( 3x - y - 1 = 0 \) 的交点且过 \( (1, 1) \) 点的直线方程。
	\begin{solution}
		\begin{align*}
			 & \begin{cases}
				   x + y - 7 = 0, \\
				   3x - y - 1 = 0
			   \end{cases}                                                                          \\
			 & \text{解得:} x = 2, \, y = 5,\text{即交点为 } (2, 5)。                                \\
			 & \text{过点 } (2,5) \text{ 和 } (1,1) \text{ 的直线斜率:} k = \frac{5 - 1}{2 - 1} = 4。 \\
			 & \text{设直线方程为:} y = 4x + a,\text{将 } (1,1) \text{ 代入,得:}                   \\
			 & a = -3。                                                                                \\
			 & \text{因此,所求直线方程为:} y = 4x - 3。
		\end{align*}
	\end{solution}

\end{questions}

\end{document}
