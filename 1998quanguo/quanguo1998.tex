%!tex program = lualatex
\documentclass[answers]{exam}
\usepackage{ctex}
\usepackage{graphicx}
\usepackage[margin=2cm]{geometry}
\usepackage{amsmath, amssymb}
\usepackage{csquotes}
\usepackage{tikz, pgfplots}
\usetikzlibrary{
	angles,
	backgrounds,
	calc,
	decorations.pathmorphing,
	decorations.pathreplacing,
	decorations.text,
	intersections,
	patterns,
	quotes,
	shapes,
	shapes.symbols,
}
\pagestyle{empty}
\newcounter{xcord}
\newcounter{ycord}
\newcounter{total}
\renewcommand{\labelenumi}{\textbf{\ifnum\value{enumi}<10 0\fi\arabic{enumi})}}

\pgfplotsset{compat=1.18}

\CorrectChoiceEmphasis{\color{blue!70!green}\bfseries}
\renewcommand{\solutiontitle}{\textbf{解:}}

\usepackage{array, tabularx}
\newcolumntype{C}{>{\centering\arraybackslash}X}
\newcolumntype{B}{>{\centering\bfseries\arraybackslash}X}
\catcode`\幺=0

\usepackage[lua]{tkz-euclide}

\renewcommand{\choicelabel}{(\Alph{choice})}
\begin{document}
\begin{center}
	\textbf{1998年普通高等学校招生考试(全国卷)}\\
	\textbf{\Large 理科数学}
\end{center}

\begin{questions}
	\question $\sin\ang{600}$的值是 \hfill (\hspace{1cm})

	\begin{oneparchoices}
		\choice $\frac12$
		\choice $-\frac12$
		\choice $\frac{\sqrt{3}}{2}$
		\CorrectChoice $-\frac{\sqrt{3}}{2}$
	\end{oneparchoices}

	\begin{solution}
		\begin{align*}
			\sin\ang{600} = \sin\ang{240} = -\frac{\sqrt{3}}{2}
		\end{align*}
	\end{solution}

	\question 函数$y=a^{|x|}(a>1)$的图像是\hfill (\hspace{1cm})

	\begin{oneparchoices}
		\choice
		\begin{tikzpicture}[scale=0.45]
			\begin{axis}[
					axis lines=middle,
					xlabel = $x$,
					ylabel = $y$,
					domain = -3:3,
					samples = 100,
					ymin = -1,
					ymax = 5,
					ticks = none
				]
				\addplot[thick, domain=0:3]{2^x - 1};
				\addplot[thick, domain=-3:0]{2^(-x) - 1};
				\node[below left] at(0,0) {$O$};
			\end{axis}
		\end{tikzpicture}
		\CorrectChoice
		\begin{tikzpicture}[scale=0.45]
			\begin{axis}[
					axis lines=middle,
					xlabel = $x$,
					ylabel = $y$,
					domain = -3:3,
					samples = 100,
					ymin = -1,
					ymax = 5,
					ticks = none
				]
				\addplot[thick, domain=0:3]{2^x};
				\addplot[thick, domain=-3:0]{2^(-x)};
				\node[below left] at(0,0) {$O$};
				\node[below left] at(0,1) {$1$};
			\end{axis}
		\end{tikzpicture}

		\choice
		\begin{tikzpicture}[scale=0.45]
			\begin{axis}[
					axis lines=middle,
					xlabel = $x$,
					ylabel = $y$,
					domain = -3:3,
					samples = 100,
					ymin = -1,
					ymax = 5,
					ticks = none
				]
				\addplot[thick, domain=0:3]{2^(-x)};
				\addplot[thick, domain=-3:0]{2^(x)};
				\node[below left] at(0,0) {$O$};
				\node[above left] at(0,1) {$1$};
			\end{axis}
		\end{tikzpicture}

		\choice
		\begin{tikzpicture}[scale=0.45]
			\begin{axis}[
					axis lines=middle,
					xlabel = $x$,
					ylabel = $y$,
					domain = -3:3,
					samples = 100,
					ymin = -1,
					ymax = 5,
					ticks = none
				]
				\addplot[thick, domain=0:3]{2-2^(-x)};
				\addplot[thick, domain=-3:0]{2-2^(x)};
				\node[below left] at(0,0) {$O$};
				\node[below left] at(0,1) {$1$};
			\end{axis}
		\end{tikzpicture}
	\end{oneparchoices}

	\begin{solution}
		$a^{|x|} (a>1)$的最小值是$1$,而且不收敛,所以答案是B。
	\end{solution}

	\question 曲线的极坐标方程$\rho=4\sin\theta$化成直角坐标方程为 \hfill (\hspace{1cm})

	\begin{oneparchoices}
		\choice $x^2 + (y+2)^2 = 4$ \CorrectChoice $x^2 + (y-2)^2 = 4$
		\choice $(x-2)^2 + y^2 = 4$ \choice $(x+2)^2 + y^2 = 4$
	\end{oneparchoices}
	\begin{solution}
		设曲线上的动点$P(x,y)$,则有:
		\begin{align*}
			y    & = \rho\sin\theta                 \\
			x    & = \rho\cos\theta                 \\
			\rho & = \sqrt{x^2 + y^2} = 4\sin\theta
		\end{align*}
		则有
		\begin{align*}
			\sqrt{x^2 + y^2} & = 4\frac{y}{\rho} \\
			x^2 + y^2        & = 4y              \\
			x^2 + (y-2)^2    & = 4
		\end{align*}
	\end{solution}

	\question 两条直线$A_1x + B_1y + C_1 = 0, A_2x + B_2y + C_2=0$垂直的充要条件是 \hfill(\hspace{1cm})

	\begin{oneparchoices}
		\choice $A_1A_2 + B_1B_2 =0$ \choice $A_1A_2 - B_1B_2 = 0$ \CorrectChoice $\dfrac{A_1A_2}{B_1B_2} = -1$
		\choice $\dfrac{A_1A_2}{B_1B_2} = 1$
	\end{oneparchoices}

	\begin{solution}
		直线的斜率分别为$-\dfrac{A_1}{B_1}$和$-\dfrac{A_2}{B_2}$,两条直线垂直的充要条件是斜率的乘积为$-1$,所以答案选择C。
	\end{solution}

	\question 函数$f(x)=\frac1x(x\neq0)$的反函数$f^{-1}(x)=$ \hfill (\hspace{1cm})

	\begin{oneparchoices}
		\choice $x ( x\neq0)$ \CorrectChoice $\frac1x(x\neq0)$ \choice $-x (x\neq0)$ \choice $-\frac1x (x\neq0)$
	\end{oneparchoices}
	\begin{solution}
		因为$y=\frac1x$关于$y=x$对称,所以反函数是其本身。
	\end{solution}

	\question 已知点$P(\sin\alpha-\cos\alpha, \tan\alpha)$在第一象限,则在$[0,2\pi]$内$\alpha$的取值范围是 \hfill
	(\hspace{1cm})

	\begin{choices}
		\choice $\left( \dfrac{\pi}{2}, \dfrac{3\pi}{4} \right) \cup \left( \pi, \dfrac{5\pi}{4} \right)$
		\CorrectChoice $\left( \dfrac{\pi}{4}, \dfrac{\pi}{2} \right) \cup \left( \pi, \dfrac{5\pi}{4} \right)$
		\choice $\left( \dfrac{\pi}{2}, \dfrac{3\pi}{4} \right) \cup \left( \dfrac{5\pi}{4}, \dfrac{3\pi}{2} \right)$
		\choice $\left( \dfrac{\pi}{4}, \dfrac{\pi}{2} \right) \cup \left( \dfrac{3\pi}{4}, \pi \right)$
	\end{choices}

	\begin{solution}
		根据条件有:
		\begin{align*}
			\sin\theta - \cos\theta > 0 \tag{a} \\
			\tan\theta = \frac{\sin\theta}{\cos\theta} > 0 \tag{b}
		\end{align*}
		根据式$(a)$有$\alpha \in \left( \dfrac{\pi}{4}, \dfrac{5\pi}{4} \right)$;根据式$(b)$有$\alpha \in \left( 0,
			\dfrac{\pi}{2} \right) \cup \left( \pi, \dfrac{3\pi}{2} \right)$

		综上可得取值范围为B。
	\end{solution}

	\question 已知圆锥的全面积是底面积的$3$倍,那么该圆锥的侧面展开图扇形的圆心角为\hfill (\hspace{1cm})

	\begin{oneparchoices}
		\choice $\ang{120}$
		\choice $\ang{150}$
		\choice $\ang{180}$
		\choice $\ang{240}$
	\end{oneparchoices}

	\begin{solution}
		设圆锥的底面半径为$r$,棱线长为$l$,则底面积为$\pi r^2$,侧面积为$\frac{2\pi r}{2\pi l}\pi l^2 = \pi
			rl$,由题意有$\pi rl = 2\pi r^2$,即$l=2r$,展开图的圆心角为$2\pi\frac{r}{l}=\pi$,选C。
	\end{solution}

	\question 复数$-i$的一个立方根是$i$,它的另外两个立方根是 \hfill (\hspace{1cm})

	\begin{oneparchoices}
		\choice $\dfrac{\sqrt{3}}{2} \pm \dfrac12i$
		\choice $-\dfrac{\sqrt{3}}{2} \pm \dfrac12i$
		\choice $\pm\dfrac{\sqrt{3}}{2} + \dfrac12i$
		\CorrectChoice $\pm\dfrac{\sqrt{3}}{2} - \dfrac12i$
	\end{oneparchoices}

	\begin{solution}
		根据欧拉恒等式$e^{i\theta} = \cos\theta + i\sin\theta$,当$\theta=\frac32\pi$时有$e^{i\frac32\pi} =
			-i$。其立方根为$e^{i(\frac{\frac32\pi + 2k\pi}{3})}, (k=0,1,2)$。

		\begin{enumerate}[label=(\arabic*)]
			\item 当$k=0$时,$e^{\frac12\pi} = \cos \left( \dfrac12\pi \right) + \sin \left( \dfrac12\pi \right)i = i$
			\item 当$k=1$时,$e^{\frac76\pi} = \cos \left( \dfrac76\pi \right) + \sin \left( \dfrac76\pi \right)i =
				      -\frac{\sqrt{3}}{2} - \frac12i$
			\item 当$k=2$时,$e^{\frac{11}6\pi} = \cos \left( \dfrac{11}6\pi \right) + \sin \left( \dfrac{11}6\pi \right)i =
				      \frac{\sqrt{3}}{2} - \frac12i$
		\end{enumerate}
		答案选D。
	\end{solution}

	\question 如果棱台的两底面积分别是$S,S'$,中截面的面积是$S_0$,那么 \hfill (\hspace{1cm})

	\begin{oneparchoices}
		\CorrectChoice $2\sqrt{S_0} = \sqrt{S} + \sqrt{S'}$
		\choice $S_0 = \sqrt{SS'}$
		\choice $2S_0 = S + S'$
		\choice $S_0^2 = 2SS'$
	\end{oneparchoices}

	\begin{solution}
		一维方向上有 $\sqrt{S_0} = \frac{\sqrt{S} + \sqrt{S'}}{2}$,所以选择A。
	\end{solution}

	\question 向高为$H$的水瓶中注水,注满为止,如果注水量$V$与水深$h$的函数关系的图像如下图所示,那么水瓶的形状是 \hfill
	(\hspace{1cm})

	\begin{tikzpicture}[scale=0.45]
		\begin{axis}[
				axis lines=middle,
				xlabel = $h$,
				ylabel = $V$,
				domain = -3:3,
				samples = 100,
				ymin = -1,
				ymax = 5,
				xmax = 3,
				xmin = -1,
				ticks = none
			]
			\addplot[thick, domain=0:2]{-x^2+ 4*x};
			\node[below left] at(0,0) {$O$};
			\draw [dashed] (2,4)  -- (2,0) node[below]{$H$};
		\end{axis}
	\end{tikzpicture}

	\tdplotsetmaincoords{80}{0}
	\begin{oneparchoices}
		\choice
		\begin{tikzpicture}[tdplot_main_coords, scale=.5]
			\tkzDefPoints{0/0/O, 1/0/A, -1/0/B}
			\coordinate(O') at (0,0,4);
			\coordinate(A') at (2,0,4);
			\coordinate(B') at (-2,0,4);

			\tkzDrawSegments(A,A' B,B')
			\tkzDrawSemiCircle(O',A')
			\tkzDrawSemiCircle(O',B')

			\tkzDrawSemiCircle[dashed](O,A)
			\tkzDrawSemiCircle(O,B)
		\end{tikzpicture}

		\CorrectChoice
		\begin{tikzpicture}[tdplot_main_coords, scale=.5]
			\tkzDefPoints{0/0/O, 2/0/A, -2/0/B}
			\coordinate(O') at (0,0,4);
			\coordinate(A') at (1,0,4);
			\coordinate(B') at (-1,0,4);

			\tkzDrawSegments(A,A' B,B')
			\tkzDrawSemiCircle(O',A')
			\tkzDrawSemiCircle(O',B')

			\tkzDrawSemiCircle[dashed](O,A)
			\tkzDrawSemiCircle(O,B)
		\end{tikzpicture}

		\choice
		\begin{tikzpicture}[tdplot_main_coords, scale=.5]
			\tkzDefPoints{0/0/O, 2/0/A, -2/0/B}
			\coordinate(O') at (0,0,4);
			\coordinate(A') at (2,0,4);
			\coordinate(B') at (-2,0,4);
			\coordinate(C) at (1.5,0, 2);
			\coordinate(C') at (-1.5,0, 2);

			\tkzDrawSegments(A,C C,A' B,C' C',B')
			\tkzDrawSemiCircle(O',A')
			\tkzDrawSemiCircle(O',B')

			\tkzDrawSemiCircle[dashed](O,A)
			\tkzDrawSemiCircle(O,B)
		\end{tikzpicture}

		\choice
		\begin{tikzpicture}[tdplot_main_coords, scale=.5]
			\tkzDefPoints{0/0/O, 2/0/A, -2/0/B}
			\coordinate(O') at (0,0,4);
			\coordinate(A') at (2,0,4);
			\coordinate(B') at (-2,0,4);

			\tkzDrawSegments(A,A' B,B')
			\tkzDrawSemiCircle(O',A')
			\tkzDrawSemiCircle(O',B')

			\tkzDrawSemiCircle[dashed](O,A)
			\tkzDrawSemiCircle(O,B)
		\end{tikzpicture}
	\end{oneparchoices}

	\begin{solution}
		由函数图像可以看出斜率是逐渐下降的,所以选择B。
	\end{solution}

	\question $3$名医生和$6$名护士被分配到$3$所学校为学生体检,每校分配$1$名医生和$2$名护士,不同的分配方法共有\hfill
	(\hspace{1cm})

	\begin{oneparchoices}
		\choice $90$种
		\choice $180$种
		\choice $270$种
		\CorrectChoice $540$种
	\end{oneparchoices}

	\begin{solution}
		$3$名医生分到三个学校共有$3!=6$种分配方法。

		$6$名护士分到三个学校,每个学校两名护士有$\displaystyle\binom{6}{2}\cdot\binom{4}{2}\cdot\binom{2}{2}=90$种分配方法。

		所以总共有$6\times90=540$种分配方法。
	\end{solution}

	\question 椭圆$\dfrac{x^2}{12} + \dfrac{y^2}{3} =
		1$的焦点为$F_1$和$F_2$,点$P$在椭圆上,如果线段$PF_1$的中点在$y$轴上,那么$|PF_1|$是$|PF_2|$的\hfill (\hspace{1cm})

	\begin{oneparchoices}
		\choice $7$倍
		\choice $5$倍
		\choice $4$倍
		\choice $3$倍
	\end{oneparchoices}

	\begin{solution}
		$c=\sqrt{12-3} = 3$,所以焦点分别为$(-3,0), (3,0)$。

		如果$PF_1$的中点在$y$轴上,则可知$P$点的$x$坐标为$3$。此时$\triangle{F_1F_2P}$是一个直角三角形,其中$|F_1F_2|=6$,$|PF_2|=\dfrac{\sqrt{3}}{2}$,则$|PF_1|=\sqrt{6^2+\frac{3}{4}}=\frac72\sqrt{3}$,所以答案选A。
	\end{solution}
\end{questions}

\end{document}
