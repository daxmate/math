%!tex program = lualatex
\documentclass[answers]{exam}
\usepackage{silence}
\WarningFilter{latexfont}{Font shape}
\WarningFilter{latexfont}{Some font}
\usepackage{ctex}
\usepackage[margin=2cm]{geometry}
\usepackage{amsmath, amssymb, amsthm, arcs}
\usepackage{siunitx}
\usepackage{csquotes}
\usepackage{tikz, pgfplots, tikz-3dplot}
\usepackage[lua]{tkz-euclide}
\usetikzlibrary{
	angles,
	backgrounds,
	calc,
	decorations.pathmorphing,
	decorations.pathreplacing,
	decorations.text,
	intersections,
	patterns,
	petri,
	positioning,
	quotes,
	shapes,
	shapes.symbols,
}
\usepackage{graphicx, wrapfig}
\usepackage{caption}
\pagestyle{empty}
\newcounter{xcord}
\newcounter{ycord}
\newcounter{total}
\renewcommand{\labelenumi}{\textbf{\ifnum\value{enumi}<10 0\fi\arabic{enumi})}}

\pgfplotsset{compat=1.18}

\CorrectChoiceEmphasis{\color{blue!70!green}\bfseries}
\renewcommand{\solutiontitle}{\textbf{解:}}
\renewcommand{\choicelabel}{(\Alph{choice})}

\usepackage{array, tabularx}
\newcolumntype{C}{>{\centering\arraybackslash}X}
\newcolumntype{B}{>{\centering\bfseries\arraybackslash}X}

\usepackage{fontawesome5}
\usepackage{enumitem}
\newcommand\regularHandPointRight{\faHandPointRight[regular]}

\newenvironment{mathenum}
{\begin{enumerate}[label=\arabic*.]}
		{\end{enumerate}}

\newenvironment{penum}
{\begin{enumerate}[label=(\arabic*)]}
		{\end{enumerate}}
\makeatletter
\providecommand\@gobblethree[3]{}
\patchcmd{\over@under@arc}
{\@gobbletwo}
{\@gobblethree}
{}{}
\makeatother

\newcommand*\circled[1]{\tikz[baseline=(char.base)]{
		\node[shape=circle,draw,inner sep=1pt] (char) {#1};}}
\setlength{\answerclearance}{6pt}
\newcommand\hfs{\hfill (\hspace{2cm})}

\begin{document}
\begin{questions}
	\question
	小明想处理掉他大部分的玩具汽车.他先给了小轩自己收藏的六分之一再加上8辆车,然后又给了小丽剩下车辆的四分之一再加上4辆,接着他把剩下的车分了一半给阿强.之后他又把剩下车辆的40\%再加上5辆给了小杰.最后他自己留下了10辆车.请问小明最初有多少辆玩具汽车?

	\begin{oneparchoices}
		\choice 116 \choice 106 \CorrectChoice 96 \choice 86 \choice 以上都不是
	\end{oneparchoices}

	\begin{solution}
		设给小杰前小明有 \( a \)辆玩具汽车,则有:
		\begin{equation*}
			a \times 0.6 - 5 = 10
		\end{equation*}

		则可以求出 \( a = 25 \)辆.

		那么在分给阿强之前小明有50辆车.

		同样的设给小丽前小明有 \( b \)辆玩具汽车,则有:
		\begin{equation*}
			b \cdot \frac{3}{4} - 4 = 50
		\end{equation*}
		可以求出 \( b=72 \)辆.

		设小明最初有 \( c \)辆汽车,则有:
		\begin{equation*}
			c \cdot \frac{5}{6} - 8 = 72
		\end{equation*}

		可以计算出小明最初有 \( c=96 \)辆车.

	\end{solution}

	\question
	设 \( k \) 是一个非零实数.两个函数 \( f(x) = \frac{k}{x} \) 和 \( g(x) = kx + k \) 被绘制在同一坐标平面上.下列哪个是正确的图像?

	\begin{oneparchoices}
		\choice
		\begin{tikzpicture}[scale=0.3]
			\begin{axis}[
					axis lines=center,
					xticklabels=none,
					yticklabels=none,
					thick
				]
				\addplot[samples=100, domain=-10:0]{1/x};
				\addplot[samples=100, domain=0:10]{1/x};
				\addplot[samples=100, domain=-7:7]{-x + 1};
			\end{axis}
		\end{tikzpicture}

		\choice
		\begin{tikzpicture}[scale=0.3]
			\begin{axis}[
					axis lines=center,
					xticklabels=none,
					yticklabels=none,
					thick
				]
				\addplot[samples=100, domain=-10:0]{-1/x};
				\addplot[samples=100, domain=0:10]{-1/x};
				\addplot[samples=100, domain=-7:7]{x - 1};
			\end{axis}
		\end{tikzpicture}

		\CorrectChoice
		\begin{tikzpicture}[scale=0.3]
			\begin{axis}[
					axis lines=center,
					xticklabels=none,
					yticklabels=none,
					thick
				]
				\addplot[samples=100, domain=-10:0]{1/x};
				\addplot[samples=100, domain=0:10]{1/x};
				\addplot[samples=100, domain=-7:7]{x + 1};
			\end{axis}
		\end{tikzpicture}
		\choice
		\begin{tikzpicture}[scale=0.3]
			\begin{axis}[
					axis lines=center,
					xticklabels=none,
					yticklabels=none,
					thick
				]
				\addplot[samples=100, domain=-10:0]{-1/x};
				\addplot[samples=100, domain=0:10]{-1/x};
				\addplot[samples=100, domain=-7:7]{-x + 1};
			\end{axis}
		\end{tikzpicture}
		\choice
		\begin{tikzpicture}[scale=0.3]
			\begin{axis}[
					axis lines=center,
					xticklabels=none,
					yticklabels=none,
					thick
				]
				\addplot[samples=100, domain=-10:0]{1/x};
				\addplot[samples=100, domain=0:10]{1/x};
				\addplot[samples=100, domain=-7:7]{-x - 1};
			\end{axis}
		\end{tikzpicture}
	\end{oneparchoices}

	\begin{solution}
		\begin{itemize}
			\item 选项A中,倒数的图像中 \( k > 0 \),而直线的图像是矛盾的,方向上可以得出 \( k < 0 \),而与 \( y
			      \)轴的交点却得到 \( k > 0 \)
			\item 选项B中, 直线的图像中 \( k \)值矛盾
			\item 选项C中,直线与倒数的图像中都可以推出 \( k > 0 \)
			\item 选项D中,直线的图像中 \( k \)值矛盾
			\item 选项E中,直线的图像中 \( k < 0 \),而倒数图像中 \( k > 0 \)
		\end{itemize}
	\end{solution}

	\question

	\begin{minipage}[t]{0.95\textwidth}
		\begin{wrapfigure}{R}{0.3\textwidth}
			\begin{tikzpicture}
				% \draw (2,0) arc [start angle=0, end angle=180, radius=2cm];
				% \draw (-2,0) -- (2,0);
				% \draw ({sqrt(2)}, {sqrt(2)}) arc [start angle=0, end angle=-180, radius={sqrt(2)}];
				% \draw ({-sqrt(2)}, {sqrt(2)}) -- ({sqrt(2)}, {sqrt(2)});
				\tkzDefPoints{0/0/O, 2/0/A, -2/0/B}
				\tkzDefPoints{sqrt(2)/sqrt(2)/C,-sqrt(2)/sqrt(2)/D, 0/sqrt(2)/E}
				\tkzDrawSector(O,A)(B)
				\tkzDrawSector(E,D)(C)

				\tkzDefShiftPoint[A](0,-0.3){A'}
				\tkzDefShiftPoint[B](0,-0.3){B'}
				\tkzDrawSegment[Stealth-Stealth](A',B')

				\tkzLabelLine[below, fill=white](A,B){4cm}
			\end{tikzpicture}
		\end{wrapfigure}

		一个小半圆内接于一个大的半圆,如右图所示.请问小半圆的半径是多少?

		\begin{oneparchoices}
			\choice \( \frac{\sqrt{2}}{2} \) cm
			\CorrectChoice \( \sqrt{2} \) cm
			\choice 1.5 cm
			\choice 2 cm
			\choice 以上都不对
		\end{oneparchoices}
	\end{minipage}

	\begin{solution}
		\begin{minipage}[t]{0.95\textwidth}
			\begin{wrapfigure}{L}{0.3\textwidth}
				\begin{tikzpicture}
					\tkzDefPoints{0/0/O, 2/0/A, -2/0/B}
					\tkzDefPoints{sqrt(2)/sqrt(2)/C,-sqrt(2)/sqrt(2)/D, 0/sqrt(2)/E}
					\tkzDrawSector(O,A)(B)
					\tkzDrawSector(E,D)(C)

					\tkzDrawSegment[dim={4cm, 0.3cm,}](A,B)
					\tkzDrawSegments[dashed](O,E)
					\tkzDrawSegments[dashed](O,C)

					\tkzLabelPoints[right](A,C)
					\tkzLabelPoints[above](E)
					\tkzLabelPoints[above left](O)
				\end{tikzpicture}
			\end{wrapfigure}

			\begin{equation*}
				OC^2 = OE^2 + EC^2
			\end{equation*}

			则有小半圆的半径为 \( \sqrt{2} \)cm
		\end{minipage}
		\vspace{2cm}

	\end{solution}

	\question 假设你知道:
	\begin{enumerate}[label=\roman*] 
		\item 如果风铃响了,那么秋千就会摇摆.
		\item 只有眼镜笑了,蚂蚁才会跳舞.
		\item 如果风铃不响,那么蚂蚁会跳舞.
	\end{enumerate}

	以下哪一项必须逻辑上是正确的?

	\begin{oneparchoices}
		\choice 如果秋千摇摆,那么眼镜笑了.
		\choice 只有眼镜笑了,秋千才不会摇摆.
		\choice 如果蚂蚁跳舞,那么秋千会摇摆.
		\choice 只有风铃响了,眼镜才会笑.
		\choice 以上都不是
	\end{oneparchoices}

	\begin{solution}
	\begin{itemize}
		\item 选项A中,如果秋千摇摆,根据 \romannumeral 1 无法推导出风铃响还是没响;
		\item 选项B中,如果眼镜笑了,蚂蚁有可能会跳舞,如果还是无法推导出秋千会不会摇摆;
		\item 选项C中,如果蚂蚁跳舞,根据\romannumeral 2,可以推导出眼镜肯定笑了,根据\romannumeral
			3可以推导出风铃没有响.风铃不响,根据\romannumeral 1 无法推导出秋千在不在摇摆;
		\item 选项D中,风铃响了,可以推导出秋千在摇摆,但是秋千会摇摆无法推导出任何其他的内容.
	\end{itemize}
	\end{solution}

\end{questions}

\end{document}
