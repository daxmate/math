%!tex program = lualatex
\documentclass[answers]{exam}
\usepackage{ctex}
\usepackage{graphicx}
\usepackage[margin=2cm]{geometry}
\usepackage{amsmath, amssymb}
\usepackage{csquotes}
\usepackage{tikz, pgfplots}
\usetikzlibrary{
	angles,
	backgrounds,
	calc,
	decorations.pathmorphing,
	decorations.pathreplacing,
	decorations.text,
	intersections,
	patterns,
	quotes,
	shapes,
	shapes.symbols,
}
\pagestyle{empty}
\newcounter{xcord}
\newcounter{ycord}
\newcounter{total}
\renewcommand{\labelenumi}{\textbf{\ifnum\value{enumi}<10 0\fi\arabic{enumi})}}

\pgfplotsset{compat=1.18}

\CorrectChoiceEmphasis{\color{blue!70!green}\bfseries}
\renewcommand{\solutiontitle}{\textbf{解:}}

\usepackage{array, tabularx}
\newcolumntype{C}{>{\centering\arraybackslash}X}
\newcolumntype{B}{>{\centering\bfseries\arraybackslash}X}
\catcode`\幺=0

\begin{document}
\begin{questions}
	\question
	小明想处理掉他大部分的玩具汽车。他先给了小轩自己收藏的六分之一再加上8辆车,然后又给了小丽剩下车辆的四分之一再加上4辆,接着他把剩下的车分了一半给阿强。之后他又把剩下车辆的40\%再加上5辆给了小杰。最后他自己留下了10辆车。请问小明最初有多少辆玩具汽车?

	\begin{oneparchoices}
		\choice 116 \choice 106 \CorrectChoice 96 \choice 86 \choice 以上都不是
	\end{oneparchoices}

	\begin{solution}
		设给小杰前小明有 \( a \)辆玩具汽车,则有:
		\begin{equation*}
			a \times 0.6 - 5 = 10
		\end{equation*}

		则可以求出 \( a = 25 \)辆。

		那么在分给阿强之前小明有50辆车。

		同样的设给小丽前小明有 \( b \)辆玩具汽车,则有:
		\begin{equation*}
			b \cdot \frac{3}{4} - 4 = 50
		\end{equation*}
		可以求出 \( b=72 \)辆。

		设小明最初有 \( c \)辆汽车,则有:
		\begin{equation*}
			c \cdot \frac{5}{6} - 8 = 72
		\end{equation*}

		可以计算出小明最初有 \( c=96 \)辆车。

	\end{solution}

	\question
	设 \( k \) 是一个非零实数。两个函数 \( f(x) = \frac{k}{x} \) 和 \( g(x) = kx + k \) 被绘制在同一坐标平面上。下列哪个是正确的图像?

	\begin{oneparchoices}
		\choice
		\begin{tikzpicture}[scale=0.3]
			\begin{axis}[
					axis lines=center,
					xticklabels=none,
					yticklabels=none,
					thick
				]
				\addplot[samples=100, domain=-10:0]{1/x};
				\addplot[samples=100, domain=0:10]{1/x};
				\addplot[samples=100, domain=-7:7]{-x + 1};
			\end{axis}
		\end{tikzpicture}

		\choice
		\begin{tikzpicture}[scale=0.3]
			\begin{axis}[
					axis lines=center,
					xticklabels=none,
					yticklabels=none,
					thick
				]
				\addplot[samples=100, domain=-10:0]{-1/x};
				\addplot[samples=100, domain=0:10]{-1/x};
				\addplot[samples=100, domain=-7:7]{x - 1};
			\end{axis}
		\end{tikzpicture}

		\CorrectChoice
		\begin{tikzpicture}[scale=0.3]
			\begin{axis}[
					axis lines=center,
					xticklabels=none,
					yticklabels=none,
					thick
				]
				\addplot[samples=100, domain=-10:0]{1/x};
				\addplot[samples=100, domain=0:10]{1/x};
				\addplot[samples=100, domain=-7:7]{x + 1};
			\end{axis}
		\end{tikzpicture}
		\choice
		\begin{tikzpicture}[scale=0.3]
			\begin{axis}[
					axis lines=center,
					xticklabels=none,
					yticklabels=none,
					thick
				]
				\addplot[samples=100, domain=-10:0]{-1/x};
				\addplot[samples=100, domain=0:10]{-1/x};
				\addplot[samples=100, domain=-7:7]{-x + 1};
			\end{axis}
		\end{tikzpicture}
		\choice
		\begin{tikzpicture}[scale=0.3]
			\begin{axis}[
					axis lines=center,
					xticklabels=none,
					yticklabels=none,
					thick
				]
				\addplot[samples=100, domain=-10:0]{1/x};
				\addplot[samples=100, domain=0:10]{1/x};
				\addplot[samples=100, domain=-7:7]{-x - 1};
			\end{axis}
		\end{tikzpicture}
	\end{oneparchoices}

	\begin{solution}
		\begin{itemize}
			\item 选项A中,倒数的图像中 \( k > 0 \),而直线的图像是矛盾的,方向上可以得出 \( k < 0 \),而与 \( y
			      \)轴的交点却得到 \( k > 0 \)
			\item 选项B中, 直线的图像中 \( k \)值矛盾
			\item 选项C中,直线与倒数的图像中都可以推出 \( k > 0 \)
			\item 选项D中,直线的图像中 \( k \)值矛盾
			\item 选项E中,直线的图像中 \( k < 0 \),而倒数图像中 \( k > 0 \)
		\end{itemize}
	\end{solution}

\end{questions}

\end{document}
