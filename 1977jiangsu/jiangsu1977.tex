%!tex program = lualatex
\documentclass[answers]{exam}
\usepackage{ctex}
\usepackage{graphicx}
\usepackage[margin=2cm]{geometry}
\usepackage{amsmath, amssymb}
\usepackage{csquotes}
\usepackage{tikz, pgfplots}
\usetikzlibrary{
	angles,
	backgrounds,
	calc,
	decorations.pathmorphing,
	decorations.pathreplacing,
	decorations.text,
	intersections,
	patterns,
	quotes,
	shapes,
	shapes.symbols,
}
\pagestyle{empty}
\newcounter{xcord}
\newcounter{ycord}
\newcounter{total}
\renewcommand{\labelenumi}{\textbf{\ifnum\value{enumi}<10 0\fi\arabic{enumi})}}

\pgfplotsset{compat=1.18}

\CorrectChoiceEmphasis{\color{blue!70!green}\bfseries}
\renewcommand{\solutiontitle}{\textbf{解:}}

\usepackage{array, tabularx}
\newcolumntype{C}{>{\centering\arraybackslash}X}
\newcolumntype{B}{>{\centering\bfseries\arraybackslash}X}
\catcode`\幺=0

\usepackage[lua]{tkz-euclide}

\begin{document}
\begin{center}
	\textbf{1977年普通高等学校招生考试(江苏卷)}
	\\ \textbf{\Large{数学试卷}}
\end{center}

\begin{questions}
	\question
	\begin{enumerate}[label=(\arabic*)]
		\item 计算:$ \left(2\frac14\right)^\frac12 + \left(\frac{1}{10}\right) - (3.14)^0 +
			      \left(-\frac{27}{8}\right)^{\frac13} $
		      \begin{solution}
			      \begin{align*}
				       & = \left(\frac32\right)^{2\times\frac12} + 10^2 - 1 - \left(\frac32\right)^{3\times(\frac13)} \\
				       & = \frac32 + 99 - \frac32                                                                     \\
				       & = 99
			      \end{align*}
		      \end{solution}

		\item 求函数$y=\sqrt{x-2} + \dfrac{1}{x-3} + \lg(5-x)$的定义域。
		      \begin{solution}
			      根据题意有:
			      \begin{math}
				      \begin{cases}
					      x - 2 >= 0   \\
					      x - 3 \neq 0 \\
					      5 - x > 0
				      \end{cases}
			      \end{math}
			      则有定义域为$[2,3),(3,5]$

		      \end{solution}
		\item 解方程:$ 5^{x^2+2x} = 125 $.
		      \begin{solution}
			      由 $5^{x^2 + 2x} = 125 = 5^3 $得: $x^2 + 2x = 3$

			      分解因式得$(x+3)(x-1) = 0$,所以$x_1 = -3, x_2 = 1$
		      \end{solution}
		\item 计算:$-\log_3(\log_3\sqrt[3]{\sqrt[3]{\sqrt[3]{3}}})$
		      \begin{solution}
			      \begin{align*}
				      -\log_3(\log_3\sqrt[3]{\sqrt[3]{\sqrt[3]{3}}}) & = -\log_3(\log_3\sqrt[3]{\sqrt[3]{3^\frac13}}) \\
				                                                     & = -\log_3(\log_3\sqrt[3]{3^\frac19})           \\
				                                                     & = -\log_3(\log_33^\frac1{27})                  \\
				                                                     & = -\log_3\left(\frac{1}{27}\right)             \\
				                                                     & = -\log_3\left(3^{-3}\right)                   \\
				                                                     & = 3
			      \end{align*}
		      \end{solution}
		\item 把直角坐标方程 \( (x-3)^2 + y^2 = 9 \)化为极坐标方程。
		      \begin{solution}
			      设动点为$(\rho, \theta)$,则有$x=\rho\cos\theta, y=\rho\sin\theta$,代入直角坐标方程得:
			      \begin{equation*}
				      (\rho\cos\theta - 3)^2 + (\rho\sin\theta)^2 = 9
			      \end{equation*}
			      展开得
			      \begin{equation*}
				      \rho^2\cos^2\theta - 6\rho\cos\theta + 9 + \rho^2\sin^2\theta = 9
			      \end{equation*}
			      合并同类项得:
			      \begin{equation*}
				      \rho^2(\cos^2\theta + \sin^2\theta) - 6\rho\cos\theta = 0
			      \end{equation*}
			      因为$\cos^2\theta + \sin^2\theta = 1$,简化为:
			      \begin{equation*}
				      \rho(\rho - 6\cos\theta) = 0
			      \end{equation*}
			      则极坐标方程为
			      \begin{equation*}
				      \rho = 6\cos\theta
			      \end{equation*}
		      \end{solution}
		\item 计算: \( \displaystyle \lim_{n\to\infty}\frac{1+2+3+\cdots+n}{n^2} \)
		      \begin{solution}
			      根据等差数列求和公式得
			      \begin{equation*}
				      1 + 2 + 3 + \cdots + n = \frac{n(n+1)}{2}
			      \end{equation*}
			      则原式
			      \begin{equation*}
				      \lim_{n\to\infty}\frac{1+2+3+\cdots+n}{n^2} = \lim_{n\to\infty}(\frac12 + \frac{1}{2n}) = \frac12
			      \end{equation*}
		      \end{solution}
		\item 分解因式:$ x^4 - 2x^2y - 3y^2 + 8y - 4 $.
		      \begin{solution}
			      \begin{align*}
				      x^4 - 2x^2y - 3y^2 + 8y - 4 & = x^4 - 2x^2y + y^2 - (4y^2 - 8y + 4)  \\
				                                  & = (x^2 - y)^2 - (2y - 2)^2             \\
				                                  & = (x^2 - y + 2y - 2)(x^2 - y - 2y + 2) \\
				                                  & = (x^2 + y - 2)(x^2 - 3y + 2)
			      \end{align*}
		      \end{solution}
	\end{enumerate}
	\question 过抛物线$ y^2 = 4x
	$的焦点作倾斜角为$\frac34\pi$的直线,它与抛物线相交于$A$、$B$两点。求$A$、$B$两点间的距离。
	\begin{solution}
		抛物线的焦点为$(0,1)$,斜角为$\frac34\pi$的直线的斜率为$-1$,设直线的方程为$ y = -x + b $,将点$(0,1)$代入解得
		$b = 1$,则直线方程为$ y=-x+1 $。将直线方程代入抛物线方程得:
		\begin{equation*}
			(-x + 1)^2 = 4x
		\end{equation*}
		展开得
		\begin{align*}
			x^2 - 2x + 1 & = 4x \\
			x^2 -6x + 1  & = 0  \\
		\end{align*}
		解得$ x_1,x_2 = \frac{6 \pm \sqrt{32}}{2} = 3 \pm 2\sqrt{2} $,代入直线方程可得$ y_1,y_2 = -(3\pm2\sqrt{2}) +
			1$,则点$A$的坐标为$(3+2\sqrt{2}, -2 - 2\sqrt{2})$,$B$点的坐标为$(3-2\sqrt{2}, -2 + 2\sqrt{2})$,$AB$的距离为:
		\begin{align*}
			\sqrt{(3-2\sqrt{2} - 3 - 2\sqrt{2})^2 + (-2 + 2\sqrt{2} + 2 + 2\sqrt{2})^2}
			 & = \sqrt{32 + 32} \\
			 & = 8
		\end{align*}
	\end{solution}
	\question 在直角三角形$ \triangle{ABC} $中,$ \angle{ACB} = 90^\circ
	$,$CD$、$CE$分别为斜边$AB$上的高和中线,且$\angle{BCD}$与$\angle{ACD}$之比为$3:1$,求证:$CD=DE$。

	\begin{tikzpicture}
		\tkzDefPoints{-1/0/A, 0/0/D, 0/2/C}
		\tkzDefLine[orthogonal=through C](A,C) \tkzGetPoint{x}
		\tkzInterLL(C,x)(A,D) \tkzGetPoint{B}
		\tkzDefPointOnLine[pos=0.5](A,B) \tkzGetPoint{E}

		\tkzDrawPolygon(A,B,C)
		\tkzDrawSegments(C,D C,E)

		\tkzLabelPoints[below](A,D,E,B)
		\tkzLabelPoint[above](C){$C$}
	\end{tikzpicture}
	\begin{solution}
		\begin{align*}
			 & \text{设}\angle{ACD} = \alpha                               \\
			 & \because CE\text{是直角三角形的中线}                                \\
			 & \therefore CE = EB                                         \\
			 & \therefore \angle{ECB} = \angle{CBE}                       \\
			 & \because \angle{DCB} = 3\angle{ACD}                        \\
			 & \because \angle{ACD} = \angle{CBE}                         \\
			 & \therefore \angle{DCE} = 3\alpha - \alpha = 2\alpha        \\
			 & \because \angle{DEC} = \angle{EBC} + \angle{ECB} = 2\alpha \\
			 & \therefore \angle{DEC} = \angle{DCE}                       \\
			 & \therefore DE = DC
		\end{align*}

	\end{solution}
	\pagebreak
	\question 在周长为 $300$ cm 的圆周上, 有甲、乙两球以大小不等的速度作匀速圆周 运动. 甲球从 $A
	$点出发按逆时针方向运动, 乙球从 $B$ 点出发按顺时针方向 运动, 两球相遇于 $C$ 点相遇后, 两球各自反方向作匀速圆周运动,
	但这时 甲球速度的大小是原来的 $2$ 倍, 乙球速度的大小是原来的一半, 以后他们 第二次相遇于 $D$ 点. 已知
	\overarc{$AmC$}  = 40 厘米,  \overarc{$BnD$} = 20 厘米, 求 \overarc{$ACB$} 的 长度.
	\begin{figure*}[ht]
		\centering
		\begin{tikzpicture}
			\coordinate(O) at (0,0);
			\coordinate(m) at (210:2);
			\coordinate(n) at (300:2);
			\coordinate(A) at (180:2);
			\coordinate(B) at (280:2);
			\coordinate(C) at (240:2);
			\coordinate(D) at (330:2);

			\draw[radius=2] circle;
			\draw[-Latex](190:1.7)node[above]{甲}  arc[start angle=190, end angle=230, radius=1.7cm];
			\draw[-Latex](280:1.7)node[right]{乙}  arc[start angle=280, end angle=250, radius=1.7cm];

			\tkzDrawPoints(m, n, A, B, C, D)
			\tkzAutoLabelPoints[center=O](m,n,A,B,C,D)
		\end{tikzpicture}
	\end{figure*}
	\begin{solution}
		设甲的速度为$x$,乙的速度为$y$,\overarc{$ACB$}的长度为$a$,则第一次相遇有如下时间相等的关系:
		\begin{equation*}
			\frac{40}{x} = \frac{a-40}{y} \tag{1}
		\end{equation*}
		根据反弹后的速度变化有如下等式:
		\begin{equation*}
			\frac{300 - 20 - (a - 40)}{2x} = \frac{a - 40 + 20}{y/2} \tag{2}
		\end{equation*}
		将(1)式与(2)式相除可得:
		\begin{equation*}
			\frac{80}{320 -a} = \frac{a-40}{2(a-20)}
		\end{equation*}
		整理可得
		\begin{equation*}
			(a-40)(a+240) = 0
		\end{equation*}
		所以 $a = 40$厘米。
	\end{solution}
	\question
	\begin{enumerate}[label=(\arabic*)]
		\item 若三角形三内角成等差数列,求证:必有一内角为$60^\circ$
		      \begin{solution}
			      因为三个内角成等差数列,可以设中间的角为$x$,则其余两个角可以分别表示为$x-a,
				      x+a$,那么三个角的和可以表示为$x-a + x + x + a = 3x = 180^\circ$,则有$x=60^\circ$
		      \end{solution}
		\item 若三角形三内角成等差数列,而且三边又成等比数列,求证:三角形三内角都是$60^\circ$.
		      \begin{solution}
			      \begin{tikzpicture}
				      \tkzDefPoints{0/0/A, 2/0/B}
				      \tkzDefPoint(60:2){C}
				      \tkzDefLine[orthogonal=through C](A,B) \tkzGetPoint{x}
				      \tkzInterLL(C,x)(A,B) \tkzGetPoint{D}

				      \tkzDrawPolygon(A,B,C)
				      \tkzDrawSegment[dashed](C,D)

				      \tkzLabelPoints(A,B,D)
				      \tkzLabelPoint[above](C){$C$}
				      \tkzMarkAngle[size=0.5](B,A,C)
				      \tkzLabelAngle(B,A,C){$60^\circ$}
				      \tkzLabelSegment[above, sloped](A,C){$a$}
			      \end{tikzpicture}

			      假设三角形不是等边三角形,并假设$\angle{A} = 60^\circ,
				      AC=a$,且$AC$是最短的一条边,则$AB$为最长边设其长为$b$。从$C$点作垂线交$AB$于$D$。根据题目中的信息有:
			      \begin{equation*}
				      \frac{AC}{BC} = \frac{BC}{AB} \tag{1}
			      \end{equation*}
			      根据三角关系得$AD=\frac{a}{2},
				      CD=\frac{\sqrt{3}}{2}a$,$DB=b-\frac{a}{2}$,则可以计算出$BC=\sqrt{(b-\frac{a}{2})^2 +
					      (\frac{\sqrt{3}}{2}a)^2}$
			      根据式(1)有
			      \begin{align*}
				      BC^2 = AC \cdot AB                                 \\
				      (b - \frac{a}{2})^2 + (\frac{\sqrt{3}}{2}a)^2 = ab \\
				      b^2 - ab + \frac{1}{4}a^2 + \frac34a^2 = ab        \\
				      (b-a)^2 = 0                                        \\
				      a = b
			      \end{align*}
			      所以假设不成立,三角形为等边三角形,所以三角形的三内角都是$60^\circ$
		      \end{solution}
	\end{enumerate}
	\question 在两条平行直线 $AB$ 和 $CD$ 上分别取定一点 $M$ 和 $N$, 在直线 $AB$ 上取 一定线段 $ME = a$; 在线段
	$MN$上取一点 $K$, 连结 $EK$ 并延长交 $CD$ 于 $F$. 试问 $K$ 取在哪里时, $\triangle{EMK}$ 与 $\triangle{FNK}$ 的面积之和最小? 最小值是多少?

	\begin{solution}
		\begin{center}
			\begin{tikzpicture}[scale=2]
				\tkzDefPoints{0/0/A, 3/0/B, 0/2/C, 3/2/D, 1/0/M, 2/2/N, 2/0/E}
				\tkzDefPointOnLine[pos=0.4](M,N) \tkzGetPoint{K}
				\tkzInterLL(E,K)(C,D) \tkzGetPoint{F}
				\tkzDefLine[orthogonal=through K](A,B) \tkzGetPoint{x}
				\tkzInterLL(K,x)(A,B) \tkzGetPoint{G}
				\tkzInterLL(K,x)(C,D) \tkzGetPoint{H}

				\tkzDrawSegments(A,B C,D M,N E,F)
				\tkzLabelPoint[below left](M){$M$}
				\tkzLabelPoint[below right](E){$E$}
				\tkzLabelPoints[above](F,N)
				\tkzLabelPoint[right](K){$K$}
				\tkzMarkSegment[dim={$a$, -16pt, above=6pt}, mark=](M,E)
				\tkzDrawSegment(G,H)
				\tkzLabelPoint[above right](G){$G$}
				\tkzLabelPoint[above](H){$H$}
				\tkzMarkSegment[dim={$h$, 36pt, right=12pt}, mark=](H,G)
			\end{tikzpicture}
		\end{center}

		假设$KG=xh$,则$HK=(1-x)h$。由$\triangle{MKE} \sim \triangle{NKF}$可得$FN =
			\frac{1-x}{x}a$,则$\triangle{EMK}$与$\triangle{NKF}$的面积和为:
		\begin{align*}
			\frac12axh + \frac12\cdot\frac{1-x}{x}a\cdot(1-x)h
			\\                  & = \frac12ah(x + \frac{(1-x)^2}{x})
			\\                  & = \frac12ah(\frac{x^2 + 1 - 2x + x^2}{x})
			\\                & = \frac12ah\frac{2x^2 - 2x + 1}{x}
			\\ & = \frac12ah(2x -2 + \frac1x)
		\end{align*}
		对其求导数得:
		\begin{align*}
			\frac12ah(2 - \frac{1}{x^2})
		\end{align*}
		此导数在$x =
			\frac{\sqrt{2}}{2}$时等于$0$,所以此时面积有极值。所以$K$在$MN$的$\frac{\sqrt{2}}{2}$处$\triangle{EMK}$和$\triangle{NKF}$有最小值。

	\end{solution}

	\begin{center}
		\textbf{附加题}
	\end{center}

	\question 求极限:$ \displaystyle \lim_{n\to\infty}\sqrt{x}(\sqrt{x+1} - \sqrt{x}) $.
	\begin{solution}
		\begin{align*}
			\text{原式} & = \lim_{n\to\infty}\frac{\sqrt{x}}{\sqrt{x+1} + \sqrt{x}} \\
			          & = \lim_{n\to\infty}\frac{1}{\sqrt{1 + \frac1x} + 1}       \\
			          & = \frac12
		\end{align*}
	\end{solution}
	\question 求不定积分:$\int\dfrac{\mathrm{d}x}{(1+e^x)^2}$.
	\begin{solution}
		设$u = 1 + e^x$,则有$\mathrm{d}u = e^x\mathrm{d}x$,即$\mathrm{d}x = \frac{\mathrm{d}u}{u-1}$

		代入得:
		\begin{align*}
			\int\frac{\mathrm{d}x}{(1+e^x)^2} & = \int\frac{1}{(u-1)u^2}\mathrm{d}u     \\
			                                  & = \int\frac{1}{u^3} - \int\frac{1}{u^2} \\
			                                  & = \frac{1}{-2u^2} + \frac{1}{u} + C     \\
			                                  & = \frac{2u - 1}{2u^2}  + C              \\
			                                  & = \frac{2(1+e^x) + 1}{2(1+e^x)^2} + C   \\
			                                  & = \frac{2e^x + 3}{2(1+e^x)^2}  + C
		\end{align*}
	\end{solution}
\end{questions}

\end{document}
