%!tex program = lualatex
\documentclass[answers]{exam}
\usepackage{ctex}
\usepackage{graphicx}
\usepackage[margin=2cm]{geometry}
\usepackage{amsmath, amssymb}
\usepackage{csquotes}
\usepackage{tikz, pgfplots}
\usetikzlibrary{
	angles,
	backgrounds,
	calc,
	decorations.pathmorphing,
	decorations.pathreplacing,
	decorations.text,
	intersections,
	patterns,
	quotes,
	shapes,
	shapes.symbols,
}
\pagestyle{empty}
\newcounter{xcord}
\newcounter{ycord}
\newcounter{total}
\renewcommand{\labelenumi}{\textbf{\ifnum\value{enumi}<10 0\fi\arabic{enumi})}}

\pgfplotsset{compat=1.18}

\CorrectChoiceEmphasis{\color{blue!70!green}\bfseries}
\renewcommand{\solutiontitle}{\textbf{解:}}

\usepackage{array, tabularx}
\newcolumntype{C}{>{\centering\arraybackslash}X}
\newcolumntype{B}{>{\centering\bfseries\arraybackslash}X}
\catcode`\幺=0


\begin{document}
\begin{center}
	\textbf{1977年普通高等学校招生考试(江苏卷)}
	\\ \textbf{\Large{数学试卷}}
\end{center}

\begin{questions}
	\question
	\begin{enumerate}[label=(\arabic*)]
		\item 计算:$ \left(2\frac14\right)^\frac12 + \left(\frac{1}{10}\right) - (3.14)^0 +
			      \left(-\frac{27}{8}\right)^{\frac13} $
		      \begin{solution}
			      \begin{align*}
				       & = \left(\frac32\right)^{2\times\frac12} + 10^2 - 1 - \left(\frac32\right)^{3\times(\frac13)} \\
				       & = \frac32 + 99 - \frac32                                                                     \\
				       & = 99
			      \end{align*}
		      \end{solution}

		\item 求函数$y=\sqrt{x-2} + \dfrac{1}{x-3} + \lg(5-x)$的定义域。
		      \begin{solution}
			      根据题意有:
			      \begin{math}
				      \begin{cases}
					      x - 2 >= 0   \\
					      x - 3 \neq 0 \\
					      5 - x > 0
				      \end{cases}
			      \end{math}
			      则有定义域为$[2,3),(3,5]$

		      \end{solution}
		\item 解方程:$ 5^{x^2+2x} = 125 $.
		      \begin{solution}
			      由 $5^{x^2 + 2x} = 125 = 5^3 $得: $x^2 + 2x = 3$

			      分解因式得$(x+3)(x-1) = 0$,所以$x_1 = -3, x_2 = 1$
		      \end{solution}
		\item 计算:$-\log_3(\log_3\sqrt[3]{\sqrt[3]{\sqrt[3]{3}}})$
		      \begin{solution}
			      \begin{align*}
				      -\log_3(\log_3\sqrt[3]{\sqrt[3]{\sqrt[3]{3}}}) & = -\log_3(\log_3\sqrt[3]{\sqrt[3]{3^\frac13}}) \\
				                                                     & = -\log_3(\log_3\sqrt[3]{3^\frac19})           \\
				                                                     & = -\log_3(\log_33^\frac1{27})                  \\
				                                                     & = -\log_3\left(\frac{1}{27}\right)             \\
				                                                     & = -\log_3\left(3^{-3}\right)                   \\
				                                                     & = 3
			      \end{align*}
		      \end{solution}
		\item 把直角坐标方程 \( (x-3)^2 + y^2 = 9 \)化为极坐标方程。
		      \begin{solution}
			      设动点为$(\rho, \theta)$,则有$x=\rho\cos\theta, y=\rho\sin\theta$,代入直角坐标方程得:
			      \begin{equation*}
				      (\rho\cos\theta - 3)^2 + (\rho\sin\theta)^2 = 9
			      \end{equation*}
			      展开得
			      \begin{equation*}
				      \rho^2\cos^2\theta - 6\rho\cos\theta + 9 + \rho^2\sin^2\theta = 9
			      \end{equation*}
			      合并同类项得:
			      \begin{equation*}
				      \rho^2(\cos^2\theta + \sin^2\theta) - 6\rho\cos\theta = 0
			      \end{equation*}
			      因为$\cos^2\theta + \sin^2\theta = 1$,简化为:
			      \begin{equation*}
				      \rho(\rho - 6\cos\theta) = 0
			      \end{equation*}
			      则极坐标方程为
			      \begin{equation*}
				      \rho = 6\cos\theta
			      \end{equation*}
		      \end{solution}
		\item 计算: \( \displaystyle \lim_{n\to\infty}\frac{1+2+3+\cdots+n}{n^2} \)
		      \begin{solution}
			      根据等差数列求和公式得
			      \begin{equation*}
				      1 + 2 + 3 + \cdots + n = \frac{n(n+1)}{2}
			      \end{equation*}
			      则原式
			      \begin{equation*}
				      \lim_{n\to\infty}\frac{1+2+3+\cdots+n}{n^2} = \lim_{n\to\infty}(\frac12 + \frac{1}{2n}) = \frac12
			      \end{equation*}
		      \end{solution}
		\item 分解因式:$ x^4 - 2x^2y - 3y^2 + 8y - 4 $.
		      \begin{solution}
			      \begin{align*}
				      x^4 - 2x^2y - 3y^2 + 8y - 4 & = x^4 - 2x^2y + y^2 - (4y^2 - 8y + 4)  \\
				                                  & = (x^2 - y)^2 - (2y - 2)^2             \\
				                                  & = (x^2 - y + 2y - 2)(x^2 - y - 2y + 2) \\
				                                  & = (x^2 + y - 2)(x^2 - 3y + 2)
			      \end{align*}
		      \end{solution}
	\end{enumerate}
\end{questions}

\end{document}
