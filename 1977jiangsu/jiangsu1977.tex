%!tex program = lualatex
\documentclass[answers]{exam}
\usepackage{silence}
\WarningFilter{latexfont}{Font shape}
\WarningFilter{latexfont}{Some font}
\usepackage{ctex}
\usepackage[margin=2cm]{geometry}
\usepackage{amsmath, amssymb, amsthm, arcs}
\usepackage{siunitx}
\usepackage{csquotes}
\usepackage{tikz, pgfplots, tikz-3dplot}
\usepackage[lua]{tkz-euclide}
\usetikzlibrary{
	angles,
	backgrounds,
	calc,
	decorations.pathmorphing,
	decorations.pathreplacing,
	decorations.text,
	intersections,
	patterns,
	petri,
	positioning,
	quotes,
	shapes,
	shapes.symbols,
}
\usepackage{graphicx, wrapfig}
\usepackage{caption}
\pagestyle{empty}
\newcounter{xcord}
\newcounter{ycord}
\newcounter{total}
\renewcommand{\labelenumi}{\textbf{\ifnum\value{enumi}<10 0\fi\arabic{enumi})}}

\pgfplotsset{compat=1.18}

\CorrectChoiceEmphasis{\color{blue!70!green}\bfseries}
\renewcommand{\solutiontitle}{\textbf{解:}}
\renewcommand{\choicelabel}{(\Alph{choice})}

\usepackage{array, tabularx}
\newcolumntype{C}{>{\centering\arraybackslash}X}
\newcolumntype{B}{>{\centering\bfseries\arraybackslash}X}

\usepackage{fontawesome5}
\usepackage{enumitem}
\newcommand\regularHandPointRight{\faHandPointRight[regular]}

\newenvironment{mathenum}
{\begin{enumerate}[label=\arabic*.]}
		{\end{enumerate}}

\newenvironment{penum}
{\begin{enumerate}[label=(\arabic*)]}
		{\end{enumerate}}
\makeatletter
\providecommand\@gobblethree[3]{}
\patchcmd{\over@under@arc}
{\@gobbletwo}
{\@gobblethree}
{}{}
\makeatother

\newcommand*\circled[1]{\tikz[baseline=(char.base)]{
		\node[shape=circle,draw,inner sep=1pt] (char) {#1};}}
\setlength{\answerclearance}{6pt}
\newcommand\hfs{\hfill (\hspace{2cm})}

\usepackage[lua]{tkz-euclide}

\begin{document}
\begin{center}
	\textbf{1977年普通高等学校招生考试(江苏卷)}
	\\ \textbf{\Large{数学试卷}}
\end{center}

\begin{questions}
	\question
	\begin{enumerate}[label=(\arabic*)]
		\item 计算:$ \left(2\frac14\right)^\frac12 + \left(\frac{1}{10}\right) - (3.14)^0 +
			      \left(-\frac{27}{8}\right)^{\frac13} $
		      \begin{solution}
			      \begin{align*}
				       & = \left(\frac32\right)^{2\times\frac12} + 10^2 - 1 - \left(\frac32\right)^{3\times(\frac13)} \\
				       & = \frac32 + 99 - \frac32                                                                     \\
				       & = 99
			      \end{align*}
		      \end{solution}

		\item 求函数$y=\sqrt{x-2} + \dfrac{1}{x-3} + \lg(5-x)$的定义域。
		      \begin{solution}
			      根据题意有:
			      \begin{math}
				      \begin{cases}
					      x - 2 >= 0   \\
					      x - 3 \neq 0 \\
					      5 - x > 0
				      \end{cases}
			      \end{math}
			      则有定义域为$[2,3),(3,5]$

		      \end{solution}
		\item 解方程:$ 5^{x^2+2x} = 125 $.
		      \begin{solution}
			      由 $5^{x^2 + 2x} = 125 = 5^3 $得: $x^2 + 2x = 3$

			      分解因式得$(x+3)(x-1) = 0$,所以$x_1 = -3, x_2 = 1$
		      \end{solution}
		\item 计算:$-\log_3(\log_3\sqrt[3]{\sqrt[3]{\sqrt[3]{3}}})$
		      \begin{solution}
			      \begin{align*}
				      -\log_3(\log_3\sqrt[3]{\sqrt[3]{\sqrt[3]{3}}}) & = -\log_3(\log_3\sqrt[3]{\sqrt[3]{3^\frac13}}) \\
				                                                     & = -\log_3(\log_3\sqrt[3]{3^\frac19})           \\
				                                                     & = -\log_3(\log_33^\frac1{27})                  \\
				                                                     & = -\log_3\left(\frac{1}{27}\right)             \\
				                                                     & = -\log_3\left(3^{-3}\right)                   \\
				                                                     & = 3
			      \end{align*}
		      \end{solution}
		\item 把直角坐标方程 \( (x-3)^2 + y^2 = 9 \)化为极坐标方程。
		      \begin{solution}
			      设动点为$(\rho, \theta)$,则有$x=\rho\cos\theta, y=\rho\sin\theta$,代入直角坐标方程得:
			      \begin{equation*}
				      (\rho\cos\theta - 3)^2 + (\rho\sin\theta)^2 = 9
			      \end{equation*}
			      展开得
			      \begin{equation*}
				      \rho^2\cos^2\theta - 6\rho\cos\theta + 9 + \rho^2\sin^2\theta = 9
			      \end{equation*}
			      合并同类项得:
			      \begin{equation*}
				      \rho^2(\cos^2\theta + \sin^2\theta) - 6\rho\cos\theta = 0
			      \end{equation*}
			      因为$\cos^2\theta + \sin^2\theta = 1$,简化为:
			      \begin{equation*}
				      \rho(\rho - 6\cos\theta) = 0
			      \end{equation*}
			      则极坐标方程为
			      \begin{equation*}
				      \rho = 6\cos\theta
			      \end{equation*}
		      \end{solution}
		\item 计算: \( \displaystyle \lim_{n\to\infty}\frac{1+2+3+\cdots+n}{n^2} \)
		      \begin{solution}
			      根据等差数列求和公式得
			      \begin{equation*}
				      1 + 2 + 3 + \cdots + n = \frac{n(n+1)}{2}
			      \end{equation*}
			      则原式
			      \begin{equation*}
				      \lim_{n\to\infty}\frac{1+2+3+\cdots+n}{n^2} = \lim_{n\to\infty}(\frac12 + \frac{1}{2n}) = \frac12
			      \end{equation*}
		      \end{solution}
		\item 分解因式:$ x^4 - 2x^2y - 3y^2 + 8y - 4 $.
		      \begin{solution}
			      \begin{align*}
				      x^4 - 2x^2y - 3y^2 + 8y - 4 & = x^4 - 2x^2y + y^2 - (4y^2 - 8y + 4)  \\
				                                  & = (x^2 - y)^2 - (2y - 2)^2             \\
				                                  & = (x^2 - y + 2y - 2)(x^2 - y - 2y + 2) \\
				                                  & = (x^2 + y - 2)(x^2 - 3y + 2)
			      \end{align*}
		      \end{solution}
	\end{enumerate}
	\question 过抛物线$ y^2 = 4x
	$的焦点作倾斜角为$\frac34\pi$的直线,它与抛物线相交于$A$、$B$两点。求$A$、$B$两点间的距离。
	\begin{solution}
		抛物线的焦点为$(0,1)$,斜角为$\frac34\pi$的直线的斜率为$-1$,设直线的方程为$ y = -x + b $,将点$(0,1)$代入解得
		$b = 1$,则直线方程为$ y=-x+1 $。将直线方程代入抛物线方程得:
		\begin{equation*}
			(-x + 1)^2 = 4x
		\end{equation*}
		展开得
		\begin{align*}
			x^2 - 2x + 1 & = 4x \\
			x^2 -6x + 1  & = 0  \\
		\end{align*}
		解得$ x_1,x_2 = \frac{6 \pm \sqrt{32}}{2} = 3 \pm 2\sqrt{2} $,代入直线方程可得$ y_1,y_2 = -(3\pm2\sqrt{2}) +
			1$,则点$A$的坐标为$(3+2\sqrt{2}, -2 - 2\sqrt{2})$,$B$点的坐标为$(3-2\sqrt{2}, -2 + 2\sqrt{2})$,$AB$的距离为:
		\begin{align*}
			\sqrt{(3-2\sqrt{2} - 3 - 2\sqrt{2})^2 + (-2 + 2\sqrt{2} + 2 + 2\sqrt{2})^2}
			 & = \sqrt{32 + 32} \\
			 & = 8
		\end{align*}
	\end{solution}
	\question 在直角三角形$ \triangle{ABC} $中,$ \angle{ACB} = 90^\circ
	$,$CD$、$CE$分别为斜边$AB$上的高和中线,且$\angle{BCD}$与$\angle{ACD}$之比为$3:1$,求证:$CD=DE$。

	\begin{tikzpicture}
		\tkzDefPoints{-1/0/A, 0/0/D, 0/2/C}
		\tkzDefLine[orthogonal=through C](A,C) \tkzGetPoint{x}
		\tkzInterLL(C,x)(A,D) \tkzGetPoint{B}
		\tkzDefPointOnLine[pos=0.5](A,B) \tkzGetPoint{E}

		\tkzDrawPolygon(A,B,C)
		\tkzDrawSegments(C,D C,E)

		\tkzLabelPoints[below](A,D,E,B)
		\tkzLabelPoint[above](C){$C$}
	\end{tikzpicture}
	\begin{solution}
		\begin{align*}
			 & \text{设}\angle{ACD} = \alpha                               \\
			 & \because CE\text{是直角三角形的中线}                                \\
			 & \therefore CE = EB                                         \\
			 & \therefore \angle{ECB} = \angle{CBE}                       \\
			 & \because \angle{DCB} = 3\angle{ACD}                        \\
			 & \because \angle{ACD} = \angle{CBE}                         \\
			 & \therefore \angle{DCE} = 3\alpha - \alpha = 2\alpha        \\
			 & \because \angle{DEC} = \angle{EBC} + \angle{ECB} = 2\alpha \\
			 & \therefore \angle{DEC} = \angle{DCE}                       \\
			 & \therefore DE = DC
		\end{align*}

	\end{solution}
\end{questions}

\end{document}
