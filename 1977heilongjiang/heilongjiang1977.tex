%!tex program = lualatex
\documentclass[answers]{exam}
\usepackage{silence}
\WarningFilter{latexfont}{Font shape}
\WarningFilter{latexfont}{Some font}
\usepackage{ctex}
\usepackage[margin=2cm]{geometry}
\usepackage{amsmath, amssymb, amsthm, arcs}
\usepackage{siunitx}
\usepackage{csquotes}
\usepackage{tikz, pgfplots, tikz-3dplot}
\usepackage[lua]{tkz-euclide}
\usetikzlibrary{
	angles,
	backgrounds,
	calc,
	decorations.pathmorphing,
	decorations.pathreplacing,
	decorations.text,
	intersections,
	patterns,
	petri,
	positioning,
	quotes,
	shapes,
	shapes.symbols,
}
\usepackage{graphicx, wrapfig}
\usepackage{caption}
\pagestyle{empty}
\newcounter{xcord}
\newcounter{ycord}
\newcounter{total}
\renewcommand{\labelenumi}{\textbf{\ifnum\value{enumi}<10 0\fi\arabic{enumi})}}

\pgfplotsset{compat=1.18}

\CorrectChoiceEmphasis{\color{blue!70!green}\bfseries}
\renewcommand{\solutiontitle}{\textbf{解:}}
\renewcommand{\choicelabel}{(\Alph{choice})}

\usepackage{array, tabularx}
\newcolumntype{C}{>{\centering\arraybackslash}X}
\newcolumntype{B}{>{\centering\bfseries\arraybackslash}X}

\usepackage{fontawesome5}
\usepackage{enumitem}
\newcommand\regularHandPointRight{\faHandPointRight[regular]}

\newenvironment{mathenum}
{\begin{enumerate}[label=\arabic*.]}
		{\end{enumerate}}

\newenvironment{penum}
{\begin{enumerate}[label=(\arabic*)]}
		{\end{enumerate}}
\makeatletter
\providecommand\@gobblethree[3]{}
\patchcmd{\over@under@arc}
{\@gobbletwo}
{\@gobblethree}
{}{}
\makeatother

\newcommand*\circled[1]{\tikz[baseline=(char.base)]{
		\node[shape=circle,draw,inner sep=1pt] (char) {#1};}}
\setlength{\answerclearance}{6pt}
\newcommand\hfs{\hfill (\hspace{2cm})}

\usepackage[lua]{tkz-euclide}

\begin{document}
\begin{center}
	\textbf{1977年普通高等学校招生考试(黑龙江卷)}

	\textbf{\Large{数学试卷}}
\end{center}

\begin{questions}
	\question 解答下列各题:
	\begin{enumerate}[label=(\arabic*)]
		\item 解方程:$\sqrt{3x+4} = 4$。
		      \begin{solution}
			      方程两边平方得:
			      \begin{math}
				      3x + 4 = 16
			      \end{math}
			      移项整理得
			      \begin{align*}
				      3x & = 12 \\
				      x  & = 4
			      \end{align*}

		      \end{solution}
		\item 解不等式:$|x| < 5$
		      \begin{solution}
			      \begin{math}
				      -5 < x < 5
			      \end{math}

		      \end{solution}
		\item 已知正三角形的外接圆半径为$6\sqrt{3}$cm,求它的边长。
		      \begin{solution}
			      \begin{tikzpicture}[scale=.3]
				      \tkzDefPoints{0/0/O, 6sqrt(3)/0/A}
				      \tkzDefPoint(90:6sqrt(3)){B}
				      \tkzDefPoint(210:6sqrt(3)){C}
				      \tkzDefPoint(330:6sqrt(3)){D}
				      \tkzDefLine[orthogonal=through O](B,C) \tkzGetPoint{x}
				      \tkzInterLL(O,x)(B,C) \tkzGetPoint{E}

				      \tkzDrawCircle(O,A)
				      \tkzDrawPolygon(B,C,D)
				      \tkzDrawPoint(O)
				      \tkzLabelPoint[below right](O){$O$}
				      \tkzDrawSegments(B,O C,O O,E)

				      \tkzMarkSegment[dim={$6\sqrt{3}$, 16pt, right=6pt}, mark=](B,O)
				      \tkzMarkAngle[size=2](C,B,O)
				      \tkzLabelAngle[pos=3.5](C,B,O){$30^\circ$}
				      \tkzLabelPoint[left](E){$E$}
				      \tkzLabelPoint[above](B){$B$}
			      \end{tikzpicture}
			      可以计算出$BE=6$,则正三角形的边长为$12$。
		      \end{solution}

	\end{enumerate}

	\question 计算下列各题:
	\begin{enumerate}[label=(\arabic*)]
		\item $\sqrt{m^2 - 2ma + a^2} $
		      \begin{solution}
			      \begin{align*}
				      \sqrt{m^2 - 2ma + a^2} & = \sqrt{(m-a)^2} \\
				                             & = |m-a|
			      \end{align*}
		      \end{solution}
		\item $\cos78^\circ\cdot\cos3^\circ + \cos12^\circ\cdot\sin3^\circ$。
		      \begin{solution}
			      \begin{align*}
				      \cos78^\circ & = \cos(90^\circ - 12^\circ)                           \\
				                   & = \cos90^\circ\cos12^\circ + \sin90^\circ\sin12^\circ \\
				                   & = \sin12^\circ
			      \end{align*}
			      代入原式得:
			      \begin{align*}
				      \cos78^\circ\cdot\cos3^\circ + \cos12^\circ\cdot\sin3^\circ
				       & = \sin12^\circ\cos3^\circ + \cos12^\circ\sin3^\circ \\
				       & = \sin15^\circ                                      \\
				       & = \sin\left(\frac{30^\circ}{2}\right)               \\
				       & = \sqrt{\frac{1-\cos30^\circ}{2}}                   \\
				       & = \frac{\sqrt{2-\sqrt{3}}}{2}
			      \end{align*}
		      \end{solution}
		\item $\arcsin\left(\cos\dfrac\pi6\right)$.
		      \begin{solution}
			      \begin{align*}
				      \arcsin\left(\cos\dfrac\pi6\right) & = \arcsin(\dfrac{\sqrt{3}}2) \\
				                                         & = \frac{\pi}{3}
			      \end{align*}
		      \end{solution}
	\end{enumerate}
	\question 解下列各题:
	\begin{enumerate}[label=(\arabic*)]
		\item 解方程: $3^{x+1} - 9^{\frac{x}{2}} = 18$.
		      \begin{solution}
			      \begin{align*}
				      3^{x+1} - 9^{\frac{x}{2}}        & = 18 \\
				      3\cdot 3^x - (3^2)^{\frac{x}{2}} & = 18 \\
				      3^x                              & = 9  \\
				      x                                & = 2
			      \end{align*}
		      \end{solution}
		\item 求数列$2,4,8,16,\cdots $前十项的和。
		      \begin{solution}
			      \begin{align*}
				      S_n    & = a_0\frac{1-q^n}{1-q}, (q=2, a_0=2) \\
				      S_{10} & = 2\frac{1-2^{10}}{1-2}              \\
				             & = 2^{11} - 2                         \\
				             & = 2046
			      \end{align*}
		      \end{solution}
	\end{enumerate}

	\question 解下列各题:
	\begin{enumerate}[label=(\arabic*)]
		\item 圆锥的高为$6$cm,母线和底面半径的夹角为$30^\circ$,求它的侧面积。
		      \begin{solution}
			      根据题目中提供的信息可计算得底面半径为$r = 6\sqrt{3}$,母线的长度为$l = 12$,则侧面积为
			      \begin{math}
				      S = \pi r l = 72\sqrt{3} \text{cm}^2
			      \end{math}
		      \end{solution}
		\item 求过点$(1,4)$且与直线$2x - 5y + 3 = 0$垂直的直线方程。
		      \begin{solution}
			      直线$2x - 5y + 3 + 0$的斜率$k=\frac25$,则与之垂直的直线的斜率为$-\frac52$。

			      设所求直线方程为$y =
				      -\frac52x+b$, 并将点$(1,4)$代入得$b = \frac{13}{2}$。
			      所以直线方程为:$2y + 5x - 13 = 0$。
		      \end{solution}
	\end{enumerate}

	\question 如果$\triangle{ABC}$的$\angle{A}$的平分线交$BC$于$D$,交它的外接圆于$E$,那么$AB\cdot AC = AD \cdot AE$。
	\begin{figure*}[htbp]
		\centering
		\begin{tikzpicture}
			\tkzDefPoints{0/0/A, 3/0/B, 2.5/2.5/C}
			\tkzDefLine[bisector](B,A,C) \tkzGetPoint{x}
			\tkzDefCircle(A,B,C) \tkzGetPoint{O}
			\tkzInterLL(A,x)(B,C) \tkzGetPoint{D}
			\tkzInterLC(A,x)(O,A) \tkzGetSecondPoint{E}

			\tkzDrawPolygon(A,B,C)
			\tkzDrawCircle(O,A)
			\tkzDrawSegment(A,E)

			\tkzLabelPoint[left](A){$A$}
			\tkzLabelPoint[right](B){$B$}
			\tkzLabelPoint[above](C){$C$}
			\tkzLabelPoint[above left](D){$D$}
			\tkzLabelPoint[right](E){$E$}

			\tkzDrawSegments[red, dashed](C,E E,B)

		\end{tikzpicture}
	\end{figure*}
	\begin{solution}
		\begin{mathenum}
			\item 作辅助线$CE$和$BE$
			\item \because $\overset{\frown}{CE}$对应圆周角$\angle{CAE}$和$\angle{CBE}$
			\\ \therefore\ $\angle{CAE} = \angle{CBE}$
			\\ \because\ $\angle{CDA} = \angle{EDB} $
			\\ \therefore\ $\triangle{CAD} \sim \triangle{EBD}$
			\item \because\ $\overset{\frown}{BE}$对应圆周角$\angle{BAE} = \angle{BCE}$
			\\ \therefore\ $\angle{BAE} = \angle{EAC}$
			\\ \because $\angle{BAE} = \angle{BAC}$
			\\ \therefore\ $\overset{\frown}{CE} = \overset{\frown}{BE}$
			\\ \therefore\ $\angle{BCE} = \angle{CBE}$
			\\ \therefore\ $\angle{EBD} = \angle{BAE}$
			\\ \because\ $\angle{BEA} = \angle{AEB}$
			\\ \therefore\ $\triangle{EBD} \sim \triangle{EAB}$
			\\ \therefore\ $\triangle{EAB} \sim \triangle{CAD}$
			\\ \therefore\ $\dfrac{AC}{AE}={AD}{AB}$
			\\ \therefore\ $AC\cdot AB = AD\cdot AE$
		\end{mathenum}

	\end{solution}
	\question
	前进大队响应毛主席关于\enquote{绿化祖国}的伟大号召,1975年造林$200$亩,又知1975年到1977年三年内共造林$728$亩,求后两年造林面积的年平均增长率是多少?
	\begin{solution}
		设年平均增长率为$q$,根据等比数列求和公式
		\begin{math}
			S_n = a_0\frac{1-q^n}{1-q} (a_0 = 200, n = 3, S_3 = 728)
		\end{math}

		代入得:
		\begin{math}
			728 = 200 \frac{1-q^3}{1-q} = 200 \frac{(1-q)(1 + q + q^2)}{1-q} = 200(1 + q + q^2)
		\end{math}
		化简得:
		\begin{align*}
			25q^2 + 25q - 66  & = 0 \\
			(5q - 6)(5q + 11) & = 0 \\
		\end{align*}
		因为$q > 0$所以$q=1.2$
	\end{solution}

	\question 解方程:
	\begin{math}
		\lg(2^x + 2x - 16) = x(1-\lg5)
	\end{math}.
	\begin{solution}
		\begin{align*}
			\lg(2^x + 2x - 16) & = x(\lg10 - \lg5) \\
			                   & = x\lg2           \\
			                   & = \lg2^x          \\
			2^x + 2x - 16      & = 2^x             \\
			2x - 16            & = 0               \\
			x                  & = 8
		\end{align*}
	\end{solution}

\end{questions}

\end{document}
