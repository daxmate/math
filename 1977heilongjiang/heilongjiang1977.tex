%!tex program = lualatex
\documentclass[answers]{exam}
\usepackage{silence}
\WarningFilter{latexfont}{Font shape}
\WarningFilter{latexfont}{Some font}
\usepackage{ctex}
\usepackage[margin=2cm]{geometry}
\usepackage{amsmath, amssymb, amsthm, arcs}
\usepackage{siunitx}
\usepackage{csquotes}
\usepackage{tikz, pgfplots, tikz-3dplot}
\usepackage[lua]{tkz-euclide}
\usetikzlibrary{
	angles,
	backgrounds,
	calc,
	decorations.pathmorphing,
	decorations.pathreplacing,
	decorations.text,
	intersections,
	patterns,
	petri,
	positioning,
	quotes,
	shapes,
	shapes.symbols,
}
\usepackage{graphicx, wrapfig}
\usepackage{caption}
\pagestyle{empty}
\newcounter{xcord}
\newcounter{ycord}
\newcounter{total}
\renewcommand{\labelenumi}{\textbf{\ifnum\value{enumi}<10 0\fi\arabic{enumi})}}

\pgfplotsset{compat=1.18}

\CorrectChoiceEmphasis{\color{blue!70!green}\bfseries}
\renewcommand{\solutiontitle}{\textbf{解:}}
\renewcommand{\choicelabel}{(\Alph{choice})}

\usepackage{array, tabularx}
\newcolumntype{C}{>{\centering\arraybackslash}X}
\newcolumntype{B}{>{\centering\bfseries\arraybackslash}X}

\usepackage{fontawesome5}
\usepackage{enumitem}
\newcommand\regularHandPointRight{\faHandPointRight[regular]}

\newenvironment{mathenum}
{\begin{enumerate}[label=\arabic*.]}
		{\end{enumerate}}

\newenvironment{penum}
{\begin{enumerate}[label=(\arabic*)]}
		{\end{enumerate}}
\makeatletter
\providecommand\@gobblethree[3]{}
\patchcmd{\over@under@arc}
{\@gobbletwo}
{\@gobblethree}
{}{}
\makeatother

\newcommand*\circled[1]{\tikz[baseline=(char.base)]{
		\node[shape=circle,draw,inner sep=1pt] (char) {#1};}}
\setlength{\answerclearance}{6pt}
\newcommand\hfs{\hfill (\hspace{2cm})}

\usepackage[lua]{tkz-euclide}

\begin{document}
\begin{center}
	\textbf{1977年普通高等学校招生考试(黑龙江卷)}

	\textbf{\Large{数学试卷}}
\end{center}

\begin{questions}
	\question 解答下列各题:
	\begin{enumerate}[label=(\arabic*)]
		\item 解方程:$\sqrt{3x+4} = 4$。
		      \begin{solution}
			      方程两边平方得:
			      \begin{math}
				      3x + 4 = 16
			      \end{math}
			      移项整理得
			      \begin{align*}
				      3x & = 12 \\
				      x  & = 4
			      \end{align*}

		      \end{solution}
		\item 解不等式:$|x| < 5$
		      \begin{solution}
			      \begin{math}
				      -5 < x < 5
			      \end{math}

		      \end{solution}
		\item 已知正三角形的外接圆半径为$6\sqrt{3}$cm,求它的边长。
		      \begin{solution}
			      \begin{tikzpicture}[scale=.3]
				      \tkzDefPoints{0/0/O, 6sqrt(3)/0/A}
				      \tkzDefPoint(90:6sqrt(3)){B}
				      \tkzDefPoint(210:6sqrt(3)){C}
				      \tkzDefPoint(330:6sqrt(3)){D}
				      \tkzDefLine[orthogonal=through O](B,C) \tkzGetPoint{x}
				      \tkzInterLL(O,x)(B,C) \tkzGetPoint{E}

				      \tkzDrawCircle(O,A)
				      \tkzDrawPolygon(B,C,D)
				      \tkzDrawPoint(O)
				      \tkzLabelPoint[below right](O){$O$}
				      \tkzDrawSegments(B,O C,O O,E)

				      \tkzMarkSegment[dim={$6\sqrt{3}$, 16pt, right=6pt}, mark=](B,O)
					  \tkzMarkAngle[size=2](C,B,O)
					  \tkzLabelAngle[pos=3.5](C,B,O){$30^\circ$}
					  \tkzLabelPoint[left](E){$E$}
					  \tkzLabelPoint[above](B){$B$}
			      \end{tikzpicture}
				  可以计算出$BE=6$,则正三角形的边长为$12$。
		      \end{solution}

	\end{enumerate}

\end{questions}

\end{document}
