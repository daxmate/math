%!tex program = lualatex
\documentclass[answers]{exam}
\usepackage{silence}
\WarningFilter{latexfont}{Font shape}
\WarningFilter{latexfont}{Some font}
\usepackage{ctex}
\usepackage[margin=2cm]{geometry}
\usepackage{amsmath, amssymb, amsthm, arcs}
\usepackage{siunitx}
\usepackage{csquotes}
\usepackage{tikz, pgfplots, tikz-3dplot}
\usepackage[lua]{tkz-euclide}
\usetikzlibrary{
	angles,
	backgrounds,
	calc,
	decorations.pathmorphing,
	decorations.pathreplacing,
	decorations.text,
	intersections,
	patterns,
	petri,
	positioning,
	quotes,
	shapes,
	shapes.symbols,
}
\usepackage{graphicx, wrapfig}
\usepackage{caption}
\pagestyle{empty}
\newcounter{xcord}
\newcounter{ycord}
\newcounter{total}
\renewcommand{\labelenumi}{\textbf{\ifnum\value{enumi}<10 0\fi\arabic{enumi})}}

\pgfplotsset{compat=1.18}

\CorrectChoiceEmphasis{\color{blue!70!green}\bfseries}
\renewcommand{\solutiontitle}{\textbf{解:}}
\renewcommand{\choicelabel}{(\Alph{choice})}

\usepackage{array, tabularx}
\newcolumntype{C}{>{\centering\arraybackslash}X}
\newcolumntype{B}{>{\centering\bfseries\arraybackslash}X}

\usepackage{fontawesome5}
\usepackage{enumitem}
\newcommand\regularHandPointRight{\faHandPointRight[regular]}

\newenvironment{mathenum}
{\begin{enumerate}[label=\arabic*.]}
		{\end{enumerate}}

\newenvironment{penum}
{\begin{enumerate}[label=(\arabic*)]}
		{\end{enumerate}}
\makeatletter
\providecommand\@gobblethree[3]{}
\patchcmd{\over@under@arc}
{\@gobbletwo}
{\@gobblethree}
{}{}
\makeatother

\newcommand*\circled[1]{\tikz[baseline=(char.base)]{
		\node[shape=circle,draw,inner sep=1pt] (char) {#1};}}
\setlength{\answerclearance}{6pt}
\newcommand\hfs{\hfill (\hspace{2cm})}

\usepackage[lua]{tkz-euclide}
\begin{document}
\begin{center}
	\textbf{1977年普通高等学校招生考试(福建卷)}

	\textbf{\huge{理科数学}}
\end{center}
\begin{questions}
	\question
	\begin{enumerate}[label=(\arabic*)]
		\item 计算: \( 5 - 3 \times \left[(-3\frac38)^{-\frac13} + 1031 \times (0.25 - 2^{-2})\right] \div 9^0 \).
		      \begin{solution}
			      \begin{align*}
				      \text{原式} & = 5 - 3 \times \left[ (-3\frac38)^{-\frac13} + 1031 \times 0 \right] \div 1 \\
				                & = 5 - 3 \times (-\frac{8}{27})^{\frac13}                                    \\
				                & = 5 - \sqrt[3]{3^3 \times (-\frac{8}{27}})                                  \\
				                & = 5 + 2                                                                     \\
				                & = 7
			      \end{align*}
		      \end{solution}

		\item \( y = \dfrac{\cos{160^\circ} - \cos{170^\circ}}{\tan{155^\circ}} \)的值是正的还是负的?为什么?
		      \begin{solution}
			      \begin{tikzpicture}[scale=3]
				      \tkzDefPoints{0/0/O, 1/0/A, -1/0/A'}
				      \tkzDefShiftPointCoord[O](160:1){B}
				      \tkzDefShiftPointCoord[O](170:1){C}
				      \tkzDefShiftPointCoord[O](155:1){D}

				      \tkzDefLine[orthogonal=through B](O,A) \tkzGetPoint{x}
				      \tkzInterLL(B,x)(O,A) \tkzGetPoint{B'}
				      \tkzDefLine[orthogonal=through C](O,A) \tkzGetPoint{x}
				      \tkzInterLL(C,x)(O,A) \tkzGetPoint{C'}
				      \tkzDefLine[orthogonal=through D](O,A) \tkzGetPoint{x}
				      \tkzInterLL(D,x)(O,A) \tkzGetPoint{D'}

				      \tkzDrawSegments(A,A' O,B)
				      \tkzDrawSegment[dim={$\cos{160^\circ}$, .25cm, below=2pt}](O,B')
				      \tkzDrawSegments(A,O O,C)
				      \tkzDrawSegment[dim={$\cos{170^\circ}$, .75cm, below=2pt}](O,C')
				      \tkzDrawSegments(A,O O,D)
				      \tkzDrawSegments[blue, thick](D,D' D',O)
			      \end{tikzpicture}

			      由三角函数的定义有\( \cos{160^\circ} < 0, \cos{170^\circ} < 0, \tan{155^\circ} < 0 \),并且有 \(
			      \cos{160^\circ} > \cos{170^\circ} \),则分子为正数,而且分母为负数,所以 \( y \)的值为负。
		      \end{solution}

	\end{enumerate}
\end{questions}
\end{document}
