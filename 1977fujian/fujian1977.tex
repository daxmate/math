%!tex program = lualatex
\documentclass[answers]{exam}
\usepackage{silence}
\WarningFilter{latexfont}{Font shape}
\WarningFilter{latexfont}{Some font}
\usepackage{ctex}
\usepackage[margin=2cm]{geometry}
\usepackage{amsmath, amssymb, amsthm, arcs}
\usepackage{siunitx}
\usepackage{csquotes}
\usepackage{tikz, pgfplots, tikz-3dplot}
\usepackage[lua]{tkz-euclide}
\usetikzlibrary{
	angles,
	backgrounds,
	calc,
	decorations.pathmorphing,
	decorations.pathreplacing,
	decorations.text,
	intersections,
	patterns,
	petri,
	positioning,
	quotes,
	shapes,
	shapes.symbols,
}
\usepackage{graphicx, wrapfig}
\usepackage{caption}
\pagestyle{empty}
\newcounter{xcord}
\newcounter{ycord}
\newcounter{total}
\renewcommand{\labelenumi}{\textbf{\ifnum\value{enumi}<10 0\fi\arabic{enumi})}}

\pgfplotsset{compat=1.18}

\CorrectChoiceEmphasis{\color{blue!70!green}\bfseries}
\renewcommand{\solutiontitle}{\textbf{解:}}
\renewcommand{\choicelabel}{(\Alph{choice})}

\usepackage{array, tabularx}
\newcolumntype{C}{>{\centering\arraybackslash}X}
\newcolumntype{B}{>{\centering\bfseries\arraybackslash}X}

\usepackage{fontawesome5}
\usepackage{enumitem}
\newcommand\regularHandPointRight{\faHandPointRight[regular]}

\newenvironment{mathenum}
{\begin{enumerate}[label=\arabic*.]}
		{\end{enumerate}}

\newenvironment{penum}
{\begin{enumerate}[label=(\arabic*)]}
		{\end{enumerate}}
\makeatletter
\providecommand\@gobblethree[3]{}
\patchcmd{\over@under@arc}
{\@gobbletwo}
{\@gobblethree}
{}{}
\makeatother

\newcommand*\circled[1]{\tikz[baseline=(char.base)]{
		\node[shape=circle,draw,inner sep=1pt] (char) {#1};}}
\setlength{\answerclearance}{6pt}
\newcommand\hfs{\hfill (\hspace{2cm})}

\usepackage[lua]{tkz-euclide}
\begin{document}
\begin{center}
	\textbf{1977年普通高等学校招生考试(福建卷)}

	\textbf{\huge{理科数学}}
\end{center}
\begin{questions}
	\question
	\begin{enumerate}[label=(\arabic*)]
		\item 计算: \( 5 - 3 \times \left[(-3\frac38)^{-\frac13} + 1031 \times (0.25 - 2^{-2})\right] \div 9^0 \).
		      \begin{solution}
			      \begin{align*}
				      \text{原式} & = 5 - 3 \times \left[ (-3\frac38)^{-\frac13} + 1031 \times 0 \right] \div 1 \\
				                & = 5 - 3 \times (-\frac{8}{27})^{\frac13}                                    \\
				                & = 5 - \sqrt[3]{3^3 \times (-\frac{8}{27}})                                  \\
				                & = 5 + 2                                                                     \\
				                & = 7
			      \end{align*}
		      \end{solution}

		\item \( y = \dfrac{\cos{160^\circ} - \cos{170^\circ}}{\tan{155^\circ}} \)的值是正的还是负的?为什么?
		      \begin{solution}
			      \begin{tikzpicture}[scale=3]
				      \tkzDefPoints{0/0/O, 1/0/A, -1/0/A'}
				      \tkzDefShiftPointCoord[O](160:1){B}
				      \tkzDefShiftPointCoord[O](170:1){C}
				      \tkzDefShiftPointCoord[O](155:1){D}

				      \tkzDefLine[orthogonal=through B](O,A) \tkzGetPoint{x}
				      \tkzInterLL(B,x)(O,A) \tkzGetPoint{B'}
				      \tkzDefLine[orthogonal=through C](O,A) \tkzGetPoint{x}
				      \tkzInterLL(C,x)(O,A) \tkzGetPoint{C'}
				      \tkzDefLine[orthogonal=through D](O,A) \tkzGetPoint{x}
				      \tkzInterLL(D,x)(O,A) \tkzGetPoint{D'}

				      \tkzDrawSegments(A,A' O,B)
				      \tkzDrawSegment[dim={$\cos{160^\circ}$, .25cm, below=2pt}](O,B')
				      \tkzDrawSegments(A,O O,C)
				      \tkzDrawSegment[dim={$\cos{170^\circ}$, .75cm, below=2pt}](O,C')
				      \tkzDrawSegments(A,O O,D)
				      \tkzDrawSegments[blue, thick](D,D' D',O)
			      \end{tikzpicture}

			      由三角函数的定义有\( \cos{160^\circ} < 0, \cos{170^\circ} < 0, \tan{155^\circ} < 0 \),并且有 \(
			      \cos{160^\circ} > \cos{170^\circ} \),则分子为正数,而且分母为负数,所以 \( y \)的值为负。
		      \end{solution}

		\item 求函数 \( y = \dfrac{\lg(2-x)}{\sqrt{x-1}} \)的定义域。
		      \begin{solution}
			      根据题意有
			      \begin{math}
				      \begin{cases}
					      2 - x > 0 \\
					      x - 1 > 0,
				      \end{cases}
			      \end{math}
			      则函数的定义域为 \( 1 < x < 2 \)
		      \end{solution}
		\item
		      \begin{minipage}[t]{0.5\textwidth}
			      如图,在梯形 \( ABCD \)中, \( DM = MP = PA, MN \parallel PQ \parallel AB \), \( DC = 2\text{cm}, AB = 3.5\text{cm} \),求 \( MN \)和 \( PQ \)的长。
		      \end{minipage}\hspace{5em}
		      \begin{tikzpicture}[baseline=(current bounding box.north)]
			      \tkzDefPoints{0/0/A, 3/0/B, 0.5/3/D, 2.5/3/C}
			      \tkzDefPointOnLine[pos={1/3}](A,D)\tkzGetPoint{P}
			      \tkzDefPointOnLine[pos={2/3}](A,D)\tkzGetPoint{M}
			      \tkzDefPointOnLine[pos={1/3}](C,B)\tkzGetPoint{N}
			      \tkzDefPointOnLine[pos={2/3}](C,B)\tkzGetPoint{Q}

			      \tkzDrawPolygon(A,B,C,D)
			      \tkzDrawSegments(P,Q M,N)

			      \tkzLabelSegment[above](D,C){$2$}
			      \tkzLabelSegment[below](A,B){$3.5$}

			      \tkzLabelPoints[left](D,M,P,A)
			      \tkzLabelPoints[right](C,N,Q,B)

		      \end{tikzpicture}

		      \begin{solution}
			      \begin{align*}
				       & \because MN \parallel PQ \parallel AB \text{并且} DM = MP = PA \\
				       & \therefore \frac{DC}{MN} = \frac{MN}{PQ} = \frac{PQ}{AB}     \\
				       & \text{将数值代入得:}                                               \\
				       & \frac{2}{MN} = \frac{MN}{PQ} = \frac{PQ}{3.5}                \\
				       & \text{则有:} PQ = \frac{MN^2}{2} \text{和} PQ \cdot MN = 7      \\
				       & MN = \sqrt[3]{14}                                            \\
				       & PQ = \sqrt[3]{14^2} / 2                                      \\
			      \end{align*}
		      \end{solution}
		\item 已经 \( \lg3=0.4771, \lg{x}=-3.5229 \),求 \( x \)。
		      \begin{solution}
			      \begin{align*}
				       & \because \lg3                              = 0.4771                  \\
				       & \therefore \lg{\frac{3}{10}}               = \lg3 - \lg10 = -0.5229  \\
				       & \text{另有} \lg0.001                       = -3                        \\
				       & \text{则} \lg(0.001 \times \frac{3}{10} )  = -3 + (-0.5229) = -3.5229 \\
				       & \therefore x = \frac{3}{10000}
			      \end{align*}
		      \end{solution}
		\item 求 \(\displaystyle \lim_{x\to1}\frac{x-1}{x^2-3x+2} \)。
		      \begin{solution}
			      \begin{align*}
				      \frac{x-1}{x^2 - 3x + 2} & = \frac{x-1}{(x-2)(x-1)} \\
				                               & = \frac{1}{x-2}
			      \end{align*}
			      则 \( \displaystyle \lim_{x\to1}\frac{x-1}{x^2-3x+2}=-1 \)
		      \end{solution}
		\item 解方程: \( \sqrt{4x+1} - 2x + 1 = 0 \)。
		      \begin{solution}
			      移项得:
			      \begin{math}
				      \sqrt{4x+1} = 2x - 1
			      \end{math}

			      两边平方得:
			      \begin{math}
				      4x + 1 = 4x^2 - 4x + 1
			      \end{math}

			      整理得:
			      \begin{math}
				      4x^2 -8x = 0
			      \end{math}

			      提取同类项得:
			      \begin{math}
				      4x(x-2) = 0
			      \end{math}

			      则
			      \begin{math}
				      x_1 = 0, x_2 = 2
			      \end{math}

			      代入验算 \( x_1 = 0  \)不符合条件,所以解为 \( x=2 \)。
		      \end{solution}
		\item 化简: \( \displaystyle \frac{a^{2n+1} - 6a^{2n} + 9a^{2n-1}}{a^{n+1} - 4a^n + 3a^{n-1}} \)。
		      \begin{solution}
			      \begin{align*}
				      \frac{a^{2n+1} - 6a^{2n} + 9a^{2n-1}}{a^{n+1} - 4a^n + 3a^{n-1}} & = \frac{a^{2n-1}(a^2 - 6a + 9)}{a^{n-1}(a^2-4a+3)}
				      \\
				                                                                       & = \frac{a^n(a-3)^2}{(a-1)(a-3)}
				      \\
				                                                                       & = \frac{a^n(a-3)}{a-1}
			      \end{align*}
		      \end{solution}
		\item 求函数 \( y = 2 - 5x - 3x^2 \)的极值。
		      \begin{solution}
			      \begin{enumerate}[label=\alph*.]
				      \item 对函数求导
				            \begin{math}
					            \frac{\text{d}y}{\text{d}x} = -5 - 6x
				            \end{math}
				      \item 则有 \( x = -\frac56 \)时函数有极值。
				      \item 将 \( x = -\frac56 \)代入函数,得到极值:
				            \begin{math}
					            y = 2 - 5(-\frac56) - 3(-\frac56)^2 = \frac{49}{12}
				            \end{math}

			      \end{enumerate}
		      \end{solution}
	\end{enumerate}
\end{questions}
\end{document}
