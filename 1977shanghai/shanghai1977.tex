%!tex program = lualatex
\documentclass[answers]{exam}
\usepackage{silence}
\WarningFilter{latexfont}{Font shape}
\WarningFilter{latexfont}{Some font}
\usepackage{ctex}
\usepackage[margin=2cm]{geometry}
\usepackage{amsmath, amssymb, amsthm, arcs}
\usepackage{siunitx}
\usepackage{csquotes}
\usepackage{tikz, pgfplots, tikz-3dplot}
\usepackage[lua]{tkz-euclide}
\usetikzlibrary{
	angles,
	backgrounds,
	calc,
	decorations.pathmorphing,
	decorations.pathreplacing,
	decorations.text,
	intersections,
	patterns,
	petri,
	positioning,
	quotes,
	shapes,
	shapes.symbols,
}
\usepackage{graphicx, wrapfig}
\usepackage{caption}
\pagestyle{empty}
\newcounter{xcord}
\newcounter{ycord}
\newcounter{total}
\renewcommand{\labelenumi}{\textbf{\ifnum\value{enumi}<10 0\fi\arabic{enumi})}}

\pgfplotsset{compat=1.18}

\CorrectChoiceEmphasis{\color{blue!70!green}\bfseries}
\renewcommand{\solutiontitle}{\textbf{解:}}
\renewcommand{\choicelabel}{(\Alph{choice})}

\usepackage{array, tabularx}
\newcolumntype{C}{>{\centering\arraybackslash}X}
\newcolumntype{B}{>{\centering\bfseries\arraybackslash}X}

\usepackage{fontawesome5}
\usepackage{enumitem}
\newcommand\regularHandPointRight{\faHandPointRight[regular]}

\newenvironment{mathenum}
{\begin{enumerate}[label=\arabic*.]}
		{\end{enumerate}}

\newenvironment{penum}
{\begin{enumerate}[label=(\arabic*)]}
		{\end{enumerate}}
\makeatletter
\providecommand\@gobblethree[3]{}
\patchcmd{\over@under@arc}
{\@gobbletwo}
{\@gobblethree}
{}{}
\makeatother

\newcommand*\circled[1]{\tikz[baseline=(char.base)]{
		\node[shape=circle,draw,inner sep=1pt] (char) {#1};}}
\setlength{\answerclearance}{6pt}
\newcommand\hfs{\hfill (\hspace{2cm})}

\usepackage[lua]{tkz-euclide}

\begin{document}
\begin{center}
	\textbf{1977普通高等学校招生考试(上海卷)}

	\textbf{\Huge 理科数学}
\end{center}
\begin{questions}
	\question
	\begin{enumerate}[label=(\arabic*)]
		\item 化简:
		      \begin{math} \displaystyle
			      \left(\frac{a}{a+b} - \frac{a^2}{a^2 + 2ab + b^2}\right) \div \left(\frac{a}{a+b} - \frac{a^2}{a^2 - b^2}\right)
		      \end{math}.
		      \begin{solution}
			      \begin{align*}
				      \left(\frac{a}{a+b} - \frac{a^2}{a^2 + 2ab + b^2}\right) \div \left(\frac{a}{a+b} - \frac{a^2}{a^2 - b^2}\right)
				       & = \left[ \frac{a(a+b)}{(a+b)^2} - \frac{a^2}{(a+b)^2} \right] \div \frac{a(a-b) - a^2}{a^2-b^2}
				      \\
				       & = \frac{ab}{(a+b)^2} \cdot \frac{a^2 - b^2}{-ab}                                                \\
				       & = \frac{b-a}{a+b}
			      \end{align*}
		      \end{solution}
		\item 计算:
		      \begin{math}
			      \displaystyle
			      \frac12\lg25+\lg2 - \lg\sqrt{0.1} - \log_29 \times \log_32.
		      \end{math}
		      \begin{solution}
			      \begin{align*}
				       & = \frac12\lg5^2 + \lg2 - \lg(10)^{-\frac12} - \log_23^2 \times \log_32 \\
				       & = \lg5 + \lg2 + \frac12 - 2 \log_23 \times \log_32                     \\
				       & = \frac32 - 2 \cdot \frac{\ln3}{\ln2}\cdot \frac{\ln2}{\ln3}           \\
				       & = \frac12
			      \end{align*}
		      \end{solution}
		\item $\sqrt{-1} = i$,验算$i$是否方程$2x^4 + 3x^3 - 3x^2 + 3x - 5 = 0$的解。
		      \begin{solution}
			      将$i$代入方程得:
			      \begin{align*}
				      2i^4 + 3i^3 - 3x^2 + 3x - 5 & = 2 - 3i + 3  + 3i - 5 \\
				                                  & = 0
			      \end{align*}
			      所以$i$是原方程的解。
		      \end{solution}
		\item 求证:$\displaystyle
			      \frac{\sin\left(\frac{\pi}{4} + \theta \right)}{\sin \left( \frac{\pi}{4} - \theta \right)} +
			      \frac{\cos\left(\frac{\pi}{4} + \theta \right)}{\cos \left( \frac{\pi}{4} - \theta \right)} =
			      \frac{2}{\cos2\theta}
		      $.
		      \begin{solution}
			      根据和角公式
			      \[
				      \begin{array}{l}
					      \sin(\alpha+\beta) = \sin\alpha\cos\beta + \cos\alpha\sin\beta, \\
					      \cos(\alpha+\beta)=\cos\alpha\cos\beta - \sin\alpha\sin\beta
				      \end{array}
			      \]
			      及差角公式$$
				      \begin{array}{l}
					      \sin(\alpha-\beta) = \sin\alpha\cos\beta - \cos\alpha\sin\beta, \\
					      \cos(\alpha-\beta)=\cos\alpha\cos\beta + \sin\alpha\sin\beta
				      \end{array}
			      $$来化简等式的左边得:
			      \begin{align*}
				       & = 	\frac{\sin\frac{\pi}{4}\cos\theta +
					      \cos\frac{\pi}{4}\sin\theta}{\sin\frac{\pi}{4}\cos\theta-\cos\frac{\pi}{4}\sin\theta} +
				      \frac{\cos\frac{\pi}{4}\cos\theta - \sin\frac{\pi}{4}\sin\theta}{\cos\frac{\pi}{4}\cos\theta +
				      \sin\frac{\pi}{4}\sin\theta}                                                              \\
				       & =
				      \frac{\sin\theta + \cos\theta}{\cos\theta - \sin\theta} + \frac{\cos\theta -
				      \sin\theta}{\cos\theta + \sin\theta}                                                      \\
				       & =
				      \frac{1 + 2\sin\theta\cos\theta + 1 - 2\sin\theta\cos\theta}{\cos^2\theta - \sin^2\theta} \\
				       & = \frac{2}{\cos2\theta}
			      \end{align*}
			      因此原式成立。
		      \end{solution}
	\end{enumerate}
	\question 在$\triangle{ABC}$中,$\angle{C}$的平分线交$AB$于$D$,过$D$作$BC$的平行线交$AC$于$E$,已知$BC=a, AC=b$,求$DE$的长。

	\begin{figure}[htbp]
		\centering
		\begin{tikzpicture}
			\tkzDefPoints{0/0/A, 3/0/C, 2/2.5/B}
			\tkzDefLine[bisector](B,C,A) \tkzGetPoint{x}
			\tkzInterLL(C,x)(A,B) \tkzGetPoint{D}
			\tkzDefLine[parallel=through D](B,C) \tkzGetPoint{x}
			\tkzInterLL(D,x)(A,C) \tkzGetPoint{E}

			\tkzDrawPolygon(A,B,C)
			\tkzDrawSegments(D,E D,C)

			\tkzLabelPoints[below](A,E,C)
			\tkzLabelPoints[above left](D,B)
		\end{tikzpicture}
	\end{figure}

	\begin{solution}
		\begin{align*}
			 & \because \angle{ACD} = \angle{BCD}                  \\
			 & \therefore AD = BD                                  \\
			 & \because DE \parallel BC                            \\
			 & \therefore
			\begin{array}{l}
				AE = EC \\
				\dfrac{DE}{BC} = \dfrac{AE}{AC}
			\end{array}                         \\
			 & \therefore DE = BC\frac{\frac{1}{2}a}{a} = \frac12b
		\end{align*}
	\end{solution}
	\pagebreak
	\question
	已知圆$A$的直径为$2\sqrt{3}$,圆$B$的直径为$4-2\sqrt{3}$,圆$C$的直径为$2$,圆$A$与圆$B$外切,圆$A$又与圆$C$外切,$\angle{A}=60^\circ$,求$BC$及$\angle{C}$。

	\begin{figure*}[ht]
		\centering
		\begin{tikzpicture}
			\tkzDefPoints{{sqrt(3)}/0/A, {sqrt(3)-2}/0/B, 0/0/O}
			\tkzDefShiftPoint[A](120:{sqrt(3)+1}){C}
			\tkzDefLine[orthogonal=through C](B,A) \tkzGetPoint{x}
			\tkzInterLL(C,x)(A,B) \tkzGetPoint{D}
			\tkzInterLC(A,C)(A,O) \tkzGetSecondPoint{x}

			\tkzDrawCircles(A,O B,O)
			\tkzDrawCircle(C,x)
			\tkzDrawPolygon(A,B,C)

			\tkzLabelPoints(B,A,D)
			\tkzLabelPoint[above](C){$C$}
			\tkzMarkAngle[size=.5](C,A,B)
			\tkzLabelAngle(C,A,B){$60^\circ$}

			\tkzDrawSegment[dashed](C,D)
		\end{tikzpicture}
	\end{figure*}

	\begin{solution}
		由题意得:
		\begin{align*}
			AB & = 2            \\
			AC & = 1 + \sqrt{3}
		\end{align*}
		由余弦定理有:
		\begin{align*}
			BC^2 & = AB^2 + AC^2 - 2\cdot AB \cdot AC \cdot \cos60^\circ           \\
			     & = 4 + 2\sqrt{3} + 4 - 2\cdot 2 \cdot (1+\sqrt{3}) \cdot \frac12 \\
			     & = 6
		\end{align*}
		所以
		\begin{equation*}
			BC = \sqrt{6}
		\end{equation*}

		再由余弦定理有:
		\begin{align*}
			\cos{C} & = \frac{AC^2 + BC^2 - AB^2}{2AC\cdot BC}                            \\
			        & = \frac{(1+\sqrt{3})^2 + (\sqrt{6})^2 - 2^2}{2(1+\sqrt{3})\sqrt{6}} \\
			        & = \frac{4 + 2\sqrt{3} + 6 - 4}{2\sqrt{2}(3+\sqrt{3})}               \\
			        & = \frac{2(3+\sqrt{3})}{2\sqrt{2}(3+\sqrt{3})}                       \\
			        & = \frac{\sqrt{2}}{2}
		\end{align*}
		所以有:
		\begin{equation*}
			\angle{C} = 45^\circ
		\end{equation*}
	\end{solution}

	\question 正六棱锥$V-ABCDEF$的高为$2$cm,底面边长为$2$cm。
	\begin{enumerate}[label=(\arabic*)]
		\item 按$1:1$画出它的二视图;
		\item 求其侧面积;
		\item 求它的侧棱和底面的夹角。
	\end{enumerate}

	\begin{solution}

		\begin{minipage}[b]{0.4\textwidth}
			\tdplotsetmaincoords{90}{0}
			\centering
			\begin{tikzpicture}[tdplot_main_coords]
				\tkzDefPoint(0:2){A}
				\tkzDefPoint(60:2){B}
				\tkzDefPoint(120:2){C}
				\tkzDefPoint(180:2){D}
				\tkzDefPoint(240:2){E}
				\tkzDefPoint(300:2){F}
				\coordinate(V) at (0,0,2);

				\draw (A) -- (B) -- (C) -- (D) -- (E) -- (F) -- cycle;
				\tkzDrawSegments(V,A V,B V,C V,D V,E V,F)

			\end{tikzpicture}
			\captionof*{figure}{正视图}

		\end{minipage}
		\begin{minipage}[b]{0.4\textwidth}
			\centering
			\tdplotsetmaincoords{0}{0}
			\begin{tikzpicture}[tdplot_main_coords]
				\tkzDefPoint(0:2){A}
				\tkzDefPoint(60:2){B}
				\tkzDefPoint(120:2){C}
				\tkzDefPoint(180:2){D}
				\tkzDefPoint(240:2){E}
				\tkzDefPoint(300:2){F}
				\coordinate(V) at (0,0,2);

				\draw (A) -- (B) -- (C) -- (D) -- (E) -- (F) -- cycle;
				\tkzDrawSegments(V,A V,B V,C V,D V,E V,F)
			\end{tikzpicture}
			\captionof*{figure}{顶视图}
		\end{minipage}

		\begin{minipage}{\textwidth}
			\tdplotsetmaincoords{70}{120}
			\centering
			\begin{tikzpicture}[tdplot_main_coords, scale=2]
				\tkzDefPoint(0,0){O}
				\tkzDefPoint(0:2){A}
				\tkzDefPoint(60:2){B}
				\tkzDefPoint(120:2){C}
				\tkzDefPoint(180:2){D}
				\tkzDefPoint(240:2){E}
				\tkzDefPoint(300:2){F}
				\coordinate(V) at (0,0,2);
				\tkzDefPointOnLine[pos=0.5](B,C) \tkzGetPoint{G}

				\draw (F) -- (A) -- (B) -- (C);
				\draw[dashed] (C)-- (D) -- (E) -- (F);
				\tkzDrawSegment[blue](V,G)
				\tkzDrawSegment[dashed, blue](O,G)
				\tkzDrawSegments(V,A V,B V,C  V,F)
				\tkzDrawSegments[dashed](V,D V,E)
				\tkzDrawSegments[dashed](V,O O,A O,B)
				\tkzLabelSegment[above, sloped](O,V){$h=2\text{cm}$}
				\tkzLabelSegment[above, sloped](O,A){$2\text{cm}$}
				\tkzLabelPoint(O){$O$}
				\tkzLabelPoint(G){$G$}
				\tkzLabelPoint(B){$B$}
				\tkzLabelPoint(A){$A$}
				\tkzLabelPoint(C){$C$}
				\tkzLabelPoint[above](V){$V$}
			\end{tikzpicture}
		\end{minipage}

		\begin{align*}
			 & \because \text{底面是正六边形}                                \\
			 & \therefore \angle{AOB} = 60^\circ                      \\
			 & \because AO = BO                                       \\
			 & \therefore \triangle{AOB} \text{是正三角形}                 \\
			 & \therefore AO = AB = 2cm                               \\
			 & \because \angle{AOV} = 90^\circ                        \\
			 & \therefore VA = \sqrt{2^2 + 2^2} = 2\sqrt{2} \text{cm} \\
			 & \therefore \text{侧棱长为}2\sqrt{2}\mathrm{cm}
		\end{align*}

		取$BC$的中点为$G$,连接$OG$和$VG$。
		\begin{align*}
			 & \because \angle{BOG} = 30^\circ                                             \\
			 & \therefore BG = 2\sin30^\circ = 1 \text{cm}                                 \\
			 & \therefore VG = \sqrt{VG^2 - BG^2} = \sqrt{7}\text{cm}                      \\
			 & \therefore S_{\text{侧}} = \frac12VG\cdot BC \times 6 = 6\sqrt{7}\text{cm}^2 \\
			 & \because
			\begin{array}{l}
				VO = OA \\
				\angle{VOA} = 90^\circ
			\end{array}                                                          \\
			 & \therefore \triangle{VOA}\text{是等腰直角三角形}                                    \\
			 & \therefore \angle{VAO} = 45^\circ (\text{即侧棱与底面的夹角为}45^\circ)
		\end{align*}
	\end{solution}

	\question 解不等式: 	\begin{math}
		\begin{cases}
			16 - x^2 \geqslant 0 \\
			x^2 - x - 6 > 0
		\end{cases},并在数轴上把它的解表示出来。
	\end{math}
	\begin{solution}
		\begin{enumerate}[label=\Roman*.]
			\item 解不等式$16 - x^2 \geqslant 0$
			      \begin{align*}
				      16  \geqslant x^2 \\
				      -4 \leqslant x \leqslant 4
			      \end{align*}
			\item 解不等式$x^2 - x - 6 > 0$
			      \begin{align*}
				      (x-3)(x+2) > 0 \\
				      x > 3, x < -2
			      \end{align*}
			\item 绘制数轴

			      \begin{tikzpicture}
				      % 绘制数轴
				      \draw[->] (-5,0) -- (5,0) node[right] {$x$};

				      % 标记关键点
				      \foreach \x in {-4, -2, 3, 4} {
						      \path  (\x, 0) node[below] {\x};
					      }

				      % 绘制闭合圆圈
				      \filldraw[black] (-4,0) circle (1pt);
				      \filldraw[black] (4,0) circle (1pt);

				      % 绘制开口圆圈
				      \draw[black] (-2,0) circle (1pt);
				      \draw[black] (3,0) circle (1pt);

				      % 绘制范围
				      \draw[thick] (-4,0)--(-4,0.5) -- (-2,0.5) -- (-2,0);
				      \draw[thick] (3,0)--(3,0.5) -- (4,0.5) -- (4,0);

			      \end{tikzpicture}
		\end{enumerate}
	\end{solution}

\end{questions}

\end{document}
