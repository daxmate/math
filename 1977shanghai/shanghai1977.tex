%!tex program = lualatex
\documentclass[answers]{exam}
\usepackage{ctex}
\usepackage{graphicx}
\usepackage[margin=2cm]{geometry}
\usepackage{amsmath, amssymb}
\usepackage{csquotes}
\usepackage{tikz, pgfplots}
\usetikzlibrary{
	angles,
	backgrounds,
	calc,
	decorations.pathmorphing,
	decorations.pathreplacing,
	decorations.text,
	intersections,
	patterns,
	quotes,
	shapes,
	shapes.symbols,
}
\pagestyle{empty}
\newcounter{xcord}
\newcounter{ycord}
\newcounter{total}
\renewcommand{\labelenumi}{\textbf{\ifnum\value{enumi}<10 0\fi\arabic{enumi})}}

\pgfplotsset{compat=1.18}

\CorrectChoiceEmphasis{\color{blue!70!green}\bfseries}
\renewcommand{\solutiontitle}{\textbf{解:}}

\usepackage{array, tabularx}
\newcolumntype{C}{>{\centering\arraybackslash}X}
\newcolumntype{B}{>{\centering\bfseries\arraybackslash}X}
\catcode`\幺=0

\usepackage[lua]{tkz-euclide}

\begin{document}
\begin{center}
	\textbf{1977普通高等学校招生考试(上海卷)}

	\textbf{\Huge 理科数学}
\end{center}
\begin{questions}
	\question
	\begin{enumerate}[label=(\arabic*)]
		\item 化简:
		      \begin{math} \displaystyle
			      \left(\frac{a}{a+b} - \frac{a^2}{a^2 + 2ab + b^2}\right) \div \left(\frac{a}{a+b} - \frac{a^2}{a^2 - b^2}\right)
		      \end{math}.
		      \begin{solution}
			      \begin{align*}
				      \left(\frac{a}{a+b} - \frac{a^2}{a^2 + 2ab + b^2}\right) \div \left(\frac{a}{a+b} - \frac{a^2}{a^2 - b^2}\right)
				       & = \left[ \frac{a(a+b)}{(a+b)^2} - \frac{a^2}{(a+b)^2} \right] \div \frac{a(a-b) - a^2}{a^2-b^2}
				      \\
				       & = \frac{ab}{(a+b)^2} \cdot \frac{a^2 - b^2}{-ab}                                                \\
				       & = \frac{b-a}{a+b}
			      \end{align*}
		      \end{solution}
		\item 计算:
		      \begin{math}
			      \displaystyle
			      \frac12\lg25+\lg2 - \lg\sqrt{0.1} - \log_29 \times \log_32.
		      \end{math}
		      \begin{solution}
			      \begin{align*}
				       & = \frac12\lg5^2 + \lg2 - \lg(10)^{-\frac12} - \log_23^2 \times \log_32 \\
				       & = \lg5 + \lg2 + \frac12 - 2 \log_23 \times \log_32                     \\
				       & = \frac32 - 2 \cdot \frac{\ln3}{\ln2}\cdot \frac{\ln2}{\ln3}           \\
				       & = \frac12
			      \end{align*}
		      \end{solution}
		\item $\sqrt{-1} = i$,验算$i$是否方程$2x^4 + 3x^3 - 3x^2 + 3x - 5 = 0$的解。
		      \begin{solution}
			      将$i$代入方程得:
			      \begin{align*}
				      2i^4 + 3i^3 - 3x^2 + 3x - 5 & = 2 - 3i + 3  + 3i - 5 \\
				                                  & = 0
			      \end{align*}
			      所以$i$是原方程的解。
		      \end{solution}
		\item 求证:$\displaystyle
			      \frac{\sin\left(\frac{\pi}{4} + \theta \right)}{\sin \left( \frac{\pi}{4} - \theta \right)} +
			      \frac{\cos\left(\frac{\pi}{4} + \theta \right)}{\cos \left( \frac{\pi}{4} - \theta \right)} =
			      \frac{2}{\cos2\theta}
		      $.
		      \begin{solution}
			      根据和角公式 $\sin(\alpha+\beta) = \sin\alpha\cos\beta + \cos\alpha\sin\beta,
				      \cos(\alpha+\beta)=\cos\alpha\cos\beta - \sin\alpha\sin\beta$及差角公式$\sin(\alpha-\beta) =
				      \sin\alpha\cos\beta - \cos\alpha\sin\beta, \cos(\alpha-\beta)=\cos\alpha\cos\beta +
				      \sin\alpha\sin\beta$来化简等式的左边得:
			      \begin{align*}
				       & = 	\frac{\sin\frac{\pi}{4}\cos\theta +
					      \cos\frac{\pi}{4}\sin\theta}{\sin\frac{\pi}{4}\cos\theta-\cos\frac{\pi}{4}\sin\theta} +
				      \frac{\cos\frac{\pi}{4}\cos\theta - \sin\frac{\pi}{4}\sin\theta}{\cos\frac{\pi}{4}\cos\theta +
				      \sin\frac{\pi}{4}\sin\theta}                                                              \\
				       & =
				      \frac{\sin\theta + \cos\theta}{\cos\theta - \sin\theta} + \frac{\cos\theta -
				      \sin\theta}{\cos\theta + \sin\theta}                                                      \\
				       & =
				      \frac{1 + 2\sin\theta\cos\theta + 1 - 2\sin\theta\cos\theta}{\cos^2\theta - \sin^2\theta} \\
				       & = \frac{2}{\cos2\theta}
			      \end{align*}
			      因此原式成立。
		      \end{solution}
	\end{enumerate}
	\question 在$\triangle{ABC}$中,$\angle{C}$的平分线交$AB$于$D$,过$D$作$BC$的平行线交$AC$于$E$,已知$BC=a, AC=b$,求$DE$的长。

	\begin{figure}[htbp]
		\centering
		\begin{tikzpicture}
			\tkzDefPoints{0/0/A, 3/0/C, 2/2.5/B}
			\tkzDefLine[bisector](B,C,A) \tkzGetPoint{x}
			\tkzInterLL(C,x)(A,B) \tkzGetPoint{D}
			\tkzDefLine[parallel=through D](B,C) \tkzGetPoint{x}
			\tkzInterLL(D,x)(A,C) \tkzGetPoint{E}

			\tkzDrawPolygon(A,B,C)
			\tkzDrawSegments(D,E D,C)

			\tkzLabelPoints[below](A,E,C)
			\tkzLabelPoints[above left](D,B)
		\end{tikzpicture}
	\end{figure}

	\begin{solution}
		\begin{align*}
			 & \because \angle{ACD} = \angle{BCD}                  \\
			 & \therefore AD = BD                                  \\
			 & \because DE \parallel BC                            \\
			 & \therefore
			\begin{cases}
				AE = EC \\
				\frac{DE}{BC} = \frac{AE}{AC}
			\end{cases}                           \\
			 & \therefore DE = BC\frac{\frac{1}{2}a}{a} = \frac12b
		\end{align*}
	\end{solution}

\end{questions}

\end{document}
