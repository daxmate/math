%!tex program = lualatex
\documentclass[answers]{exam}
\usepackage{silence}
\WarningFilter{latexfont}{Font shape}
\WarningFilter{latexfont}{Some font}
\usepackage{ctex}
\usepackage[margin=2cm]{geometry}
\usepackage{amsmath, amssymb, amsthm, arcs}
\usepackage{siunitx}
\usepackage{csquotes}
\usepackage{tikz, pgfplots, tikz-3dplot}
\usepackage[lua]{tkz-euclide}
\usetikzlibrary{
	angles,
	backgrounds,
	calc,
	decorations.pathmorphing,
	decorations.pathreplacing,
	decorations.text,
	intersections,
	patterns,
	petri,
	positioning,
	quotes,
	shapes,
	shapes.symbols,
}
\usepackage{graphicx, wrapfig}
\usepackage{caption}
\pagestyle{empty}
\newcounter{xcord}
\newcounter{ycord}
\newcounter{total}
\renewcommand{\labelenumi}{\textbf{\ifnum\value{enumi}<10 0\fi\arabic{enumi})}}

\pgfplotsset{compat=1.18}

\CorrectChoiceEmphasis{\color{blue!70!green}\bfseries}
\renewcommand{\solutiontitle}{\textbf{解:}}
\renewcommand{\choicelabel}{(\Alph{choice})}

\usepackage{array, tabularx}
\newcolumntype{C}{>{\centering\arraybackslash}X}
\newcolumntype{B}{>{\centering\bfseries\arraybackslash}X}

\usepackage{fontawesome5}
\usepackage{enumitem}
\newcommand\regularHandPointRight{\faHandPointRight[regular]}

\newenvironment{mathenum}
{\begin{enumerate}[label=\arabic*.]}
		{\end{enumerate}}

\newenvironment{penum}
{\begin{enumerate}[label=(\arabic*)]}
		{\end{enumerate}}
\makeatletter
\providecommand\@gobblethree[3]{}
\patchcmd{\over@under@arc}
{\@gobbletwo}
{\@gobblethree}
{}{}
\makeatother

\newcommand*\circled[1]{\tikz[baseline=(char.base)]{
		\node[shape=circle,draw,inner sep=1pt] (char) {#1};}}
\setlength{\answerclearance}{6pt}
\newcommand\hfs{\hfill (\hspace{2cm})}


\begin{document}
\begin{center}
	\textbf{1977年普通高等学校招生考试(河北卷)}

	\section*{数学试卷}
\end{center}
\begin{questions}
	\question 解答下列各题
	\begin{enumerate}[label=(\arabic*)]
		\item 叙述函数的定义
		      \begin{solution}
			      函数是从一个集合到另一个集合之间的映射,使得每个输入值有且公有一个输出值。
		      \end{solution}
		\item 求函数$y=1-\frac{1}{\sqrt{2-3x}}$的定义域。
		      \begin{solution}
			      由函数的定义有:
			      \begin{math}
				      2-3x > 0
			      \end{math},则函数的定义域为:$x<\frac23$
		      \end{solution}
		\item 计算:$\left[1-(0.5)^{-2}\right] \div \left(-\frac{27}{8}\right)^{\frac13}$.
		      \begin{solution}
			      \begin{align*}
				      \left[1-(0.5)^{-2}\right] \div \left(-\frac{27}{8}\right)^{\frac13} & = \left(1-2^{-1\times (-2)}\right)
				      \div \left(-\frac32\right)^{3\times\frac13}                                                              \\
				                                                                          & = (1 - 4) \cdot (-\frac32)         \\
				                                                                          & = \frac92
			      \end{align*}
		      \end{solution}
		\item 计算:$\log_42$。
		      \begin{solution}
			      \begin{align*}
				      \log_42 & = \log_44^\frac12 \\
				              & = \frac12
			      \end{align*}
		      \end{solution}
		\item 分解因式:$x^2y - 2y^3$。
		      \begin{solution}
			      \begin{align*}
				      x^2y - 2y^3 & = y(x^2 - 2y^2)               \\
				                  & = y(x+\sqrt{2}y)(x-\sqrt{2}y)
			      \end{align*}
		      \end{solution}
		\item 计算:$\sin\dfrac{4\pi}{3}\cdot\cos\dfrac{25\pi}{6}\cdot\tan\left(-\dfrac{3\pi}{4}\right)$。
		      \begin{solution}
			      \begin{align*}
				      \sin\frac{4\pi}{3}\cdot\cos\frac{25\pi}{6}\cdot\tan\left(-\frac{3\pi}{4}\right)
				       & = -\sin\frac{\pi}{3}\cdot\cos\frac{\pi}{6}\cdot \tan\frac{3\pi}{4} \\
				       & = -\frac{\sqrt{3}}{2} \cdot \frac{\sqrt{3}}{2} \cdot 1             \\
				       & = - \frac43
			      \end{align*}
		      \end{solution}
	\end{enumerate}

\end{questions}

\end{document}
