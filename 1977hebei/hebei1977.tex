%!tex program = lualatex
\documentclass[answers]{exam}
\usepackage{ctex}
\usepackage{graphicx}
\usepackage[margin=2cm]{geometry}
\usepackage{amsmath, amssymb}
\usepackage{csquotes}
\usepackage{tikz, pgfplots}
\usetikzlibrary{
	angles,
	backgrounds,
	calc,
	decorations.pathmorphing,
	decorations.pathreplacing,
	decorations.text,
	intersections,
	patterns,
	quotes,
	shapes,
	shapes.symbols,
}
\pagestyle{empty}
\newcounter{xcord}
\newcounter{ycord}
\newcounter{total}
\renewcommand{\labelenumi}{\textbf{\ifnum\value{enumi}<10 0\fi\arabic{enumi})}}

\pgfplotsset{compat=1.18}

\CorrectChoiceEmphasis{\color{blue!70!green}\bfseries}
\renewcommand{\solutiontitle}{\textbf{解:}}

\usepackage{array, tabularx}
\newcolumntype{C}{>{\centering\arraybackslash}X}
\newcolumntype{B}{>{\centering\bfseries\arraybackslash}X}
\catcode`\幺=0

\usepackage[lua]{tkz-euclide}

\begin{document}
\begin{center}
	\textbf{1977年普通高等学校招生考试(河北卷)}

	\section*{数学试卷}
\end{center}
\begin{questions}
	\question 解答下列各题
	\begin{enumerate}[label=(\arabic*)]
		\item 叙述函数的定义
		      \begin{solution}
			      函数是从一个集合到另一个集合之间的映射,使得每个输入值有且公有一个输出值。
		      \end{solution}
		\item 求函数$y=1-\frac{1}{\sqrt{2-3x}}$的定义域。
		      \begin{solution}
			      由函数的定义有:
			      \begin{math}
				      2-3x > 0
			      \end{math},则函数的定义域为:$x<\frac23$
		      \end{solution}
		\item 计算:$\left[1-(0.5)^{-2}\right] \div \left(-\frac{27}{8}\right)^{\frac13}$.
		      \begin{solution}
			      \begin{align*}
				      \left[1-(0.5)^{-2}\right] \div \left(-\frac{27}{8}\right)^{\frac13} & = \left(1-2^{-1\times (-2)}\right)
				      \div \left(-\frac32\right)^{3\times\frac13}                                                              \\
				                                                                          & = (1 - 4) \cdot (-\frac32)         \\
				                                                                          & = \frac92
			      \end{align*}
		      \end{solution}
		\item 计算:$\log_42$。
		      \begin{solution}
			      \begin{align*}
				      \log_42 & = \log_44^\frac12 \\
				              & = \frac12
			      \end{align*}
		      \end{solution}
		\item 分解因式:$x^2y - 2y^3$。
		      \begin{solution}
			      \begin{align*}
				      x^2y - 2y^3 & = y(x^2 - 2y^2)               \\
				                  & = y(x+\sqrt{2}y)(x-\sqrt{2}y)
			      \end{align*}
		      \end{solution}
		\item 计算:$\sin\dfrac{4\pi}{3}\cdot\cos\dfrac{25\pi}{6}\cdot\tan\left(-\dfrac{3\pi}{4}\right)$。
		      \begin{solution}
			      \begin{align*}
				      \sin\frac{4\pi}{3}\cdot\cos\frac{25\pi}{6}\cdot\tan\left(-\frac{3\pi}{4}\right)
				       & = -\sin\frac{\pi}{3}\cdot\cos\frac{\pi}{6}\cdot \tan\frac{3\pi}{4} \\
				       & = -\frac{\sqrt{3}}{2} \cdot \frac{\sqrt{3}}{2} \cdot 1             \\
				       & = - \frac43
			      \end{align*}
		      \end{solution}
	\end{enumerate}

	\question
	\begin{minipage}[t]{.3\textwidth}
		证明:如图,$AB$是圆$O$的直径,$CB$是圆$O$的切线,切点为$B$,$OC$平行于弦$AD$,求证:$DC$是圆$O$的切线。
	\end{minipage}\hspace{3cm}
	\begin{tikzpicture}[scale=.5, baseline=(current bounding box.center)]
		\tkzDefPoints{-2/0/A, 0/0/O, 2/0/B, 2/5/C}
		\tkzDefLine[parallel=through A](O,C) \tkzGetPoint{x}
		\tkzInterLC(A,x)(O,B) \tkzGetPoints{D}{A}

		\tkzDrawCircle(O,B)
		\tkzDrawSegment(A,B)
		\tkzDrawSegment(C,B)
		\tkzDrawSegment(C,O)
		\tkzDrawSegment(A,D)
		\tkzDrawSegment(C,D)
		\tkzDrawSegment[dashed, red](D,O)

		\tkzMarkAngle[mark=|, size=15pt](O,A,D)
		\tkzMarkAngle[mark=|, size=15pt](B,O,C)
		\tkzMarkAngle[mark=||, size=20pt](A,D,O)
		\tkzMarkAngle[mark=||, size=20pt](C,O,D)

		\tkzLabelPoint[left](A){$A$}
		\tkzLabelPoint[right](B){$B$}
		\tkzLabelPoint[above](C){$C$}
		\tkzLabelPoint[below](O){$O$}
		\tkzLabelPoint[left](D){$D$}
	\end{tikzpicture}

	\begin{solution}
		\begin{enumerate}[label=\arabic*.]
			\item 因为 $CB$ 是圆 $O$ 的切线且切点为 $B$,所以有 $AB \perp CB$。
			\item 由于 $AD \parallel OC$,可以得到 $\angle DAO = \angle COB$。
			\item 作辅助线 $DO$。因为 $\angle ADO$ 和 $\angle COD$ 是平行线的内错角,所以 $\angle ADO = \angle COD$。
			\item 在 $\triangle AOD$ 中,由于 $AD \parallel OC$,且 $OA = OD$(都是圆的半径),所以 $\triangle AOD$ 是等腰三角形,由此得到 $\angle OAD = \angle ODA$。
			\item 进一步分析 $\triangle DOC$ 和 $\triangle BOC$:
			      \begin{itemize}
				      \item $OB = OD$(都是圆的半径);
				      \item $OC = OC$(公共边);
				      \item $\angle BOC = \angle DOC$(由 $\angle COB = \angle DAO$ 和 $\angle ADO = \angle COD$ 推导)。
			      \end{itemize}
			      因此,$\triangle BOC \cong \triangle DOC$(根据边角边准则)。
			\item 由 $\triangle BOC \cong \triangle DOC$,可以得到 $\angle ODC = \angle CBO = 90^\circ$。
			\item 因为 $\angle ODC = 90^\circ$,所以 $DC$ 是圆 $O$ 的切线。
		\end{enumerate}
	\end{solution}

	\question 证明: $\dfrac{\sin2\alpha + 1}{1+\cos2\alpha + \sin2\alpha} = \dfrac12\tan\alpha + \dfrac12$.
	\begin{mathenum}
		\item 根据倍角公式$\sin2\alpha = 2\sin\alpha\cos\alpha$和$\cos2\alpha=\cos^2\alpha -
			\sin^2\alpha$来化简等式的左边得:
		\begin{align*}
			\frac{\sin2\alpha + 1}{1+\cos2\alpha + \sin2\alpha}
			 & = \frac{2\sin\alpha\cos\alpha + 1}{1 + \cos^2\alpha - \sin^2\alpha + 2\sin\alpha\cos\alpha} \\
			 & = \frac{\cos^2\alpha + \sin^2\alpha + 2\sin\alpha\cos\alpha}{\cos^2\alpha + \sin^2\alpha +
			\cos^2\alpha - \sin^2\alpha + 2\sin\alpha\cos\alpha}                                           \\
			 & = \frac{(\sin\alpha + \cos\alpha)^2}{2\cos\alpha(\sin\alpha + \cos\alpha)}                  \\
			 & = \frac{\sin\alpha + \cos\alpha}{2\cos\alpha}                                               \\
			 & = \frac12\tan\alpha + \frac12
		\end{align*}
		\item 综上,等式成立。
	\end{mathenum}

	\question 已知$2\lg{x} + \lg{2} = \lg(x+6)$,求$x$。
	\begin{solution}
		\begin{align*}
			2\lg{x} + \lg2 & = \lg{x^2} + \lg2 \\
			               & = \lg{2x^2}
		\end{align*}
		根据等式则有:
		\begin{equation*}
			2x^2 = x + 6 \tag{a}
		\end{equation*}
		对等式(a)整理得:
		\begin{align*}
			2x^2 - x - 6  & = 0 \\
			(2x + 3)(x-2) & = 0 \\
		\end{align*}
		因为$x>0$所以$x=2$。
	\end{solution}

	\question
	\begin{minipage}[t]{.6\textwidth}
		某生产队要建立一个形状是直角梯形的苗圃,其两邻边借用夹角为$135^\circ$的两面墙,另外两边是总长为$30$米的篱笆(如图,$AD$和$DC$为墙),问篱笆的两边各多长时,苗圃的面积最大?最大面积是多少?
	\end{minipage}
	\hspace{1cm}
	\begin{tikzpicture}[baseline=(current bounding box.north)]
		\tkzDefPoints{0/0/A, 3/0/B}
		\tkzDefPoint(45:2){D}
		\tkzDefLine[parallel=through D](A,B) \tkzGetPoint{x}
		\tkzDefLine[orthogonal=through B](A,B) \tkzGetPoint{y}
		\tkzInterLL(D,x)(B,y) \tkzGetPoint{C}

		\tkzDefLine[orthogonal=through D](A,B) \tkzGetPoint{x}
		\tkzInterLL(D,x)(A,B) \tkzGetPoint{E}

		\tkzDrawPolygon(A,B,C,D)
		\tkzLabelPoints[below](A,B,E)
		\tkzLabelPoints[above](C,D)
		\tkzDrawSegment[blue, dashed](D,E)
	\end{tikzpicture}

	\begin{solution}
		\begin{mathenum}
			\item 从$D$点向$AB$作垂线,垂足为$E$.
			\item 从条件可知$AE=ED=BC$
			\item 设$AE=x$,则有$DC=EB=30-2x$
			\item 梯形的面积可以表示为:
			\begin{align*}
				S & = (30 - 2x + x + 30 - 2x)\cdot x / 2 \\
				  & = (60 - 3x)x/2                       \\
				  & = -\frac32x^2 + 30x \tag{1}
			\end{align*}
		\item 根据抛物线的性质在$x=-\frac{b}{2a}$时有极值,代入可得$x=10$,此时$AB=20,BC=10$
		\item 将$x=10$代入式(1)可得最大面积为$150\mathrm{m}^2$
		\end{mathenum}
	\end{solution}

\end{questions}

\end{document}
